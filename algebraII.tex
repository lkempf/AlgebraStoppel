\documentclass[12pt,a4paper]{scrartcl}

\usepackage{includes}
\usepackage{shortcuts}
\usepackage{numbering}

%%%%%%%%%%%%%%%%%%%%%%%%%%%%%%%%%%%%%%%%%%%%%%%%%%%%%%%%%%%%%%
% Stroppel hat jetzt leider andere Nummerierungsvorlieben... %
%%%%%%%%%%%%%%%%%%%%%%%%%%%%%%%%%%%%%%%%%%%%%%%%%%%%%%%%%%%%%%

\setlist[enumerate,1]{label=\textup{\arabic*)}}

\renewcommand{\thethmcounter}{\Roman{section}.\arabic{thmcounter}}

\counterwithin{thmcounter}{section}
\counterwithin{subsection}{section}

\theoremstyle{cplain}
\newtheorem{cor}[thmcounter]{Corollary}
\crefname{cor}{Corollary}{Corollaries}

\theoremstyle{cplain}
\newtheorem{thm}[thmcounter]{Theorem}
\crefname{thm}{Theorem}{Theorems}

\theoremstyle{cplain}
\newtheorem{prop}[thmcounter]{Proposition}
\crefname{prop}{Proposition}{Propositions}

\theoremstyle{definition}
\newtheorem*{deff}{Definition}

% Literatur
\usepackage[backend=biber,sorting=none,style=alphabetic]{biblatex}
\addbibresource{literatur.bib}

\title{Algebra II}
\subtitle{Winter Semester 2018/19}
\date{\lastcompiled}

\begin{document}
\begin{otherlanguage}{english}

\maketitle
\tableofcontents
\newpage

\noindent
These are notes of the lecture \enquote{Algebra II}, taught by Prof. Dr. Catharina Stroppel at the University of Bonn in the winter semester 2018/19.

\bigskip

\noindent
Lecture website:\\
\url{http://guests.mpim-bonn.mpg.de/enorton/alg2.html}

\nocite{hungerford}
\nocite{knapp-basic}
\nocite{knapp-advanced}
\nocite{procesi}
\nocite{borel}
\nocite{humphreys}
\nocite{springer}
\printbibliography

\newpage

\lecture{October 8, 2018}

\section{Group actions}
If $G$ is a group, denote by $e \in G$ the neutral element, by $g^{-1}$ the inverse of $g\in G$ and by $gh$ the composition $g \circ h$.
\begin{deff}
  Given a group $G$ and a set $X$, an \emph{action} of $G$ on $X$ is a map
  \begin{eqnarray*}
    G \times X &\to& X \\
    (g,x) &\mapsto& g.x
  \end{eqnarray*}
  such that
  \begin{description}
   \item[(A1)] $e.X = x$ and
   \item[(A2)] $(gh).x = g.(h.x)$
  \end{description}
  for all $x \in X$ and $g,h \in G$. We call then $X$ a \emph{$G$-set}.
\end{deff}
\begin{deff}
  Given a set $X$, define \[ S(X) := \set{f\colon X \to X \given f \text{ bijective}}, \] the \emph{symmetric group} of $X$ (with composition as group multiplication).
  
  Given a $G$-set $X$ and $g\in G$, let $\pi_g \in S(X)$ be defined as $\pi_g(x) = g.x$.
\end{deff}
\begin{lem}
  For any group $G$ and set $X$ we have a bijective correspondence
  \begin{eqnarray*}
    \set{\text{$G$-actions on $X$}} &\xlongleftrightarrow{1:1}& \set{\text{Group homomorphisms $G \to S(X)$}} \\
    \pi &\mapsto& \hat\pi = (g \mapsto (x \mapsto \pi(g,x) = g.x)) \\
    ((g,x)\mapsto \phi(g)(x))=\mathring \phi &\mapsfrom& \phi.
  \end{eqnarray*}
\end{lem}
\begin{proof}
  Left to the reader.
\end{proof}

\paragraph{Examples.}
Let $G$ be a group.
\begin{enumerate}
  \item $G$ acts on itself by
  \begin{itemize}
    \item left multiplication: $g.x = gx$ (left regular action)
    \item \enquote{right multiplication}: $g.x = xg^{-1}$ (right regular action)
    \item conjugation $g.x = gxg^{-1}$
  \end{itemize}
  \item Any set $X$ is a $G$-set via the \emph{trivial action} $g.x = x$.
  \item Let $X,Y$ be $G$-sets. then $G$ acts on $\Maps(X,Y) := \set{f\colon X\to Y}$ via $(g.f)(x) = g.(f(g^{-1}.x))$. Special case: the action $Y$ is trivial, then $(g.f)(x) = f(g^{-1}.x)$.
\end{enumerate}

\begin{deff}
  Let $X,Y$ be $G$-sets. A map $f\colon X\to Y$ is called \emph{$G$-equivariant} if $f(g.x) = g.f(x)$ for all $g\in G$ and $x \in X$. We write \[\Hom_G(X,Y) := \set{f\colon X\to Y \given \text{$f$ is $G$-equivariant}}.\]
\end{deff}
\begin{lem}
  Let $G$ be a group.
  \begin{enumerate}
    \item If $X$ is a $G$-set then $\id_X\in \Hom_G(X,X)$.
    \item If $X,Y,Z$ are $G$-sets, $f_1 \in \Hom_G(X,Y)$ and $f_2 \in \Hom_G(Y,Z)$ then $f_2 \circ f_1 \in \Hom_G(X,Z)$.
  \end{enumerate}
\end{lem}
\begin{proof}
  Left to the reader.
\end{proof}

\paragraph{Examples.}
Let $G$ be a group.
\begin{enumerate}
  \item If $G$ acts on itself by left multiplication then
  \begin{eqnarray*}
    \Hom_G(G,G) &\cong& G \quad\text{(as sets)} \\
    f &\mapsto& f(e) \\
    (x\mapsto xa) = m_a &\mapsfrom& a.
  \end{eqnarray*}
  \item If $X,Y$ are trivial $G$-sets then $\Hom_G(X,Y) = \Maps(X,Y)$.
\end{enumerate}

% TODO besseres Symbol für Menge der Orbits
\begin{deff}
  Let $X$ be a $G$-set. For $x\in X$ let $G_x = \set{g.x \given g \in G}$ be the \emph{orbit} of $x$. We write \[ \orbits GX := \set{G_x \given x \in X}.\]
\end{deff}
\medskip
Note that $G_x = G_y$ iff $y \in G_x$.

\paragraph{Remark.}
We can view $\orbits GX$ as a $G$-set via the trivial action. Then $\can\colon X \to \orbits GX,x \mapsto G_x$ is $G$-equivariant.

\begin{deff}
  Let $X$ be a $G$-set. Then \[X^G := \set{x \in X \given \forall g\in G: g.x = x}\] is the \emph{set of $G$-fixed points} or \emph{$G$-invariants} in $X$.
\end{deff}
\begin{lem}
  Let $X,Y$ be $G$-sets and $f\in \Hom_G(X,Y)$. Then, $f(X^G) \subseteq Y^G$.
\end{lem}
\begin{proof}
  Let $x\in X^G$. For all $g\in G$, we have $g.f(x) = f(g.x) = f(x)$. Therefore, $f(x) \in Y^G$.
\end{proof}

\medskip
Thus, $f$ induces a map $f^G\colon X^G \to Y^G$ by restriction.

\begin{lem}
  Let $G$ be a group.
  \begin{enumerate}
    \item If $X$ is a $G$-set then $\id_X^G = \id_{X^G}$.
    \item If $X,Y,Z$ are $G$-sets, and $f_1 \in \Hom_G(X,Y)$ and $f_2\in \Hom_G(Y,Z)$ then $(f_2\circ f_1)^G = f_2^G \circ f_1^G$.
  \end{enumerate}
\end{lem}
\begin{proof}
  Left to the reader.
\end{proof}
\begin{lem}
  Let $X,Y$ be $G$-sets. Then $\Hom_G(X,Y) = \Maps(X,Y)^G$.
\end{lem}
\begin{proof}
  $f \in \Hom_G(X,Y) \Leftrightarrow \forall g\in G,x \in X: f(g.x) = g.f(x) \Leftrightarrow \forall g\in G,x \in X:g^{-1}.f(g.x) = g^{-1}.(g.f(x)) = f(x) \Leftrightarrow \forall g\in G,x \in X: g.f(g^{-1}.x) = f(x) \Leftrightarrow f\in \Maps(X,Y)^G$.
\end{proof}
\begin{deff}
  Let $X$ be a $G$ set and $k$ a field. A map $f\colon X\to k $ is \emph{$G$-invariant} if $f(g.x) = f(x)$ for all $g \in G$ and $x\in X$.
\end{deff}

\paragraph{Example.}
Let $G= \fak\IZ{2\IZ} = \set{e,s}$ and $k=\IR$. Let $G$ act on $\IR$ by $s.\lambda = -\lambda$. Any polynomial $p(t) \in \IR[t]$ can be viewed as an element in $\Maps(\IR,\IR)$. Then $p(t) = \sum a_it^i$ is $G$-invariant iff $p(t)$ is even (i.e. $a_i=0$ for odd $i$).
\begin{proof}
  \begin{align*}
    \qedherea
    &\phantom{{}\Leftrightarrow{}}\quad \text{$p(t)$ is $G$-invariant} \\
    &\Leftrightarrow\quad \forall \lambda \in \IR: p(s.\lambda) = p(\lambda) \\
    &\Leftrightarrow\quad \forall \lambda \in \IR: p(-\lambda) = p(\lambda) \\
    &\Leftrightarrow\quad \forall \lambda \in \IR: \sum_i (-1)^ia_i \lambda^i = \sum_i a_i \lambda^i \\
    &\Leftrightarrow\quad \forall \lambda \in \IR: 2 \sum_{\text{$i$ odd}} a_i\lambda^i = 0 \\
    &\Leftrightarrow\quad \text{$a_i=0$ for all odd $i$}
    \qedhere
  \end{align*}
\end{proof}

\paragraph{Remark.}
$f\colon X \to k$ is $G$-invariant iff $f\in \Maps(X,k)^G$ where we have trivial $G$-action on $k$.

\begin{lem}[Universal property of invariant maps]
  Let $X$ be a $G$-set, $k$ a field (or a commutative ring with $1$). Then $f\colon X \to k$ is $G$-invariant iff $f$ factors through $\can$ (i.e. $\exists! \ol f\colon \orbits GX \to k$ such that $f = \ol f \circ \can$).
  \begin{center}
    \begin{tikzcd}
      X \arrow{r}{f} \arrow[swap]{d}{\can} & k \\
      \orbits GX \arrow[dashed,swap]{ru}{\exists!\ol f}
    \end{tikzcd}
  \end{center}
\end{lem}
\begin{proof}
  \begin{align*}
    \qedherea
    &\phantom{{}\Leftrightarrow{}}\quad \text{$f$ is $G$-invariant} \\
    &\Leftrightarrow\quad \forall g\in G, x \in X : f(g.x) = f(x) \\
    &\Leftrightarrow\quad \text{$f$ is constant on orbits} \\
    &\Leftrightarrow\quad \text{$\ol f$ exists (namely $\ol f(G_x) = f(x)$, obviously unique)}
    \qedhere
  \end{align*}
\end{proof}
\begin{lem} \label{lem:I.7}
  Let $X$ be a finite $G$-set and $k$ a field (or commutative ring with $1$). Then:
  \begin{enumerate}
    \item\label{lem:I.7:1} $\Maps(X,k)$ is a $k$-vector space (or $k$-module) with pointwise addition and scalar multiplication.
    \item\label{lem:I.7:2} A $k$-basis of $\Maps(X,k)$ is given by \[ \Xs_x\colon y \mapsto \begin{cases*} 1 & if $x = y$ \\ 0 & otherwise \end{cases*}\] where $x \in X$.
    \item\label{lem:I.7:3} $\Maps(X,k)^G$ forms a subspace (or submodule) with basis \[ \Xs_\Gs \colon y \mapsto \begin{cases*} 1 & if $y \in \Gs$ \\ 0 & otherwise \end{cases*} \] where $\Gs \in \orbits GX$.
  \end{enumerate}
\end{lem}
\begin{proof}
  \leavevmode
  \begin{enumerate}[label=\ref{lem:I.7:\arabic*}]
    \item Clear.
    \item \begin{description}
            \item[Generating system:] Let $f \in \Maps(X,k)$. Then $f= \sum_{x\in X}f(x)\Xs_x$, as we have $\sum_{x\in X}f(x)\Xs_x(y) = f(y)$ for all $y \in X$.
            \item[Linear independence:] Let $\sum_{x \in X} a_x\Xs_x = 0$ for some $a_x \in k$. Thus, $\sum_{x \in X} a_x\Xs_x(y) = 0$ for all $y \in X$, and we have $a_y = 0$ for all $y \in X$.
          \end{description}
    \item \begin{description}
            \item[Generating system:] Let $f \in \Maps(X,k)^G$. Hence, $f$ is constant on orbits, and we have $f = \sum_{\Gs \in \orbits GX} a_\Gs \Xs_\Gs$ with $a_\Gs = f(x)$ for $x \in \Gs$.
            \item[Linear independence:] As in \ref{lem:I.7:2}.
            \qedhere
          \end{description}
  \end{enumerate}
\end{proof}

If $X$ is an infinite set we often replace $\Maps(X,k)$ by \[kX := \set{f\colon X \to k \given \text{$\supp f$ is finite}} \] where $\supp f := \set{ x \in X \given f(x) \neq 0}$ is the \emph{support} of $f$.

\paragraph{Note.}
We have
\begin{align*}
  \supp(f_1+f_2) &\subseteq \supp f_1 \cup \supp f_2, \\
  \supp(\lambda f) &\subseteq \supp f
\end{align*}
for all $f_1,f_2,f\in \Maps(X,k)$ and $\lambda \in k\setminus \set0$. Thus, $kX \subseteq \Maps(X,k)$ together with the $0$-function is a vector space (usually just call it $kX$ as well).

$kX$ is preserved under $G$-action. Let $f \in kX, g \in G$. Then 
\begin{align*}
  &\phantom{{}\Leftrightarrow{}}\quad (g.f)(x) \neq 0 \\
  &\Leftrightarrow\quad f(g^{-1} .x ) \neq 0 \\
  &\Leftrightarrow\quad g^{-1}.x \in \supp f\\
  &\Leftrightarrow\quad x \in \underbrace{\set{g.y \given y \in \supp f}}_{\text{finite}}.
\end{align*}

\cref{lem:I.7} generalizes to $kX$.

\begin{lem}
  Let $G$ be a group and $R$ a ring. Let $G$ act on $R$ by ring homomorphisms (i.e. if $\pi\colon R \to R$ is the action then $\pi_g\colon R\to R$ is a ring homomorphism for all $g\in G$) then $R^G$ is a subring of $R$.
\end{lem}
\begin{proof}
  Let $r_1,r_2 \in R^G$. To show: $r_1+r_2,r_1r_2 \in R^G$. For $g \in G$ we have $g.(r_1+r_2) = \pi_g(r_1+r_2) = \pi_g(r_1) + \pi_g(r_2) = g.r_1 + g.r_2 = r_1+r_2$. Similarly, $g.(r_1r_2) = r_1r_2$.
\end{proof}

\paragraph{Example.} Even polynomials form a subring of $\IR[t]$.

\begin{deff}
  If $G,H$ are groups and $X$ a $G$-set and an $H$-set then the two actions \emph{commute} if \[g.(h.x) = h.(g.x)\] for all $g\in G$, $h\in H$ and $x \in X$.
\end{deff}

\lecture{October 11, 2018}

\section{Representations of groups}
\begin{deff}
  Let $G$ be a group, $V$ a $k$-vector space and $G\times V \to V$ an action. This action is linear if $\pi_g\colon V\to V$ is a linear map for all $g\in G$. Then $V$ is called a $G$-space or a \emph{representation} of $G$.
\end{deff}

\paragraph{Example.} If $V$ is a $k$-vector space then $\GL(V)$ acts linearly on $V$ by $g.v = g(v)$ for all $g \in \GL(V)$ and $v\in V$. We call this the \emph{standard representation}.

\paragraph{Remark.}
We have a bijection
\begin{eqnarray*}
  \set{\text{linear $G$-actions on $V$}} &\xleftrightarrow{1:1}& \set{\text{group homomorphisms $G \to \GL(V)$}}, \\
  \pi &\mapsto& (g \mapsto \pi_g).
\end{eqnarray*}

\paragraph{Examples.}
\begin{enumerate}
  \item Let $X$ be a $G$-set. Then $kX$ is a representation (the \emph{regular representation} of $kX$) of $G$ via \[ g.\left(\sum_{x\in X} a_x \Xs_x\right) = \sum_{x \in X} a_x \Xs_{g.x}. \]
  \item Let $V$ and $W$ be representations of $G$ over $K$. Then the $G$-action on $\Maps(V,W)$ induces a $G$-action on $\Hom_k(V,W) = \set{f\colon V\to W \given \text{$f$ $k$-linear}}$.
  \item Let $V$ and $W$ be representations of $G$ over $k$. Then $V \oplus W$ and $V\tp W$ are representations of $G$, called direct sum and tensor product via $g.(v,w) = (g.v,g.w)$ and $g.(v\tp w) = (g.v)\tp (g.w)$ extended linearly.
\end{enumerate}

\begin{deff}
  Let $V$ be a representation of $G$ over $k$.
  \begin{itemize}
    \item A \emph{subrepresentation} of $V$ is a vector subspace $U$ of $V$ such that $g.u \in U$ for all $g \in G$ and $u \in U$. It is \emph{proper} if $0 \neq U \neq V$.
    \item $V$ is \emph{irreducible} if $V\neq 0$ and there is no proper subrepresentation.
    \item $V$ is \emph{indecomposable} if it cannot be written as a decomposition $V=U_1 \oplus U_2$ such that $U_1$ and $U_2$ are proper subrepresentations.
    \item $V$ is \emph{completely reducible} if $V = \sum_{i \in I} V_i$ where $V_i$ are irreducible subrepresentations (for some set $I$).
  \end{itemize}
\end{deff}

\paragraph{Example.}
Let \[ G = \set*{\begin{pmatrix}a&b\\0&c\end{pmatrix} \given a,b,c \in \IC, a,c \neq 0 } \] act on $V = \IC^2$ by standard action. Then $U = \gen{\begin{pmatrix}1\\0\end{pmatrix}}$ is a proper subrepresentation of $V$, but $V$ is not irreducible. But $V$ is indecomposable since $U$ is the unique proper subrepresentation. To see this, assume $U' = \gen{\begin{pmatrix}x\\y\end{pmatrix}}$ to be a proper subrepresentation. Then \[ \begin{pmatrix}1&1\\0&1\end{pmatrix} \begin{pmatrix}x\\y\end{pmatrix} = \begin{pmatrix}x+y\\y\end{pmatrix} \in U', \] and as $U'$ is a subspace, we have $\begin{pmatrix}y\\0\end{pmatrix} \in U'$ and therefore $U' = U$. $V$ is also not completely irreducible.

\begin{deff}
  Let $G$ be a group and $k$ a field. The group algebra of $G$ over $k$ is the $k$-algebra given by the $k$-vector space \[ kG = \set{f \colon G \to k \given \text{$\supp f$ is finite}} \] with multiplication given by convolution of functions: \[ (f_1\cdot f_2)(x) = \sum_{y \in G} f_1(y)f_2(y^{-1}x) \] with unit $1 = \Xs_e$.
\end{deff}
Indeed, we have
\begin{align*}
  (f\cdot \Xs_e)(x) &= \sum_{y \in G} f(y) \underbrace{\Xs_e(y^{-1}x)}_{\mathclap{\text{nonzero iff $y = x$}}} = f(x)  &&\text{and}
  &(\Xs_e \cdot f)(x) &= \sum_{y \in G} \underbrace{\Xs_e(y)}_{\mathclap{\text{nonzero iff $y = 1$}}}f(y^{-1}x) = f(x)
\end{align*}
for all $f \in kG$. It remains to check associativity and distributivity.

\paragraph{Remark.} The group algebra can be defined in the same way over any commutative ring with $1$. We write \[ \sum_{g \in G} a_g g := \sum_{g \in G} a_g \Xs_g \] where $a_g \in k$ and almost all $a_g = 0$.

\begin{lem}
  The algebra structure on $kG$ is given by extending the multiplication on $G$ bilinearly.
\end{lem}
\begin{proof}
  We have \[ (\Xs_g \cdot \Xs_h) (x) = \sum_{y \in G} \Xs_g(y) \Xs_h(y^{-1}x) = \begin{cases*}
                                                                                  1 & if $h = g^{-1}x$ \\
                                                                                  0 & otherwise
                                                                                \end{cases*} = \Xs_{gh}(x). \]
  By definition the convolution product extends this bilinearly.
\end{proof}

\paragraph{Note.} $kG$ is commutative iff $G$ is abelian.

\begin{lem} \label{lem:II.2}
  Let $G$ be a group and $V$ a $k$-vector space. Then
  \begin{eqnarray*}
    \set{\text{linear $G$-actions on $V$}} &\xleftrightarrow{1:1}& \set{\text{$kG$-module structures on $V$}}, \\
    (G \times V \to V) &\mapsto& \left(\left(\sum_{g\in G} a_g g\right).v := \sum_{g\in G} a_g (g.v)\right).
  \end{eqnarray*}
\end{lem}
\begin{proof}
  Left to the reader.
\end{proof}

\begin{deff}
  Let $V$ and $W$ be representations of $G$ over $k$. A \emph{morphism} (of representations) from $V$ to $W$ si a linear, $G$-equivariant map $f\colon V\to W$. Denote $\Hom_G(V,W) := \set{f \colon V \to W \text{ morphisms of representations}}$ and $\End_G(V) := \Hom_G(V,V)$.
\end{deff}

\paragraph{Note.} $\Hom_G(V,W)$ is a vector space. Write $V \cong W$ if there exists an isomorphism $V \to W$.

\begin{lem}
  Let $G$ be a group and $k$ a field. Representations of $G$ over $k$ together with morphisms of representations  form a category $\Rep_k(G)$.
\end{lem}
\begin{proof}
  See \cref{lem:II.2}.
\end{proof}

\paragraph{Example.} For a field $k$, the $k$-vector spaces together with $k$-linear maps form a category $\Vect_k$.

\begin{cor}
  Let $k$ be a field. The assignments
  \begin{eqnarray*}
    F \colon \Rep_k(G) &\to& \Vect_k\\
    V &\mapsto& V^G \\
    f &\mapsto& f^G\colon V^G \to W^G
  \end{eqnarray*}
  define a functor from $\Rep_k(G)$ to $\Vect_k$, the functor of $G$-invariants.
\end{cor}
\begin{proof}
  Left to the reader.
\end{proof}

\begin{lem} \label{lem:II.5}
  If $f\colon V \to W$ is a morphism of representations of $G$ then $\ker f$ and $\im f$ are subrepresentations of $V$ respectively $W$.
\end{lem}
\begin{proof}
  $\ker f$ and $\im f $ are subspaces since $f$ is linear. Let $g \in G$ and $x \ker f$. Then $f(g.x) = g.f(x) = g.0 = 0$ and $g.x \in \ker f$, thus $\ker f $ is a subrepresentation.
  
  Let $y \im f$ and $x \in V$ with $f(x) = y$. We get $g.y = g.(f(x)) = f(g.x) \im f$.
\end{proof}

\paragraph{Remark.} It can be shown that $\Rep_k(G)$ is an abelian category.

\begin{lem}[Schur's lemma] \label{lem:schur}
  Let $g$ be a group and $V,W$ irreducible representations of $G$ over $k$.
  \begin{enumerate}
    \item\label{lem:schur:1} $\Hom_G(V,W) = 0$ if $V \ncong W$. If $V \cong W$, we have $\Hom_G(V,W) \neq 0 $ and every non-zero morphism is an isomorphism.
    \item\label{lem:schur:2} If $k = \ol k$ and $V$ and $W$ are finite dimensional then \[ \Hom_G(V,W) \cong \begin{cases*}
                                                                                            k & if $V \cong W$ \\
                                                                                            0 & if $V \ncong W$
                                                                                          \end{cases*} \]
    as representations.
  \end{enumerate}
\end{lem}
\begin{proof}
  \leavevmode
  \begin{enumerate}[label=\ref{lem:schur:\arabic*}]
    \item Assume $V \cong W$ and $0\neq f \in \Hom_G(V,W)$. This implies $\ker f \neq V$ and $\im F \neq 0$. By \cref{lem:II.5} it follows $\ker f = 0$ and $\im f = W$, since $f$ is a morphism and $V$ and $W$ are irreducible. As $f$ is linear, $f$ is an isomorphism.
    \item Assume $V \cong W$ and $0\neq \alpha,\beta \in \Hom_G(V,W)$. It is enough to show $\beta = \lambda\alpha$ for some $\lambda \in k$. By \ref{lem:schur:1} $\alpha$ has an inverse $\alpha^{-1}$ (which is again a morphism) and we have $\alpha^{-1}\circ\beta \in \End_G(V)$. If $k=\ol k$ and $V$ is finite dimensional $\alpha^{-1}\circ\beta$ has eigenvectors. We define $K := \ker(\alpha^{-1}\circ\beta -\lambda\id_V) \neq 0$ for some $\lambda \in k$. Now $\alpha^{-1}\circ\beta - \lambda \id_v \in \End_G(V)$ (the reader may check this statement), thus $K$ is a subrepresentation of $V$, hence $K=V$, since $V$ is irreducible and $K \neq 0$. Therefore, $\alpha^{-1}\circ\beta = \lambda\id_V$ and $\beta = \lambda \alpha$.
    \qedhere
  \end{enumerate}
\end{proof}

\begin{cor}
  Let $k=\ol k$ and $V_i$ ($1 \le i \le r$) be pairwise non-isomorphic irreducible finite dimensional representations of $G$ over $k$. Let $W_i := V_i^{\oplus n_i} := V_i \oplus \ldots \oplus V_i$ for some $n_i \in \Z_{>0}$ (a representation of $G$). Then \[ \End_G(W_1 \oplus \ldots \oplus W_r) \cong \M_{n_1\times n_1}(k) \oplus \ldots \oplus M_{n_r \times n_r}(k) \] as algebras.
\end{cor}
\begin{proof}
  We have
  \begin{align*}
    \End_G(W_1\oplus \ldots\oplus W_r) &= \Hom_G\left(\bigoplus_{i=1}^r\bigoplus_{j=1}^{n_i} V_i,\bigoplus_{i=1}^r\bigoplus_{j=1}^{n_i} V_i \right) \\
    \intertext{and by \namereff{lem:schur}, since $V_i \cong V_i$, and $\End_G(V_i)\cong k$, we get}
    &\cong \End_G(V_1^{\oplus n_1}) \oplus \ldots \oplus \End_G(V_r^{\oplus n_r})\\ & \cong \M_{n_1\times n_1}(k) \oplus \ldots \oplus M_{n_r \times n_r}(k).
    \qedhere
  \end{align*}
\end{proof}

\lecture{October 15, 2018}

\begin{thm}[Maschke's theorem] \label{thm:maschke}
  Let $G$ be a finite group and $k$ a field such that $\chr k \nmid \abs G$ (in particular $\chr k = 0$ is allowed). The the finite dimensional representations of $G$ over $k$ are completely reducible.
\end{thm}
\begin{proof}
  It is enough to show that for any finite dimensional representation $V$ of $G$ the following holds: any subrepresentation $U$ of $V$ has a complement in $W$ in $V$ which is again a subrepresentation; so $V = U \oplus W$ as representations. Let $U$ be such a subrepresentation and choose a vector space complement $U'$ so $V = U \oplus U'$ as vector spaces.
  
  Define now $\hat p\colon V \to U $ by \[ \hat p(v) = \frac1{\abs G} \sum_{g \in G}\underbrace{g^{-1}.\underbrace{p(g.v)}_{\in U}}_{\in U} \in U. \]
  Now:
  \begin{itemize}
    \item We have ${\displaystyle\hat p(u) = \frac1{\abs G} \sum_{g\in G} g^{-1} .p(g.h.v) = \frac1{\abs G}\sum_{g \in G}g^{-1}.g.u = u }$ for all $u\in U$.
    \item $\hat p$ is $G$-equivariant, as for any $h\in G$ and $v\in V$
    \begin{align*}
      \hat p(h.v) &= \frac1{\abs G} \sum_{g \in G}g^{-1}.p(g.h.v) = \frac1{\abs G} \sum_{g\in G} h.(h^{-1}.(g^{-1}.p(g.h.v)) \\
      &= h.\left(\frac1{\abs G} \sum_{g \in G} (gh)^{-1}.p((gh).v)\right) = h.\left(\frac1{\abs G} \sum_{g \in G} g^{-1}.p(g.v)\right) = h.\hat p(v).
    \end{align*}
  \end{itemize}
  Therefore, $V = \im \hat p \oplus \ker \hat p = U \oplus \ker \hat p$ since $\hat p$ is $G$-equivariant. $W := \ker \hat p$ is a subrepresentation of $V$.
\end{proof}
\paragraph{Warning.}
\namereff{thm:maschke} does not hold in general if $\chr k \mid \abs G$. For example, take $G = \fak \IZ{2\IZ} = \set{e,s}$, $k = \IF_2$ and $V = kG$ the regular representation. Then $\gen{e+s}_k$ is a $1$-dimensional subrepresentation, but in fact the unique one. Therefore, it has no complement. (Note: if $\chr k \neq 2$ then $\gen{e+s}_k$ is also a $1$-dimensional subrepresentation and a complement of the above one).

\section{Invariant polynomial functions}

\subsection{Gradings and filtrations}
\begin{deff}
  Let $A$ be a $k$-algebra. A \emph{grading} (or $\IZ$-grading) on $A$ is a decomposition \[ A = \bigoplus_{i\in IZ} A_i\] into vector subspaces $A_i$ such that $A_iA_j \subseteq A_{i+j}$ for all $i,j\in \IZ$. We call then $A$ a \emph{graded algebra}. The $A_i$ ($i \in \IZ$) are the \emph{graded} (or \emph{homogeneous}) \emph{components}. An element $a_i \in A_i$ is called \emph{homogeneous} (of degree $i$).
\end{deff}
\begin{deff}
  A \emph{grading} of a ring $R$ is a decomposition $R = \sum_{i \in \IZ} R_i$ into $\IZ$-modules such that $R_iR_j \subseteq R_{i+j}$ for all $i,j\in \IZ$. We call then $R$ a \emph{graded} ring and the $R_i$ the \emph{graded}/\emph{homogeneous components}.
\end{deff}

\begin{lem}
  Let $k$ be a field and $A$ a $k$-algebra with $1$.
  \[ A = \bigoplus_{i \in \IZ} A_i\text{ is a graded algebra.} \quad \Longleftrightarrow \quad A = \bigoplus_{i \in \IZ} A_i \text{ is a graded ring and $k1 \subseteq A_0$.} \]
\end{lem}
\begin{proof}
  \leavevmode
  \begin{description}
    \item[\enquote{$\Leftarrow$}] $A= \bigoplus _{ i\in\IZ}A_i$ is a decomposition into $k$-vector spaces; in particular into $\IZ$-modules. We have to show $k1 \subseteq A_0$.
    
    Write $1 = \sum_{i \in \IZ} e_i$ with $e_i \in A_i$ and almost all $e_i=0$. Then for any $a \in A_j$ we have $a = a1 = \sum_{i \in \IZ}ae_i$. As $ae_i\in A_{j+i}$, we have $a = ae_0$ because the sum $A=\bigoplus_{i \in \IZ}A_i$ is direct. Similarly we get $e_0a = a$. Thus, $e_0 = a = ae_0$ for all $a\in A$, and we have $1 = e_0 \in A_0$ and finally $k1 = ke_0 \subseteq A_0$ since $A_0$ is a vector space.
    \item[\enquote{$\Rightarrow$}] We have to show that $A_i$ is closed under scalar multiplication for all $i \in \IZ$. Let $\lambda \in k$ and $i\in \IZ$. Then $\lambda A_i = (\lambda1)A_i \subseteq A_0A_i \subseteq A_{0+i} = A_i$.
    \qedhere
  \end{description}
\end{proof}

\paragraph{Examples.}
\begin{enumerate}
  \item Let $A$ be any $k$-algebra. It is a graded algebra via the \enquote{stupid grading} $A = \bigoplus_{i \in \IZ} A_i$ where \[ A_i = \begin{cases*}
                                 A & if $i=0$, \\
                                 0 & if $i \neq 0$.
                               \end{cases*} \]
  \item Let $R= \IZ$ or $R = k$ for a field. Then $A= R[X_1,\ldots,X_n]$ is a graded ring respectively a graded algebra where $A= \sum_{i \in \IZ} A_i$ is given by \[ A_i = \begin{cases*}
                                                                 0 & if $i<0$, \\
                                                                 \gen{\set*{X_1^{a_1}\cdots X_n^{a_n} \given \sum_{j=1}^n a_j =i}}_R & else,
                                                               \end{cases*} \]
  because clearly the monomials $X_1^{a_1}\cdots X_n^{a_n}$ with $a_i \in \IZ_{\ge0}$ (and by convention $X_1^0\cdots X_n^0=1$) form an $R$-basis of $R[X_1,\ldots,X_n]$ and $(X_1^{a_1}\cdots X_n^{a_n})(X_1^{b_1}\cdots X_n^{b_n}) = (X_1^{a_1+b_1}\cdots X_n^{a_n+b_n})$, so that $a_ia_j \in A_{i+j}$ for all basis elements $a_i \in A_i$ and $a_j \in A_j$ (then also $A_iA_j \subseteq A_{i+j}$).
  \item Let $V$ be a $k$-vector space. Consider the vector space \[ \T(V) := k \oplus V \oplus (V \tp V) \oplus \ldots = k \oplus \bigoplus_{d\ge1} V^{\tp d} =: \bigoplus_{d \ge 0} V^{\tp d}. \]
  We claim that $\T(V)$ is an algebra by setting \[ (\underbrace{v_{i_1} \tp \ldots \tp v_{i_d}}_{\in V^{\tp d}})(\underbrace{v_{j_1} \tp \ldots \tp v_{j_{d'}}}_{\in V^{\tp d'}}) = \underbrace{v_{i_1} \tp \ldots \tp v_{i_d} \tp v_{j_1} \tp \ldots \tp v_{j_{d'}}}_{\in V^{\tp (d+d')}} \] for any $v_{i_r}, v_{j_s}$ in a chosen basis $\set{v_i \given i \in I}$ of $V$ ($1 \le r \le d$, $1 \le s \le d'$) and extended linearly to $\T(V)$ with \begin{align*} \underbrace{\lambda}_{\in V^{\tp 0}} \cdot \underbrace{v}_{\in V^{\tp d}} & := \underbrace{\lambda v}_{\in V^{\tp d}} &&\text{and} & \underbrace{v}_{\in V^{\tp d}} \cdot \underbrace{\lambda}_{\in V^{\tp 0}} & := \underbrace{\lambda v}_{\in V^{\tp d}} . \end{align*}
  
  We also claim that $\T(V) = \bigoplus_{i \in \IZ} \T(V)_i$ with \[ \T(V)_i = \begin{cases*}
                                                                                 V^{\tp i} & if $i \ge 0$ \\
                                                                                 0 & otherwise
                                                                               \end{cases*} \]
  is then a graded algebra.
\end{enumerate}

\begin{deff}
  Let $A$ be a $k$-algebra. A \emph{filtration} of $A$ is a (possibly infinite) sequence $F_\bullet(A)$ of vector subspaces of the form \[ 0 = F_{-1}(A) \subseteq F_0(A) \subseteq F_1(A) \subseteq \ldots \subseteq A \] such that
  \begin{enumerate}
    \item\label{def:filtered algebra:1} $F_i(A)F_j(A) \subseteq F_{i+j}(A)$ for all $i,j \in \IZ_{\ge -1}$ and
    \item\label{def:filtered algebra:2} $\displaystyle \bigcup_{i \ge -1} F_i(A) = A$.
  \end{enumerate}
  An algebra with a filtration is a \emph{filtered} algebra.
\end{deff}

\begin{prop}
  If $A$ is a filtered algebra with filtration $F_\bullet (A)$ then we can consider the vector space \[ \gr A := \bigoplus_{i \in \IZ} (\gr A)_i \quad\text{where}\quad (\gr A)_i = \begin{cases*}
                                                                          \fak{F_i(A)}{F_{i-1}(A)} & if $i \ge 0$, \\
                                                                          0 & if $i<0$.
                                                                        \end{cases*} \]
  Then $\gr A$ becomes a graded algebra by defining the multiplication \[ (a + F_{i-1}(A))(b+F_{j-1}(A)) := ab + F_{i+j-1}(A) \] for any $a \in F_i(A)$ and $b \in F_j(A)$.
  
  This algebra is called the \emph{associated graded algebra} to the filtered algebra $(A,F_\bullet(A))$.
\end{prop}
\begin{proof}
  We have to show that the multiplication is well-defined. Note that we have
  \begin{alignat*}{2}
    F_{i-1}(A) b &\subseteq F_{i-1}(A)F_j(A) &&\subseteq F_{i+j-1}(A), \\
    aF_{j-1}(A) &\subseteq F_i(A)F_{j-1}(A)  &&\subseteq F_{i+j-1}(A), \\
    F_{i-1}(A) F_{j-1}(A) & \subseteq F_{i+j-2}(A) && \subseteq F_{i+j-1}(A).
  \end{alignat*}
  Therefore, we have \[(a + F_{i-1}(A))(b+F_{j}(A))=(c + F_{i-1}(A))(d+F_{j}(A))\] if $a+F_{i-1}(A) = c+F_{i-1}(A)$ in $\fak{F_{i+j}(A)}{F_{i+j-1}(A)}$ and $b+F_j(A) = d+F_j(A)$ for all $a,c \in F_j(A)$ and $b,d \in F_j(A)$.
  
  Associativity and distributivity follow from the same properties in $A$.
\end{proof}

\begin{prop}
  Let $A = \bigoplus_{i \in IZ}A_i$ be a graded algebra such that $A_i = 0$ for $i <0$. Then define \[ F_j(A) = \bigoplus_{\mathclap{0\le i \le j}} A_i \] for all $j \ge 0$. Then \begin{equation} 0 =: F_{-1}(A) \subseteq F_0(A) \subseteq F_1(A) \subseteq \ldots \subseteq A \tag{*}\label{prop:III.3:eq} \end{equation} turns into a filtered algebra.
\end{prop}
\begin{proof}
  Obviously $F_j(A) \subseteq A$ are vector subspaces for all $j \ge -1$ and \eqref{prop:III.3:eq} is a sequence of nested vector spaces.
  \begin{enumerate}
    \item[\ref{def:filtered algebra:2}] Any $a \in A$ can be written as $a = \sum_{i=0}^\infty a_i$ with $a_i \in A_i$ where almost all $a_i = 0$. There exists $j>0$ such that $a \in F_j(A)$ and we have \[ A \subseteq \bigcup_{j \ge -1} F_j(A). \]
    \item[\ref{def:filtered algebra:1}] Let $A \in F_r(A)$ and $b \in F_s(A)$. We can write $a = \sum_{i=1}^r a_i$ and $b = \sum_{i=1}^sb_i$ for some $a_i,b_i \in A_i$. Thus we get \begin{align*} ab \in \sum_{{\substack{0\le i \le r\\0\le j \le s}}} \underbrace{a_ib_j}_{A_{i+j}} \in \bigoplus_{l=0}^{r+s}A_l = F_{r+s}(A). \qedhereb \end{align*}
  \end{enumerate}
\end{proof}

\paragraph{Examples.}
\begin{enumerate}
  \item Let $R = \IZ$ or $R=k$ a field. Consider $A=R[X_1,\ldots,X_n]$. This is a filtered algebra by setting \[ F_j(A) = \gen{\set*{X_1^{a_1}\cdots X_n^{a_n} \given \sum_{i=1}^na_i = j}}_R \] for $j\ge 0$ ($F_{-1}(A) = 0$).
  \item Let $R=k[t]$ for any field $k$. Consider $\End_k(k[T])$ (linear endomorphisms). There are the two following interesting elements in $\End_k(k[t])$:
  \begin{xalignat*}{2}
    X\colon k[t] &\to k[t] & \qquad \partial\colon k[t] &\to k[t] \\
    p & \mapsto tp & \qquad p &\mapsto p' := \text{formal derivation}
  \end{xalignat*}
  Let $A$ be the subalgebra of $\End_k(k[t])$ generated by $X$ and $\partial$. This is called the (first) \emph{Weyl algebra} $A_1$.
  
  We claim that $A$ has basis $\set{X^a\partial^b \given a,bc \in \IZ_{\ge 0}}$ (with $X^0\partial^0 = 1$). The reader may check this using the formula $\partial X = X\partial + \id$. Furthermore, one can define a filtration on $A$ via $F_j(A) = \gen{\set{X^a\partial^b \given a+b \le j}}$ for $j \ge 0$.
\end{enumerate}

\end{otherlanguage}
\end{document}
