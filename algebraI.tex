\documentclass[12pt,a4paper]{scrartcl}

\usepackage{includes}
\usepackage{shortcuts}
\usepackage{numbering}

%---------------------------%
% Franzen-spezifisches Zeug %
%---------------------------%

% numbering
\counterwithin{subsection}{section}
\renewcommand{\thesection}{\arabic{section}}
\setcounter{section}{-1}
\setlist[enumerate,1]{label=\textup{\roman*)}}
\setlist[enumerate,2]{label=\textup{\alph*)}}

\counterwithout{thmcounter}{subsection}
\counterwithout{defcounter}{subsection}
\counterwithout{bspcounter}{subsection}
\counterwithin{thmcounter}{section}
\counterwithin{defcounter}{section}
\counterwithin{bspcounter}{section}
\renewcommand{\thethmcounter}{\arabic{section}.\arabic{thmcounter}}
\renewcommand{\thedefcounter}{\arabic{section}.\arabic{defcounter}}
\renewcommand{\thebspcounter}{\arabic{section}.\arabic{bspcounter}}

\makeatletter
\let\c@thmcounter=\c@defcounter
\let\c@bspcounter=\c@defcounter
\makeatother

\theoremstyle{cplain}
\newtheorem{prop}[thmcounter]{Proposition}
\crefname{prop}{Proposition}{Propositionen}

\theoremstyle{cdef}
\newtheorem{beme}[thmcounter]{Bemerkung}
\crefname{beme}{Bemerkung}{Bemerkungen}

% bibliography
\usepackage[backend=biber,sorting=none]{biblatex}
\addbibresource{literatur.bib}


\author{}
\title{Algebra I}
\subtitle{Sommersemester 2018}    

\begin{document}
\maketitle
\tableofcontents
\newpage

\noindent
Dies ist eine Mitschrift der Vorlesung \enquote{Algebra I} von Dr. Hans Franzen an der Universität Bonn, gehalten im Sommersemester 2018.

\bigskip

\noindent
Vorlesungswebsite:\\
\url{http://www.math.uni-bonn.de/ag/stroppel/Franzen_Algebra_1.htmpl}

\nocite{atiyah-macdonald}
\nocite{eisenbud}
\nocite{matsumura}
\printbibliography

\newpage

\lecture{9. April 2018}

\section{Motivation}
Sei $k=\overline k$ (algebraisch abgeschlossener Körper). Seien $f_1, \dots f_m \in k[t_1,\dots t_n]$. Dann sind \[ V(f_1,\dots, f_m) := \{x=(x_1,\dots, x_n) \in k^n \mid f_i(x)=0 \text{ für alle $1\le i \le m$} \}\] affine algebraische Varietäten.


\emph{Beispiel.} Sei $k=\IC$, $n=2$ und $m=1$. Wir erhalten die folgenden Bilder:
\begin{figure}[H]
    \begin{subfigure}[b]{.5\linewidth}
        \centering
        \begin{tikzpicture}
            \begin{axis}[width=7cm,xmax=1,xmin=-1,ymax=1,ymin=-1,axis equal,xtick={-1,0,1},ytick={-1,0,1}]
                \draw (axis cs:0,0) circle[radius=1];
            \end{axis}
        \end{tikzpicture}
        \caption*{$f=x_1^2+x_2^2-1$}
    \end{subfigure}
    \begin{subfigure}[b]{.5\linewidth}
        \centering
        \begin{tikzpicture}
            \begin{axis}[width=7cm,xmax=1,xmin=-1,ymax=1,ymin=-1,axis equal,xtick={-1,0,1},ytick={-1,0,1}]
                \addplot[black,samples=100,domain=0:1] {x^(3/2)};
                \addplot[black,samples=100,domain=0:1] {-x^(3/2)};
            \end{axis}
        \end{tikzpicture}
        \caption*{$f=x_1^3-x_2^2$}
    \end{subfigure}
\end{figure}
\begin{figure}[H]
    \begin{subfigure}[b]{.5\linewidth}
        \centering
        \begin{tikzpicture}
            \begin{axis}[width=7cm,xmax=1,xmin=-1,ymax=1,ymin=-1,axis equal,xtick={-1,0,1},ytick={-1,0,1}]
                \addplot[black] coordinates{(-2,0) (2,0)};
                \addplot[black] coordinates{(0,-2) (0,2)};
            \end{axis}
        \end{tikzpicture}
        \caption*{$f=x_1\cdot x_2$}
    \end{subfigure}
    \begin{subfigure}[b]{.5\linewidth}
        \centering
        \begin{tikzpicture}
            \begin{axis}[width=7cm,xmax=1,xmin=-1,ymax=1,ymin=-1,axis equal,xtick={-1,0,1},ytick={-1,0,1}]
                \addplot[black] coordinates{(-2,0) (2,0)};
                \addplot[black] coordinates{(0,-2) (0,2)};
            \end{axis}
        \end{tikzpicture}
        \caption*{$f=x_1^{21}\cdot x_2^{138}$}
    \end{subfigure}
\end{figure}

In der algebraischen Geometrie betrachten wir die folgende Beziehung:
\begin{eqnarray*}
    \text{Affine Varietät $X$} & \longleftrightarrow & \text{Ring $A(X)$} \\
    \text{Studium der Geometrie von $X$} & \cong & \text{Studium des Ringes $A(X)$}
\end{eqnarray*}
Hierzu ist das Studium von kommutativer Algebra notwendig.
\newpage

\section{Primideale und maximale Ideale}
\subsection{Grundbegriffe}
\begin{konv}
	In der gesamten Vorlesung sind Ringe immer kommutativ und haben Eins. Ringhomomorphismen erhalten die Eins.
\end{konv}
\begin{defi}
	Sei $A$ ein Ring und $I\subsetneq A$ ein Ideal.
	\begin{enumerate}
        \item $I$ heißt Primideal, falls \begin{enumerate}
            \item $ab \in A \setminus I$ für alle $a,b \in A \setminus I$ gilt, oder äquivalent
            \item $\fak A I$ ein Integritätsbereich ist.
        \end{enumerate}
        \item $I$ heißt maximales Ideal, falls \begin{enumerate}
            \item für jedes Ideal $J \subsetneq A$ aus $I \subset J$ bereits $I=J$ folgt, oder äquivalent
            \item $\fak A I$ ein Körper ist.
        \end{enumerate}
		\item $\Spec A := \{\fp \mid \fp \subsetneq A \text{ Primideal}\}$
		\item $\Max A:=\{\fm \mid \fm \subsetneq A \text{ maximales Ideal}\}$
	\end{enumerate}
\end{defi}
\begin{defi} Sei $f: A \to B$ ein Ringhomomorphismus.
	\begin{enumerate}
		\item Sei $J \subset B$ ein Ideal. Dann ist $J \cap A:=f^{-1}(J)$ ein Ideal von $A$, genannt \emph{Kontraktion}. Ist $J$ ein Primideal, so ist $J \cap A$ ebenfalls ein Primideal.
		\item Sei $I \subset A$ ein Ideal. Dann ist $f(I)$ nicht notwendigerweise ein Ideal. Setze $I \cdot B := \left( f(I) \right)$, das von $f(I)$ erzeugtes Ideal. Wir nennen dieses \emph{Ausdehnung} von $I$.
	\end{enumerate}
\end{defi}
\begin{bem}
	Es kann sein, dass $I \in \Max A$, aber $I \cdot B \notin \Spec B$. Sei dazu $A=\IZ$ und $B=\IZ[i]$. Sei $I=(2) \in \Max(\IZ)$, aber $I \cdot B = (2) \notin \Spec \IZ[i]$, da $2=(1-i)(1+i)$ ist.
\end{bem}
\begin{satz} \label{thm:existenz maximaler ideale}
	Sei $A$ ein Ring und $I \subsetneq A$ ein Ideal. Dann existiert ein $\fm \in \Max A$ mit $I \subset \fm$.
	\begin{proof}
		Siehe Vorlesung Einführung in die Algebra, Satz 9.1, oder \cite{atiyah-macdonald}, Theorem 1.3.
	\end{proof}
\end{satz}
\begin{kor} \label{kor:einheiten und maximale ideale}
    Sei $A$ ein Ring, bez. $A^*=\{\text{Einheiten von }A\}$. Dann gilt \[A^*=A \setminus \bigcup_{\mathclap{\fm \in \Max A}} \fm.\]
	\begin{proof}
		Ist $a \in A^*$, so gilt $(a)=(1)=A$ und damit $a \notin \fm$ für alle $\fm \in \Max A$.

		Sei $a \notin \fm$ für alle $\fm \in \Max A$. Dann ist $(a) \subset \fm$ für alle $\fm \in \Max A$; es folgt also $(a)=A=(1)$ mit \cref{thm:existenz maximaler ideale} und damit $a \in A^*$.
	\end{proof}
\end{kor}
\subsection{Lokale Ringe}
\begin{defi} \label{def1.5}
	Sei $A$ ein Ring. $A$ heißt \emph{lokal}, falls $A$ genau ein maximales Ideal $\fm$ hat (d.h. $\Max A = \{\fm\}$).
\end{defi}
\begin{lem} \label{lem:lokale ringe}
	Sei $A$ ein Ring.
	\begin{enumerate}
		\item Sei $I \subsetneq A$ ein Ideal. Dann sind äquivalent:
		      \begin{enumerate}
			      \item $\Max A = \{I\}$
			      \item $A \setminus I \subset A^*$
			      \item $A \setminus I = A^*$
		      \end{enumerate}
		\item Sei $\fm \in \Max A$. Falls $1+x \in A^*$ für alle $x \in \fm$ gilt, so ist $\Max A=\{ \fm \}$.
	\end{enumerate}
    \begin{proof}
        \leavevmode
		\begin{enumerate}
			\item \enquote{a) $\Rightarrow$ b)} folgt aus \cref{kor:einheiten und maximale ideale}. Für \enquote{b) $\Rightarrow$ a)} sei $J \subsetneq A$ ein Ideal. Dann liegt $J$ in $I$, und es folgt $I \in \Max A$ mit $\Max = \{I\}$.
			\item Sei $b \in A \setminus \fm$. Es ist $b \in A^*$ zu zeigen. Da $\fm \in \Max A$, gilt $(b)+m=A$, es existieren folglich $a \in A, x\in m$ mit $ab+x=1$. Daraus folgt $ab=1-x \in A^*$ und dann $b \in A^*$.
            \qedhere
        \end{enumerate}
	\end{proof}
\end{lem}

\begin{bsp}
	\begin{enumerate}
        \leavevmode
		\item[0)] Körper sind lokale Ringe.
        \item Ist $(A,\fm)$ ein lokaler Ring. Dann ist $A\llbracket t \rrbracket$ ein lokaler Ring. Dabei ist \[ A\llbracket t \rrbracket=\left\{\sum_{i=0}^\infty a_it^i \,\middle|\, a_i \in A\right\} \] der Ring der formalen Potenzreihen mit kanonischer Addition und Multiplikation. Betrachte die Komposition $A\llbracket t \rrbracket \to A \to A/\fm$ gegeben durch $\phi=\can \circ \ev_0$. Diese bildet surjektiv auf den Körper $A/\fm$ ab, also ist $\ker \phi \in \Max A\llbracket t \rrbracket$.
        
        Sei $f \in \ker \phi$ und betrachte $1+ f$. Aus $f \in \ker \phi$ folgt $f=\sum_{i=0}^\infty a_it^i$ mit $a_0 \in \fm$. Wir wollen ein $g \in A\llbracket t\rrbracket$ mit $g=\sum_{i=0}^\infty b_it^i$ mit $(1+f)g=1$ finden, d.h. $(1+a_0)b_0$ und $(1+a_0)b_n+a_1b_{n-1}+\dots a_nb_0$ für alle $n > 0$. Erstere Gleichung ist lösbar, da $a_0 \in m$ liegt und damit $1+a_0 \in A^*$ folgt. Wir führen nun eine Induktion nach $n$. Nehme an, dass $b_0, \dots b_{n-1}$ bereits bekannt sind. Dann folgt, dass $b_n=-(1+a_0)^{-1}(a_1b_{n-1}+\dots+a_nb_0)$. Nach \cref{lem:lokale ringe} ist $A\llbracket t \rrbracket$ lokal mit dem maximalen Ideal $(t)+\fm$. Der Residuenkörper ist $A\llbracket T\rrbracket /((t)+\fm) \cong A/\fm$.

        Spezialfall: $k\llbracket t_1,\dots t_n\rrbracket := k\llbracket t_1,\dots,t_{n-1}\rrbracket\llbracket t_n\rrbracket$ lokaler Ring mit maximalen Ideal $(t_1,\dots,t_n)$ und Residuenkörper isomorph zu $k$.
        
		\item Sei $X \subset \IR^n$ offen, $0 \in X$. Betrachte Paare $(U,f)$ mit $U \subset X$ offen, $0 \in U$ und $f:U \to R$ stetig. Definiere Äquivalenzklassen durch $(U_1,f_1) \sim (U_2,f_2)$ genau dann, wenn ein offenes $W \subset U_1 \cap U_2$ mit $0 \in W$ und $f_1|_W=f_2|_W$ existiert. Die Äquivalenzklasse $\langle U,f \rangle$ heißt Funktionenkeim. Definiere $A := \{\langle U,f \rangle \mid (U,f) \text{ wie oben}\}$ mit  punktweiser Addition und Multiplikation.

		Wir zeigen nun, dass $A$ ein lokaler Ring ist. Betrachte $\phi: A \to \mathbb{R}, \; \langle U,f \rangle \to f(0)$, einen wohldefinierten Ringhomomorphismus. $\phi$ surjektiv und damit folgt $\ker \phi \in \Max A$.

        Sei $s = \langle U,f \rangle \in \ker \phi$. Es bleibt $1+s \in A^*$ zu zeigen. Da $1+f(0)=1$, existiert eine offene Umgebung $W$ von $0$, sodass $f(x) \neq 0$ für alle $x \in W$ ist. Dann ist $y: W \to \IR, \; x \mapsto \frac{1}{1+f(0)}$ stetig und es gilt $(1+s)\langle W,y \rangle = 1$. Somit ist $A$ ein lokaler Ring.
        
        \item Sei $A$ ein Ring und $\fp \in \Spec A$. Dann ist $S := A \setminus \fp$ ein multiplikatives System. Wir definieren $A_\fp := S^{-1}A$ (Lokalisation).
        
        Später werden wir zeigen, dass $A_\fp$ ein lokaler Ring mit maximalem Ideal $\fp \cdot A_\fp$ ist.
	\end{enumerate}
\end{bsp}

\lecture{12. April 2018}

\subsection{Nilradikal und Jacobson-Radikal}
\begin{defi}
    Sei $A$ ein Ring sowie $I\subset A$ ein Ideal. Definiere das \emph{Radikal} von $I$ durch \[\sqrt{I} := \{x\in A \mid \exists n>0: x^n\in I\}.\]
    $\Nil(A) := \sqrt{0}$ heißt \emph{Nilradikal} von $A$.
\end{defi}
\begin{lem}
    \leavevmode
    \begin{enumerate}
        \item $\sqrt{I}$ ist ein Ideal von $A$.
        \item $I\subset \sqrt{I} = \sqrt{\sqrt{I}}$
        \item $\sqrt{I} = (1) \Leftrightarrow I = (1)$
        \item $\fak A{\sqrt{I}}$ hat keine nilpotente Elemente (außer 0).
    \end{enumerate}
    \begin{proof}
        \leavevmode
        \begin{enumerate}
            \item Seien $x, y\in \sqrt{I}$, es existieren also $m, n > 0$ mit $x^m, y^n\in I$. Dann gilt
            \[(x+y)^{m+n-1}=\sum_{r=0}^{m+n-1}{m+n-1 \choose r}\overbrace{x^r}^{\in I \text{ falls } r\geq m}\underbrace{y^{n+m-1-r}}_{\in I \text{ falls } r<m},\]
            also liegt auch $x+y$ in $I$. Die anderen Eigenschaften sind schnell nachgeprüft.
            \item Da $I\subset \sqrt{I}$, folgt sofort $\sqrt{I} \subset \sqrt{\sqrt{I}}$. Ist nun umgekehrt $x\in \sqrt{\sqrt{I}}$, so existiert ein $n>0$ mit $x^n\in \sqrt{I}$, und dann existiert ein $m>0$ mit $(x^n)^m = x^{nm} \in I$ Also liegt $x$ in $\sqrt{I}$.
            \item Die Rückrichtung ist klar. Sei also $\sqrt I=(1)$. Dann exisitert ein $n>0$ mit $1^n \in I$, also liegt $1$ in $I$.
            \item Sei $z\in \fak A{\sqrt{I}}$ mit $z^n=0$. Schreibe $z=x+\sqrt{I}$ für ein $x\in A$. Dann ist $x^n+\sqrt{I} = 0$, also liegt $x^n$ in $\sqrt{I}$ und es folgt $z=0$.
            \qedhere
        \end{enumerate}
    \end{proof}
\end{lem}
\begin{bsp}
    Sei $A=\IZ$ und $I=(a)$. Was ist $\sqrt{(a)}$?

    Für $a=0$ ist $\sqrt{(a)}=(0)$. Sei nun $a\neq 0$, o.B.d.A. $a>0$. Schreibe $a=p_1^{m_1}p_2^{m_2}\ldots p_l^{m_l}$ mit positiven Primzahlen $p_1, \ldots, p_l$ und $p_i\neq p_j$ für $i\neq j$. Dann ist $\sqrt{(a)}=(p_1p_2\ldots p_l)$.

    \begin{proof}
        \leavevmode
        \begin{description}
            \item[\enquote{$\subseteq$}] Ist $x\in \sqrt{(a)}$, so existiert ein $n>0$ mit $x^n \in (a)$. $a$ teilt also $x^n$, weshalb jedes $p_i$ und somit auch ihr Produkt $x$ teilt, woraus $x \in (p_1\ldots p_l)$ folgt.
            \item[\enquote{$\supseteq$}] Ist $x \in (p_1\ldots p_l)$, so wähle $n \ge \max\{m_1,\ldots,m_l\}$. Dann liegt $x^n$ in $(p_1^n\ldots p_l^n \subset (p_1^{m_1}\ldots p_l^{m_l}) = (a)$.
            \qedhere
        \end{description}
    \end{proof}
\end{bsp}
\begin{prop}
    Sei $A$ ein Ring und $I$ ein Ideal in $A$. Dann gilt
    \[\sqrt{I} = \bigcap_{\mathclap{\substack{\fp\in \Spec A,\\ \fp\supset I}}} \fp.\]
\end{prop}
\begin{proof}
    \leavevmode
    \begin{description}
        \item[\enquote{$\subseteq$}] Sei $x\in \sqrt{I}$. Dann existiert ein $n>0$ mit $x^n\in I$.
        Sei $\fp\in \Spec A$ mit $\fp\supset I$. Dann liegt $x^n$ auch in $\fp$, und da $\fp$ ein Primideal ist, folgt $x \in \fp$.
        \item[\enquote{$\supseteq$}] Sei $x\in \bigcap_{\fp\in \Spec A, \fp\supset I}\fp$. Angenommen, $x$ wäre nicht in $\sqrt{I}$; für alle $n>0$ gilt also $x^n\not \in I$.

        Definiere \[\Sigma := \{J \mid J\subset A \text{ Ideal}, I\subset J,\, \forall n>0: x^n\not \in J\}.\]
        Dann ist $(\Sigma, \subset)$ angeordnet (partiell geordnet).
        
        Wir wollen das \textsc{Lemma von Zorn} anwenden.
        \begin{itemize}
            \item $\Sigma \neq \emptyset$ (da $I\in \Sigma$)
            \item Sei $(J_t)_{t\in T}$ mit $T \neq \emptyset$ eine Kette in $\Sigma$, es gilt also $J_t \in \Sigma$ sowie $J_t \subset J_{t'}$ oder $J_t \supset J_{t'}$ für alle $t,t'\in T$.
            Dann ist $\bigcup_{t\in T} J_t$ ein Ideal in $A$ und enthält $I$, aber kein $x^n$ mit $n>0$. Also liegt $\bigcup_{t\in T} J_t$ in $\Sigma$.
        \end{itemize}
        
        Nach dem \textsc{Lemma von Zorn} existiert nun ein $\fp \in \Sigma$, welches maximal in $\Sigma$ bezüglich Inklusion ist. Wir behaupten, dass $\fp$ ein Primideal in $A$ ist.
        \begin{proof}
            Seien $a, b \in A \setminus \fp$. Es ist $ab\in A\setminus \fp$ zu zeigen.
            
            Da $(a)+\fp$ und $(b)+\fp$ echt größer als $\fp$ sind, liegen sie nicht in $\Sigma$. Somit existieren $m,n>0$ mit $x^m \in (a) + \fp$ und $x^n \in (b) + \fp$, weshalb wiederum Elemente $c,d\in A$ und $r,s \in \fp$ existieren, sodass $x^m = ac+r$ und $x^n = bd+s$ gilt. Wir erhalten \[x^{m+n} = (ac+r)(bd+s) = \underbrace{abcd}_{\in (ab)} + \underbrace{rbd + sac + rs}_{\fp} \in (ab)+\fp. \]
            Wäre $ab \in \fp$, so läge $x^{m+n}$ auch in $\fp$, was aber $\fp \notin \Sigma$ bedeuten würde, ein Widerspruch.
        \end{proof}

        Nach unserer Annahme liegt $x=x^1$ in $\fp$, was aber wegen $\fp \in \Sigma$ einen Widerspruch darstellt.
        \qedhere
    \end{description}
\end{proof}

\begin{defi}
    Sei $A$ ein Ring. Dann heißt \[\Jac(A) := \bigcap_{\mathclap{\fm\in \Max{A}}} \fm\] das \emph{Jacobson-Radikal} von $A$.
\end{defi}
\begin{beme}
    \leavevmode
    \begin{enumerate}
        \item $\Jac(A)$ ist ein Ideal von $A$.
        \item Sei $(A, \fm)$ ein lokaler Ring und ein Integritätsbereich. Dann gilt $\Nil(A)=(0)$ und $\Jac(A) = \fm$, d.h. $\Nil(A) \subsetneq \Jac(A)$ falls $A$ kein Körper ist.
        
        Beispielsweise ist $k\llbracket t\rrbracket$ ein lokaler Ring mit $\fm=(t)\neq (0)$ und ist nullteilerfrei, da mit $f=\sum_{i=m}^\infty a_it^i,g=\sum_{i=n}^\infty b_it^i \in k\llbracket t\rrbracket$ mit $a_m,b_n \neq 0$ dann \[ fg = \underbrace{a_mb_nt^{m+n}}_{\neq 0} + \text{ Terme höheren Grades} \neq 0 \] gilt.
    \end{enumerate}
\end{beme}
\begin{prop} \label{prop:jacobson}
    Sei $A$ ein Ring. Dann gilt $\Jac(A) = \{x\in A \mid \forall a\in A: 1-ax\in A^* \}$.
\end{prop}
\begin{proof}
    \leavevmode
    \begin{description}
        \item[\enquote{$\subseteq$}] Sei $x\in \Jac(A)$ und $a\in A$. Angenommen, $1-ax\not \in A^*$. Dann existiert ein $\fm\in \Max(A)$ mit $1-ax\in \fm$, sodass $1=(1-ax)+ax\in \fm$ gilt, Widerspruch.
        \item[\enquote{$\supseteq$}] Sei $x \in A$ mit $1-ax\in A^*$ für alle $a \in A$ sowie $\fm\in \Max(A)$. Angenommen, $x\not \in \fm$. Dann gilt $(x)+\fm=(1)$, es existieren also $a\in A, y\in \fm$ mit $1=ax+y$. Dann folgt aber $\fm \ni y = 1-ax \in A^*$, ein Widerspruch.
        \qedhere
    \end{description}
\end{proof}

\section{Moduln}
\subsection{Grundbegriffe}
\begin{defi}
    Sei $A$ ein Ring. Ein \emph{$A$-Modul} ist Tripel $(M, +, \cdot)$ besteht aus einer Menge $M$ und Abbildungen $+: M\times M \rightarrow M$, $\cdot: A\times M \rightarrow M$, sodass folgendes für alle $a,b\in A$ und $x,y \in M$ gilt:
    \begin{enumerate}
        \item $(M,+)$ ist eine abelsche Gruppe.
        \item $(a+b)x=ax+bx$
        \item $a(x+y)=ax+ay$
        \item $a(bx)=(ab)x$
        \item $1\cdot x=x$
    \end{enumerate}
\end{defi}
\begin{bsp} \label{bsp:moduln}
    \leavevmode
    \begin{enumerate}
        \item Falls $A=k$ Körper: $A$-Modul $=$ $k$-Vektorraum \label{bsp:moduln:i}
        \item $A$ ist selbst ein $A$-Modul, genannt ${}_AA$. \label{bsp:moduln:ii}
        \item Ist $\phi: A\rightarrow B$ ein Ringhomomorphismus, so wird $B$ ein $A$-Modul durch $a\cdot b = \phi(a)\cdot b$.
        \item Ist $A=\IZ$, so sind $A$-Moduln abelsche Gruppen (für $n\in \IZ, n>0: nx=\underbrace{x+x+\ldots+x}_{n\text{ mal}}$).
        \item Sei $A=k[t]$ ($k$ Körper). Sei $V$ ein $k$-Vektorraum und $\psi \in \End_k(V)$. Definiere $k[t] \times V \rightarrow V$ durch $(P,v) \mapsto (P(\psi))v:=\sum_{\nu=0}^n a_\nu(\psi^\nu)(v)$ mit $P(t)=\sum_{\nu=0}^n a_\nu t^\nu$. Dann wird $V$ ein $k[t]$-Modul.
    \end{enumerate}
\end{bsp}
\begin{defi}
    Sei $\phi: A\rightarrow B$ ein Ringhomomorphismus und $N$ ein $B$-Modul. Dann wird $N$ mit $+$ und $\cdot_A: A\times N\rightarrow N$ definiert durch $(a, y) \mapsto \phi(a)\cdot y$ zu einem $A$-Modul. Bezeichnung: ${}_AN$.
\end{defi}
\begin{bsp}
    \leavevmode
    \begin{enumerate}
        \item Mit \cref{bsp:moduln} \ref{bsp:moduln:i} und \ref{bsp:moduln:ii} gilt ${}_AB={}_A({}_BB)$.
        \item Sei $A=k$, $B=k[t]$ und $N$ $k[t]$-Modul.
        Dann ist ${}_kN$ ein $k$-Vektorraum.
        Betrachte $\psi: {}_kN \to {}_kN, y\mapsto t\cdot y$. Dann gilt $\psi\in \End_k({}_kN)$.
        Sei $P(t)=\sum_{\nu=0}^n a_\nu t^\nu\in k[t]$. Dann ist $P\cdot y = \sum_{\nu = 0}^n a_\nu \psi^n(y) = \sum_{\nu = 0}^n a_\nu (t\cdot y) = P\cdot y$ und wir erhalten einen $N = $ $k[t]$-Modul entstanden aus $({}_kN, \psi)$.
    \end{enumerate}
\end{bsp}
\begin{defi}
Seien $M$ und $N$ $A$-Moduln. Eine Abbildung $M\rightarrow N$ heißt \emph{$A$-linear}, wenn $f(x+x')=f(x)+f(x')$ und $f(ax)=af(x)$ für alle $x,x'\in M$ und $a\in A$ gilt.
Bezeichne $\Hom_A(M, N) := \{f \mid  \text{$f: M\rightarrow N$ ist $A$-linear}\}$.
\end{defi}


\lecture{16. April 2018}
\begin{beme}
	\leavevmode
	\begin{enumerate}
		\item Seien $M\xrightarrow{f}N\xrightarrow{g}P$ $A$-linear. Dann ist auch $g\circ f$ $A$-linear.
		\item Seien $M$ und $N$ $A$-Moduln. Dann wird $\Hom_A(M,N)$ zu einem $A$-Modul mit
		\begin{eqnarray*}
			f+g & \colon & M\to N,\, x\mapsto f(x) + g(x)\\
			af & \colon & M\to N,\, x\mapsto af(x)\; (= f(ax))
		\end{eqnarray*}
		für alle $f,g\in\Hom_A(M,N)$ und $a\in A$.
		
		Für die $A$-Linearität von $af$ wird benötigt, dass $A$ ein kommutativer Ring ist.
		\item Sei $\phi\colon A\to B$ ein Ringhomomorphismus sowie $f\colon M\to N$ $B$-linear. Dann ist $f\colon {}_AM\to {}_AN$ $A$-linear. Wir erhalten also eine Injektion $\Hom_B(M,N)\hookrightarrow \Hom_A({}_AM,{}_AN)$.
	\end{enumerate}
\end{beme}
\begin{defi}
	Sei $f\colon M\to N$ eine $A$-lineare Abbildung. $f$ heißt \emph{Isomorphismus} (von $A$-Moduln), wenn $f$ bijektiv ist.
\end{defi}
\begin{beme} Ist $f\colon M\to N$ ein Isomorphismus von $A$-Moduln, so ist $f^{-1}$ ebenfalls $A$-linear. Das heißt, $f$ ist genau dann ein Isomorphismus, wenn ein $A$-lineares $g\colon N\to M$ existiert, sodass $g\circ f = \id_M$ und $f\circ g = \id_N$ gilt.
\end{beme}
\begin{bsp}
	Sei $M$ ein $A$-Modul. Betrachte die Abbildung \begin{eqnarray*}
		M &\to& \Hom_A(A,M)\\
		x&\mapsto & [a\mapsto ax]\enspace .
	\end{eqnarray*}
	Das ist ein Isomorphismus von $A$-Moduln mit Umkehrabbildung
	\begin{eqnarray*}
		\Hom_A(A,M)&\to& M\\
		f &\mapsto& f(1) \enspace .
	\end{eqnarray*}
\end{bsp}
\subsection{Untermoduln}
\begin{defi}
	Sei $M$ ein $A$-Modul. Eine Teilmenge $M'\subseteq M$ heißt \emph{$A$-Untermodul}, falls
	\begin{enumerate}
		\item $0\in M'$ \label{defi:untermodul:i}
		\item $x+x'\in M'$ \label{defi:untermodul:ii}
		\item $ax\in M'$ \label{defi:untermodul:iii}
    \end{enumerate}
    für alle $x,x' \in M'$ und $a \in A$ gelten.

	Falls $M'\subseteq M$ ein $A$-Untermodul ist, ist $M'$ mit den Einschränkungen der Addition und der skalaren Multiplikation von $M$ wieder ein $A$-Modul; die Inklusion $M'\hookrightarrow M$ ist $A$-linear.
	
	Sei $M'\subseteq M$ ein $A$-Untermodul. Dann ist $\fak M{M'}$ (als abelsche Gruppe) mit 
	\begin{eqnarray*}
		A\times \fak M{M'} & \to & \fak M{M'}\\
		(a, x+M') &\mapsto& ax + M'
	\end{eqnarray*}
	ein $A$-Modul. Die Wohldefiniertheit ist schnell nachgerechnet.
	
	Die Quotientenabbildung $\pi\colon M\to \fak M{M'}$ ist $A$-linear (nach der Definition der Skalarmultiplikation).
\end{defi}
\begin{bsp}
	\leavevmode
	\begin{enumerate}
		\item Für $M = {}_AA$ sind die $A$-Untermoduln von ${}_AA$ genau die Ideale von $A$.
		\item Sei $A = k[t]$ und $M$ ein $A$-Modul. Dann ist $M$ eindeutig bestimmt durch $({}_kM, \psi)$, wobei $\psi\in\End_k(M)$ mit $\psi(x) = tx$. Sei $M'\subseteq M$ eine Teilmenge. Dann ist $M'$ genau dann ein $k[t]$-Untermodul, wenn $M'$ ein $k$-Untervektorraum von ${}_kM$ ist und $\psi(M')\subseteq M'$ gilt.
		\begin{proof}
		\leavevmode
		\begin{description}
			\item[\glqq $\Rightarrow$\grqq:] $\checkmark$
			\item[\glqq $\Leftarrow$\grqq:]
            Die Axiome \ref{defi:untermodul:i} und \ref{defi:untermodul:ii} von Untermoduln sind erfüllt, da $M'$ ein Untervektorraum ist. Für Axiom \ref{defi:untermodul:iii} sei $x\in M'$ und $P\in k[t]$ mit $P(t) = \sum a_\nu t^\nu$ mit $a_\nu\in k$. Dann folgt $P\cdot x = \sum a_\nu\psi^\nu(x)\in M'$, da $\psi^\nu(x)$ in $M'$ liegt.
            \qedhere
		\end{description}
		\end{proof}
		\item Sei $I\subseteq A$ ein Ideal, $M$ ein $A$-Modul. Definiere $I\cdot M \coloneqq \{\sum a_ix_i|a_i\in I, x_i\in M\}\subseteq M$. Das ist ein $A$-Untermodul. Beipsielsweise $I = (a)$; dann ist $(a)\cdot M = \{ax|x\in M\}$.
	\end{enumerate}
\end{bsp}
\begin{defi}
	Sei $f\colon M\to N$ eine $A$-lineare Abbildung. Definiere
	\begin{itemize}
		\item $\ker f \coloneqq \{x\in M\mid f(x) = 0\}$,
		\item $\im f \coloneqq f(M)$,
		\item $\coker f\coloneqq N/\im (f)$.
	\end{itemize}
	Der Kern und das Bild von $f$ sind dabei Untermoduln von $M$ bzw. $N$.
\end{defi}
\begin{lem}
	Sei $f\colon M\to N$ eine $A$-lineare Abbildung.
	\begin{enumerate}
        \item Sei $M'\subseteq M$ ein $A$-Untermodul mit $M'\subseteq \ker f$. Dann existiert genau eine Abbildung $\overline{f}\colon \fak M{M'}\to N$, sodass folgendes Diagramm kommutiert.
        \begin{figure}[H]
            \centering
            \begin{tikzcd}
                    M \arrow{r}{f} \arrow{d}{\pi} & N \\
                    \fak M{M'} \arrow[dashrightarrow]{ur}{\overline{f}}
            \end{tikzcd}
        \end{figure} \label{lem:homosatz fuer moduln:i}
        \item Es existiert genau eine $A$-lineare Abbildung $\tilde{f}\colon \fak M{\ker f}\to \im f$, sodass das folgende Diagramm kommutiert.
        \begin{figure}[H]
            \centering
            \begin{tikzcd}
                M \arrow{r}{f} \arrow{d}{\pi} & N \\
                \fak M{\ker f} \arrow[dashrightarrow]{r}{\tilde{f}} & \im f \arrow[hookrightarrow]{u}
            \end{tikzcd}
        \end{figure}
	\end{enumerate}
\end{lem}
\begin{proof}
	\leavevmode
	\begin{enumerate}
		\item Definiere $\overline{f}\colon \fak M{M'}\to N$ durch $\overline{f}(x+M') = f(x)$ (wohldefiniert, da $M'\subseteq \ker f$). Außerdem ist $\overline{f}$ auch $A$-linear, da $\overline{f}(ax+M') = f(ax) = a(f(x)) = a\overline{f}(x+M')$ gilt.
        \item Wenden wir \ref{lem:homosatz fuer moduln:i} auf $M' = \ker f$ an, so erhalten wir eine eindeutige $A$-lineare Abbildung $\overline{f}\colon \fak M{\ker f}\to N$, sodass
        \begin{figure}[H]
            \centering
            \begin{tikzcd}
                M \arrow{r}{f} \arrow{d}{\pi} & N \\
                M/\ker f \arrow[dashrightarrow]{ur}{\overline{f}}
            \end{tikzcd}
        \end{figure}
        
        kommutiert. Da $\im\overline{f} = \im f$, folgt
        \begin{figure}[H]
            \centering
            \begin{tikzcd}
                & N  \\
                M/\ker f \arrow{ur}{\overline{f}}\arrow[dashrightarrow]{r}{\exists!\tilde{f}}	& \im f\arrow[hookrightarrow]{u}
               \end{tikzcd}
        \end{figure}
		
        $\tilde{f}$ ist analog zum Homomorphiesatz für Gruppen bijektiv.
        \qedhere
	\end{enumerate}
\end{proof}

\begin{beme}
	Sei $M$ ein $A$-Modul sowie $M_i\subseteq M$ Untermoduln mit $i\in I$.
	\begin{enumerate}
		\item $\bigcap\limits_{i\in I} M_i$ ist ein Untermodul.
		\item Sei $T\subseteq M$ eine Teilmenge. Definiere
		\[\<T\> \coloneqq \bigcap_{\mathclap{\substack{T\subseteq M'\subseteq M\\\text{$M'$ Untermodul}}}} M'\] als den von $T$ erzeugten Untermodul.
		\item Die Menge $\sum\limits_{i\in I} M_i  \coloneqq \left\{\sum x_i \,\middle|\, \substack{x_i \in M_i \text{ mit }i\in I,\\\text{ nur endlich viele } x_i\neq 0}\right\}$ ist ein Untermodul.
	\end{enumerate}
\end{beme}
\begin{beme}
	Seien $M_i$ $A$-Moduln mit $i\in I$.
	\begin{enumerate}
		\item $\prod\limits_{i\in I} M_i\coloneqq \{(x_i)_{i\in I}\mid x_i\in M_i\}$ ist ein $A$-Modul mit $a(x_i) = (ax_i)$.
		\item $\bigoplus\limits_{i\in I} M_i\coloneqq \{(x_i)_{i\in I}\mid\text{nur endlich viele }x_i\neq 0\}\subseteq \prod\limits_{i\in I}M_i$ ist ein Untermodul.
        \item Sind $M_i\subseteq M$ Untermoduln und $M_i\cap \sum\limits_{j\neq i} M_j = \{0\}$, so ist $\sum\limits_{i\in I} M_i\cong \bigoplus M_i$.
        \item Falls $M_i = M$ für alle $i \in I$, schreibe $M^I \coloneqq \prod\limits_{i\in I} M_i$ und $M^{(I)}\coloneqq \bigoplus\limits_{i\in I}M_i$.
	\end{enumerate}
\end{beme}

\subsection{Endlich erzeugte Moduln}
\begin{defi}
	Sei $M$ ein $A$-Modul. Sei $\{x_i\}_{i\in I}\subseteq M$. $\{x_i\}$ heißt 
	\begin{itemize}
		\item linear unabhängig (linearly independent)
		\item Erzeugendensystem (generating system)
		\item Basis
	\end{itemize}
	analog zur linearen Algebra.
	\begin{enumerate}
        \item $M$ heißt \emph{frei}, falls eine der folgenden, äquivalenten Aussagen gilt:
        \begin{itemize}
            \item $M$ hat eine Basis.
            \item Es existiert eine Menge $I$, sodass ein Isomorphismus $A^{(I)}\xrightarrow{\cong}M$ existiert.
        \end{itemize}
        \item $M$ heißt \emph{endlich frei}, falls eine der folgenden, äquivalenten Aussagen gilt:
        \begin{itemize}
            \item $M$ hat eine endliche Basis.
            \item Es existiert ein $n \ge 0$, sodass ein Isomorphismus $A^{(I)}\xrightarrow{\cong}M$ existiert.
        \end{itemize}
        \item $M$ heißt \emph{endlich erzeugt}, falls eine der folgenden, äquivalenten Aussagen gilt:
        \begin{itemize}
            \item $M$ hat ein endliches Erzeugendensystem.
            \item Es existiert ein $n \ge 0$, sodass eine surjektive $A$-lineare Abbildung $A^{(I)}\to M$ existiert.
        \end{itemize}
        \item $M$ heißt \emph{endlich präsentiert}, falls ein $n \ge 0$ existiert, sodass eine surjektive $A$-lineare Abbildung $A^{(I)}\xrightarrow{f} M$ existiert, sodass $\ker f$ endlich erzeugt ist.
	\end{enumerate}
\end{defi}
\begin{bem}
	\leavevmode
	\begin{enumerate}
		\item endlich frei $\Rightarrow$ endlich präsentiert $\Rightarrow$ endlich erzeugt
		\item Falls $A = k$ Körper: endlich erzeugt $\Rightarrow$ endlich frei
		\item Sei $A = \IZ$, $M = \fak \IZ{m\IZ}$. Dann ist $M$ endlich präsentiert, aber nicht endlich frei.
		\item Sei $A = \IZ[T_1,T_2,T_3,\dots]$. Betrachte $\ev_{(T_i = 0)}\colon \IZ[T_1,T_2,\dots]\to \IZ$ und $M = {}_A\IZ$. Dann ist $M$ endlich erzeugt, aber nicht endlich präsentiert, denn $\ker\ev_{(T_i = 0)}= (T_1,T_2,\dots)$, was nicht endlich erzeugt ist.
	\end{enumerate}
\end{bem}
\begin{lem}[Nakayamas Lemma]
	Sei $M$ ein endlich erzeugter $A$-Modul und $I\subseteq A$ ein Ideal mit $I \cdot M = M$.
	\begin{enumerate}
		\item \label{lem:nakayama:i} Es existiert ein $a\in I$ mit $(1+a)\cdot M = \{0\}$ \textup(das heißt, für alle $x\in M: (1+a)x = 0$\textup)
		\item Falls $I\subseteq\Jac(A)$, so gilt $M = \{0\}$.
	\end{enumerate}
\end{lem}
\begin{proof}
	\leavevmode
	\begin{enumerate}
		\item Induktion nach $n$, der minimalen Anzahl der Erzeuger von $M$.
        \begin{description}
            \item[$n = 0$:] $M = 0$, $\checkmark$
            \item[$n-1\to n$:] Sei $x_1,\dots, x_n$ ein Erzeugendensystem von $M$. Definiere $N\coloneqq \fak M{\<x_n\>}$. Sei $\pi \colon M\to N$ die Quotientenabbildung. Dann folgt $N = \<\pi(x_1),\dots,\pi(x_{n-1})\>$. Nun gilt $I\cdot N = N$, da für $y\in N$ ein $x\in M$ mit $\pi(x) = y$ und $x\in I\cdot M$ existiert, und somit $a_i\in I$ sowie $\tilde{x_i}\in M$ mit $x =\sum a_i\pi(\tilde{x_i})\in I\cdot N$ existieren.
		
            Nach der Induktionsannahme existiert ein $b\in I$, sodass $(1+b)\cdot N = \{0\}\subseteq N = \fak M{\<x_n\>}$. Dann gilt
            \begin{align*}
                &\phantom{{}\Rightarrow{}}(1+b)\cdot M\subseteq \<x_n\>\\
                &\Rightarrow (1+b)\cdot M = (1+b)\cdot (I\cdot M) = I\cdot((1+b)\cdot M)\subseteq I \cdot \<x_n\>\\
                &\Rightarrow \exists c\in I: (1+b) x_n = cx_n \\
                &\Rightarrow (1+b-c)x_n = 0\\
                &\Rightarrow (1+b-c)(1+b)M = \{0\}
            \end{align*}
            Dabei ist $(1+b-c)(1+b) = (1+a)$ mit $a = b-c+b^2-bc\in I$.
        \end{description}
        \item Sei $x\in M$. Nach Punkt \ref{lem:nakayama:i} existiert ein $a\in I$ mit $(1+a)x = 0$. Da $a\in \Jac(A)$, folgt nach \cref{prop:jacobson} dann $1+a\in A^{*}$; also gilt $x = 0$.
        \qedhere
	\end{enumerate}
\end{proof}

	

\end{document}


%\begin{bsp}
%	Das hier ist ein Beispiel-Diagramm:

%	\begin{tikzcd}
%		X \arrow{rd}[swap]{g\circ f} \arrow{r}{f} & Y \arrow{d}{g} \\
%		W \arrow{u}	& Z \\
%	\end{tikzcd}
%\end{bsp}