\documentclass[12pt,a4paper]{scrartcl}

\usepackage{includes}
\usepackage{shortcuts}
\usepackage{numbering}

\author{Alexander Esgen \and Lukas Kempf \and Luise Puhlmann}
\title{Einführung in die Algebra}
\subtitle{Wintersemester 2017/18}


\begin{document}
\maketitle
\tableofcontents
\newpage

\lecture{9. Oktober 2017}

\url{http://www.math.uni-bonn.de/people/palmer/A1.html}

\paragraph {Organisatorisches}
\begin{itemize}
	\item Assistent: Martin Palmer
	\item Abgabe der Übungsblätter Donnerstag vor der Vorlesung
	\item Übungsgruppen Beginn nächste Woche
	\item Literatur siehe Homepage
\end{itemize}

\section{Gruppen}
\subsection{Grundlegendes} \label{ch:grp_basics}
\begin{defi} Eine Gruppe ist eine Menge $G$ zusammen mit einer Abbildung
\begin{align*}
	\circ\colon G\times G &\longto G\\
	 (g,h)&\longmapsto g\circ h
\end{align*}
(genannt Gruppenoperation), sodass gilt:
\begin{itemize}
	\item[(G1)]$(a\circ b)\circ c = a \circ (b\circ c) ~ \forall a,b,c\in G$ (Assoziativität)
	\item[(G2)] $\exists e\in G$ mit $g\circ e = g = e\circ g ~ \forall g\in G$ (Neutrales Element)
	\item[(G3)] $\forall g\in G\ \exists g^{-1}$ sodass $g\circ g^{-1} = e = g^{-1}\circ g$ (Inverse Elemente)
\end{itemize}
\end{defi}

\begin{bem}
\leavevmode
\begin{itemize}
	\item Neutrales Element $e$ ist eindeutig
	\item Inverse Elemente $g^{-1}$ sind eindeutig
	\item Es reicht sogar zu fordern: Existenz von Linksneutralem und Linksinversem oder Existenz von Rechtsneutralem und Rechtsinversem.
	\item Es gelten die Kürzungsregeln:
		\begin{eqnarray*}
			a\circ c = b\circ c &\Leftrightarrow& a = b\qquad \forall a,b,c\in G\\
			c\circ a = c\circ b &\Leftrightarrow& a = b \qquad \forall a,b,c\in G
		\end{eqnarray*}
\end{itemize}
\end{bem}

\begin{defi}
	$(G,\circ)$ heißt abelsch, falls $g\circ h = h\circ g$ für alle $g,h\in G$.
\end{defi}

\begin{bsp}
\leavevmode
\begin{itemize}
	\item $(\mathbb Z, +)$
	\item $(K,+,\cdot)$ Körper $\Rightarrow (K,+)$ und $(K^*=K\setminus \{0\}, \cdot)$ sind Gruppen
	\item $(V,+,\cdot)$ $K$-Vektorraum, dann ist $(V,+)$ eine Gruppe
	\item $K$ Körper, $n\in\mathbb N$; $G = \GL_n(K)$ ist Gruppe mit Matrixmultiplikation
	\item $M$ nichtleere Menge; $S_M := \{f\colon M\to M|f ~ \text{invertierbar}\}$ mit $\circ = $ Komposition von Abbildungen ist eine Gruppe; Spezialfall: $M = \{1,\dots n\},\ n\in\mathbb N$ ergibt die symmetrische Gruppe $S_n$ der Ordnung $n!$.
	\item Sei $(G,\circ)$ eine Gruppe und $a\in G$ fest gewählt. Dann ist $(G,\circ_a)$ eine Gruppe, wobei $g\circ_a h = g\circ a\circ h$.
\end{itemize}
\end{bsp}

\begin{defi} 
	$(G,\circ)$ Gruppe. Dann ist die Anzahl $\abs G$ der Elemente von $G$ die Ordnung von $G$.
\end{defi}

\begin{defi}
	 Sei $(G,\circ)$ Gruppe. Eine Teilmenge $H\subseteq G$ heißt Untergruppe (kurz UG), falls $H\neq\emptyset$ und $h_1,h_2\in H\Rightarrow h_1\circ h_2^{-1}\in H$. Wir schreiben dann: $H<(G,\circ)$ oder $H<G$.
\end{defi}

\begin{bem} $H<(G,\circ)$ gilt genau dann, wenn gilt:
	\begin{description}
	\item[(UG0)] $e\in H$
	\item[(UG1)] $h_1,h_2\in H\Rightarrow h_1\circ h_2\in H$
	\item[(UG2)] $h\in H\Rightarrow h^{-1}\in H$
	\end{description}
	Klar: Untergruppen sind Gruppen
\end{bem}

\begin{bsp}[selber nachprüfen!!!]
	\leavevmode
	\begin{itemize}
		\item $2\IZ < (\IZ, +)$
		\item $n\in\mathbb N$; $O(n) = \{A\in \GL_n(\IR)|AA^{T} = \mathds 1_n\}< \GL_n(\IR)$ die orthogonale Gruppe
		\item  $n\in\mathbb N$; $U(n) = \{A\in GL_n(\IC)|A\overline A^{T} = \mathds 1_n\}< \GL_n(\IC)$ die unitäre Gruppe
		\item $SL_n(K) = \{A\in \GL_n(K)|\det(A)=1\}<\GL_n(K)$
		\item $SO(n) = O(n)\cap SL_n(\IR)<O(n)$
		\item Spezielle Unitäre Gruppe
		\item $H(3,\IR) = \left\{\left(\begin{array}{ccc}
			1 & a & b \\ 
			0 & 1 & c \\ 
			0 & 0 & 1
		\end{array}\right) \right\}$: Obere Dreiecksmatrizen, nur 1en auf der Diagonalen (Heisenberggruppe)
	
	\end{itemize}
\end{bsp}

\begin{defi}
	Sei $(G,\circ)$ eine Gruppe. Sei $\emptyset\neq N\subseteq G$. Dann ist $\<N\>$ die kleinste (bzgl. Inklusion) Untergruppe von $G$, die $N$ enthält (also: $H<G$ mit $N\subseteq H\Rightarrow \<N\>\subseteq H$). Wir nennen $\<N\>$ die von $N$ erzeugte Untergruppe von $(G,\circ)$.
\end{defi}

\begin{bem}
	$\<N\>$ ist wohldefiniert, denn seien $H_1, H_2<G$ mit $N\subseteq H_1, N\subseteq H_2$, dann $N\subseteq H_1\cap H_2$ und $H_1\cap H_2<G$. Also existiert kleinste Untergruppe, die $N$ enthält; $\<N\>$ ist wohldefiniert.
\end{bem}

\begin{defi}
	$G$ Gruppe, $N\subseteq G$
	\begin{enumerate}
		\item $N$ erzeugt die Gruppe $G$, falls $\<N\> = G$. In diesem Fall heißt $N$ Erzeugendensystem der Gruppe $G$
		\item $(G,\circ)$ heißt endlich erzeugt als Gruppe, falls $\exists N\subseteq G$ mit $|N|$ endlich und $G = \<N\>$.
	\end{enumerate}
\end{defi}

\begin{bem}
	$(G,\circ)$ Gruppe, sei $N\subseteq G$. Dann gilt: $N$ erzeugt $G$ (also $G = \<N\>$) genau dann, wenn $\forall g\in G : \exists n_1,\dots,n_r\in G$ (mit $r\in \IN_0$), sodass $g = n_1\circ \dots \circ n_r$ (mit $g=e$, falls $r=0$) und $n_i\in N$ oder $n_i^{-1}\in N$ für alle $1\leq i\leq r$ (*).
\end{bem}

\begin{proof}
	\glqq$\Leftarrow$\grqq: Sei $g\in G$ und $g = n_1\circ\dots \circ n_r$ wie in (*). Daraus folgt $g\in \<N\>$, da $n_1,\dots,n_r\in \<N\>$ und dann auch $g$, weil $\<N\>$ Gruppe. Dadurch ist $G\subseteq \<N\>$, also $G = \<N\>$.\\
	\glqq$\Rightarrow$\grqq: Sei $G = \<N\>$. Behauptung: $H:=\{g\in G| g \mbox{ von der Form (*)}\}<G$. (dkddiermsü)
	
	Da $\<N\>\subseteq H$ nach Definition von $\<N\>$ gilt und $\<N\>$ eine Gruppe ist, muss also $\<N\> = H$ wegen Minimalität gelten, da $N\subseteq H$ gilt. Nach Voraussetzung folgt $G = H$. Also hat jedes $g\in G$ die Form (*).
\end{proof}

\begin{bsp}
	\leavevmode
	\begin{itemize}
		\item $\{$Transpositionen$\}\subseteq S_n$, d.h. $(i,j)$ mit $1\leq i<j\leq n$ erzeugen die Gruppe $S_n$
		\item $\{$Einfache Transpositionen$\}\subseteq S_n$, d.h. $(i,j)$ mit $1\leq i<j=i+1\leq n$ erzeugt $S_n$
	\end{itemize}
	
\end{bsp}

\begin{defi}
	Eine Gruppe $G$ heißt zyklisch, falls $\exists g\in G$, sodass $\gen{\{g\}} = G$ (d.h. falls $G$ von einem Element erzeugt wird).
\end{defi}

\noindent Beachte: $\gen{\{g\}} = \{e, g, g^{-1}, g^2, g^{-2},\dots\} = \{g^i|i\in\IZ\}$

\begin{bsp}

 $(\IZ,+)$ ist zyklisch mit $\IZ =\genSet{1} = \genSet{-1}$

\end{bsp}

\begin{defi}
	$(G,\circ)$ und $(G',\circ')$ seien Gruppen. Ein Gruppenhomomorphismus von $G$ nach $G'$ ist eine Abbildung $f\colon G\to G'$ mit $f(g\circ h) = f(g)\circ'f(h)\enspace \forall g, h\in G$.
	
	Er ist ein Gruppenisomorphismus, falls zusätzlich $f$ invertierbar ist. Wir schreiben $(G,\circ)\simeq (G',\circ')$, falls ein Gruppenisomorphismus von $G$ nach $G'$ existiert und nennen die Gruppen isomorph.
\end{defi}

\paragraph{Eigenschaften von Gruppenhomomorphismen} $f\colon G\to G'$ von $G$ nach $G'$ sei ein Gruppenhomomorphismus. Dann gilt:
\begin{description}
	\item[(E1)] $f$ Gruppenisomorphismus $\Leftrightarrow$ $f^{-1}$ Gruppenisomorphismus: Nach Definition existiert $f^{-1}$. Zu zeigen: $f^{-1}(g'\circ' h')= f^{-1}(g')\circ f^{-1}(h')$ für alle $g',h'\in G$. Sei $g', h' \in G' $. Daraus folgt $ \exists g, h\in G : f(g) = g', f(h) = h'$. Also:
	\begin{equation*}
		f^{-1}(g'\circ'h')= f^{-1}(f(g)\circ'f(h)) = f^{-1}(f(g\circ h))= g\circ h  = f^{-1}(g')\circ f^{-1}(h')
	\end{equation*} 
	
	\item[(E2)] $f$ bildet Neutrales auf Neutrales ab
	
	\lecture{12. Oktober 2017}
	\item[(E3)] $f$ bildet Inverse auf Inverse ab
	\item[(E4)] Sei $(G'',\circ'')$ eine weitere Gruppe; $f'\colon G'\to G''$ Gruppenhomomorphismus von $(G',\circ')$ nach $(G'',\circ'')$, dann ist $f'\circ f$ Gruppenhomomorphismus. Denn: 
	\begin{equation*}
		(f'\circ f)(g\circ h) = f'(f(g\circ h)) = f'(f(g)\circ'f(h)) = (f'\circ f)(g)\circ''(f'\circ f)(h)
	\end{equation*}
\end{description}

\begin{bsp}[Gruppenhomomorphismus]
	\leavevmode
	\begin{enumerate}
		\item $(G,\circ)$ mit $\mbox{id}\colon G\to G,\ g\mapsto g$ Gruppenhomomorphismus von $(G,\circ)$ nach $(G,\circ)$
		
		\textbf{Achtung} $\mbox{id}\colon G\to G,\ g\mapsto g$ ist kein Gruppenhomomorphismus von $(G,\circ)$ nach $(G,\circ_a)$, falls $a\neq e$
		
		\item $\det\colon \GL_n(K)\to K^*$ für einen Körper $K$ ist ein Gruppenhomomorphismus
		\item $f\colon \IR^*\to \IR_{\geq 0},\ x\mapsto |x|$ Gruppenhomomorphismus von $(\IR^*,\cdot)$ nach $(\IR_{\geq 0}, \cdot)$
		\item $x\mapsto \exp(x)$ Gruppenhomomorphismus von $(\IZ,+)$ nach $(\IR^*,\cdot)$
		\item Betrachte $G = \left\{\left.\left(\begin{array}{cc}
		1 & a \\ 
		0 & 1
		\end{array} \right)\right\vert a\in\IZ\right\}<\GL_n(\IR,\cdot)$ und $f\colon \IZ\to G,\ a\mapsto\left(\begin{array}{cc}
		1 & a \\ 
		0 & 1
		\end{array} \right)$. Das ist ein Gruppenhomomorphismus von $(\IZ,+)$ nach $(G,\mbox{Matrixmultiplikation})$. Es ist sogar ein Gruppenisomorphismus mit Inversem:$\left(\begin{array}{cc}
		1 & a \\ 
		0 & 1
		\end{array} \right)\mapsto a$
		\item \textit{Trivialer Gruppenhomomorphismus}: Schicke alles auf das neutrale Element
		\item Gegeben $(G,\circ)$ Gruppe, $a\in G$. Dann ist $f\colon G\to G,\ g\mapsto g\circ a^{-1}$ ein Gruppenhomomorphismus von $(G,\circ)$ nach $(G,\circ_a)$
	\end{enumerate}
\end{bsp}

\begin{lem}
	Sei $n\in \IZ$.
	\begin{enumerate}
		\item Dann $\exists!$ Gruppenhomomorphismus $can\colon \IZ\to \IZ/n\IZ$ von $(\IZ,+)$ nach $(\IZ/n\IZ,+)$ mit $can(1)=\overline{1}$
		\item Falls $n\neq 0$, existiert kein nichttrivialer Gruppenhomomorphismus $f\colon \IZ/n\IZ\to\IZ$ 
	\end{enumerate}
\end{lem}
\begin{proof}
	\leavevmode
	\begin{enumerate}
		\item
	Eindeutigkeit: Sei $f\colon \IZ\to \IZ/n\IZ$ ein Gruppenhomomorphismus. Dann $f(0)=\overline{0}$ nach (E2) und falls $f(1) = \overline{1}$, dann gilt $f(n)= f(1+\dots 1) = n\cdot f(1)$ für alle $n\in\IN$ und damit auch $f(-n) = -nf(1)$ nach (E5) $\Rightarrow$ $f$ eindeutig.
	
	\noindent Gruppenhomomorphismus: Es gilt dann $\text{can}(x) = \overline{x}$ für alle $x\in\IZ$ und da $\text{can}(x+y) = \overline{x+y} = \overline{x}+\overline{y} = \text{can}(x)+\text{can}(y)$ ist das auch ein Gruppenhomomorphismus.
	
	\item Sei $n\neq 0$. Sei $f\colon \IZ/n\IZ\to \IZ$ ein Gruppenhomomorphismus. Sei $f(\overline{1})= x$. Dann: (oBdA $n\in\IN$) $0=f(0)=f(\overline{n})= f(\overline{1}+\dots \overline{1})=nf(\overline{1})= nx\Rightarrow x=0$. Somit ist $f$ ein trivialer Gruppenhomomorphismus.
	\qedhere
	\end{enumerate}
\end{proof}

\begin{lem}
	Sei $(G,\circ)$ eine Gruppe.
	\begin{enumerate}
		\item Sei $\Aut(G) = \{f\colon G\to G| f \text{ Gruppenisomorphimus von $(G,\circ)$ nach }(G,\circ)\}$. Dann ist $\Aut(G)$ eine Gruppe, die Automorphismengruppen von $G$
		\item Betrachte die Abbildung $\Konj\colon G\to \Aut(G) ,\ g\mapsto \Konj(g)$, wobei $\Konj(g)(h)= g\circ\ h\circ g^{-1}$ für alle $h\in G$. Dann ist Konj ein Gruppenhomomorphismus von $G$ nach $\Aut(G)$. \textup(Im Allgemeinen nicht injektiv.\textup)
	\end{enumerate}
\end{lem}

\begin{proof}
	einfach nachrechnen
\end{proof}

\begin{bem}
	\leavevmode
	\begin{enumerate}
		\item 	Falls $(G,\circ)$ abelsch, dann ist jede Konjugation die Identität.
		\item $\Konj(g) = \text{id}_G \Leftrightarrow g\in Z(G):=\{x\in G|x\circ y = y\circ x\enspace \forall y\in G\}$
	\end{enumerate}
\end{bem}

\noindent\textbf{Konvention:} Von jetzt an schreiben wir meist $gh$ statt $g\circ h$ und $G$ statt $(G,\circ)$.

\begin{satz} \label{thm:kerim_g}
	Sei $f\colon G\to G'$ Gruppenhomomorphismus. Dann gilt:	
	\begin{equation*}
	\begin{array}{llll}
		\ker(f) & := \{g\in G|f(g) = e\}                  & <G & \mbox{ Kern von }f\\
		\im(f)  & := \{g'\in G'|\exists g\in G\ f(g)=g'\} & <G'& \text{ Bild von }f
	\end{array}
	\end{equation*}
\end{satz}

\begin{proof}
	einfach nachrechnen
\end{proof}
\begin{bsp}
	\leavevmode
	\begin{enumerate}
		\item $\ker(\text{can}\colon \IZ\to \IZ/n\IZ) = n\IZ<\IZ$
		\item $\ker(\Konj\colon G\to \Aut(G)) = Z(G)<G$
		\item $\ker(\det\colon \GL_n(K)\to K^*) = SL_n(K)$
	\end{enumerate}
\end{bsp}

\paragraph{Übung:} $f$ Gruppenhomomorphismus; $f$ ist injektiv genau dann, wenn $\ker f = \{e\}$.

\begin{satz}[Satz von Cayley] \label{thm:cayley}
	Sei $G$ eine Gruppe. Dann ist
	\begin{align*}
		\Phi \colon G&\longto S_G\\
		g&\longmapsto \Phi(g)
	\end{align*} mit $\Phi(g)(h) = gh$ für alle $h\in G$ ein injektiver Gruppenhomomorphismus. \textup(Damit kann man $G$ als Untergruppe einer Permutationsgruppe \glqq realisieren\grqq.\textup)
\end{satz}

\begin{proof}
	\leavevmode
	\begin{enumerate}
	\item Wohldefiniert: $\Phi(g)$ ist invertierbar mit Inversem $h\mapsto g^{-1}h$.	
	\item Gruppenhomomorphismus: Zu zeigen: $\Phi(g_1g_2) = \Phi(g_1)\circ \Phi(g_2)$, also $\Phi(g_1g_2)(h) = \Phi(g_1)(\Phi(g_2)(h))$ für alle $h\in G$. Es gilt $\Phi(g_1g_2)(h) = g_1g_2h$ und $\Phi(g_1)(\Phi(g_2)(h))  = \Phi(g_1)(g_2h) = g_1g_2h$, was zu zeigen war.
	
	\item Injektiv: Es reicht zu zeigen, dass der Kern trivial ist. Sei $g\in \ker\Phi\Leftrightarrow \Phi(g) = e = \text{id}_G \Leftrightarrow \Phi(g)(h)= h ~ \forall h\in G\Leftrightarrow gh = h ~ \forall h\in G\Leftrightarrow g= e$
	\qedhere
	\end{enumerate}
\end{proof}


\subsection{Satz von Lagrange und Normalteiler}
\begin{defi}
	Sei $G$ eine Gruppe, $H<G$ eine Untergruppe und $a\in G$. Dann ist:
	\begin{itemize}
		\item[] $aH = \{ah|h\in H\}\subseteq G$ die Linksnebenklasse von $H$ zu $a$
		\item[] $Ha = \{ha|h\in H\}\subseteq G$ die Rechtsnebenklasse von $H$ zu $a$
	\end{itemize}
	Meist arbeiten wir mit Linksnebenklassen und nennen sie einfach Nebenklassen.
\end{defi}

\noindent
Aus der Linearen Algebra wissen wir folgendes: \begin{enumerate}
	\item Zwei Nebenklassen sind gleich oder disjunkt d.h. $aH\cap bH \neq \emptyset \Leftrightarrow aH = bH\Leftrightarrow b^{-1}a \in H$
	\item Die Abbildung $aH\to H,\ ah\mapsto h$ ist bijektiv $\Rightarrow$ alle Nebenklassen haben dieselbe Kardinalität
	\item $$ G = \bigcup\limits_{g\in G}gH = \overset{.}{\bigcup\limits_{b\in R} }bH, $$ wobei $R\subseteq G$, sodass die $bH$ mit $b\in R$ genau ein Repräsentantensystem für die verschiedenen Nebenklassen bilden.
	\item $g\in aH\Leftrightarrow g^{-1}\in Ha^{-1}$ (dadurch ergibt sich eine Bijektion zwischen Links- und Rechtsnebenklassen)
	
\end{enumerate}

\begin{defi}
	Bezeichne mit $G/H$ die Menge der (Links)Nebenklassen von $G$ bezüglich $H$ und mit $ H\backslash G$ die Menge der Rechtsnebenklassen. Dann gilt $|G/H| = |H\backslash G|$ (nach (4)). Wir nennen diese Zahl den Index, auch $(G:H)$, von $H$ in $G$
\end{defi}

\begin{satz}[Satz von Lagrange] \label{thm:lagrange}
	Sei $G$ eine Gruppe, $H<G$ eine Untergruppe und gelte $|G|<\infty$. Dann gilt
	\begin{equation}
		|G| = |H|\cdot (G:H)
	\end{equation}
	Insbesondere: $|G| = p$ Primzahl $\Rightarrow H = \{e\}$ oder $H = G$.
\end{satz}

\begin{proof}
	Die Formel folgt direkt aus (3), (2) und der Definition des Index.
	Falls nun $|G| = p$ gilt, so muss auch $|H| = 1$ oder $|H| = p$ gelten, woraus $H = \{e\}$ oder $H = G$ folgt.
\end{proof}

\noindent Noch mehr Wissen aus der Linearen Algebra: Falls $G$ abelsch ist, dann ist $G/H$ wieder eine Gruppe mit Gruppenoperation
\begin{align*}
	\circ \colon G/H\times G/H &\longto G/H\\
	(aH,bH)&\longmapsto abH
\end{align*}
Im Allgemeinen (falls $G$ nicht abelsch ist) ist $\circ$ nicht wohldefiniert (siehe Übungsblatt 2).

\begin{defi}
	Sei $G$ eine Gruppe. Eine Untergruppe $H<G$ heißt Normalteiler, falls gilt: 
	\begin{equation*}
	\forall g\in G, h\in H: g\circ h\circ g^{-1}\in H
	\end{equation*}
	Wir schreiben dann: $H\vartriangleleft G$.
\end{defi}
\begin{bem}
	Falls $G$ abelsch, dann ist jede Untergruppe Normalteiler.
\end{bem}

\begin{lem} \label{lem:ker_nt}
	Sei $f\colon G\to G'$ ein Gruppenhomomorphismus.  Dann: $\ker(f)\nt G$.
\end{lem}
\begin{proof}
	Sei $g\in G$ und $h\in \ker f$. Dann gilt:
	\begin{align*}
		&f(ghg^{-1}) = f(g)f(h)f(g)^{-1} = f(g)f(g)^{-1} = e \\
		&\Rightarrow ghg^{-1}\in \ker f \\
		&\Rightarrow \ker f\nt G
  \qedhere
	\end{align*}
\end{proof}



\lecture{16. Oktober 2017}

\begin{satz} \label{thm:nt}
	Sei $G$ eine Gruppe, $N\nt G$ ein Normalteiler. Dann gilt:
	\begin{enumerate}
		\item $G/N$ bildet Gruppe mit $\circ\colon G/N\times G/N \to G/N,\enspace (aN, bN)\mapsto abN$.
		\item Die Abbildung 
		\begin{eqnarray*}
			\can\colon G \longto& G/N\\
			g \longmapsto& gN
		\end{eqnarray*}
	ist ein surjektiver Gruppenhomomorphismus.
	\end{enumerate}
\end{satz}

\begin{proof}
	\leavevmode
	\begin{enumerate}
		\item Es gilt $(aN\circ bN)\circ cN = abN\circ cN = abc N = aN\circ (bN\circ cN) \Rightarrow$ (G1).
		Offensichtlich ist $eN = N$ neutrales Element $\Rightarrow$ (G2).
		$a^{-1}N$ ist das Inverse zu $aN$ $\Rightarrow$ (G3).
		
		Jetzt ist noch die Wohldefiniertheit zu zeigen. Sei also $a_1N = a_2N$ und $b_1N = b_2N$. Daraus sollte $a_1b_1N = a_2b_2N$ folgen.
		
		Tatsächlich gilt $a_1^{-1}a_2 \in N$ und $b_1^{-1}b_2\in N$. Dann gilt auch $(a_1b_1)^{-1}(a_2b_2) = b_1^{-1}a_1^{-1} a_2b_2$, wobei $a_1^{-1}a_2\in N$ und 
		\begin{align*}
		&b_1^{-1}a_1^{-1} a_2b_2 = b_1^{-1}b_2(b_2^{-1}a_1^{-1}a_2b_2)\in N \\
		&\Rightarrow (a_1b_1)^{-1}a_2b_2\in N \\
		&\Rightarrow a_1b_1N = a_2b_2N
		\end{align*}
		
		
		\item Surjektivität ist klar nach (3); um zu zeigen, dass das ein Gruppenhomomorphismus ist, muss man das einfach nachrechnen.
  \qedhere
	\end{enumerate}
\end{proof}

\begin{bem}
	Somit gilt, dass Normalteiler genau die Kerne von Gruppenhomomorphismen sind.
\end{bem}

\begin{satz}[Homomorphiesatz] \label{thm:homsatz_g}
	Sei $f\colon G\to H$ ein Gruppenhomomorphismus. Sei $N\nt G$ ein Normalteiler. Dann: $N\subseteq \ker(f)\Leftrightarrow \exists!$ Gruppenhomomorphismus $\overline{f}\colon G/N\to H$, sodass $\overline{f}\circ \can = f$. Also 
	
	\begin{center}
	\begin{tikzcd}
		G \arrow{rd}[swap]{\can} \arrow{r}{f} & H  \\
				 	& G/N \arrow{u}[swap]{\exists! \overline{f}\text{ Gruppenhom}} \\
	\end{tikzcd}
	\end{center}
\end{satz}


\begin{proof}
	\glqq $\Leftarrow$\grqq: $\ker(\can) = \{g\in G|gN = N\} = \{g\in G|g\in N\} = N \Rightarrow f(N) = \overline{f}(\can(N)) = \overline{f}(e) = e\Rightarrow N\subseteq \ker(f)$.
	
	\noindent \glqq $\Rightarrow$\grqq: Eindeutigkeit: Es muss für $\overline{f}$ gelten: $\overline{f}(aN)=\overline{f}(\can(a)) = f(a)\enspace \forall aN\in G/N\Rightarrow \overline{f}$ eindeutig bestimmt durch $f$.
	
	Existenz: Wir setzen $\overline{f}(aN): = f(a)\enspace \forall aN\in G/N$. Das ist offensichtlich wohldefiniert. Nachrechnen ergibt, dass es auch ein Gruppenhomomorphismus ist.
\end{proof}

\begin{kor}
	Sei $f\colon G\to H$ ein Gruppenhomomorphismus. Dann gilt $G/\ker f \cong \im f$.
\end{kor}
\begin{proof}
	$\ker f\nt G$ nach \cref{lem:ker_nt} $\Rightarrow G/\ker f$ ist eine Gruppe nach \cref{thm:nt}. $\im f$ ist eine Gruppe nach \cref{thm:kerim_g}. Setze $N:= \ker f$. Klar: $N\subseteq \ker f$. Also existiert nach \cref{thm:homsatz_g} ein $\overline{f}$, sodass
	
	\begin{center}
	\begin{tikzcd}
		G \arrow{rd}[swap]{\can} \arrow{r}{f} & H  \\
		& G/\ker f \arrow{u}[swap]{\exists! \overline{f}\text{ Gruppenhom}} \\
	\end{tikzcd}
	\end{center}
	
	Also haben wir $\overline{f}\colon G/\ker f\to \mbox{im}f$ ein Gruppenhomomorphismus. Er ist surjektiv, weil $\can$ surjektiv ist. \\
	Behauptung: $\overline{f}$ ist injektiv.
	
	Es gilt $\overline{f}(aN)=f(a) = e \Leftrightarrow a\in \ker f = N$. Also $\ker \overline{f} = \{N\}$, was das neutrale Element in $G/\ker f$ ist. Also ist $\overline{f}$ injektiv. $\Rightarrow \overline{f}$ ist Gruppenisomorphismus.
\end{proof}

\begin{satz}[1. Isomorphiesatz] \label{thm:iso1_g}
	Sei $G$ eine Gruppe, $H<G$, $N\nt G$. Es gilt:
	\begin{enumerate}
		\item $HN:=\{hn|h\in H, n\in N\}<G$
		\item $N\nt HN$, $(H\cap N)\nt H$
		\item $H/(H\cap N) \cong HN/N$ mit dem Gruppenisomorphismus $h(H\cap N)\mapsto hN$
	\end{enumerate}
\end{satz}	
\begin{proof}
	\leavevmode
	\begin{enumerate}
		\item $HN\neq \emptyset$, da $e = ee\in HN$. Seien $h_1n_1,h_2n_2\in HN$ ($h_i\in H, n_i\in N$). Dann ist $h_1n_1(h_2n_2)^{-1} = h_1n_1n_2^{-1}h_2^{-1} = h_1h_2^{-1}h_2n_1n_2^{-1}h_2^{-1}$, wobei $n_1n_2^{-1}\in N$. Somit gilt auch $h_2n_1n_2^{-1}h_2^{-1}\in N$, da $N\nt G$. Da $h_1h_2^{-1}\in H$ ist der gesamte Ausdruck Element von $HN$.
		\item Zunächst zeigen wir, dass $N\nt HN$. Es gilt $N\subseteq HN$, da $n = en$. Daraus folgt, dass $N<HN$ weil $N<G$; analog auch $N\nt HN$, weil $N\nt G$.
		
		Noch zu zeigen: $(H\cap N)\nt H$. Es ist offensichtlich, dass $(H\cap N)\subseteq H$ und $(H\cap N)<H$, weil $(H\cap N)<G$. Sei $x\in H\cap N$, $h\in H$. Dann gilt $hxh^{-1}\in H$, weil $H<G$ und $hxh^{-1} \in N$ gilt, da $N\nt G$. Also $hxh^{-1}\in (H\cap N) \Rightarrow H\cap N\nt H$
		\item Betrachte 
		\begin{align*}
			f\colon H \longto& HN \xrightarrow{\can} HN/N\\
			h \longmapsto &he
		\end{align*}
		Es lässt sich leicht nachprüfen, dass $f$ ein Gruppenhomomorphismus ist. Für $x\in H$ gilt $x\in \ker(f)\Leftrightarrow xeN = N\Leftrightarrow x = xe\in \ker(\can) = N\Leftrightarrow x\in (H\cap N)$. Also existiert nach dem \namereff{thm:homsatz_g} ein Gruppenhomomorphismus $\overline{f}$:
		\begin{equation*}
			\overline{f}\colon H/(H\cap N)\longto (HN)/N
		\end{equation*}
		Dieser ist nach Konstruktion injektiv.
		
		Surjektiv: Sei $hnN\in (HN)/N$ mit $h\in H, n\in N$. Dann gilt aber: $hnN = hN$ und dann $f(h)=hN$. Somit gilt $\overline{f}\circ\can(h) = \overline{f}(\can(h)) = hN$ woraus folgt, dass $hN\in \im f$. Folglich ist $\overline{f}$ surjektiv und deshalb ein Gruppenisomorphismus.
  \qedhere
	\end{enumerate}
\end{proof}

\noindent Anmerkung zu Beweis des Homomorphiesatzes: Wo wird in \glqq$\Rightarrow$\grqq verwendet, dass $N\subseteq \ker f$? Es wird benötigt für die Wohldefiniertheit von $\overline{f}$.

\begin{satz}[2. Isomorphiesatz] \label{thm:iso2_g}
	Sei $G$ eine Gruppe; $N_1\nt G$, $N_2\nt G$, $N_1\subseteq N_2$. Dann gilt $N_1\nt N_2$ und $N_2/N_1\nt G/N_1$ und es gilt:
	$$(G/N_1)/(N_2/N_1) \cong G/N_2$$ durch den Isomorphismus $(gN_1)N_2/N_1\mapsto gN_2$.
\end{satz}	

\begin{proof}
	$G/N_1$ ist eine Gruppe, weil $N_1\nt G$. Analog für $N_2$. Auch gilt $N_2/N_1\subseteq G/N_1$. Aus $N_1\subseteq N_2$ folgt, dass $N_1\nt N_2$, weil $N_1\nt G$. Sei
	\begin{align*}
		f\colon G/N_1  \longto &G/N_2\\
		gN_1  \longmapsto & gN_2
	\end{align*}
	Das ist wohldefiniert: Seien $g, h\in G$.
	\begin{align*}
		&gN_1 = hN_1 \\
		&\Rightarrow g^{-1}h\in N_1\subseteq N_2 \\
		&\Rightarrow gN_2 = hN_2 \\
		&\Rightarrow \text{wohldefiniert}
	\end{align*}
	Klar: $f$ ist surjektiv und $gN_1\in \ker(f)\Leftrightarrow gN_2 = N_2\Leftrightarrow g\in N_2$. Also gilt $\ker(f) = \{gN_1|g\in N_2\} = N_2/N_1$. Also insbesondere $N_2/N_1\nt G/N_1$.
	Nach dem Korollar des Homomorphiesatzes erhalten wir einen Gruppenhomomorphismus
	$$ \overline{f}\colon (G/N_1)/\ker f(=N_2/N_1)\longto \im f = G/N_2 \mbox{ (da $f$ surjektiv)}$$
	Nach Kosntruktion ist $\overline{f}$ injektiv, also erhalten wir den gewünschten Gruppenisomorphismus mit $\overline{f}(gN_1\cdot (N_2/N_1)) = f(gN_1) = gN_2$.
\end{proof}

\paragraph{Anwendungen}
\begin{enumerate}
	\item \textit{Anzahlformel:} Sei $G$ eine endliche Gruppe, $H<G$, $N\nt G$. Dann gilt $$|HN| =\frac{ |H||N|}{|H\cap N|}$$ 
	Denn nach dem \namereff{thm:lagrange} ist $|H| = |H\cap N|(H:H\cap N)$ und $|HN| = |N|(HN:N)$. Nach dem \namereff{thm:iso1_g} ist $(H:H\cap N)=(HN:N)$. \hfill $\checkmark$
	\item Sie $(G,\circ) = (\IZ, +)$, $m,n\in\IN$ und $m|n$. Wir wissen: $m\IZ<\IZ$ und $n\IZ<\IZ$ (sogar Normalteiler, weil $G$ abelsch ist). Klar ist: $n\IZ\subseteq m\IZ$ (insbesondere auch $n\IZ\nt m\IZ$). Dann gilt 
	$$(\IZ/n\IZ)/(m\IZ/n\IZ) \cong \IZ/m\IZ$$
\end{enumerate}

\subsection{Zyklische Gruppen} \label{ch:zyklisch}
Wir schreiben kurz $\<g\>$ statt $\<\{g\}\>$.
\begin{satz}
	Untergruppen von zyklischen Gruppen sind zyklisch.
\end{satz}

\begin{proof}
	Sei $G$ eine zyklische Gruppe; $G = \<g\>$ mit $g\in G$. Sei $H<G$.\begin{description}
		\item[Fall 1] $H = \{e\} = \<e\>$, also zyklisch
		\item[Fall 2] $H\neq \{e\}\Rightarrow \exists m\in \IZ\setminus\{0\}: e\neq g^m\in H\Rightarrow \exists n\in \IN: e\neq g^n\in H$ (weil $H<G$). Wähle $n := \min\{j\in \IN|e\neq g^j\in H\}$. Behauptung: $H = \<g^n\>$.
		
		\glqq$\supseteq$\grqq: Klar, da $g^n\in H$
		
		\glqq$=$\grqq: Angenommen, Gleichheit gilt nicht. Also $\exists s\in \IZ: g^s\in H\setminus\<g^n\>$ (beachte $G = \<g\>$). Schreibe $s = an+r$ für $a,r\in \IZ$ und $0\leq r<n$. Falls $r = 0$, dann $s = an$ und $g^s = g^{an} = (g^n)^a\in \<g^n\>$ Widerspruch!
		
		Falls $r>0$: Dann $g^r = (g^{an})^{-1}g^{an}g^r = ((g^n)^a)^{-1}g^s\in H$ (Widerspruch zur Minimalität)
		
		Somit war die Annahme falsch und $H$ ist zyklisch.
  \qedhere
	\end{description}
\end{proof}

\lecture{19. Oktober 2017}

\begin{lem}
	Bilder von zyklischen Gruppen unter Gruppenhomomorphismen sind zyklisch.
\end{lem}

\begin{proof}
	Sei $f: G \rightarrow G'$ ein Gruppenhomomorphismus und sei $G$ zyklisch, also $G = \<g\>$ für ein $g \in G$ $\Rightarrow G = \left\lbrace g^i | i \in \IZ \right\rbrace$ also $f(G) = \left\lbrace f(g^i) | i\in \IZ \right\rbrace = \left\lbrace (f(g^i)) | i \in \IZ)\right\rbrace = \<f(g)\> \Rightarrow \im f = \<f(g)\>$ zyklisch.
\end{proof}

\begin{lem} \label{lem:ord}
	Sei $G$ endliche Gruppe $\abs G = n < \infty$. Sei $g \in G$ mit $G=\<g\>$ \textup(also $G$ zyklisch\textup).
	Sei $\ord(g) = \min \left\lbrace j \in \IN | g^j = e\right\rbrace $.
	Dann gilt: $\ord(g) = n$. 
\end{lem}

\begin{defi}
	Allgemeiner: Sei $G$ irgendeine Gruppe, $g \in G$. Dann definiere \begin{equation*}
		\ord(g) := \begin{cases*}
		\min \left\lbrace j \in \IN | g^j = e\right\rbrace &falls das existiert \\
		\infty &sonst
		\end{cases*}
	\end{equation*}
	Wir nennen $\ord(g)$ die Ordnung von $g \in G$.
\end{defi}

\begin{proof} [Beweis von Lemma \ref{lem:ord}]
	\begin{enumerate}
	\item Behauptung: $\ord(g)$ existiert. Angenommen es existiert nicht, also 
	$
		g^j \not = g ~ \forall j \in \IN \Rightarrow g^i \neq g^j \text{ falls } i \neq j, ~ i,j \in \IN 
	$
	(denn sonst gilt $g^{i-j}=e=g^{j-i}$ mit $i-j \in \IN$ oder $j-i \in \IN$).
	Also $\abs G = \infty \Rightarrow$ Widerspruch.
	
	Jetzt ist noch zu zeigen, dass $n = \ord(g)$ gilt. Dazu sei $S := \left\lbrace g, g^2, \ldots, g^{\ord(g)} = e \right\rbrace \subset G$.
	
	\item Behauptung: $S < G$. Klar: $e \in S$. Sei $g^a, g^b \in S$. Schreibe $a-b = k \cdot \ord(g) + r$, wobei $k, r \in \IZ, 0 \leq r < \ord(g)$. Daraus folgt
	\begin{equation*}
		g^a\left( g^b \right)^{-1}=g^{a-b}=g^{k \cdot \ord(g)+r}
		= \left(g^{\ord(g)} \right)^kg^r = e^kg^r=eg^r=g^r \in S
	\end{equation*}
	weil $0 \leq r < \ord(g)$.
	Da $g \in S$, gilt $\<g\>\subset S$. Weil $S < G$ ist klar, dass $S \subset \<g\>$, also $\<g\> = S$.
	\item Behauptung: $\abs S = \ord(g)$. Seien $g^i, g^j \in S$ mit $1 \leq i,j \leq \ord(g)$ und $g^i=g^j$. Also $g^{i-j} = e = g^{j-i}$, was ein Widerspruch zur Minimaltität von $\ord(g)$ ist außer $i=j$. Folglich sind die $g^i (1\leq i \leq \ord(g))$ paarweise verschieden, was die Behauptung zeigt.
	\qedhere
	\end{enumerate}
\end{proof}

\begin{bem}
	Sei $G$ irgendeine Gruppe, $g\in G$. Dann gilt: $\ord(g) = \abs{\<g\>}$ und nach \namereff{thm:lagrange} denn $\ord(g)$ teilt $\abs G$, falls $\abs G$ endlich.
\end{bem}

\begin{satz}[Klassifikation zyklischer Gruppen] \label{thm:klasszykl}
	Je zwei zyklische Gruppen der selben Ordnung sind isomorph. Genauer gilt für $G$ zyklische Gruppe: 
	\begin{equation*}
		G \cong \begin{cases*}
			\IZ &falls $\abs G = \infty$ \\
			\IZ/n\IZ &falls $\abs G = n$
		\end{cases*}
	\end{equation*}
\end{satz}

\begin{proof}
	Sei $G = \<g\>$ mit $g \in G$. Sei $f: \IZ \rightarrow G : j \mapsto g^j$. Dann ist $f$ ein Gruppenhomomorphismus (nachrechnen) und surjektiv, da $G = \<g\>$.
	\begin{itemize}
		\item[Fall 1] $\abs G = \infty$. Dann muss $f$ injektiv sein, damit $f$ ein Isomorphismus ist und damit $\IZ \cong G$. Falls $f$ nicht injektiv ist, dann $\exists i,j \in \IZ, i\neq j$ mit $g^i=g^j$, als $g^{i-j} = e = g^{j-i}$. Folglich ist $\ord(g) < \infty$. Damit wäre $G$ nach \ref{lem:ord} endlich, was ein Widerspruch ist.
		\item[Fall 2] $\abs G = n$ endlich. Dann folgt aus \ref{lem:ord}: 
		\begin{equation*}
			\ord(g)=n \Rightarrow g^n = e \Rightarrow g^{nk} = (g^n)^k = e^k = e ~ \forall k\in \IZ \Rightarrow n\IZ \subset \ker F
		\end{equation*}
		Nach dem Homomorphiesatz gilt dann: 
		\begin{center}
		\begin{tikzcd}
					\IZ \arrow{rd}[swap]{\text{can}} \arrow{r}{f} & G  \\
						& \IZ/n\IZ \arrow{u}[swap]{\exists!\ \overline{f}\text{ Gruppenhom}} \\
		\end{tikzcd}
		\end{center}
		 Also $\overline{f} : \IZ/n\IZ \rightarrow G$. Da $\abs{\IZ/n\IZ} = n = \abs G$ muss diese surjektive Abbildung schon ein Isomorphismus sein.
  \qedhere
	\end{itemize}
\end{proof}

\subsection{Auflösbare Gruppen}
\begin{defi}
	Eine Normalreihe eine Gruppe $G$ ist eine Kette von Untergruppen der Form $\left\lbrace e\right\rbrace = G_0 \nt G_1 \nt \ldots \nt G_n = G$. Man nennt die Quotientengruppe $G_i/G_{i-1}$ die Faktoren der Normalreihe.
\end{defi}

\begin{defi}
	Eine Gruppe heißt auflösbar, falls eine Normalreihe mit abelschen Faktoren existiert.
\end{defi}

\begin{bsp}
	\leavevmode
	\begin{enumerate}
		\item Abelsche Gruppen sind auflösbar: $\left\lbrace e \right\rbrace \nt G$ und $G/\left\lbrace e \right\rbrace \cong G $, also abelsch
		\item Sei $G = \left\lbrace \begin{pmatrix}
		a & b \\ 
		0 & d
		\end{pmatrix} \in \GL_2(K) \right\rbrace < \GL_2(K) $. Behauptung: $G$ ist auflösbar. Dazu betrachtet man $G' = \left\lbrace \begin{pmatrix}
		a & 0 \\
		0 & d
		\end{pmatrix} \in \GL_2(K)\right\rbrace < \GL_2(K)$, wobei $G'$ insbesondere eine Gruppe ist. 
		\begin{equation*}
			f: G \longto G': \begin{pmatrix}
			a & b \\ 
			0 & d
			\end{pmatrix} \longmapsto \begin{pmatrix}
			a & 0 \\ 
			0 & d
			\end{pmatrix}
		\end{equation*}
		was ein Gruppenepimorphismus ist (nachrechnen). Es gilt: 
		\begin{equation*}
			\ker f = \left\lbrace \begin{pmatrix}
			1 & b \\ 
			0 & 1
			\end{pmatrix} \lvert b \in K \right\rbrace \nt G
		\end{equation*}
		Folglich gilt $\ker f \cong (K, +)$, sodass $\begin{pmatrix}
		1 & b \\ 
		0 & 1
		\end{pmatrix} \mapsto b$, weil $\begin{pmatrix}
		1 & b \\ 
		0 & 1
		\end{pmatrix}\begin{pmatrix}
		1 & b' \\ 
		0 & 1
		\end{pmatrix}=\begin{pmatrix}
		1 & b + b' \\ 
		0 & 1
		\end{pmatrix}$ als Gruppenhomomorphismus offensichtlich bijektiv ist. Damit ist $\ker f$ abelsch und $G'$ somit auch.
		\begin{equation*}
			\Rightarrow \left\lbrace e\right\rbrace  = G_0 \nt \ker f = G_1 \nt G_2 = G
		\end{equation*}
		und $\ker f /\left\lbrace e\right\rbrace$ abelsch, sowie auch $G/\ker f \cong \im f = G'$ abelsch. Somit ist $G$ auflösbar.
		\item $S_4$ ist auflösbar. Betrachte
		\begin{equation*}
			S_4 > A_4 := \left\lbrace \pi \in S_4 | \sgn(\pi) = 1 \right\rbrace 
		\end{equation*}
		Nach LA 1 ist $\sgn$ ein Gruppenhomomorphismus und damit $A_4 = \ker(\sgn) < S_4$. Es gilt $S_4 \nt A_4$, weil $A_4 = \ker(\sgn)$ oder weil $(S_4 - A_4) = 2$, was dann nach Blatt 2 folgt. Betrachte nun 
		\begin{equation*}
			A_4 > V_4 := \left\lbrace e, \underbrace{(1,2)(3,4)}_a, \underbrace{(1,3)(2,4)}_b, \underbrace{(1,4)(2,3)}_c\right\rbrace
		\end{equation*}
		Gruppentafel:
		$\begin{array}{c||c|c|c}
			& a & b & c \\ \hline \hline 
			a & e & c & b \\ 
			\hline 
			b & c & e & a \\ 
			\hline 
			c & b & a & e \\ 
		\end{array}$ \\
		Dann gilt $A_4 \nt V_4$, da folgendes gilt: 
		\begin{equation*}
			\forall  \pi \in S_4: \pi \circ \underbrace{(a_1, a_2)(a_3, a_4)}_\tau \circ \pi^{-1} = (\pi(a_1), \pi(a_2))(\pi(a_3), \pi(a_3))
		\end{equation*}weil
		\begin{alignat*}{3}
			\pi(a_1)&\xmapsto{\pi^{-1}}&a_1&\xmapsto{\tau}&a_2&\xmapsto{\pi}\pi(a_2) \\
			\pi(a_2)&\longmapsto&a_2&\mapsto& a_1&\mapsto\pi(a_1) \\
			\pi(a_3)&\longmapsto&a_3&\mapsto& a_4&\mapsto\pi(a_4) \\
			\pi(a_4)&\longmapsto&a_4&\mapsto& a_3&\mapsto\pi(a_3) \\
		\end{alignat*} also $V_4 \nt A_4$. Folglich haben wir
		\begin{equation*}
			\left\lbrace e\right\rbrace = G_0 \nt V_4 = G_1 \nt A_4 = G_2 \nt S_4 = G_3
		\end{equation*}
		\begin{description}
			\item[$G_1/G_0 \cong V_4$] ablesch
			\item[$G_2/G_1 \cong \IZ/2\IZ$] also abelsch, da jede Gruppe $H$ der Ordunung 2 zyklisch mit $H = \<g\> (g \neq e)$ ist und dann nach Klassifikationssatz $H \cong \IZ/2\IZ$
			\item[$G_3/G_2$] Wir wissen, dass $\abs{G_3/G_2} = 3$. Dann behaupten wir, dass $G_3/G_2 \cong \IZ/3\IZ$. Jede Gruppe $H$ mit $\abs H = 3$ ist zyklisch, denn $\<g\> < H (g \neq e)$. Nach dem \namereff{thm:lagrange} gilt $\<g\> = H$, weil $\<g\> \neq e$ und 3 prim ist. Also folgt die Aussage aus dem Klassifikationssatz.
		\end{description}
		Daraus folgt, dass $S_4$ auflösbar ist.
	\end{enumerate}
\end{bsp}

	\begin{satz}
		Untergruppen und Bilder unter Gruppenhomomorphismen von auflösbaren Gruppen sind auflösbar.
	\end{satz}
	\begin{proof}
		Sei $G$ auflösbar. Dann existiert eine Auflösung 
		\begin{equation*}
			\left\lbrace e\right\rbrace = G_0 \nt G_1 \nt \ldots \nt G_n = G, \qquad  G_i/G_{i-1} \text{ abelsch.}
		\end{equation*}
		\begin{description}
			\item[Untergruppe:]~
				\begin{enumerate}
				\item Sei $U < G$. Behauptung: $\left\lbrace e\right\rbrace = G_0 \cap U \nt (G_1 \cap U) \nt \ldots \nt (G_n \cap U) = U $. Es ist klar, dass $(G_{i-1} \cap U) \subset (G_i \cap U)$. Auch klar ist, dass $G_i \cap U$ eine Gruppe ist und $(G_{i-1} ) < (G_i \cap U)$. Jetzt ist noch zu zeigen, dass $(G_{i-1} \cap U) \nt (G_i \cap U)$. Sei $x \in G_{i-1} \cap U$ und sei $y \in G_i \cap U$. Dann folgt, dass $\underbrace{yxy^{-1}}_{\in G_{i-1}} \in U$, weil $x, y \in U, U < G$, weil $x \in G_{i-1}, y \in G_i$ und $G_{i-1} \nt G_i$. Daraus folgt, dass $yxy^{-1} \in U \cap G_{i-1}$, was zu zeigen war.
				\item Behauptung: $G_i \cap U / G_{i-1} \cap U$ abelsch. Nach dem \namereff{thm:iso1_g} gilt $G_i \cap U / G_{i-1} \cap U \cong (U \cap G_i)G_{i-1}/G_{i-1} \nt G_i/G_{i-1}$ abelsch. Daraus folgt die Behauptung.
			\end{enumerate}
\lecture{23. Oktober 2017}
			\item[Bild:]
				Sei $f\colon G\to G'$ Gruppenhomomorphismus. Behauptung: $\{e\} = f(G_0)\nt f(G_1)\nt \dots \nt f(G_n ) = f(G)$ ist eine Normalreihe mit $f(G_i)/f(G_{i-1})$ abelsch.
				
				Sei $y' = f(y)\in f(G_i)$ mit $y\in G_i$. Dann gilt $y'f(G_{i-1})(y')^{-1} = f(yG_{i-1}y^{-1})\subseteq f(G_i)\Rightarrow f(G_{i-1})\nt f(G_i)$ für alle $i$. Betrachte nun 
\[\alpha\colon G_i\xto{f} f(G_i)\xto{\text{can}}f(G_i)/f(G_{i-1}).\] Dies ist ein Gruppenhomomorphismus, welcher offensichtlich surjektiv ist. Da $G_{i-1}\subseteq \ker \alpha$, existiert Gruppenhomomorphismus $\overline{\alpha}\colon G_i/G_{i-1}\to f(G_i)/f(G_{i-1})$ nach dem \namereff{thm:homsatz_g}. $\overline{\alpha}$ ist surjektiv, da auch $\alpha$ surjektiv ist. Weil $G_i/G_{i-1}$ abelsch ist, ist auch $f(G_i)/f(G_{i-1})$ abelsch. Somit folgt die Behauptung.
		\end{description}
\end{proof}


\begin{defi}
	Sei $G$ eine Gruppe, $M := \{ghg^{-1}h^{-1}|g,h\in G\}$; dann heißt $[G,G] = \<M\>$ Kommutatorgruppe.
\end{defi}

\begin{bem}
	Nach dem 2. Übungsblatt gilt $[G,G] \nt G$. $[G,G]\nt G$ ist sogar der kleinster Normalteiler, sodass $G/[G,G]$ abelsch (denn: sei $N\nt G, a,b\in G, aNbN = bNaN\Leftrightarrow abN = baN\Leftrightarrow a^{-1}b^{-1}ab\in N\Leftrightarrow [G,G]\subseteq N$).
\end{bem}

\noindent
Betrachte zu einer Gruppe die abgeleitete Reihe:

\begin{equation}
\underbrace{G}_{D^0(G)} \tn \underbrace{[G,G]}_{D^1(G)} \tn \underbrace{[D^1(G),D^1(G)]}_{D^2(G)} \tn \ldots . \tag{*}\label{eq:abgeleitetereihe}
\end{equation}

\begin{satz}
	$G$ auflösbar $\Leftrightarrow \exists m\in\IN : D^m(G) = \{e\}$.
\end{satz}
\begin{proof}
	\leavevmode
	\begin{itemize}
		\item [\glqq $\Leftarrow$\grqq] Die abgeleitete Reihe \eqref{eq:abgeleitetereihe} ist nach Definition eine Normalreihe und die Faktoren sind abelsch nach der Bemerkung.
		\item [\glqq $\Rightarrow$\grqq] Sei $G$ auflösbar und $\{e\} = G_0\nt G_1\nt\dots G_n = G$ mit abelschen Faktoren. Nach der Bemerkung gilt $G_n/G_{n-1}$ abelsch $\Rightarrow [G_n,G_n]\subseteq G_{n-1}$.
		
		Behauptung: $D^i(G) \subseteq G_{n-i}$. Dies ist für $i = 0;1$. $D^{i+1}(G) = [D^i(G), D^i(G)]\subseteq [G_{n-1},{n-1}]\subseteq G_{n-i-1}$ nach Bemerkung. Also existiert ein $n\in N$ sodass $D^n(G) \subseteq G_0 = \{e\}\Rightarrow \exists m:=n$ mit $D^m(G) =\{e\}$.
  \qedhere
	\end{itemize}
\end{proof}


\subsection{Gruppenoperationen}
\begin{defi}
	Sei $G$ eine Gruppe, $X\neq \emptyset$ eine Menge. Eine Operation von $G$ auf $X$ ist eine Abbildung
	\begin{align*}
		\Phi\colon G\times X \longto & X\\
		(g,x) \longmapsto & g.x = \Phi(g,x)
	\end{align*}
	sodass
	\begin{description}
		\item[(O1)] $e.x = x$ für alle $x\in X$
		\item[(O2)] $g.(h.x) = (gh).x$ für alle $g,h\in G, x\in X$
	\end{description}
	Kurz: $G$ operiert auf $X$; wir schreiben $G\operates X$.
\end{defi}

\begin{bem}
	Existenz von $\Phi$ ist äquivalent zur Existenz von $\Phi'\colon G\to S_X$ Gruppenhomomorphismus mit $\Phi'(g)(x) := g.x$ (nachprüfen!)
\end{bem}

\begin{defi}
	Gegeben $G\operates X$, $G\operates Y$, $f\colon X\to Y$ Abbildung. $f$ heißt $G$-Homo"-morphismus, falls $f(g.x) = g.f(x)$ für alle $g\in G$ und $x\in X$.
\end{defi}

\begin{defi}
	$G\operates X$, $x\in X$. Dann\begin{enumerate}
		\item $G.x = \{g.x\mid g\in G\}$ Bahn von $x$
		\item $G_x = \{g\in G\mid g.x = x\}$ Stabilisator von $x$
		\item $X^G = \{x\in X\mid \forall g\in G\enspace g.x = x\}$ Menge der Fixpunkte
	\end{enumerate}
\end{defi}

\begin{bem}
	$x\sim y \Leftrightarrow : y\in G.x$ ist eine Äquivalenzrelation:\begin{itemize}
		\item $x\sim x$ klar, weil $x = e.x\in G.x$
		\item $x\sim y\Rightarrow \exists g\in G: g.x = y\Rightarrow x = g^{-1}.y\Rightarrow x\in G.y\Rightarrow y\sim x$
		\item $x\sim y, y\sim z\Rightarrow x\sim z$ klar nach (O2)
	\end{itemize}
	Also $X=\dot\bigcup\substack{\text{versch.}\\\text{Bahnen}}$.
\end{bem}

\begin{defi}
	$G$ operiert transitiv, falls genau eine Bahn existiert. Die Operation $\Phi$ heißt treu, falls $\Phi'$ (siehe obige Bemerkung) injektiv ist.
\end{defi}


\begin{bsp}~
	\begin{enumerate}
		\setcounter{enumi}{-1}
		\item $SO(\IR) \operates \IR^2$ durch Drehungen um $(0,0)$. Hier gibt es unendlich viele Bahnen.
		\begin{center}
			TOLLES BILD -  to be inserted
		\end{center}
		\item $G$ Gruppe, $H<G$, $X = G$. Es sei $H \operates G$ durch
		\begin{enumerate}[label=\alph*)]
			\item $h.x\coloneqq hx$ (linksreguläre Operation)
			\item $h.x\coloneqq xh^{-1}$ (rechtsreguläre Operation)
			\item $h.x\coloneqq hxh^{-1}$ (Konjugation)
		\end{enumerate}
		für alle $x\in X, h \in H$ definiert, wobei sich die folgenden Eigenschaften ergeben:
		\begin{enumerate}[label=\alph*)]
			\item \begin{itemize}[itemsep=-3pt]
				\item treu (nach dem \namereff{thm:cayley})
				\item transitiv $\Leftrightarrow G = H$
				\item Bahnen $=$ Rechtsnebenklassen
				\item $X^H = \emptyset \Leftrightarrow H \neq \{e\}$, sonst sind alle $x \in X$ Fixpunkte
			\end{itemize}
			\item wie a), außer Links- statt Rechtsnebenklassen als Bahnen
			\item \begin{itemize}[itemsep=-3pt]
				\item Bahnen $=$ Konjugationsklassen
				\item ${\displaystyle X^H = \{x \in X \mid \forall h \in H: h.x = x \} = \underbrace{\{x \in X \mid \forall h \in H: hxh^{-1} = x \}}_{\substack{\text{\emph{Zentralistator}}\\\text{bzgl. Konjugation auf }H}}}$
				\item Spezialfall $H=G$: $X^H = Z(G)$.
			\end{itemize}
		\end{enumerate}
		\item Sei $G$ eine Gruppe, $X = \{H < G \}$ und sei $G \operates X$ definiert durch Konjugation als \[ g.H \coloneqq gHg^{-1} = \{ ghg^{-1} \mid h \in H \} \in X. \]
		\begin{itemize}[itemsep=-3pt]
			\item Bahnen $=$ Konjugationsklassen von Untergruppen
			\item Stabilisator von $H\in X$: $G_H = \{ g.H = H \}$, heißt auch \emph{Normalisator} von $H$ in $G$, schreib $N_G(H)$.
			\item $X^G = \{ H < G \mid \forall g \in G: g.H = H \}=\{ H < G \mid \forall g \in G: gHg^{-1} = H \} = \{H \nt G \}$
		\end{itemize}
		\item Sei $G$ eine Gruppe, $H < G$, $X = G/H$. Dann $G \operates X$ durch $g.(aH) = gaH$ für alle $g\in G, a \in G$; heißt \emph{Linkstranslation}. \label{bsp:linkstrans}
		\begin{itemize}[itemsep=-3pt]
			\item transitiv, da $\forall a,b \in G: \exists g \in G: g(aH)=bH$.
			\item Im Allgemeinen nicht treu, da \[\ker\Phi' = \underbrace{\bigcap_{x \in G}xHx^{-1}}_{\mathclap{\text{kleinster Normalteiler}}}.\]
		\end{itemize}
	\end{enumerate}
\end{bsp}

\begin{lem}
	$G\operates X$. Dann
	\begin{enumerate}
		\item $\forall x\in X: G_x<G$
		\item $f\colon G/G_x\to G.x, gG_x\mapsto g.x$ ist wohldefiniert, bijektiv und ein $G$-Homomorphismus \textup(wobei $G$ links wie in Beispiel 2 oben und rechts durch $G\operates X$ operiert\textup).
		\item $|G.x| = (G:G_x)$, wobei $(G:G_x) = \infty$, falls $|G/G_x |=\infty$.
	\end{enumerate}
\end{lem}

\begin{proof}
	\leavevmode
	\begin{enumerate}
		\item Übung
		\item Es ist klar, dass $f$ surjektiv ist. Zur Injektivität: Sei $f(g_1G_x) = f(g_2G_x)$. Das ist äquivalent zu $ g_1.x = g_2.x\Leftrightarrow g_1^{-1}g_2.x = x \Leftrightarrow g_1^{-1}g_2\in G_x \Leftrightarrow g_1G_x = g_2G_x$ für alle $g_1,g_2 \in G, x\in X$. Also ist $f$ wohldefiniert und bijektiv.
		
		Nun muss noch gezeigt werden, dass $f$ ein $G$-Homorphismus ist: Es gilt $f(h.(gG_x)) = h.f(gG_x)$ für alle $x\in X, h,g\in G$. Aber $f(h.(gG_x)) = hgG_x = (hg).x = h.g.x = h.f(gG_x)$
		\item Es gilt nun $|G.x|\xeq{2.} |G/G_x| = (G:G_x)$
  \qedhere
	\end{enumerate}
\end{proof}

\begin{satz}[Bahnenformel] \label{thm:bahnformel}
	Sei $G$ eine Gruppe, $X$ eine endliche Menge und sei $G\operates X$ eine Gruppenoperation. Dann gilt: $$|X| = \sum_{i\in I}(G:G_{x_i}) = \abs*{X^G} + \sum_{\mathclap{\substack{i\in I,\\x_i\notin X^G}}}(G:G_{x_i})\enspace,$$ wobei $(x_i)_{i\in I}$ Elemente in $X$ sind, sodass die Bahnen ein Repräsentantensystem der Bahnen bilden.
\end{satz}

\begin{proof} Es gilt $|X| = \abs*{\bigcup\limits_{i\in I} G.x_i} = \sum_{i\in I}|G.x_i| = \sum (G:G_{x_i})$, woraus der 1. Teil der Gleichung folgt. Teilt man nun die Bahnen $G.x_i$ in solche auf mit genau einem Element ($\Leftrightarrow x_i\in X^G$) und solchen mit $\geq 2$ Elementen, so folgt $x_i\in X^G\Leftrightarrow G_{x_i} = G\Leftrightarrow (G:G_{x_i}) = 1$. Daraus folgt sofort der 2. Teil der Gleichung.
\end{proof}

\begin{satz}
	Sei $G$ eine endliche Gruppe. $G\operates G$ durch Konjugation. Sei $(x_i)_{i\in I}$ so gewählt, dass die Bahnen ein Repräsentantensystem für Konjugationsklassen sind. Dann gilt: \[ |G|  = |Z(G)| + \sum_{\mathclap{\substack{i\in I,\\x_i\notin Z(G)}}}(G: C_G(x_i))\enspace ,\] wobei $C_G(x_i) = \{g\in G\mid gx_ig^{-1} = x_i\}$ der Zentralisator von $x_i$ in $G$ ist.
\end{satz}

\begin{proof}
	Die Formel folgt direkt aus der Bahnenformel, da $C_G(x_i) = G_{x_i}$ mit der Konjugation als Operation und $x\in X^G\Leftrightarrow g.x = x~\forall g\in G\Leftrightarrow gxg^{-1} = x~\forall g\in G_{x_i}\Leftrightarrow x\in Z(G)$.
\end{proof}

\lecture{26. Oktober 2017}


\subsection{$p$-Gruppen und Sylow-Sätze}
\begin{defi}
	Sei $p$ prim (insbesondere $\geq 2$). Eine $p$-Gruppe ist eine Gruppe $G$ mit $|G| = p^r$ für ein $r\in\IN_0$. Insbesondere ist $|G|$ endlich.
\end{defi}
\begin{satz}
	Sei $G\neq\{e\}$ eine $p$-Gruppe. Dann gilt $\abs{Z(G)}\neq \abs*{\{e\}} = 1$. Insbesondere hat $G$ eine nicht-triviale abelsche Untergruppe.
\end{satz}
\begin{proof}
	Nach Satz 5.3 hat man \[ \underbrace{|G|}_{\mathclap{\text{durch $p$ teilbar}}}  = |Z(G)| + \underbrace{\sum_{\substack{i\in I,\\x_i\notin Z(G)}}(G: G_{x_i})}_{A}.\] Nach dem \namereff{thm:lagrange} ist $A$ durch $p$ teilbar oder gleich $1$, weil $G$ eine $p$-Gruppe ist. Letzters kann aber nicht sein, da $(G:G_{x_i}) = 1 \Leftrightarrow G = G_{x_i}\Leftrightarrow x_i\in Z(G)$, Widerspruch. Also sind die Summanden $(G:G_{x_i})$ und somit auch $A$ durch $p$ teilbar. Damit teilt $p$ auch $|Z(G)|\Rightarrow |Z(G)|\geq 2\Rightarrow Z(G)\neq \{e\}$.
\end{proof}

\begin{satz}
	Sei $G$ eine $p$-Gruppe. Dann existiert eine Normalreihe der Form
	$$ \{e\}\nt G_0\nt\dots\nt G_n = G$$ für ein $n\in\IN$, sodass $G_i/G_{i-1}\cong \IZ/p\IZ$ \textup($1\le i\le n$\textup).
	Insbesondere ist $G$ auflösbar.
\end{satz}
\begin{proof}
	Übungsblatt 3.
\end{proof}

\begin{defi}
	Sei $G$ eine endliche Gruppe, $p$ eine Primzahl. Sei $|G| = p^rm$ mit $p\nmid m$. $H<G$ heißt $p$-Sylowgruppe, falls $|H| = p^r$. Wir definieren  $\Syl_p(G):=\{H<G\mid H\text{ ist Sylowgruppe}\}$.
\end{defi}

\begin{satz}[Sylowsätze] \label{thm:sylow}
	Sei $p$ eine Primzahl, $G$ eine endliche Gruppe, $|G| = p^rm$ mit $p\nmid m$.\begin{enumerate}
		\item $\forall 0\le k\le r \colon\exists H<G$ mit $|H| = p^k$
		\item Sei $U<G$ eine $p$-Gruppe. Dann $\exists g\in G$, sodass $U<gSg^{-1}$ für alle $S\in \Syl_p(G)$ gilt.
		\item Sei $n_p = |Syl_p(G)|$. Dann gilt \begin{itemize}
			\item $n_p\equiv 1\pmod p$
			\item $n_p\mid m$
		\end{itemize}
	\end{enumerate}
\end{satz}

\begin{proof}
	\leavevmode
	\begin{enumerate}
		\item Sei $1\leq k \leq r$. Der Fall $k = 0$ ist klar mit $H = \{e\}$. Sei $X = \{A\subseteq G \mid |A| = p^k\}$, wobei $|X| = \binom{p^rm}{p^k}$. Nach Übungsblatt 3 gilt: $p^{r-k+1}\nmid |X|$.
		
		Nun $G \operates X$ durch $g.A := gA=\{ga|a\in A\}$ für alle $g\in G, A\in X$. (klar: $|gA| = p^k $, also $gA\in X$). Nachrechnen: (O1), (O2) gelten (offensichtlich). 
		
		Nach \cref{thm:bahnformel} folgt $|X| = \sum_{i\in I}(G:G_{x_i})$, wobei $\exists i\in I$, sodass $p^{r-k+1} \nmid (G:G_{x_i})$, weil $p^{r-k+1}\nmid |X|$. Wähle solch ein $x_i = : A'\in X$.
		
		Behauptung: Es gilt $G_{A'}<G$ mit $|G_{A'}| = p^k$. Dann würde 1) folgen mit $H = G_{A'}$. Es ist klar, dass $G_{A'}<G$ gilt. Nach dem \namereff{thm:lagrange} folgt dann: $|G| = |G_{A'}|(G:G_{A'})$, wobei $p^r$ die linke Seite der Gleichung teilt, und im Index auf der rechten Seite $p$ höchstens $r-k$-mal vorkommt.
		
		Deshalb gilt: $p^k$ teilt $|G_{A'}|$ und somit $p^k\leq |G_{A'}|$. Sei $a\in A'$. Dann muss $G_{A'}.a = \{g.a\mid g\in G_{A'}\}\subseteq G_{A'}.A'\subseteq A'$ nach Definition von $G_{A'}$ gelten.
		
		Also folgert man $|G_{A'}| = |G_{A'}.a|\leq |A'| = p^k$.  (nach der Definition von $G_{A'}.a$ und $A'\in X$).
		Somit erhält man $|G_{A'}| = p^k$, wodurch die Behauptung und somit auch die erste Aussage gezeigt ist.
		
		%TODO: Stimmt die ref? Sie war ursprünglich (B3)
		\item Sei $U<G$ mit $|U| = p^s$ für ein $s\in\IN$. Sei $S\in \Syl_p(G)$. Sei $U \operates G/S$ wie in \cref{bsp:linkstrans} durch Linksmultiplikation gegeben.
		\begin{align*}
			u.(gS) &= ugS\qquad \forall u\in U, g\in G \\
			m &= |G/S| = \sum_{i\in I}(U:U_{x_i})
		\end{align*}
		nach Definition von $S\in Syl_p(G)$ und dem \namereff{thm:lagrange}. Die zweite Gleichheit folgt aus \cref{thm:bahnformel}.
		
		Weil $p\nmid m$, existiert ein $i\in I$ sodass $p\nmid (U:U_{x_i})$. Wähle ein solches $x_i =: aS$. Nach dem \namereff{thm:lagrange} ist
		\[ p^s = |U| = |U_{aS}|(U:U_{aS}).\] Also $(U:U_{aS}) = 1$. Also gilt $U = U_{aS}$. Damit folgt
		\begin{center}
			$ \begin{array}{crclc}
			&u.aS &=& aS      						\qquad& \forall a\in U\\
			\Leftrightarrow& (ua)S &=& as		\qquad &\forall u\in U\\
			\Leftrightarrow& a^{-1}uaS &=& S \qquad &\forall u\in U\\
			\Leftrightarrow& a^{-1}ua&\in& S  \qquad &\forall u\in U\\
			\Leftrightarrow& u&\in& aSa^{-1}  \qquad &\forall u\in U
			\end{array}$
		\end{center}
		Setze $g := a$ und erhalte $U<gSg^{-1}$.
		
		\item Übungsblatt 3
  \qedhere
	\end{enumerate}
\end{proof}

\paragraph{Konsequenzen.}
Sei $G$ eine endliche Gruppe, $p$ eine Primzahl.
\begin{enumerate}
	\item Je zwei $p$-Sylowuntergruppen in $G$ sind zueinander konjugiert, also \[S, S'\in \Syl_p(G) \Rightarrow \exists g\in G: S' = gSg^{-1}.\] \label{kons:kong}
	\begin{proof}
		Nach Sylowsatz 2 folgt $\exists g\in G$ mit $S'<gSg^{-1}$. Da $|S'| = |gSg^{-1}|$ nach Definition der $p$-Sylowgruppe gilt, folgt $S' = gSg^{-1}$.
	\end{proof}
	
	Beachte: Falls $n_p  = |\Syl_p(G)| = 1$ gilt, also $\exists!$ $p$-Sylowgruppe $S$, dann ist $S\nt G$. Denn $\forall g\in G$ ist $gSg^{-1}$ wieder eine $p$-Sylowgruppe, also $gSg^{-1} = S$.
	
	\item (Cauchy) Falls $p\mid|G|$ gilt, dann existiert ein $g\in G$ mit $\ord(g) = p$.
	\begin{proof}
		Nach Sylowsatz 1 existiert $H<G$ mit $|H| = p$. Wähle ein $g\in H$, $g\neq e$. Dann ist $\<g\> <H$ und $\<g\> \neq \{e\}$, also $\<g\> = H$ nach dem \namereff{thm:lagrange}. Aus \cref{ch:zyklisch} folgt $\ord(g) = |H| = p$.
	\end{proof}
	\item Es gilt: $G$ ist $p$-Gruppe $\Leftrightarrow $ Jedes Element $g\in G$ hat Ordnung $p^s$ für ein geeignetes $s\in \IN_0$ (abhängig von $g$).
	\begin{proof}~
		\begin{description} %TODO: Satz 3.3 durch Ch 3 ersetzt. Passt das?
			\item[\glqq$\Rightarrow$\grqq] Sei $g\in G$, sei $\ord(g) = n$. Aus \cref{ch:zyklisch} folgt $|\<g\>| = n$. Daraus folgt $n\mid |G|$ nach dem \namereff{thm:lagrange}. Da $G$ eine $p$-Gruppe ist, muss nun $n = p^s$ für ein $s \in \IN_0$ gelten.
			\item[\grqq $\Leftarrow$\grqq] Zu zeigen: $|G| = p^r$ für ein $r\in\IN_0$.
			Angenommen $q\mid|G|$ für $q$ Primzahl $p\neq q$. Nach dem Satz von \textsc{Cauchy} existiert $g\in G$ mit $\ord(g) = q$. Das ist ein Widerspruch.
    \qedhere
		\end{description}
	\end{proof}
	\begin{bem}
		$p$-Gruppen mit unendlicher Ordnung kann man definieren als Gruppen mit $\ord(g) = p^r$, $r\in \IN_0$ von $p$ für alle $g\in G$.
	\end{bem}
\end{enumerate}

\paragraph{Anwendungen.}
Vorbemerkung: Sei $G$ eine Gruppe, $|G| = p$ prim. Dann folgt $G\cong \IZ/p\IZ$. (Denn wähle $g\in G$, $g\neq e$. Es gilt $\<g\> <G$ und nach dem \namereff{thm:lagrange} ist $|\<g\>| = p = |G|$, also ist $G = \<g\>$ zyklisch und somit $G\cong \IZ/p\IZ$ nach Klassifikation von zyklischen Gruppen.)
\begin{satz}
	Sei $G$ eine Gruppe, $|G| = pq$ mit $p\neq q$ Primzahl. Dann ist $G$ auflösbar.
\end{satz}	
\begin{proof}
	Ohne Beschränkung der Allgemeinheit sei $p>q$. Nach Sylowsatz 3 gilt: $n_p\mid q$, also $n_p\in\{1,q\}$ und $n_p\equiv 1 \pmod p$.
	
	Dann gilt $n_p = 1$, weil $p>q$. Nach \cref{kons:kong} der Konsequenzen der Sylowsätze gilt $\exists!$ $p$-Sylowgruppe $S$ und $S\nt G$. Nach der Definition von $p$-Sylowgruppe und weil $|G| = pq$ ist, gilt $|S| = p$. Also erhalten wir eine Normalreihe 
	$$ \{e\}\nt S\nt G$$ mit $S/\{e\}\cong S\cong \IZ/p\IZ$ und $|G/S| = q$, also $G/S\cong \IZ/q\IZ$.
	
	Die Faktoren sind folglich abelsch und somit ist $G$ auflösbar.
\end{proof}

\begin{satz}
	Sei $G$ eine Gruppe, $|G| = pq$ mit $p,q$ prim sowie $p<q$ und $p \nmid q-1$. Dann folgt $G\cong\IZ/p\IZ \times \IZ/ q\IZ$.
\end{satz}
\begin{proof}
	Nach Sylowsatz 3 gilt $n_p\in \{1,q\}, n_q\in\{1,p\}$ und $n_p\equiv 1\pmod p$, $n_q\equiv 1\pmod q$. Da $p<q$ ist, gilt $n_q = 1$. Also existiert genau eine $q$-Sylowgruppe $Q\nt G$. Falls $n_p = q$ dann gilt $q\equiv 1\pmod p$. Daraus folgt $p\mid (q-1)$, was im Widerspruch zur Voraussetzung steht. Also ist $n_p = 1$, womit genau eine $p$-Sylowgruppe mit $P\nt G$ existiert.
	
	Behauptung: Sei $x\in P ,y\in Q$. Dann gilt $xy = yx$. $xyx^{-1}y^{-1}$ liegt in $Q$, da $xyx^{-1}\in Q$, weil $Q$ ein Normalteiler ist und $y^{-1}\in Q$ per Definition. Analog gilt $xyx^{-1}y^{-1}\in P$, da $x\in P$ und $yx^{-1}y^{-1} \in P$.
	Somit liegt $xyx^{-1}y^{-1}$ in $P\cap Q$. Aber es gilt $P\cap Q = \{e\}$, da $|P\cap Q|\mid p = |P|$ und $|P\cap Q| \mid q = |Q|$. Somit folgt die Behauptung.
	
	Betrachte nun 
	\begin{align*}
		\Phi\colon P\times Q&\longto G\\
		 (x,y)&\longmapsto xy
	\end{align*}
	$\Phi$ ist ein wohldefinierter Gruppenhomomorphismus, denn 
	\begin{gather*}
		\Phi((x,y)\circ (x',y')) = \Phi ((xx', yy')) = xx'yy' \\ \Phi((x,y))\circ \Phi((x',y')) = xyx'y' = xx'yy' ~\text{(nach Behauptung)}
	\end{gather*}
	Außerdem ist $\Phi$ injektiv, denn $\Phi((x,y)) = e\Leftrightarrow xy = e\Leftrightarrow x = y^{-1} = e$, weil $P\cap Q = \{e\}$.
	$\Phi$ ist surjektiv, weil $|P\times Q| = |P|\cdot |Q| = pq = |G|$.
	Somit liefert $\Phi$ einen Gruppenisomorphismus $P\times Q \cong G$, also $\IZ/p\IZ\times \IZ/q\IZ\cong G$.
\end{proof}


\begin{kor}
	Sei $G$ eine Gruppe, $|G| = 15$. Dann gilt $G\cong \IZ/3\IZ\times \IZ/5\IZ \cong \IZ/15\IZ$ und $G$ ist zyklisch.
\end{kor}

\begin{proof}
	Wir wissen $G \cong \IZ/3\IZ\times \IZ/5\IZ$. Behauptung: $ \IZ/3\IZ\times \IZ/5\IZ\cong \IZ/15\IZ$. Sei nämlich $g = (\overline{1}, \overline{1})\in  \IZ/3\IZ\times \IZ/5\IZ$. Dann gilt: $$\ord(g) = \min \{j\mid (\overline{1},\overline{1})+\dots +(\overline{1},\overline{1}) = (\overline{0}, \overline{0})\in  \IZ/3\IZ\times \IZ/5\IZ\}  =15$$
	Folglich gilt $|\<g\>| = 15$ und damit ist $\IZ/3\IZ\times \IZ/5\IZ$ zyklisch. Der Isomorphismus ist durch 
	\begin{align*} %TODO: Klingt fishy. Gilt immer noch g=(1,1)?
		\IZ/3\IZ\times \IZ/5\IZ&\longto \IZ/15\IZ\\
		g=(\overline{1},\overline{1})&\longmapsto \overline{1}
	\end{align*}
	gegeben.
\end{proof}

\lecture{30. Oktober 2017}

\section{Ringe}
\subsection{Allgemeines}
\begin{defi} Ein Ring (mit 1) ist eine Menge $R$ zusammen mit zwei Abbildungen
	\begin{alignat*}{2}
		+,\cdot \colon R\times R &\longto  R &&\quad\\
		(a,b)&\longmapsto a+b \qquad &&\text{Addition}\\
		\text{bzw. }(a,b)&\longmapsto a\cdot b &&\text{Multiplikation,}
	\end{alignat*}
sodass gilt:
\begin{itemize}
	\item[(R1)] $(R,+)$ ist eine abelsche Gruppe.
	\item[(R2)] $\forall a,b,c\in R$ gilt $(a\cdot b)\cdot c = a\cdot (b\cdot c)$ (also $\cdot$ ist assoziativ)
	\item[(R3)] $\forall a,b,c\in R$ gilt:
	\begin{equation*}
		\begin{aligned}
		a\cdot(b+c) &= (a\cdot b)+ (a\cdot c)\\
		(b+c)\cdot a &= (b\cdot a)+(c\cdot a)
		\end{aligned} \tag{Distributivität}
	\end{equation*}
	\item[(R4)] $\exists 1 = 1_R\in R$, sodass $a\cdot 1 = a = 1\cdot a$ für alle $a\in R$ (Neutrales bezüglich $\cdot$)
\end{itemize}
\end{defi}

\begin{bem} \leavevmode
	\begin{enumerate}
		\item Wir bezeichnen mit $0$ oder $0_R$ das neutrale Element und mit $(-a)$ das Inverse zu $a\in R$ bezüglich $+$.
		\item Das Element $1\in R$ ist eindeutig (denn sei $1'$ ein anderes, dann ist $1 = 1\cdot 1' = 1'$).
		\item In einem Ring gilt: $a\cdot 0 = 0 = 0\cdot a$ für alle $a\in R$, denn $a\cdot 0 = a\cdot (0+0 = (a\cdot 0)+(a\cdot 0)\Rightarrow 0 = a\cdot 0$; analog für $0\cdot a$.
	\end{enumerate}
\end{bem}

\begin{defi}
	Ein Ring $(R, +,\cdot)$ heißt kommutativ, falls $a\cdot b = b\cdot a$ für alle $a,b\in R$
\end{defi}	

\begin{bsp}
	\leavevmode
	\begin{enumerate}
		\item Jeder Körper $(K,+,\cdot)$ ist ein kommutativer Ring (aber Ringe haben im Allgemeinen keine multiplikativ Inversen).
		\item (aus LA) Sei $V$ ein $K$-Vektorraum, $K$ ein Körper, dann ist $(\text{End}_K(V),+,\cdot)$ ein Ring mit $(f+g)(v) = f(v)+g(v)$ und $(f\cdot g)(v) = (f\circ g)(v)$ (Hintereinanderausführung) mit $f,g\in \text{End}_K(V), v\in V$ mit $0_{\text{End}_K(V)} =$ Nullabbildung; $1_{\text{End}_K(V)} = \text{id}_V$.
		\item Nullring: $R = \{0=1\}$ mit $0+0 = 0$ und $0\cdot 0 = 0$.
		\item Es gilt folgende Umkehrung von $1.$: wenn $(R,+,\cdot)$ ein kommutativer Ring ist, $R\neq \{0\}$, jedes $x\in R$ mit $x\neq 0$ besitzt Inverses $x^{-1}$ bezüglich $\cdot$; dann ist $(R, +, \cdot )$ Körper
		\item $(R, +, \cdot)$ Ring. Betrachte
		\[ R[t] = \left\{ \left. \sum_{i=0}^{\infty} a_it^i \right| a_i\in R, \text{ nur endlich viele }a_i\neq 0\right\}  = \left\{\left. \sum_{i=0}^{n} a_it^i \right| a_i\in R, n\in \IN_0\right\},\]
		die Polynome mit Koeffizienten in $R$. Dann ist $(R[t], +, \cdot)$ ein Ring mit $0_{R[t]} =$ Nullpolynom, d.h. $a_i = 0$ für alle $i$. $1_{R[t]}$ ist das Polynom $p(t) = \sum_{i = 0}^{\infty}a_it^i$ mit $a_0 = 1$ und $a_i = 0$ für $i\geq 1$.  Es gilt: $(R[t], +,\cdot)$ ist kommutativ $\Leftrightarrow$ $(R,+,\cdot)$ ist kommutativ.
	\end{enumerate}
\end{bsp}	

\begin{defi}
	Sei $(R,+,\cdot)$ ein Ring. $R'\subseteq R$ heißt Unterring, falls
	\begin{description}
		\item[(UR1)] $1_R\in R'$
		\item[(UR2)] $\forall a, b\in R': a+(-b)\in R', a\cdot b \in R'$
	\end{description}
\end{defi}

\begin{bsp}
	Sei $(R,+,\cdot)$ ein Ring. $Z(R) := \{a\in R|a\cdot x = x\cdot a \enspace \forall x\in R\}$ ist das Zentrum des Ringes; dieses ist ein Unterring. Warnung: $Z(R) \neq Z((R,+))$ im Allgemeinen.
\end{bsp}	

\begin{defi}
	Seien $(R,+,\cdot)$ und $(S,+,\cdot)$ Ringe. Eine Abbildung $\phi\colon R\to S$ ist Ringhomomorphismus, falls gilt:
	\begin{description}
		\item[(RH1)] $\phi (a+b) = \phi (a)+\phi(b)$
		\item[(RH2)]$\phi(a\cdot b) = \phi(a)\cdot \phi(b)$
		\item[(RH3)]$\phi(1_R) = \phi(1_S)$
	\end{description}
	für alle $a,b\in R$.
	
	Falls $\phi$ zusätzlich bijektiv ist, ist $\phi$ ein Ringisomorphismus.
\end{defi}

\begin{bem}
	Sei $\phi\colon R\to S$ ein Ringhomomorphismus. Dann ist $R\to S$ ein Gruppenhomomorphismus von $(R,+)$ nach $(S,+)$ wegen (RH1).
\end{bem}

\begin{lem}
	\leavevmode
	\begin{enumerate}
		\item Ist $\phi\colon R\to S$ ein Ringisomorphismus, dann ist auch $\phi^{-1}\colon S\to R$ ein Ringisomorphismus
		\item Seien $\phi_1\colon R\to S, \phi_2\colon S\to T$ Ringhomomorphismen. Dann ist auch $\phi_2\circ\phi_1\colon R\to T$ ein Ringhomomorphismus
	\end{enumerate}
\end{lem}
\begin{proof}
	Nachrechnen.
\end{proof}


\begin{lem}
	Sei $\phi\colon R\to S$ Ringhomomorphismus. Dann ist $\im\phi\subseteq S$ ein Unterring.
\end{lem}

\begin{proof}
	Es gilt $\phi(1_R) = 1_S\in\im\phi\Rightarrow$ (UR1).
	
	Seien $s_1, s_2\in\im\phi$.
	\begin{align*}
		&\Rightarrow \exists r_1,r_2\in R: \phi (r_1) = s_1, \phi (r_2 )= s_2 \\
		&\Rightarrow s_1\cdot s_2 = \phi (r_1)\cdot \phi (r_2) = \phi(r_1\cdot r_2)\in \im\phi \\
		&\Rightarrow s_1\cdot s_2\in \im\phi
	\end{align*}
	Außerdem gilt:
	\begin{equation*}
		s_1+(-s_2) = \phi (r_1) + (-\phi (r_2)) = \phi (r_1)+\phi (-r_2) = \phi (r_1+(-r_2))\in\im\phi
	\end{equation*}
	Somit gilt (UR2) und es sind alle Unterringaxiome erfüllt.
\end{proof}

\noindent
\textbf{Warnung:} Wir setzen für $\phi\colon R\to S$ als Ringhomomorphismus
\[\ker\phi := \{r\in R\mid\phi(r)= 0_S\}.\] Dann ist $\ker\varphi\subseteq R$ genau dann ein Unterring, falls $S$ der Nullring ist. Denn:

\glqq $\Rightarrow$\grqq: Sei $\ker\phi$ ein Unterring. Dann gilt per Definition $1_R\in \ker \phi$. Somit muss auch $0_S = \phi (1_R) = 1_S$ gelten. Deshalb gilt für alle $s\in S$, dass $s = s\cdot 1_S = s\cdot 0_S = 0_S$.

\glqq $\Leftarrow$\grqq: Sei $S = \{0\}$. Dann ist $\ker\phi$ bereits ganz $R$ und dies ist offensichtlich ein Unterring.

\begin{defi}
	Sei $(R,+,\cdot)$ ein Ring. $I\subseteq R$ heißt Ideal, falls gilt:
	\begin{itemize}
		\item[(I1)] $I<(R,+)$
		\item[(I2)] \begin{itemize}
			\item[a)] $a\cdot x \in I$ für alle $x\in I, a\in R$
			\item[b)] $x\cdot a \in I$ für alle $x\in I, a\in R$
		\end{itemize}
	\end{itemize}
	Falls nur (I1), (I2a) erfüllt sind, heißt $I$ Linksideal; falls nur (I1) und (I2b) erfüllt sind, heißt $I$ Rechtsideal.
\end{defi}

\begin{bsp} \label{bsp:rt_r}
	\leavevmode
	\begin{enumerate}
		\item $(\IZ,+,\cdot)$ ist Ring. Sei nun $n\in\IZ$, dann ist $I= n\IZ = \{nk\mid k\in\IZ\}\subseteq \IZ$ ein Ideal, denn: $n\IZ<(\IZ,+)$, also folgt (I1); und für $a\in\IZ$ und $x = nk\in n\IZ$ gilt: $ax = ank = nak\in I$; $xa = nka  = nak \in I$ und damit folgt (I2).
		
		\item $(R,+,\cdot)$ Ring; $(R[t],+,\cdot)$ wie in Beispiel oben; \[I = \left\{p(t)\in R[t]\left| p(t) = \sum_{i = 0}^{\infty}a_it^i \wedge a_0 = 0\right.\right\}\] enthält die Polynome ohne konstanten Term. Dann ist $I\subseteq R[t]$ ein Ideal (nachprüfen).
	\end{enumerate}
\end{bsp}


\begin{lem} \label{lem:kerid_r}
	Sei $\phi\colon R\to S$ ein Ringhomomorphismus. Dann ist $\ker\phi \subseteq R$ ein Ideal.
\end{lem}

\begin{proof}
	Es gilt $\ker \phi<(R,+)$ nach \cref{thm:kerim_g}, womit (I1) erfüllt ist. Sei nun $a\in R, x\in\ker \phi$. Dann gilt $\phi(ax) = \phi(a)\phi(x) = \phi(a)\cdot 0_S = 0_S$ und somit liegt $ax$ im Kern. Das funktioniert analog auch für $xa$, womit auch (I2) erfüllt ist.
\end{proof}

\begin{bsp}
	Sei $(R[t],+,\cdot)$ wie in \cref{bsp:rt_r}. Sei $a\in R$.\begin{eqnarray*}
		\ev_a\colon R[t] & \longto & R\\
		p(t) = \sum_{i=0}^{\infty}b_it^i & \longmapsto & p(a) = \sum_{i=0}^{\infty} b_ia^i\\
		(b_i\in R; \text{ fast alle $b_i = 0$})&&\left(\text{mit }a^i = \prod_{k=1}^ia\right)
	\end{eqnarray*}
	heißt Auswertungs- oder Evaluationsabbildung. Nachrechnen ergibt, dass $\ev_a$ ein Ringhomomorphismus ist.
	
	Es gilt $\ker (\ev_a) = \{p(t)\in R[t]\mid p(a) = 0_R\}$. Das sind genau die Polynome, die $a$ als Nullstelle haben. Wir wissen, dass $\ker (\ev_a) \subseteq R[t]$ ein Ideal ist nach \cref{lem:kerid_r}.
	
	Spezialfall: Sei $a = 0_R$. Dann gilt $\ker \ev_0  = I$ wie in \cref{bsp:rt_r} Teil 2). Insbesondere ist $I$ ein Ideal.
\end{bsp}

\begin{bem}
Seien nun ein Ring $(R,+,\cdot)$ und ein Ideal $I\subseteq R$ gegeben. Insbesondere, nach $(I1)$, ist $I<(R,+)$ und sogar $I\nt (R,+)$, weil $(R,+)$ abelsch ist.
Somit ist $R/I$ wieder eine Gruppe mit den Nebenklassen in $(R,+)$ bezüglich $I$ als Elemente. Nebenklassen sind von der Form $\overline{a} = \{a+x\mid x\in I\}\enspace, a\in R$ und die Gruppenoperation auf $R/I$ ist $\overline{a}\circ\overline{b} = \overline{a+b}$.
\end{bem}

\begin{satz}
	Seien $R, I$ wie in der Bemerkung gegeben. Dann wird $(R/I, \circ)$ zu einem Ring $(R/I,\circ,\odot)$, wobei die Multiplikation gegeben ist durch $\overline{a}\odot \overline{b} = \overline{a\cdot b}$. Letzteres ist dabei die Multiplikation in $R$.
\end{satz}
\begin{proof}
	\leavevmode
	\begin{itemize}
		\item[(R1)] $(R/I,+)$ ist eine abelsche Gruppe (nach \cref{ch:grp_basics}).
		\item[(R2)] Seien $\overline{a},\overline{b},\overline{c}\in R/I$. $(\overline{a}\odot\overline{b})\odot\overline{c}  = (\overline{ab})\odot \overline{c} = \overline{(ab)c} = \overline{a(bc)}  = \overline{a}\odot \overline{bc} = \overline{a}\odot(\overline{b}\odot\overline{c})$. 
		\item[(R3)] Seien $\overline{a},\overline{b},\overline{c}\in R/I$. Dann $\overline{a}\odot(\overline{b}\circ \overline{c}) = \overline{a}\odot\overline{b+c} = \overline{a\cdot(b+c)} =\overline{ab+ac} = \overline{ab}\circ\overline{ac} = \overline{a}\odot \overline{b}\circ\overline{a}\odot\overline{c}$. Analog für den zweiten Teil von (R3).
		\item[(R4)] Sei $\overline{a}\in R/I$. Dann gilt $\overline{a}\odot\overline{1_R}  = \overline{a1_R} = \overline{a} = \overline{1_R\cdot a} = \overline{1_R}\odot \overline{a}$. Somit ist $\overline{1_R}$ das neutrale Element für $\odot$.
		
	\end{itemize}
	
	Nun ist noch die Wohldefiniertheit von $\odot$ zu prüfen. Also ist zu zeigen: für $\overline{a} = \overline{a'}$ und $\overline{b} = \overline{b'}$ folgt $\overline{a}\odot\overline{b} = \overline{a'}\odot\overline{b'}$ mit $\overline{a},\overline{b},\overline{a'},\overline{b'}\in R/I$. Sei also $\overline{a} = \overline{a'}$ und $\overline{b} = \overline{b'}$. 
	Dann existieren $x,y\in I$ mit $a+(-a') = x$ und $b+(-b') = y$ nach (I1).
	Es gilt $$a\cdot b = (a'+x)\cdot (b'+y)  = (a'b')+(a'y)+(xb')+(xy)$$ wobei $(a'y)+(xb')+(xy)\in I$, weil $I$ ein Ideal ist. Somit liegt auch $(ab)+(-a'b')$ in $I$ und es folgt $\overline{ab} = \overline{a'b'}$.
\end{proof}



\lecture{2. November 2017}

\noindent Wir wissen: Sei $(R,+,\cdot)$ ein Ring, $I\subseteq R$ ein Ideal. Dann ist $(R/I,+,\odot)$ ein Ring mit $\overline{a}\odot\overline{b} = \overline{ab}$. Wir nennen $(R/I,+,\odot)$ den Quotientenring von $R$ nach/modulo $I$. Im Folgenden schreiben wir kurz $R$ statt $(R,+,\cdot)$ etc.

\begin{satz}[Homomorphiesatz] \label{thm:homsatz_r}
	Sei $R$ ein Ring, $I\subseteq R$ ein Ideal. \begin{enumerate}
		\item Die Abbildung $\can\colon R\to R/I, a\mapsto \overline{a}$ ist Ringhomomorphismus.
		\item Sei $\phi\colon R\to S$ Ringhomomorphismus mit $I\subseteq \ker \phi$, dann $\exists!\;\overline{\phi}\colon R/I\to S$ Ringhomomorphismus, sodass $\overline{\phi}\circ \can  = \phi$. Also:
		\begin{center}
		\begin{tikzcd}
					R \arrow{rd}[swap]{\can} \arrow{r}{\phi} & S \\
					& R/I\arrow{u}[swap]{\exists! \; \overline{\phi}\text{ Ringhom}} \\
		\end{tikzcd}
		\end{center}
	\end{enumerate}
\end{satz}

\begin{proof}
	Übungsblatt 3.
\end{proof} 

\begin{defi}
	Sei $R$ ein kommutativer Ring und $I, J\subseteq R$ Ideale. Dann ist \begin{itemize}
		\item $I+J = \{x+y \mid x\in I, y\in J\}\subseteq R$ die Summe von $I$ und $J$
		\item $I\cap J = \{x \mid x\in I, x\in J\}\subseteq R$ der Schnitt von $I$ und $J$
		\item $I\cdot J = \{\sum_{i = 1}^na_ib_i \mid a_i\in I, b_i\in J, n\in\IN\}$ das Produkt von $I$ und $J$
	\end{itemize}
	Das sind alles Ideale in $R$ (nachprüfen!).
\end{defi}

\begin{defi}
	Sei $R$ ein kommutativer Ring, $a,b\in R$. \begin{enumerate}
		\item $(a) = \{ra\mid r\in R\}$ ist das von $a$ erzeugte Ideal.
		\item $b$ teilt $a$ (in $R$), falls $\exists r\in R: a = br = rb$
		\item 
		$a$ ist Nullteiler, wenn $\exists r\in R, r \neq 0$ mit $ra = ar = 0$. $R$ ist nullteilerfrei, falls 0 der einzige Nullteiler ist.
		$R$ heißt Integritätsbereich, falls $R$ nullteilerfrei und $R \neq \{0\}$ ist.
		\item $a$ heißt Einheit, falls $\exists r\in R$ mit $ar = 1 = ra$ (also $a$ ein multiplikatives Inverses hat).
		$R^{\times} = \{c\in R \mid c \text{ Einheit in } R\}$ ist die Einheitengruppe von $R$.
	\end{enumerate}
\end{defi}

\begin{bem}
	$(a)$ ist in der Tat ein Ideal:\begin{itemize}
		\item[(I2)] Sei $r' \in R, x = ra\in (a)$. Dann gilt $r'x  = r' (ra) = (r'r)a \in (a)$. Genauso für $xr'\in (a)$.
		\item[(I1)] Seien $x,y\in I$. Dann lassen sich $x$ und $y$ in der Form $x = r'a, y = ra$ schreiben. Folglich gilt $x+(-y) = (r'a)+(-ra) = (r'+(-r))a \in (a)$. Wir benutzen dabei $-(ra) = (-r)a$, weil $(ra)+((-r)a) = (r+(-r))a = 0a = 0$ gilt.
	\end{itemize}
\end{bem}
\begin{bem}
	$R^{\times}$ ist eine Gruppe bezüglich $\cdot$ (Übung).
\end{bem}

\begin{defi}
	Ein Ideal der Form $(a)$ wie oben heißt Hauptideal. Ein kommutativer Ring $R\neq\{0\}$ heißt Hauptidealring, wenn $R$ nullteilerfrei und jedes Ideal in $R$ ein Hauptideal ist (also von der Form $(a)$ ist): $\forall I\subseteq R\text{ Ideal }\exists a\in R: I = (a)$
\end{defi}

\begin{bsp}
	Wir betrachten $(\IZ,+,\cdot)$. Behauptung: Das ist ein Hauptidealring.
	\begin{itemize}
		\item Es ist klar, dass $\IZ \neq \{0\}$ und dass $\IZ$ nullteilerfrei ist.
		\item Sei $I \subseteq \IZ$ ein Ideal. Falls $I = \{0\}$, dann muss $I = (0)$ gelten. Es sei also $I \neq \{0\}$. Dann $\exists x\in I, x\neq 0$. Wähle $n\in\IN$ minimal mit $n\in I$. Denn wenn $x \in I$, so liegt nach (I1) auch das Inverse bezüglich der Addition $-x$ in $I$.
		
		Wir behaupten nun, dass $(n)$ bereits $I$ ist.
		\begin{description}
			\item[\glqq $\supseteq$\grqq:] Das gilt nach (I2)
			\item[\glqq $\subseteq$\grqq:] Sei $y\in I$. Wir schreiben $y = bn+r$ mit $b,r\in\IZ$ mit $0\leq r< n$. Somit gilt $ y+(-bn) = r$ und damit würde $r$ in $I$ liegen. Das ist ein Widerspruch zur Minimalität von $n$ außer $r = 0$. Deshalb gilt $y = bn$ und folglich liegt $y$ in $(n)$.
		\end{description} 
	\end{itemize}
\end{bsp}

\begin{bsp}
	$(1) = R$, $(0) = \{0\}$ sind Hauptideale.
\end{bsp}

\begin{lem} \label{lem:komring}
	Sei $R$ ein kommutativer Ring, $R\neq \{0\}$.\begin{enumerate}
		\item $R$ ist genau dann ein Körper, wenn $R^{\times} = R\setminus\{0\}$
		\item Für $a,b,c\in R$ gilt \begin{enumerate}
			\item $(a)\subseteq (b)\Leftrightarrow$ $b$ teilt $a$
			\item $(a) = R \Leftrightarrow a\in R^{\times} \Leftrightarrow (a) = (1)$
			\item Falls $c$ nicht Nullteiler: $ac =bc \Rightarrow a =b$
			\item $a$ Nullteiler $\Rightarrow a\notin R^{\times}$
		\end{enumerate}
	\end{enumerate}
\end{lem}
\begin{proof}
	\leavevmode
	\begin{enumerate}
		\item Sei $R$ Körper, dann gilt: Für alle $x\in R$, wobei $x\neq 0$, existiert ein $x^{-1}$ mit $xx^{-1} = 1 =x^{-1}x$. Somit liegt $x$ in $R^{\times}$ und es gilt $R\setminus \{0\}\subseteq R^{\times}$.
		
		Angenommen $R^{\times} = R\setminus\{0\}$. Sei $x\in R$, $x\neq 0$. Dann gilt $x\in R^{\times}$ und somit existiert ein $r \in R$ mit $xr = 1 = rx$. Folglich ist $r = x^{-1}$ und damit das Inverses zu $x$. Also ist $R$ ein Körper.
		
		\item \begin{enumerate}
			\item 
				\begin{description}
					\item[\glqq $\Rightarrow$\grqq:] Sei $(a)\subseteq (b)$. Dann liegt $a$ in $(b)$ und es existiert ein $r\in R$ mit $a = rb$. Folglich gilt $b$ teilt $a$.
					
					\item[\glqq $\Leftarrow$\grqq:] Angenommen $b$ teilt $a$. Dann existiert ein $r\in R$, sodass $a =rb$. Folglich gilt $r'a = r'(rb) = (r'r)b\in (b)~\forall r'\in R$. Somit gilt $(a)\subseteq (b)$.
				\end{description}
				
			\item %TODO: Fix this shit
				Sei $(a) = R\Rightarrow 1\in (a)\Rightarrow \exists r\in R: 1 = ra = ar\Rightarrow a\in R^{\times}$.
			
				Sei $a\in R^{\times}\Rightarrow \exists r\in R: ra = 1 = ar\Rightarrow 1\in (a) \Rightarrow r'\cdot 1 = r'\in (a) \forall r'\in R \Rightarrow (a) = R$.
				
				Sei $a \in R^{\times}\Rightarrow 1\in (a) \Rightarrow R = (1) \subseteq (a) \Rightarrow (1)  = (a) $.
				
				Sei $(1) = (a) \Rightarrow1\in (a) \Rightarrow \exists r\in R: ra = 1 = ar\Rightarrow a\in R^{\times}$.
			\item 
				\begin{align*}
					&ac = bc \\
					\Rightarrow& (a+(-b))c = 0 \\
					\Rightarrow& a+(-b) = 0 \\
					\Rightarrow& a = b
				\end{align*}
			\item 
				Sei $a$ Nullteiler. Dann existiert ein $r\neq 0$ mit $ra = 0$. Sei nun $a\in R^{\times}$, dann $\exists b$ mit $ab = 1$. Somit gilt $r = r\cdot 1 = rab = 0 \cdot b = 0$. Somit wäre $r = 0$, was ein Widerspruch ist.
    \qedhere
		\end{enumerate}
	\end{enumerate}
\end{proof}


\begin{bsp}
	Sei $R = \IR[t]$, $I = (t^2)$. Wir behaupten: $R/I$ ist kein Körper. Denn es gilt $\overline{t} \neq \overline{0}$, da $t\notin I$. Andererseits gilt $\overline{t^2} = \overline{0}$, weil $t^2\in I$. Somit gilt auch $\overline{0} = \overline{t^2} = \overline{t}\odot\overline{t}$, weshalb $\overline{t}$ ein Nullteiler in $R/I$ ist. Insbesondere ist $\overline{t}$ nach \cref{lem:komring} keine Einheit. Folglich gilt $(R/I)^{\times} \neq (R/I)\setminus\{0\}$ und $R/I$ kann kein Körper sein.
\end{bsp}

\begin{bsp}
	Sei $R = \IR [t], I = (t^2+1)$. Dann ist $R/I$ ein Körper.
\end{bsp}

\begin{proof}
	Es ist klar, dass $R/I$ kommutativ ist, weil $R$ kommutativ ist. Sei $x\in R/I, x\neq 0$. Es ist zu zeigen, dass $x\in (R/I)^{\times}$. In $R/I$ gilt $\overline{t^2+1} = \overline{0}$, weil $t^2+1$ in $I$ liegt und damit für $j\in \IZ_{\geq 2}$ folgendes gilt:
	\begin{align*}
		\overline{t ^j} &= \overline{t^{j-2}(t^2+1)+(-t^{j-2})} = \overline{t^{j-2}(t^2+1)} \overline{-t^{j-2}} \\
		&= \overline{t^{j-2}}\odot \overline{t^2+1} + \overline{-t^{j-2}} = \overline{-t^{j-2}} = - (\overline{t^{j-2}})
	\end{align*}
	Für $p = \sum_{i = 0}^{\infty}a_it^i\in\IR[t]$ gilt: $\overline{p} = \overline{b_01+b_1t}$ für gewisse $b_0,b_1\in R$. Also sei o.B.d.A. $x = \overline{b_01+b_1t}$ mit $b_0^2+b_1^2 \neq 0$. Sei $q := \frac{1}{b_0^2b_1^2}(b_01-b_1t)\in \IR[t]$. Das ist wohldefiniert, da $b_0^2+b_1^2 \neq 0$.
	
	Sei $y := \overline{q}$. Wir behaupten nun, dass $y = x^{-1}$ in $R/I$ liegt. Es gilt: 
	$$ (b_01 +b_1t) q = \frac{1}{b_0^2+b_1^2}(b_0^2+b_0b_1t-b_1b_0t-b_1^2t^2) = \frac{1}{b_0^2+b_1^2}(b_0^2-b_1^2t^2)$$
	Da $\overline{t^2+1} = \overline{0}$ und somit $\overline{-t^2} = \overline{1}$, gilt 
	$$xy = \overline{\frac{1}{b_0^2+b_1^2}(b_0^2+b_1^2)} = \overline{1}$$
	Also ist $x\in (R/I)^{\times}$ mit dem Inversen $y$.
	
	Übung: $R/I$ ist isomorph zu $\IC$ (als Körper).
\end{proof}

\begin{defi}
	Sei $R$ ein kommutativer Ring, $R\neq \{0\}$. Ein Ideal $I\subsetneq R$ heißt maximal, falls $I\neq R$ und $\nexists J\subsetneq R$ Ideal mit $I\subsetneq J\subsetneq R$ (echte Teilmengen).
\end{defi}
\begin{lem} \label{lem:field_0_max_ideal}
	Sei $R$ kommutativer Ring, $R\neq \{0\}$. Dann ist $R$ genau dann ein Körper, wenn $\lbrace 0 \rbrace$ das einzige \textup(maximale\textup) Ideal ist.
\end{lem}
\begin{proof}
	\leavevmode
	\begin{itemize}
		\item[\glqq $\Leftarrow$\grqq:] Sei $\{0\}$ ein maximales Ideal. Dann ist es auch schon das einzige maximale Ideal, denn falls $I\neq \{0\}$ ein maximales Ideal ist, so folgt $\{0\}\subset I\neq R$. Daraus folgt $\{0\} = I$, was im Widerspruch zur Annahme steht.
		
		Sei $a\in R, a\neq 0$. Dann gilt $(0) \subset (a)$ und folglich $(a) = R$. Nach \cref{lem:komring} ist dann $R^{\times} = R\setminus\{0\}$ und somit ist $R$ ein Körper.
		\item[\glqq $\Rightarrow$\grqq:] Sei $R$ ein Körper, $I\subseteq R$ Ideal, $I\neq \{0\}$. Dann existiert ein $a$ in $I$ mit $ a\neq 0$. Weil $R$ ein Körper ist, ist $a\in R^{\times}$. Daraus folgt $(a) = R$ und somit $I = R$. Somit ist $\{0\}$ das einzige Ideal und somit maximal.
  \qedhere
	\end{itemize}
\end{proof}

\begin{bsp} \label{bsp:IZ_max_ideal_klas}
	Betrachte $R = \IZ$ sowie ein Ideal $I\subseteq \IZ$. Dann ist $I$ genau dann maximal, wenn $I = (p)$, wobei $p$ prim ist.
	\begin{proof}
		\leavevmode
		\begin{itemize}
			\item[\glqq $\Rightarrow$\grqq:] Sei $I = (a)$ maximal. Jedes Ideal in $\IZ$ kann so geschrieben werden, da $\IZ$ ein Hauptidealring ist. Falls $a\neq \pm p$ mit $p$ prim, so existiert $b\in \IZ$ mit $b\mid a$ und $b\neq \pm a$, $b\neq  \pm 1$.
			Dann gilt $(a)\subsetneq (b)\subsetneq R$ ($(b)\neq R$, weil $b\neq \pm 1$, also $b\notin \IZ^{\times}$), was im Widerspruch zu $I$ maximal steht, also ist $a  = \pm p$ mit $p$ prim.
			
			\item[\glqq $\Leftarrow$\grqq:] Sei $I = (a)$ mit $a = \pm p$, $p$ prim. Sei $J$ ein Ideal in $R$ sowie $I \subsetneq J\subsetneq R$. Dann gilt $J = (b)$ für ein $b\in\IZ$, da $\IZ$ ein Hauptidealring ist, und nach \cref{lem:komring} folgt $b\mid a$, $b\neq \pm a$ (weil $I\subsetneq J$) und $b \neq \pm 1$ (weil $(b)\neq R$). Das ist ein Widerspruch zu $a = \pm p$.
    \qedhere
		\end{itemize}
	\end{proof}
\end{bsp}

\begin{satz} \label{thm:field_max_ideal}
	Sei $R$ ein kommutativer Ring, $R \neq\{0\}$, $I\subseteq R$ Ideal. $R/I$ ist genau dann ein Körper, wenn $I$ ein maximales Ideal ist.
\end{satz}

\paragraph{Als Vorbereitung:}

\begin{satz} \label{thm:ideal_bij}
	Sei $\phi\colon R\to S$ ein surjektiver Ringhomomorphismus. Dann gibt es eine Bijektion zwischen den Mengen
	\begin{alignat*}{2}
		\left\{\substack{\text{Ideale in $R$}\\\text{ mit $\ker \phi\subseteq I$} }\right\}&\longleftrightarrow \; &&\{\text{Ideale in $S$}\} \\\intertext{durch}
		I & \longmapsto &&\phi(I)\\
		\underbrace{\phi^{-1}(J)}_{\mathclap{\text{$\{r\in R\mid\phi(r) \in J\}$}}} & \longmapsfrom && J
	\end{alignat*}
\end{satz}

\begin{proof}
	Behauptung: $\phi(I)\subseteq S$ ist ein Ideal.
	\begin{itemize}
		\item[(I1)] Das ist klar, weil Bilder von Gruppen unter Gruppenhomomorphismen wieder Gruppen sind.
		\item[(I2)] Es ist zu zeigen: Falls $x\in \phi(I) , r\in S$ dann gilt $ rx\in \phi(I), xr\in \phi(I)$. Weil $\phi$ surjektiv ist, existiert ein $r'\in R$ mit $\phi(r') = r$. Nach Annahme existiert ein $y \in I$ mit $\phi(y) = x$. Dann gilt $\phi(r'y) = \phi (r')\phi(y)  = rx\in \phi(I)$, weil  $I$ ein Ideal ist. Genauso gilt $\phi(yr') = xr\in\phi(I)$. Somit ist $\phi(I)$ ein Ideal.
	\end{itemize}
	Behauptung: $\phi^{-1}(J)\subseteq R$ ist ein Ideal mit $\ker\phi \subseteq \phi^{-1}(J)$.
	\begin{itemize}
		\item[(I1)]	Sei $\phi^{-1}(J)\subseteq R$ ein Ideal. Dann gilt $\ker\phi\subseteq \phi^{-1}(J)$, weil $\ker\phi = \phi^{-1}(\{0\})$ und $0\in J$ für jedes Ideal $J$. Es ist klar, dass $\phi^{-1}(J)$ eine Untergruppe von $R$ ist, da $\phi$ ein Gruppenhomomorphismus ist. 
		\item[(I2)] Sei $r\in R, x\in\phi^{-1}(J)$; es ist zu zeigen, dass $rx, xr\in\phi^{-1}(J)$. Es gilt $\phi(rx) = \phi(r)\phi(x)\in J$. Genauso gilt $\phi(xr)\in J$. Also liegen $rx$ und $xr$ in $\phi^{-1}(J)$, womit die Behauptung gezeigt ist.
	\end{itemize}
	\lecture{6. November 2017}

	Sei nun $I \subseteq R$ ein Ideal mit $I \supseteq \ker\phi$ und $J \subseteq S$ ein Ideal. Es bleibt zu zeigen:
	\begin{description}
		\item[$\phi^{-1}(\phi(I)) = I$:] \leavevmode
		\begin{itemize}
			\item[\glqq $\supseteq$\grqq] $x\in I \Rightarrow \phi(x) \in \phi(I)\Rightarrow x\in \phi^{-1}(\phi(I))$
			\item[\glqq $\subseteq$\grqq] 
			\begin{align*}
				&x\in \phi^{-1}(\phi(I)) \\
				\Rightarrow & \phi(x)\in \phi(I) \\
				\Rightarrow & \exists y\in I \text{ mit } \phi(x) = \phi(y) \\
				\Rightarrow & \phi(x+(-y)) = 0 \\
				\Rightarrow & x\in I, \text{ da } y \in I
			\end{align*}
		\end{itemize}
		\item[$\phi(\phi^{-1}(J)) = I$:] \leavevmode
		\begin{itemize}
			\item[\glqq $\supseteq$\grqq] 
			\begin{align*}
				&x\in J \\
				\Rightarrow &\exists y\in R \text{ mit } \phi(y) = x \\
				\Rightarrow &y\in\phi^{-1}(J) \text{ und } \phi(y) = x \\
				\Rightarrow &x\in\phi(\phi^{-1}(J))
			\end{align*}
			\item[\glqq$\subseteq$\grqq] $x\in\phi(\phi^{-1}(J))\Rightarrow \exists y\in\phi^{-1}(J)\colon \phi(y) = x\Rightarrow x\in J$
    \qedhere
		\end{itemize}
	\end{description}
\end{proof}

\begin{bem} Beachte: Die Bijektion von \cref{thm:ideal_bij} ist inklusionserhaltend, also 
	\[
	\underbrace{I \subseteq I'}_{\mathclap{\substack{\text{Ideale in $R$,}\\\text{die $\ker\phi$ enthalten}}}} \quad \Longleftrightarrow \quad \phi(I) \subseteq \phi(I').
	\]
\end{bem}

\begin{proof}[Beweis von \cref{thm:field_max_ideal}] %TODO: Das hier noch sauberer machen
	Setze $\phi := \can \colon R\to R/I$ (surjektiv mit $\ker\phi = I$).
	\begin{itemize}
		\item [\glqq $\Leftarrow$\grqq ] Sei $R/I$ ein Körper. Nach \cref{lem:field_0_max_ideal} ist $\{0\}\subseteq R/I$ ein maximales Ideal. Dann gilt nach \cref{thm:ideal_bij}, dass $\phi^{-1}(\{0\}) = I$ und $\phi^{-1}(R/I) = R$ die einzigen Ideale sind, die $I$ enthalten. Somit ist $I$ ein maximales Ideal.
		\item[\glqq$\Rightarrow$\grqq] Sei $I$ ein maximales Ideal in $R$. Dann gibt es genau $I$ und $R$ als Ideale, die $I$ enthalten.
		Nach \cref{thm:ideal_bij} sind $\{0\} = \phi(I)$ und $R/I = \phi(R)$ alle Ideale von $R/I$. Aus \cref{lem:field_0_max_ideal} folgt dann, dass $R/I$ ein Körper ist.
  \qedhere
	\end{itemize}
\end{proof}

\begin{defi}
	Sei $R$ ein kommutativer Ring, $I\subseteq R$ ein Ideal. Dann heißt $I$ Primideal, falls $I\neq R$ und $\forall x, y\in R$ gilt, dass falls $xy$ in $I$ liegt, $x$ oder $y$ das auch tut, also: \[xy\in I \quad \Longrightarrow \quad x\in I\text{ oder }y\in I.\]
\end{defi}
\begin{bem}
	Falls $R$ ein Integritätsbereich ist, dann ist $\{0\}$ ein Primideal.
\end{bem}

\begin{bsp}
	Sei $R = \IZ, a\in \IZ$. Dann ist $(a)$ genau dann ein Primideal, wenn $a$ oder $-a$ prim ist, oder $a = 0$.
\end{bsp}
\begin{proof}
	\leavevmode
	\begin{itemize}
		\item [\glqq $\Leftarrow$\grqq ]Sei $a = \pm p$, $p$ prim. Seien $x,y\in R$ mit $xy\in (a)$.

		$\Rightarrow$ $p$ teilt $xy$

		$\Rightarrow$ $p$ teilt $x$ oder $p$ teilt $y$

		$\Rightarrow$ $x\in (p) = (a)$ oder $y\in (p) = (a)$

		$\Rightarrow$ $(a)$ ist ein Primideal
		\item [\glqq $\Rightarrow$\grqq ] Sei $(a) \neq (0)$ ein Primideal. Falls $a = \pm 1$, also $ (a) = R$, so ist das ein Widerspruch dazu, dass $(a)$ ein Primideal ist. Falls $a \neq \pm 1, a\neq \pm p$ mit $p$ prim, so existieren $n, m$ mit $1<\abs{n}, \abs{m}<\abs{a}$ sodass $a = nm$. 
		Daraus folgt nun $ nm\in (a)$ und $n,m\notin (a)$, was wieder $(a)$ als Primideal widerspricht.
  \qedhere
	\end{itemize}
\end{proof}

\begin{bem}
	Beachte: $(0)$ ist prim in $\IZ$, aber nicht maximal. In $\IZ$ gilt \glqq$I$ maximal $\Rightarrow I $ prim\grqq.
\end{bem}


\begin{satz} \label{thm:primitb}
	Sei $R$ ein kommutativer Ring, $R\neq\{0\}$ und $ I\subseteq R$ ein Ideal. Dann gilt
	\begin{align*} I \text{ Primideal} \quad &\Longleftrightarrow \quad R/I \text{ Integritätsbereich}. \\\intertext{Insbesondere folgt}
	 \text{$\{0\}$ Primideal} \quad&\Longleftrightarrow\quad \text{$R$ Integritätsbereich}.\end{align*}
\end{satz}

\begin{proof}
	\leavevmode
	\begin{itemize}
		\item[\glqq $\Rightarrow$\grqq] Da $R$ kommutativ ist, ist auch $R/I$ kommutativ.
		Sei $I$ ein Primideal. Dann gilt $I\neq R$ und somit $R/I \neq\{0\}$. 
		
		Seien $\overline{x}, \overline{y}\in R/I$ mit  $\overline{x}\odot\overline{y} = \overline{0}$.
		\begin{align*}
			&\overline{xy} = \overline{0} \\
			\Rightarrow& xy\in I \\
			\Rightarrow& x\in I \text{ oder } y\in I \\
			\Rightarrow& \overline{x} = \overline{0} \text{ oder } \overline{y} = \overline{0} \\
			\Rightarrow& R/I \text{ hat keine Nullteiler}
		\end{align*}
		
		\item[\glqq$\Leftarrow$\grqq] Sei $R/I$ ein Integritätsbereich. Dann ist $R/I \neq \{0\}$. Daraus folgt $I \neq R$. Seien $x,y\in R$ mit $xy\in I$. Dann gilt:
		\begin{align*}
			&\overline{x}\odot\overline{y} = \overline{xy} = \overline{y} \\
			\Rightarrow& \overline{x} = \overline{0} \text{ oder } \overline{y} = \overline{0} \\
			\Rightarrow& x\in I \text{ oder } y\in I \\
			\Rightarrow& I \text{ ist ein Primideal}
    \qedhere
		\end{align*}
	\end{itemize}
\end{proof}

\begin{kor} \label{cor:maxprim}
	Sei $R\neq\{0\}$ ein kommutativer Ring, $I\subseteq R$ ein Ideal. Dann gilt: \[\text{$I$ maximal} \quad  \Longrightarrow \quad  \text{$I $ prim} \]
\end{kor}
\begin{proof}
	$I$ maximal $\xLeftrightarrow{\ref{thm:field_max_ideal}}  R/I$ Körper $\Rightarrow R/I$ Integritätsbereich $\xLeftrightarrow{\ref{thm:primitb}} I$ Primideal
\end{proof}


\begin{kor} \label{cor:IZ_quotring_klas}
	Es sei $R=\IZ$.
	\begin{enumerate}
	\item Dann ist $\IZ/n\IZ$ Körper genau dann wenn $n = \pm p$ mit $p$ prim.
	\item  $\IZ/n\IZ$ ist Integritätsbereich genau dann wenn $n = \pm p$ und $p$ prim oder $n = 0$.
	\end{enumerate}
\end{kor}

\begin{proof}
	\leavevmode
	\begin{enumerate}
		\item Folgt aus \cref{bsp:IZ_max_ideal_klas} und \cref{thm:primitb}.
		\item Für Ringe $R$, die Integritätsbereich Hauptidealbereich sind, gilt für ein Primideal $I \neq \{0\}$, dass $I $ maximal ist.
  \qedhere
  \end{enumerate}
\end{proof}	
	
\begin{satz}[Universelle Eigenschaft von Polynomringen 2]\label{thm:unieig_polyring}
	Sei $\phi\colon R\to S$ Ringhomomorphismus. Sei $a\in S$. Dann existiert ein eindeutiger Ringhomomorphismus
	\begin{eqnarray*}
		\ev_a\colon R[t] & \longto & S\\
		t & \longmapsto & a\\
		p(t )= \sum_{i = 0}^{\infty}b_it^i & \longmapsto& \sum_{i = 0}^{\infty} \phi(b_i)a^i
	\end{eqnarray*}
	
\end{satz}

\begin{proof}
	Da wir $\ev_a$ im Fall der Existenz bereits eindeutig charakterisiert haben, genügt es, die Existenz zu zeigen.
	\begin{align*}
		\ev_a(1_{R[t]}) &= \ev_a(1t^0) = \phi(1)\cdot a^0 = 1 \\
		\ev_a\left(\sum_{i= 0}^{\infty}b_it^i + \sum_{i = 0}^{\infty}c_it^i\right) &= \ev_a\left(\sum_{i= 0}^{\infty}(b_i+c_i)t^i\right)  = \sum_{i = 0}^{\infty} \phi(b_i+c_i)a^i \\
		&= \sum_{i = 0}^{\infty} \phi(b_i)a^i +\sum_{i=0}^{\infty}\phi(c_i)a^i=\ev_a\left(\sum_{i=0}^{\infty}b_it^i\right) + \ev_a \left(\sum_{i = 0}^{\infty} c_it^i\right) \\
		\ev_a\left(\left(\sum_{i = 0}^{\infty}b_it^i\right)\left(\sum_{i = 0}^{\infty}c_it^i\right)\right) &= \ev_a\left(\sum_{i = 0}^{\infty}\left(\sum_{k = 0}^{i} b_{i-k}c_k\right)t^i\right) = \sum_{i = 0}^{\infty} \left(\sum_{k = 0}^{i}\phi(b_{i-k})\phi(c_k)\right)a^i \\
		&= \sum_{i = 0}^{\infty} \left(\sum_{k = 0}^{i}\phi(b_{i-k})a^{i-k}\phi(c_k)a^k\right) = \left(\sum_{i = 0}^{\infty} \phi(b_i)a^i\right)\left(\sum_{i = 0}^{\infty}\phi(c_i)a^i\right)\\
		& = \ev_a\left(\sum_{i = 0}^{\infty}b_it^i\right)\cdot \ev_a\left(\sum_{i = 0}^{\infty}c_it^i\right)
	\end{align*}
	Somit ist $\ev_a$ tatsächlich ein Ringhomomorphismus.
\end{proof}

\begin{bsp}
	Sei
	\begin{eqnarray*}
		\phi\colon \IR & \longhookrightarrow & \IC\\
		x & \longmapsto & x
	\end{eqnarray*}
	Es existiert ein eindeutiger Ringhomomorphismus
	\begin{eqnarray*}
		\ev_\ima\colon \IR[t] & \longto & \IC\\
		t & \longmapsto & \ima\\
		p(t) & \longmapsto & p(i)
	\end{eqnarray*}
	Es gilt $\ev_\ima(t^2+1) = \ima^2 +1 = 0$. Somit gilt $t^2+1\in\ker \ev_\ima$ und deshalb auch $(t^2+1)\subset \ker \ev_\ima$. Mit dem \namereff{thm:homsatz_r} erhalten wir einen eindeutigen Ringhomomorphismus $\overline{\ev_\ima}\colon \IR[t]/(t^2+1)\to \IC$ mit $\overline{\ev_\ima}(\overline{b_1t+b_0}) = b_1i+b_0$. Dieser ist offensichtlich surjektiv.
	
	Da sich jedes $\overline{p(t)}\in\IR[t]/(t^2+1)$ eindeutig schreiben lässt als $\overline{p(t)} = \overline{b_1t+b_0}$ gilt:
	$$\overline{\ev_\ima}(\overline{p(t)}) = \overline{\ev_\ima}(\overline{b_1t+b_0}) = b_1i +b_0 = 0\Leftrightarrow b_1 = b_0 = 0$$
	Damit ist $\overline{\ev_\ima}$ injektiv und somit ist $\overline{\ev_\ima}$ ein Isomorphismus von Ringen (insbesondere von Körpern).
\end{bsp}

\begin{defi}
	Ist $R$ ein kommutativer Ring, dann heißt $a\in R$ Nullstelle von $p(t) \in R[t]$, falls gilt: $p(a) := \ev_a(p(t)) = 0$ ($\phi\colon R\to R = \id_R$).
\end{defi}


\begin{satz}\label{thm:nullstelle_eigenschaft}
	Sei $R$ ein kommutativer Ring sowie $a\in R$. Dann gilt:
	\begin{enumerate}
		\item $a$ Nullstelle von $p(t) \Leftrightarrow t-a$ teilt $p(t)$ in $R[t]$
		\item $R$ Integritätsbereich $\Rightarrow R[t]$ ist Integritätsbereich und $\deg(p(t)q(t)) = \deg (p(t)) + \deg (q(t))$
		\item $\deg (p(t)) = n\geq 0$, $R$ Integritätsbereich $\Rightarrow$ $p(t)$ hat maximal $n$ verschiedene Nullstellen.
	\end{enumerate}
\end{satz}


\lecture{9. November 2017}

\begin{proof}
	\leavevmode
	\begin{enumerate}
		\item \begin{description}
			\item[\glqq$\Leftarrow$\grqq:] Gilt $t-a \mid p(t)$, so existiert ein $q(t) \in R[t]$, sodass $p(t) = q(t)\cdot (t-a)$. Dann folgt $\ev_a (p(t)) = \ev_a(q(t))\cdot\ev_a(t-a) = 0$, und $a$ ist somit eine Nullstelle von $p(t)$.
			\item[\glqq$\Rightarrow$\grqq:] Sei $a$ eine Nullstelle von $p(t)$ und weiterhin
			\begin{eqnarray*} %TODO: \phi is never used
				\phi\colon R&\longhookrightarrow & R[t]\\
				r & \longmapsto & r\cdot 1\\[1mm]
				\xRightarrow{\ref{cor:IZ_quotring_klas}} \ev_{t-a} \colon R[t]& \longto & R[t]\\
				t&\longmapsto &t-a\\[1mm]
				\ev_{t+a}\colon R[t] & \longto & R[t]\\
				t & \longmapsto & t+a
			\end{eqnarray*}
			Es gilt $\ev_{t-a}\circ \ev_{t+a} = \id_{R[t]} = \ev_{t+a}\circ \ev_{t-a}$
			(für alle $r\in R$ gilt $r \mapsto r1$ sowie $t\mapsto t+a\mapsto (t-a)+a = t$ und umgekehrt).
			
			%TODO: This seems fishy. Someone check this later
			Somit folgt:
			\begin{center}
				\begin{tikzcd}
					R[t] \arrow{rd}[swap]{\ev_0} \arrow{r}{\ev_{t-a}} & R[t] \arrow{d}{\ev_{a}} \\
					& R \\
				\end{tikzcd}
			\end{center}
			$\ev_a \circ \ev_{t-a} = \ev_0$ bzw. $\ev_0\circ \ev_{t+a} = \ev_a$
			
			$\Rightarrow \ev_a(p(t)) = 0\Rightarrow \ev_0(\ev_{t+a}(p(t))) = 0\Rightarrow t$ teilt $\ev_{t+a}(p(t))\Rightarrow \exists q(t)\in R[t]$ sodass $q(t)\cdot t = \ev_{t+a}(p(t))\Rightarrow p(t) = \ev_{t-a}(ev_{t+a}(p(t))) = \ev_{t-a}(q(t))\cdot \ev_{t-a}(t) = \ev_{t-a}(q(t)) \cdot(t-a)\Rightarrow t-a$ teilt $p(t)$.
		\end{description}
		\item Es seien $m,n\in\IN_0$. Dann hat man
		\[p(t)q(t) = \left(\sum_{i = 0}^{m}a_it^i\right)\left(\sum_{i = 0}^{n}b_it^i\right) = a_nb_mt^{n+m}+\dots,\] wobei $a_n$ und $b_n$ als Leitkoeffizienten nicht $0$ sind, also $a_mb_n\neq 0$, da $R$ ein Integritätsbereich ist. Es folgt $\deg(p(t)q(t)) = n+m$.
		
		\item Wir führend eine Induktion nach dem Grad von $p(t)$.
		
		
			Induktionsanfang: Für $\deg(p(t)) = 1$ ist $p(t) = b_1t+b_0$ mit $b_0,b_1\in R$. Falls $a$ eine Nullstelle von $p(t)$ ist, so existiert nach Teil 1 ein $r\in R\setminus\{0\}$, sodass $r(t-a) = b_1t+b_0$. Daraus folgt $r = b_1$ und $-ar = b_0$. Wir nehmen an, $a' \in R$ sei eine weitere Nullstelle. Dann ist $-ar = -a'r' = -a'r$, da $r = b_1 = r'$, also $r(a'+(-a)) = 0$. Da $R$ ein Integritätsbereich ist und $r\neq 0$ ist, muss $a' = a$ gelten.
			
			
			Induktionsschritt: Sei nun $\deg(p(t)) = n>1$ sowie $a$ eine Nullstelle von $p(t)$. Dann existiert wieder mit Teil 1 des Satzes ein $q(t)\in R[t]$ mit $(t-a) q(t) = p(t)$ und $\deg(q(t)) = n-1$ (nach Teil 2 des Satzes).
			Also gilt nach der Induktionsvoraussetzung, dass $q(t)$ 
			maximal $n-1$ Nullstellen hat.
			
			\emph{Behauptung}: $b$ ist genau dann eine Nullstelle von $p(t)$, wenn eine $b$ Nullstelle von $q(t)$ oder $b = a$ ist.
			
			\begin{description}
				\item[\glqq$\Leftarrow$\grqq:] Ist $b$ eine Nullstelle von $q(t)$, so erhalten wir $\ev_{b}(p(t)) = \ev_{b}(q(t))\cdot \ev_b(t-a) = 0\cdot \ev_{b}(t-a) = 0$. Ist $b=a$, so folgt ebenfalls $\ev_{b}(p(t)) = \ev_{b}(q(t))\cdot \ev_b(t-a) = \ev_{b}(q(t))\cdot 0 = 0$.
				\item[\glqq$\Rightarrow$\grqq:] Ist $b$ Nullstelle von $p(t)$, so folgt von $R$ dann $0 = \ev_{b}(p(t)) = \ev_{b}(q(t))\cdot \ev_{b}(t-a)$. Da $R$ ein Integritätsbereich ist, hat man $\ev_{b}(q(t)) = 0$, also ist $b$ eine Nullstelle von $q(t)$, oder $\ev_b(t-a) = 0$, also $b=a$.
			\end{description}
			
			Folglich hat $p(t)$ maximal $n$ verschiedene Nullstellen.
  \qedhere
	\end{enumerate}
\end{proof}


\begin{defi}
	Ein Körper $K$ heißt algebraisch abgeschlossen, falls jedes $p(t)\in K[t]$ mit $\deg(p(t))>0$ eine Nullstelle hat.
\end{defi}
\begin{bem}
	\leavevmode
	\begin{enumerate}
		\item Es sei $K$ algebraisch abgeschlossen sowie $p(t)\in K[t]$ mit $\deg(p(t)) =n > 0$. Dann existieren ein $c\in K^{\times}$ und $ a_1,\dots,a_n\in K$ (nicht zwangsweise verschieden), sodass $p(t) = c(t-a_1)\cdot \dots \cdot (t-a_n)$. (Beweis: Teil 3 von \cref{thm:nullstelle_eigenschaft} induktiv anwenden)
		\item Oft schreibt man $K = \overline{K}$ für $K$ algebraisch abgeschlossen.
	\end{enumerate}
\end{bem}


\begin{defi}
	Sei $S$ ein Ring und $R\subseteq S$ ein Unterring sowie $a_1,\dots,a_n\in S$ mit $a_ix = xa_i$ und $a_ia_j = a_ja_i$ für alle $x\in R$, $1\leq i,j\leq n$. Setze
	\[ R[a_1,\dots,a_n] := \bigcap_{\mathclap{\substack{S'\subseteq S \text{ Unterring},\\R\subseteq S',\\ a_1,\dots,a_n\in S'}}}S'.\]
	Dann heißt $R[a_1,\dots,a_n]$ der von $a_1,\dots,a_n$ erzeugte Unterring in $S$ über $R$. 
\end{defi}


\begin{bem}
	$R[a_1,\dots,a_n]$ ist der kleinste Unterring von $S$, der $R$ und $a_1,\dots,a_n$ enthält.
\end{bem}
\begin{proof}
	$R[a_1,\dots,a_n]$ enthält $R$ und $a_1,\dots,a_n$ per Konstruktion.
	Ist $S'$ ein weiterer solcher Unterring, so ist enthält er bereits $\bigcap_{S''}S'' = R[a_1,\dots,a_n]$, wobei $S''$ wie oben definiert ist.
\end{proof}

\begin{satz}\label{thm:evalpolyring}
	Sei $S$ ein Ring und $R\subseteq S$ ein Unterring sowie $a_1,\dots,a_n\in S$ mit $a_ia_j = a_ja_i$ und $a_ix = xa_i$ für alle $x\in R$ und $1\leq i,j\leq n$. Sei $\phi \colon R\hookrightarrow S$ die Inklusion und $R[t_1,\dots, t_n] := ((\dots(R[t_1])[t_2])\dots)[t_n]$. Dann ist $R[a_1,\dots,a_n]$ das Bild des Auswertungshomomorphismus 
	\begin{eqnarray*}
		\ev = \ev_{a_1,\dots,a_n}: R[t_1,\dots,t_n] & \longto & S\\
		p(t_1,\dots t_n) & \longmapsto & p(a_1, \dots, a_n)
	\end{eqnarray*}
	wobei
	\begin{align*}
		p (t_1,\dots, t_n) &= \sum_{\mathclap{(m_1,\dots,m_n)\in\IN_{0}^n}}b_{m_1,\dots,m_n} t_1^{m_1} \dots t_n^{m_n} \\
		\intertext{(fast alle $b_{m_1,\dots,m_n} = 0$) auf}
		p(a_1,\dots, a_n ) &= \sum_{\mathclap{(m_1,\dots, m_n)\in \IN_0^n}} b_{m_1,\dots,m_n} a_1^{m_1}\dots a_n^{m_n}
	\end{align*}
	abgebildet wird.
\end{satz}

\begin{proof}
		Zuerst wird Wohldefiniertheit gezeigt:
		\begin{equation*}
		\begin{gathered}
			R[t_1,\dots, t_n]\xlongrightarrow{\ev_{a_n}} R[t_1,\dots, t_{n-1}]\xlongrightarrow{\ev_{a_{n-1}}}\dots \xlongrightarrow{\ev_{a_2}}R[t_1]\xlongrightarrow{\ev_{a_1}} S \\
			\sum b_{m_1,\dots, m_n}t_1^{m_1}\dots t_n^{m_n}\longmapsto \sum b_{m_1,\dots m_n} a_n^{m_n}\dots a_1^{m_1} = \sum b_{m_1,\dots, m_n}a_1^{m_1}\dots a_n^{m_n}
		\end{gathered}
		\end{equation*}
		Somit ist $\ev$ als Verknüpfung von Ringhomomorphismen wieder ein wohldefinierter Ringhomomorphismus.
		
		Nun muss noch gezeigt werden, dass $R\lbrack a_1,\dots,a_n\rbrack = \im \ev$ gilt. Für $r\in R$ gilt $\ev(rt_1^0\dots t_n^0) = ra_1^0\dots a_n^0 = r$, also folgt $R\subseteq \im (\ev)$.
		Da nun $\ev(1t_1^0\dots t_{i-1}^0t_1^1t_{i+1}^0\dots t_n^0) = 1a_i = a_i$, erhalten wir $a_i\in\im (\ev)$ und dadurch $R[a_1,\dots, a_n]\subseteq \im(\ev)$.
		
		Sei $S'\subseteq S$ ein Unterring mit $R\subseteq S'$ und $a_1,\dots, a_n\in S'$. Dann ist $p(a_1,\dots,a_n) = \sum b_{m_1,\dots,m_n}a_1^{m_1}\dots a_n^{m_n}\in S'$, da $S'$ unter $+$ und $\cdot$ abgeschlossen ist. Man hat schließlich $R[a_1,\dots,a_n] = \im (\ev)$.
\end{proof}

\begin{defi}
	Sei $S$ ein kommutativer Ring und $R\subseteq S$ ein Unterring sowie $a_1,\dots, a_n\in S$. Dann heißen $a_1,\dots,a_n$ algebraisch unabhängig (über $R$), fallls $\ev = \ev_{a_1,\dots, a_n}$ (von \cref{thm:evalpolyring}) injektiv ist.
	
	Falls $\ev$ nicht injektiv ist, so heißen $a_1,\dots, a_n$ algebraisch abhängig (über $R$).
\end{defi}

\begin{lem}
	Sei $S$ ein kommutativer Ring und $R\subseteq S$ ein Unterring sowie $a_1,\dots, a_n\in S$. Es gilt:
	\begin{enumerate}
		\item $a_1,\dots,a_n$ sind genau dann algebraisch abhängig über $R$, wenn $p(t_1,\dots,t_n)\in R[t_1,\dots,t_n]\setminus\{0\}$ existiert sodass $p(a_1,\dots,a_n) = 0$.
		\item $a_1,\dots, a_n$ sind genau dann algebraisch unabhängig über $R$, wenn $p(a_1,\dots,a_n) = 0$ für ein Polynom $p(t_1,\dots, t_n)\in R[t_1,\dots,t_n]$ impliziert, dass $p(t_1,\dots, t_n) = 0$.
	\end{enumerate}
\end{lem}

Spezialfall: Sei $n = 1$. Dann ist $a\in S$ algebraisch abhängig über $R$, falls $p(a) = 0$ für ein $p(t) \in R[t]\setminus\{0\}$ gilt.

\begin{bsp}
	\leavevmode
	\begin{enumerate}
		\item $1,\ima\in\IC$ sind linear unabhängig über $\IR$, aber algebraisch abhängig, denn mit $p(t_1,t_2) = t_1+t_2^2$ hat man $\ev_{1,\ima}(p(t_1,t_2)) = 1+\ima^2 = 0$.
		
		Schon $\ima\in \IC$ ist algebraisch abhängig über $\IR$, man betrachte $\ev_{\ima}(1+t^2) = 1+\ima^2  = 0$.
		\item Sei $S$ ein kommutativer Ring sowie $a\in S$. Dann ist $a$ ist algebraisch abhängig über $S$: $\ev_a(t-a) = a-a = 0$.
		\item Sei $S =R[t_1,\dots, t_n]$. Dann sind $t_1,\dots t_n$ algebraisch unabhängig über $R$, weil $\ev_{t_1,\dots, t_n}\colon R[t_1,\dots,t_n]\to R[t_1,\dots, t_n] = \id_{R[t_1,\dots t_n]}$ offensichtlich injektiv ist.
		\item Algebraisch über $\IQ$ unabhängige Zahlen in $\IR$ heißen transzendente Zahlen.
	\end{enumerate}
\end{bsp}

\lecture{13. November 2017}
\subsection{Faktorielle Ringe}
Ab jetzt sind alle Ringe kommutativ.
\begin{defi} Sei $R$ ein Integritätsbereich (also $R\neq\{0\}$ und nullteilerfrei) sowie $a\in R$, $a\neq 0$, $a\notin R^{\times}$. Weiterhin seien $b,c \in R$.
\begin{itemize}
	\item $a$ heißt irreduzibel, falls \[a = b\cdot c\quad\Longrightarrow\quad \text{$b \in R^{\times}$ oder $c\in R^{\times}$}.\]
	\item $a$ heißt prim, falls \[a \mid b\cdot c\quad\Longrightarrow\quad \text{$a \mid b$ oder $a \mid c$}.\]
	\item $R$ ist ein faktorieller Ring, falls zusätzlich für $b\in R, b\neq 0$ eine eindeutige Zerlegung \glqq PZ\grqq{}
	\[b = \epsilon p_1p_2\dots p_r \text{ für ein $r\in\IN$}\]
	mit $\epsilon\in R^{\times}$ und irreduziblen $p_1,\dots, p_r\in R$ existiert. Die Eindeutigkeit besteht hierbei bis auf Einheiten und Reihenfolge der Faktoren, das heißt, falls $b = \epsilon'p_1'\dots p_s'$ für $\epsilon'\in R^{\times}$ und $p_1',\dots, p_s'$ irreduzibel, dann gilt $s = r$ und es existieren $c_1,\dots, c_r\in R^{\times}$ und ein $\pi \in S_r$, sodass $p_i' = c_ip_{\pi(i)}$ für alle $1 \le i \le s=r$ gilt.
\end{itemize}
\end{defi}

\begin{bsp}
	$R = \IZ$. Sei $a\in\IZ$ mit $a\neq 0$ und $a\notin\{\pm1\} = \IZ^{\times}$. Dann erhalten wir \[\text{$a$ prim} \quad \Longleftrightarrow \quad \text{$a=\pm p$ mit $p$ prim} \quad\Longleftrightarrow\quad \text{$a$ irreduzibel}.\] $\IZ$ ist faktoriell, denn jedes $b\in\IZ, b\neq 0$ hat eine Primzahlzerlegung $b = \epsilon p_1\dots p_r$ mit $p_i$ prim und $\epsilon \in\{\pm 1\} = \IZ^{\times}$. Diese ist \glqq eindeutig\grqq, z.B. $10 = 1\cdot 2\cdot 5 = (-1)\cdot(-2)\cdot 5 = (-1)\cdot2\cdot(-5)$ etc. 
\end{bsp}
\begin{bsp}
	$R = \IZ[\sqrt{-5}] = \{a+b\sqrt{5}\ima\mid a,b\in \IZ\}\subset \IC$. Das ist ein Unterring von $\IC$, genauer der von $\sqrt{5}\ima$ über $\IZ$ erzeugte Unterring in $\IC$, also \[\IZ[\sqrt{-5}] = \bigcap_{\mathclap{\substack{S \subseteq \IC \text{ Unterring}\\\IZ \subseteq S \\ \sqrt{5} \ima \in S }}}S. \]
	Dann sind $2$, $3$, $1+\sqrt{5}\ima$, $1-\sqrt{5}\ima$ irreduzibel (Übung). Insbesondere ist $1+\sqrt{5}\ima\notin R^{\times}$, da es sonst $a,b \in \IZ$ gäbe, sodass $(1+\sqrt 5 \ima)(a+b\sqrt 5\ima)=1$ und somit $a-5b=1$ und $a+b=0$ ist, was aber unlösbar ist.
	Außerdem: $1+\sqrt{5}i\notin \{\epsilon 2,\epsilon3 \mid\epsilon\in R^{\times}\}$ (Übung).
	
	Aber es ist $ 6 = 2\cdot 3 = (1+\sqrt{5}\ima)(1-\sqrt{5}\ima)$, also ist $R$ nicht faktoriell, da die Zerlegung nicht eindeutig ist.
\end{bsp}

\begin{lem} \label{lem:primired}
	Sei $R$ ein Integritätsbereich und $a\in R$. Dann gilt:
	\begin{enumerate}
		\item $a$ ist genau dann prim, wenn $(a) \neq \{0\}$ ein Primideal ist.
		\item $a$ ist genau dann irreduzibel, wenn $(a)\neq\{0\}$, $(a)\neq R$ und es kein $b \in R$ gibt, sodass $(a) \subsetneq (b)$ gilt, außer $(b)$ erzeugt bereits den ganzen Ring, also $\forall b\in R: (a) \subseteq (b)\Rightarrow ((a) = (b) \text{ oder } (b) = R)$.
	\end{enumerate}
\end{lem}
\begin{proof}
	\leavevmode
	\begin{enumerate}
	\item 
	\begin{description}
		\item[\glqq$\Rightarrow$\grqq:] Sei $a$ prim. Nach Definition gilt $a\neq 0, a\notin R^{\times}$. Folglich gilt auch $(a)\neq \{0\}, (a) \neq R$. Sei nun $bc\in (a)$ mit $b,c\in R$. Dann teilt $a$ $bc$. Da $a$ prim ist, teilt nun also $a$ $b$ oder $c$. Es folgt $b\in (a)$ oder $c\in (a)$ und somit ist $(a)$ ein Primideal.
		\item[\glqq$\Leftarrow$\grqq:] Sei $(a) \neq \{0\}$ (und somit $a\neq 0$) und sei $(a)$ ein Primideal. Nach Definition gilt $(a) \neq R$, also ist $a$ keine Einheit. Gelte nun $a$ teilt $bc$. Dann liegt $ bc$  in $(a)$. Da $(a)$ ein Primideal ist, gilt $b\in (a)$ oder $c\in (a)$. Somit teilt $a$ $b$ oder $c$ und ist folglich prim.
	\end{description}
	\item
	\begin{description}
		\item[\glqq$\Leftarrow$\grqq:] Sei $(a) \neq\{0\}, (a) \neq R$. Dann folgt sofort $a\neq 0$ und $a\notin R^{\times}$. Sei nun $a = bc$, also $(a)\subseteq (b)$. Nach Voraussetzung gilt dann $(a) = (b)$ oder $(b) = R$ (also $b\in R^{\times}$). Falls $b$ keine Einheit ist, existiert somit ein $r\in R$, sodass $b = ar$. Durch Umformen folgt
		\begin{align*}
			&b = ar \\
			\Rightarrow& b = bcr \\
			\Rightarrow& b(1-cr) = 0 \\
			\Rightarrow& 1-cr = 0 \text{, da R nullteilerfrei ist}
		\end{align*}
		In diesem Fall gilt $c \in R^{\times}$ und folglich ist $b$ oder $c$ eine Einheit, also $a$ irreduzibel.

		\item[\glqq$\Rightarrow$\grqq:] Sei $a$ irreduzibel. Es folgt sofort $(a) \neq \{0\}$ und $(a)\neq R$. Sei nun $b\in R$ mit $(a) \subseteq (b)$. Dann gilt $b$ teilt $a$ und somit existiert ein $r\in R$, sodass $a = br$. Aus der Irreduzibilität von $a$ folgt nun $b\in R^{\times}$ oder $r\in R^{\times}$. Folglich gilt $(b) = R$ oder $(a) = (b)$.
  \qedhere
	\end{description}
\end{enumerate}
\end{proof}



\begin{satz} \label{thm:primirred_intb} Sei $R$ ein Integritätsbereich. Dann gilt für $a\in R$: \begin{enumerate}
		\item Wenn $a$ prim ist, dann ist $a$ auch irreduzibel.
		\item Falls $R$ zusätzlich ein Hauptidealring ist:
		\begin{enumerate}
			\item $a$ ist genau dann prim, wenn $a$ irreduzibel ist.
			\item Sei $\{0\}\neq I\subseteq R$ ein Ideal. Dann ist $I$ genau dann ein Primideal, wenn $I$ ein maximales Ideal ist.
		\end{enumerate}
	\end{enumerate}
\end{satz}
\begin{proof}
	\leavevmode
	\begin{enumerate}
		\item Sei $a = bc$ mit $b,c \in R$. Da $a$ prim ist, folgt $a\mid b$ oder $a \mid c$, es ist also $b = ar$ oder $c = ar$ für ein $r\in R$.  Folglich haben wir $b = bcr$ oder $c = cbr$ und somit auch $b(1-cr)=0$ oder $c(1-br)=0$. Da $R$ nullteilerfrei ist und $b,c \neq 0$, weil $a \neq 0$, ist $c$ oder $b$ eine Einheit. Schließlich ist $a$ irreduzibel.
		
		\item \begin{enumerate}
			\item Die eine Richtung folgt nach 1. Sei für die Rückrichtung nun $a$ irreduzibel. Nach \cref{lem:primired} ist $(a)$ maximal unter allen Hauptidealen in $R$. Da $R$ ein Hauptidealring ist, ist $(a)$ sogar ein maximales Ideal. Aus \cref{cor:maxprim} folgt, dass $(a)$ ein Primideal ist. Nun ist $a$ prim nach \cref{lem:primired}, da $(a)\neq\{0\}$, weil $a$ irreduzibel, also insbesondere $a\neq 0$ ist.
			
			\item \begin{description}
					\item[\glqq$\Leftarrow$\grqq:] siehe \cref{thm:primitb}
					\item[\glqq$\Rightarrow$\grqq:] Sei $\{0\}\neq I\subseteq R$ ein Primideal. Da $R$ ein Hauptidealring ist, existiert ein $a\in R$ mit $I = (a)$, wobei $a$ nach \cref{lem:primired} prim ist. Mit Teil 1. folgt $I = (a)$ und $a$ irreduzibel. Wieder nach \cref{lem:primired} ist $I = (a) $ maximal unter allen Hauptidealen, und da $R$ ein Hauptidealring ist, ist $I$ tatsächlich ein maximales Ideal.
					\qedhere
					\end{description}
			\end{enumerate}
	\end{enumerate}
\end{proof}

\begin{defi} Sei $R$ ein Integritätsbereich. $R$ heißt euklidischer Ring, falls eine Abbildung
	$$\deg \colon R\setminus\{0\} \longto \IN$$
	existiert, sodass für alle $a,b\in R, b\neq 0$ dann $q, r\in R$ mit \[a = qb+r \text{ und $r = 0$ oder $\deg(r)<\deg(b)$}\] existieren.
\end{defi}

\begin{satz}
	Jeder euklidische Ring ist ein Hauptidealring.
\end{satz}
\begin{proof}
	Wie für $R = \IZ$.
\end{proof}

\begin{satz}\label{thm:hirfakt}
	Jeder Hauptidealring ist faktoriell.
\end{satz}

\begin{proof} Sei $R$ ein Hauptidealring und $0\neq b\in R$. Es ist nun die Existenz einer eindeutigen Primfaktorzerlegung zu zeigen.
\begin{description}
	\item[Existenz:] Wir nehmen an, dass diese nicht existiert, also insbesondere $b$ nicht irreduzibel ist. Dann existieren $b_1,c_1\notin R^{\times}$ mit $b = b_1c_1$, wobei nicht beide Faktoren eine Primfaktorzerlegung haben. O.B.d.A. habe $b_1$ keine Primfaktorzerlegung, insbesondere $b_1$ nicht irreduzibel. Folglich existieren $b_2,c_2\notin R^{\times}$ mit $b_1 = b_2c_2$ und o.B.d.A. habe $b_2$ keine (PZ) etc.
	
	Wir erhalten eine Kette echter Inklusionen
	\[(b_1)\subsetneq (b_2)\subsetneq (b_3)\subsetneq (b_4)\subsetneq \dots\] für gewisse $b_i\in R$. Es gilt tatsächlich $(b_i) \subsetneq (b_{i+1})$, denn sonst wäre $b_i = b_{i+1}c_{i+1}$ für ein $c_{i+1}\in R$ und $b_{i+1} = b_ic_i'$ für ein $c_i'\in R$, also $b_i(1-c_i'c_{i+1}) = 0$. Da $R$ nullteilerfrei ist, folgt $c_{i+1}\in R^{\times}$ für alle $i$, was im Widerspruch zur Konstruktion steht.
	
	Setze nun $I : = \bigcup\limits_{i = 1}^{\infty}(b_i)$. Dies ist ein Ideal und somit folgt $I = (d)$ für ein $d\in R$, weil $R$ ein Hauptidealring ist. Nach Definition von $I$ existiert ein $b_i$ mit $d\in (b_i)$, also $(d)\subseteq (b_i)$, weil $(b_i)$ ein Ideal ist. Wir erhalten unmittelbar $(d) \subseteq (b_i)\subseteq I = (d)$ oder $(b_i) = (d)\subseteq (b_j) = (d)$ für alle $j>i$. Das ist nun ein Widerspruch zu $(b_i) \subsetneq (b_{i+1})$. Folglich existiert für $b$ eine Primfaktorzerlegung.
	\item[Eindeutigkeit:] Seien $b = \epsilon p_1\dots p_r$ und $b = \epsilon'p_1'\dots p_s'$ zwei Zerlegungen mit $\epsilon, \epsilon'\in R^{\times}$ und irreduziblen $p_i, p_j'\in R$ für $1\leq i\leq r$, $1\leq j\leq s$. Betrachte $p_1'$. Nach Satz 8.1 2a) ist $p_1'$ als irreduzibles Element schon prim. Da $p_1'\mid b$, folgt $p_1'\mid p_i$ für ein $i$ ($1\leq i\leq r$) oder $p_1'$ teilt $\epsilon$. Letztes kann aber nicht eintreten, da sonst $p_1'\in R^{\times}$ ist, was im Widerspruch zur Irreduzibilität von $p_1'$ steht.
	
	Es sei also $i$ so fixiert, dass $p_1' \mid p_i$. O.B.d.A. sei $i = 1$ (eventuelle Umnummerierung). Also ist $p_1 = p_1'c$ für ein $c\in R$. Da $p_1$ irreduzibel ist, ist $p_1'\in R^{\times}$ oder $c\in R^{\times}$. $p_1'$ ist aber irreduzibel und damit insbesondere keine Einheit; folglich ist $c\in R^{\times}$.
	
	Wir erhalten somit $\epsilon cp_1'p_2\dots p_r = b = \epsilon 'p_1'p_2'\dots p_s'$. Da $R$ nullteilerfrei ist, darf man hier kürzen: $\epsilon c p_2\dots p_r = \epsilon 'p_2'\dots p_s'$. Wir wenden nun dieses Argument iterativ an und erhalten schließlich $r = s$ und $p_i = p_{\pi(i)}'c_i$ für $c_i\in R^{\times}$ und $\pi\in S_r$, was genau die gewünschte Eindeutigkeit darstellt.
\end{description}
Somit ist $R$ faktoriell.
\end{proof}

\begin{bem}
	Insbesondere: In $\IZ$ ist Primfaktorzerlegung \glqq eindeutig\grqq.
\end{bem}

\subsection{Maximale Ideale: Existenz}
\begin{satz}\label{thm:max_ideal_ex}
	Sei $R$ ein Ring.
	und $R \neq \{0\}$.
	\begin{enumerate}
		\item $R$ besitzt maximale Ideale \textup(und damit auch Primideale nach \cref{thm:primitb}\textup).
		\item Jedes Ideal $I\subseteq R$, $I \neq R$ ist in einem \textup(nicht notwendigerweise eindeutigen!\textup) maximalen Ideal enthalten.
	\end{enumerate}
\end{satz}
\begin{proof}
	Wir zeigen zunächst 2. unter der Annahme von 1.: Sei $I\neq R$ ein Ideal in $R$. Dann ist $R/I \neq \{0\}$  und besitzt daher nach 1. ein maximales Ideal. Nach \cref{thm:ideal_bij} existiert ein maximales Ideal $J\subseteq R$ mit $J\supseteq I$.
	
	\lecture{16. November 2017}
	
	Es bleibt die erste Aussage zu zeigen. Hierfür sei $M := \{I\subseteq R\mid I\text{ Ideal}, I \neq R\}$. Betrachte nun die partielle Ordnung $\leq$ auf $M$ gegeben durch $I_1\leq I_2$ falls $I_1\subseteq I_2$. Sei $N\subseteq M$ eine total geordnete Teilmenge. Dann betrachte $\tilde{I} := \bigcup\limits_{I\in N} I$ ist ein Ideal.
	
	Es folgt $\tilde{I}\in M$, weil $1\notin \tilde{I}$, da $1\notin I$ für alle $ I\in N$ aus unserer Definition von $N\subseteq M$ folgt.
	
	Somit ist $N$ in $M$ nach oben beschränkt. Nach dem \textsc{Lemma von Zorn} existiert ein maximales Element in $M$, woraus die Existenz eines (nicht notwendigerweise eindeutigen) maximalen Ideals in $R$ folgt.
\end{proof}
\begin{bsp}
	$R = \IC[t], a\in \IC$. Dann ist
	\begin{eqnarray*}
		\ev_a\colon \IC[t] & \longto & \IC\\
		p(t) & \longmapsto & p(a)
	\end{eqnarray*}
	ein surjektiver Ringhomomorphismus. Es gilt $p(t) \in \ker\ev_a$ genau dann, wenn $p(a) = 0$, also wenn $a$ eine Nullstelle von $p(t)$ ist. Nach \cref{thm:unieig_polyring} ist das äquivalent dazu, dass $(t-a)$ das Polynom $p(t)$ teilt; also ist $p(t)$ Element des von $(t-a)$ erzeugten Ideals. Nach dem \namereff{thm:homsatz_r} für Ringe erhalten wir einen Isomorphismus
	$$ \IC[t]/(t-a) \longto \IC$$
	Da $\IC$ ein Körper ist, ist $I = (t-a)$ maximales Ideal (für jedes $a\in \IC$).
\end{bsp}


\subsection{Einschub: Chinesischer Restsatz}
\paragraph{Mathematik in der Praxis.}
Stroppel hat eine Pralinenschachtel. Wenn jemand errät, wieviele Pralinen sich darin befinden, bekommt derjenige alle Pralinen (da niemand nachprüfen kann, ob das, was Stroppel bezüglich der Anzahl der Pralinen behauptet, stimmt, wird vermutlich niemand die Pralinen bekommen). Wir wissen über die Anzahl der Pralinen: Wenn man sie auf vier Leute aufteilt, bleiben zwei übrig; wenn man sie auf sieben Personen verteilt, bleiben drei Pralinen übrig, also
\begin{eqnarray*}
	x &\equiv & 2 \pmod 4,\\
	x &\equiv & 3 \pmod 7.
\end{eqnarray*}
Eine Lösung ist zum Beispiel $x = 10$, aber laut Stroppel ist das nicht die Anzahl der Pralinen (die Pralinenschachtel sieht aber gar nicht so groß aus, als ob da noch wirklich viel mehr Pralinen darin sein könnten). Wir vermuten nun, dass genau die Zahlen der Form $10+28k, k\in\IZ$ Lösungen sind.

\begin{satz}[Chinesischer Restsatz]
	Sei $R$ ein kommutativer Ring sowie $R\neq\{0\}$. Seien weiter $I_1,\dots,I_s$ Ideale in $R$, sodass $I_i+I_j = R$ für alle $i\neq j$ gilt. Dann existiert ein surjektiver Ringhomomorphismus
	\begin{eqnarray*}
		\Phi\colon R & \longto & R/I_1\times \dots R/I_s\\
		r & \longmapsto & (\overline{r},\dots, \overline{r})
	\end{eqnarray*}
	\textup(wobei $\overline{r}$ jeweils die Nebenklasse von $r$ bezüglich dem passenden Ideal $I_i$ ist\textup).
	Dabei gilt $\ker\Phi = I_1\cap\dots\cap I_s$. Insbesondere erhalten wir einen Isomorphismus von Ringen:
	$$ R/(I_1\cap\dots\cap I_s) \longto R/I_1 \times\dots \times R/I_s$$
	
\end{satz}


\begin{proof}
	Es lässt sich leicht nachrechnen, dass $\Phi$ tatsächlich ein Ringhomomorphismus ist. Es bleibt die Surjektivität von $\Phi$ zu zeigen. Nach Voraussetzung existieren für $1\leq i,j\leq s$ Elemente $a_{ij}\in I_i, a_{ji}\in I_j$ mit $a_{ij}+a_{ji} = 1$. Wir definieren nun
	\[b_i := \prod_{\mathclap{\substack{j\neq i\\1\leq j\leq s}}} (1-a_{ij}) = \prod_{\mathclap{\substack{j\neq i\\1\leq j\leq s}}} a_{ji}\]
	Es gilt
	\begin{equation}
	 \overline{b_i} = \begin{cases*} \text{$\overline{0}$ in $R/I_j$} & für $i\neq j$,\\
	 \text{$\overline{1}$ in $R/I_i$} & für $i=j$.
	 \end{cases*} \tag{*}\label{eq:thmchinres}
	 \end{equation}
	 Sei nun $(\overline{r_1},\dots, \overline{r_s})\in R/I_1\times \dots \times R/I_s$ beliebig ($r_i\in R$). Wir setzen
	 \[ x : = \sum_{i = 1}^{s} b_ir_i\in R.\]
	 Da $\Phi$ ein Ringhomomorphismus ist, folgt dann mit \eqref{eq:thmchinres} \[
	 \Phi(x) = \sum_{i=1}^{s}\Phi(b_ir_i) = \sum_{i=1}^{s}\Phi(b_i)\Phi(r_i) = \left(\sum_{i =1}^{s}\overline{b_i}\overline{r_i},\dots, \sum_{i = 1}^{s}\overline{b_i}\overline{r_i}\right) = (\overline{r_1},\dots, \overline{r_s})
	 .\]
	 Folglich ist $x$ Urbild von $(\overline{r_1},\dots, \overline{r_s})$. Damit ist $\Phi$ ist surjektiv.
	 
	 Schließlich ist $r\in \ker\Phi$ genau dann, wenn $(\overline{r},\dots, \overline{r}) = (\overline{0},\dots, \overline{0})\in R/I_1\times\dots\times R/I_s$, also wenn $\overline{r} = \overline{0}\in R/I_k$ für alle $1\leq k\leq s$. Das ist aber äquivalent dazu, dass $r$ im Schnitt $I_1\cap\dots\cap I_s$ liegt, also $\ker\Phi = I_1\cap\dots\cap I_s$.
\end{proof}


\subsection{Lokalisierung}
\begin{defi} Sei $R$ ein kommutativer Ring und $R\neq \{0\}$. Eine Teilmenge $S\subseteq R$ heißt multiplikativ abgeschlossen, falls $1\in S$ und $s_1,s_2\in S\Rightarrow s_1s_2\in S$ gelten. Dann nennen wir $S^{-1}R$ die Lokalisierung von $R$ an $S$. Dabei ist $S^{-1}R$ ein kommutativer Ring, wobei die Elemente Äquivalenzklassen in $R\times S$ bezüglich $\sim$ (siehe Übungsblatt 5).
Wir bezeichnen mit $\frac rs$ die Äquivalenzklasse von $(r,s)\in R\times S$.
Es gilt $\frac rs = \frac {r'}{s'}$ genau dann, wenn ein $a\in S$ existiert, sodass $a(rs'-r's) = 0$. Wir nennen $S^{-1}R$ auch den Ring der Brüche für $R$ bezüglich $S$.
\end{defi}

\begin{bem}
	\leavevmode
	\begin{enumerate}
		\item Falls $0_R\in S$ ist, so ist $S^{-1}R$ bereits der Nullring, denn für $s,s'\in S$ und $r,r'\in R$ gibt es ein $a\in S$ (nämlich z.B. $a = 0$) mit  $a(rs'-r's) = 0$.
		\item Falls $R$ ein Integritätsbereich ist, ist auch $S^{-1}R$ ein Integritätsbereich für alle $S$ mit $0\notin S$ (selber nachrechnen!).
	\end{enumerate}
\end{bem}

\begin{satz}[Universelle Eigenschaft der Lokalisierung] \label{thm:unieig_lokali}
	Sei $R$ kommutativer Ring, $R\neq\{0\}$, und $S\subseteq R$ multiplikativ abgeschlossen.
	\begin{enumerate}
		\item Die Abbildung
				\begin{eqnarray*}
					\can\colon R & \longto & S^{-1}R\\
					r & \longmapsto & \frac r1
				\end{eqnarray*}
			ist ein Ringhomomorphismus, wobei $\can(S) \subseteq(S^{-1}R)^{\times}$.
		\item Sei $\phi\colon R\to T$ ein Ringhomomorphismus, sodass $\phi(S) \subseteq T^{\times}$, dann existiert ein eindeutiger Ringhomomorphismus $\hat{\phi}\colon S^{-1}R\to T$, sodass
		$\hat{\phi}\circ \can = \phi$, also sodass das nachfolgende Diagramm kommutiert.
		\begin{center}
				\begin{tikzcd}
						R \arrow{rd}[swap]{\can} \arrow{r}{\phi} & T \\
						 	& Z\arrow{u}[swap]{\exists! \hat{\phi}\text{ Ringhomomorphismus}} \\
				\end{tikzcd}
		\end{center}
	\end{enumerate}
\end{satz}

\begin{proof}
	\leavevmode
	\begin{enumerate}
		\item Durch einfaches Nachrechnen stellt man fest, dass $\can$ ein Ringhomomorphismus ist. Weiterhin sei $s\in S$. Dann ist $\can(s) = \frac s1$ und $\frac s1\cdot \frac1s = 1$, also hat $\frac s1$ ein Inverses bezüglich der Multiplikation, ist also eine Einheit. Damit folgt die erste Aussage.
		\item \begin{description}
			\item[Existenz:] Wir definieren $\hat{\phi}(\frac rs) = \phi(r)\phi(s)^{-1}$, wobei das Inverse bezüglich $\cdot$ in $S^{-1}R$ nach Voraussetzung existiert.
			
			Wieder durch Nachrechnen erhält man, dass $\hat{\phi}$ ein Ringhomomorphismus ist. Es gilt $\hat{\phi}\circ \can(r) = \hat{\phi}(\frac r1) = \phi(r)\phi(1)^{-1} = \phi(r)\cdot 1^{-1} = \phi(r) $. Damit folgt die Existenz von $\hat{\phi}$.
			\item[Eindeutigkeit:] Wir erhalten $\hat{\phi}(\frac rs) = \hat{\phi} (\frac r1\cdot \frac1s) = \hat{\phi}(\frac r1)\cdot \hat{\phi}(\frac1s) = \hat{\phi}(\frac r1)\hat{\phi}((\frac s1)^{-1}) = \hat{\phi}(\frac r1)\hat{\phi}(\frac s1)^{-1} = \phi(r)\phi(s)^{-1}$, weil $\hat{\phi}\circ\can = \phi$, also $\hat{\phi}(\frac rs) = \phi(r)\phi(s)^{-1}$ gilt. Damit ist $\hat{\phi}$ eindeutig.
    \qedhere
		\end{description}
	\end{enumerate}
\end{proof}

\begin{defi}
	Falls $R$ ein Integritätsbereich und $R\neq\{0\}$, so ist $S = R\setminus\{0\}$ multiplikativ abgeschlossen und $S^{-1}R =: \Quot(R)$ ist nach dem Übungsblatt ein Körper. Wir nennen diesen den Quotientenkörper zu $R$.
\end{defi}
\begin{bem} Wir weisen auf die Unterscheidung zwischen $R/I$, einem Faktorring, und $S^{-1}R$, einem Quotientenring bzw. Quotientenkörper, hin.
\end{bem}
\begin{bsp}
	\leavevmode
	\begin{enumerate}
		\item $\Quot(\IR[t]) = \left\{\frac{f(t)}{g(t)}| f(t), g(t)\in \IR[t], g(t)\neq 0\right\}$, der Körper der rationalen Funktionen über $\IR$ in der Variable $t$.
		
		\item Was ist z.B. $\Quot(\IZ[t])$? Können wir diesen \glqq gut\grqq{} beschreiben? (Stroppel kündigt einen Überkill an \dots)
		
		Es sei $R$ ein Integritätsbereich, wir betrachten die Abbildung
		
		$\begin{array}{rcccccl}
			\phi\colon R & \xlongrightarrow{\can} & \Quot(R) &\longto& (\Quot(R))[t] &\longto& \Quot((\Quot(R))[t]) =: T\\
			r& \longmapsto & \frac r1 & \longmapsto & \text{konstantes Polynom } \frac r1,
		\end{array}$
		wobei wir beachten, dass mit $R'$ auch $R'[t]$ ein Integritätsbereich ist.
		
		Nach der universellen Eigenschaft des Polynomrings existiert genau ein Ringhomomorphismus
		\begin{eqnarray*}
			f = \ev_{\tilde{t}}\colon R[t] & \longto & T\\
			\sum_{i = 0}^{\infty} b_it^i & \longmapsto & \sum_{i = 0}^{\infty}\phi(b_i)\tilde{t}^i
		\end{eqnarray*}
		wobei $\tilde{t} = \frac{1_{\Quot(R)}\cdot t}{1} \in T$.
		
		Für $p(t)\neq 0$ ist klar, dass $f(p(t))$ nicht $0$, also eine Einheit ist, da $T$ ein Körper ist. Nach \cref{thm:unieig_lokali} existiert genau ein Ringhomomorphismus $\hat{f}\colon \Quot(R[t])\to T$ mit $\hat{f}\left(\frac{g_1}{g_2}\right) = f(g_1)(f(g_2))^{-1}$. Man kann zeigen, dass $\hat{f}$ ein Ringisomorphismus ist, indem man die inversen Abbildungen angibt.
		
		Also erhalten wir $\Quot(R[t])\xrightarrow{\sim} \Quot((\Quot(R))[t]) = T$, einen Isomorphismus von Ringen. Im Beispiel hier setzen wir $R = \IZ$ und es ergibt sich
		\begin{eqnarray*}
			\Quot(\IZ[t]) & \xlongrightarrow{\sim} & \Quot((\Quot(\IZ))[t])\\
			& &= \Quot(\IQ[t])\\
			&& = \text{rationale Funktionen über $\IQ$ in einer Variablen}
		\end{eqnarray*}
	\end{enumerate}
\end{bsp}

Wir definieren $K(t_1,\dots,t_n) := \Quot(K[t_1,\dots,t_n])$ für einen Körper $K$.


\lecture{20. November 2017}
\begin{bem}
	Sei $R$ ein kommutativer Ring und $S\subseteq R$ multiplikativ abgeschlossen. Falls $S$ keine Nullteiler besitzt, dann ist
	\begin{eqnarray*}
		 \can\colon R &\longto& S^{-1}R\\
		 r & \longmapsto& \frac r1
	\end{eqnarray*}
	injektiv. \emph{Begründung:} Sei $\can(r) = \can(r')$ mit $r,r'\in R$ Dann folgt $\frac r1 = \frac {r'}1$, es existiert also ein $ s\in S$ mit $ s(r1-r'1) = 0$. Da $S$ keine Nullteiler enthält, ist $r1 -r'1 = 0$, und wir erhalten $r' = r$.
	
	Insbesondere können wir $R$ als Unterring von $S^{-1}R$ vermöge $\can$ auffassen.
\end{bem}



\subsection{Satz von Gauß}
Ziel dieses Kapitels:
\begin{satz}[Satz von Gauß]\label{thm:gauss}
	Der Polynomring eines faktoriellen Ringes ist faktoriell.
	
	
	\emph{Genauer:} Sei $R$ ein faktorieller Ring. Dann ist $R[t]$ faktoriell und die irreduziblen Elemente sind bis auf Einheiten in $R$ genau die
	\begin{enumerate}[label=(\alph*)]
		\item irreduziblen Elemente in $R\subseteq R[t]$ und die \label{enumi:gauss:a}
		\item Elemente $p(t) \in R[t]$, die irreduzibel in $\Quot(R)[t]$ und primitiv \textup(siehe \cref{def:primitiv}\textup) sind. \label{enumi:gauss:b}
	\end{enumerate}
\end{satz}

\begin{kor}
	Sei $K$ ein Körper. Dann ist $K[t_1,\dots,t_n]$ faktoriell, also ist insbesondere $\IZ[t_1,\dots, t_n]$ faktoriell \textup(aber kein Hauptidealring, vgl. Übungungsblatt 5\textup)
\end{kor}

\begin{proof} $K$ und $\IZ$ sind beide Hauptidealringe, weshalb $K$ und $\IZ$ nach \cref{thm:hirfakt} also faktoriell sind. Laut \nameref{thm:gauss} sind damit $K[t_1]$ und $\IZ[t_1]$ faktoriell. Durch iterative Anwendung folgt die Aussage.
\end{proof}

\begin{bsp}
	$p(t) = 2t$ ist irreduzibel in $\IQ[t] = \Quot(\IZ)[t]$ (da für $p(t) = bc$ in $\IQ[t]$ schon $\deg(b) = 0$ oder $\deg(c) = 0$ gilt und damit $b$ oder $c$ eine Einheit in $\IQ$ ist).
	
	Aber: $p(t)$ ist nicht irreduzibel in $\IZ[t]$, da $p(t) = 2\cdot t$ und $2, t\notin \IZ[t]^{\times}$. (Einfaches Argument über den Grad der Polynome)
\end{bsp}

\begin{defi}\label{def:primitiv}
	Sei $R$ ein faktorieller Ring. Dann heißt $p(t) = \sum_{i = 0}^na_it^i$ mit $n = \deg (p(t))$ primitiv, falls kein $r\in R\setminus R^{\times}$ existiert, welche alle $a_i$ mit $1\leq i\leq n$ teilt.
\end{defi}

\begin{bsp}
	$2t^2+4t^3\in\IZ[t]$ ist nicht primitiv (man wähle $r = 2$), dafür aber in $\IQ[t]$.
\end{bsp}

\begin{lem}[Lemma von Gauß]\label{lem:gauss}
	Sei $R$ ein faktorieller Ring sowie $P,Q\in R[t]$ primitiv. Dann folgt, dass $PQ\in R[t]$ primitiv ist.	
\end{lem}

\begin{proof}
	Wir nehmen an, dass $PQ\in R[t]$ nicht primitiv ist. Dann exsitert ein $r\in R\setminus R^{\times}$ mit $r\mid PQ$. Sei ohne Beschränkung der Allgemeinheit $r$ prim. Betrachte $R/(r)$. Da $r$ prim ist, ist $(r)$ ein Primideal. Folglich ist $R/(r)$ ein Integritätsbereich, also insbesondere nullteilerfrei.
	
	Betrachte nun $\overline{PQ} \in R/(r)[t]$ (Polynome mit Koeffizienten modulo $(r)$). Dann gilt $\overline{PQ} = \overline{0}\in R/(r)[t]$, da $r$ das Polynom $PQ$ teilt. Da $\overline{PQ} = \overline{P}\,\overline{Q}$ ist, folgt damit $\overline{P}\,\overline{Q} = \overline{0}$. Da $R/(r)$ nullteilerfrei ist, ist $\overline{P} = 0$ oder $\overline{Q} = 0$ und damit $P$ nicht primitiv oder $Q$ nicht primitiv. Das ist ein Widerspruch zur Annahme.
\end{proof}

\begin{lem}\label{lem:faktquotirred}
	Sei $R$ ein faktorieller Ring, $K = \Quot (R)$ und $f\in R[t]$ primitiv. Wenn $f$ irreduzibel in $K[t]$ ist, ist $f$ auch irreduzibel in $R[t]$. 
\end{lem}

\begin{proof}
	Sei $f = bc$ in $R[t]$. Es ist zu zeigen, dass $b$ oder $c$ eine Einheit in $R[t]$ ist. Es gilt $f = bc$ auch in $K[t]$. Nach Voraussetzung ist $f$ irreduzibel in $K[t]$. Also ist $b$ oder $c\in K[t]^{\times} = K^{\times}$. Ohne Beschränkung der Allgemeinheit gelte $b\in K^{\times}$, also $b \neq 0$ (da $K$ ein Körper ist) und da nach Voraussetzung $b\in R[t]$ ist, folgt $b\in R\setminus\{0\}$. Da $R$ faktoriell ist, ist $b = \epsilon p_1\dots p_r$, wobei $\epsilon\in R^{\times}$ und die $p_i\in R$ irreduzibel für alle $1\le i \le r$ mit einem geeigneten $r\in\IN_0$ sind.
	
	Aus der Tatsache, dass $f$ primitiv ist, folgt, dass $b$ primitiv ist und somit $b = \epsilon$ (weil sonst die Primfaktoren Teiler von $b$ wären). Damit ist $b$ eine Einheit in $R$ und $f\in R[t]$ ist irreduzibel.
\end{proof}

\newcommand{\hyperhyper}[2]{\hyperref[#1]{#2}}
\begin{proof}[Beweis des \hyperhyper{thm:gauss}{Satzes von Gauß}] %TODO: herausfinden, warum hier optionale Argumente nicht funktionieren
	\leavevmode
	\begin{description}
		\item[\emph{Elemente der Form \ref{enumi:gauss:a} und \ref{enumi:gauss:b} sind irreduzibel.}] Für \ref{enumi:gauss:b} ist dies nach \cref{lem:faktquotirred} klar. Für \ref{enumi:gauss:a}: Sei $r\in R$ irreduzibel und $r = bc$ in $R[t]$. Es bleibt $b$ oder $c\in R[t]^{\times}$ zu zeigen. Da $R$ nullteilerfrei ist, gilt $\deg (b) = 0  =\deg (c)$, also $b,c\in R$. Weil $r\in R$ irreduzibel ist, folgt $b\in R^{\times}$ oder $c\in R^{\times}$.
		
		\item[\emph{Existenz einer Primfaktorzerlegung.}] Sei $K = \Quot(R)$. Da $K$ ein Körper ist, folgt mit Übungsblatt 5, dass $K[t]$ ein Hauptidealring ist, welcher nach \cref{thm:hirfakt} faktoriell ist. Sei $0\neq f\in R[t]\subseteq K[t]$, so existiert eine Primfaktorzerlegung mit $f = \epsilon p_1\dots p_r$ in $K[t]$ mit $\epsilon\in K[t]^{\times} = K^{\times}$ und irreduziblen $p_1,\dots,p_r\in K[t]$.
		
		Wir wählen für $1\leq i\leq r$ Elemente $c_i\in K$, sodass $p_i = c_i\tilde p_i$ mit primitiven $\tilde p_i\in R[t]$ ist (\glqq Hauptnenner\grqq). Ist $c := c_1\dots c_r\epsilon$, so gilt $f = c\tilde p_1\dots \tilde p_r$, wobei $\tilde p_1\dots\tilde p_r$ nach dem \nameref{lem:gauss} primitiv ist. Daraus folgt $c\in R$.
		
		Da $R$ faktoriell ist, gilt $c = \eta q_1\dots q_s$ mit $\eta\in R^{\times}$ und irreduziblen $q_1,\dots,q_2\in R$. Also ist $f = \eta q_1\dots q_s\tilde p_1\dots \tilde p_r$, wobei $q_1,\dots q_s$ Faktoren von Typ \ref{enumi:gauss:a} und $\tilde p_1,\dots, \tilde p_r$ Faktoren von Typ \ref{enumi:gauss:b} sind. Folglich existiert eine Primfaktorzerlegung
		
		\item[\emph{Typ \ref{enumi:gauss:a} und Typ \ref{enumi:gauss:b} sind genau die irreduziblen Elemente in $R[t]$ (bis auf Einheiten).}] Sei $p\in R[t]$ irreduzibel. Dann existiert ein $\epsilon \in R[t]^{\times} = R^{\times}$ und irreduzible $p_1,\dots,p_r\in R[t]$ vom Typ \ref{enumi:gauss:a} oder \ref{enumi:gauss:b}, sodass $p = \epsilon p_1\dots p_r$ (da die Primfaktorzerlegung existiert). Da $p$ irreduzibel ist, gilt $p = p_i$ für ein $i$ bis auf Einheiten.
		
		\item[\emph{Eindeutigkeit der Primfaktorzerlegung (bis auf Einheiten).}] Seien $p = \eta q_1\dots q_s\tilde {p_1}\dots \tilde{p_r}$ und $p = \eta'q_1'\dots q'_{s'}\tilde{p_1'}\dots\tilde{p'_{r'}}$ zwei Primfaktorzerlegungen wie oben.
		Weil $K[t]$ faktoriell ist, gilt dann $\eta q_1\dots q_s,\eta'q_1 '\dots q'_{s'}\in K[t]^{\times}$ und somit $r = r'$, und es existiert ein $\gamma_i\in K[t]^{\times} =K^{\times}$ für jedes $i$ mit $1\leq i \leq r$ und $\pi\in S_r$, sodass $\tilde p_{\pi(i)}' = \gamma_i\tilde{p_i}$. Da $\tilde p_i$ primitiv und $\tilde p_i'\in R[t]$ ist, gilt $\gamma_i\in R$ für alle $i$. Also existiert ein $\gamma = \gamma_1\dots \gamma_r$, sodass $\eta\gamma{q_1}\dots {q_s} = \eta'{q_1'}\dots {q_{s'}'}$ ist.
		
		Dabei sind alle Terme in $R$, und da $R$ faktoriell ist, gilt $s = s'$ und die Zerlegung ist \glqq eindeutig\grqq. Folglich ist die Zerlegung insgesamt \glqq eindeutig\grqq.
  \qedhere
	\end{description}
\end{proof}

\begin{satz}[Eisensteinsches Irreduzibilitätskriterium] \label{thm:eisenstein}
	Sei $f = \sum_{i = 1}^n a_it^i\in\IZ[t]$ primitiv sowie $\deg (f) = n>0$. Es sei $p\in \IZ$ eine Primzahl, für die Folgendes gilt: \begin{align*}
		p \nmid a_n && \text{$p \mid a_i$ für $0 \le i \le n-1$} && p^2 \nmid a_0 
	\end{align*}
	Dann ist $f$ irreduzibel in $\IZ[t]$ und somit auch in $\IQ[t]$.
\end{satz}

\begin{proof}
	Sei $f = gh$ mit $g, h\notin \IZ[t]^{\times}$, also $\deg(g)>0$, $\deg(h)>0$, da $f$ primitiv ist. Sei $g = \sum_{i = 0}^mb_it^i$, $h = \sum_{i = 0}^sc_it^i$ mit $m,s>0$, $m+s = n$. Dann gilt $a_n = b_mc_s$, also $p\nmid b_m, p\nmid c_s$, sowie $a_0 = b_0c_0$, also $p\nmid b_0$ oder $p\nmid c_0$. Ohne Beschränkung der Allgemeinheit gelte $p\mid b_0$ und $p\nmid c_0$. Sei $t$ maximal mit $p\mid b_j$ für alle $0\leq j\leq t$. Es gilt $0\leq t < m$.
	
	Nun gilt $a_{l+1} = b_0c_{l+1} +b_1c_l +b_2c_{l-1}+\dots +b_{l+1}c_0$. Dabei werden alle bis auf den letzten Summanden von $p$ geteilt und $b_{l+1}c_0$ wird nicht von $p$ geteilt. Damit teilt $p$ nicht $a_{l+1}$. Nach Voraussetzung muss also $l+1 = n$ gelten. Folglich $m = n$ und damit $s = 0$. Das ist ein Widerspruch. Damit gilt das Eisensteinsche Irreduzibilitätskriterium.
\end{proof}

\lecture{23. November 2017}

\subsection{Kreisteilungspolynome}\label{sec:kreisteil}
\begin{anw}
Sei $p$ prim. Dann ist $\Phi_p(t) := t^{p-1}+\dots +t+1$ irreduzibel in $\IZ[t]$, welches wir das $p$-te Kreisteilungspolynom nennen.
\end{anw}

\begin{proof}
	Wir betrachten
	\begin{align*}
		f = \ev_{t+1} : \IZ[t] &\longto \IZ[t] \\
		p(t) &\longmapsto p(t+1) ,
	\end{align*}
	was ein Ringhomomorphismus mit der inversen Abbildung $\ev_{t-1}$ ist. Jeder Ringhomomorphismus $\phi$ bildet Einheiten auf Einheiten ab. Falls $\phi$ ein Ringisomorphismus ist, gilt, dass für $\phi: R \to R'$ ein $x \in R$ genau dann irreduzibel ist, wenn $\phi(x) \in R$ irreduzibel ist.
	\textit{Begründung:} Sei $x \in R$ irreduzibel und $\phi(x) = b\cdot c$. Dann ist $\phi^{-1}\left( \phi (x) \right) = \phi^{-1} (b) \phi^{-1} (c) $, da $\phi^{-1}$ existiert. Also gilt $x = \phi^{-1} (b) \phi^{-1} (c) $ und damit ist $\phi^{-1}(b)$ oder $\phi^{-1}(c)$ eine Einheit. Deshalb ist bereits $b$ oder $c$ eine Einheit und somit folgt die Behauptung.
	
	Es reicht nun zu zeigen, dass $f\left(\Phi_p(t) \right) $ irreduzibel in $\IZ[t]$ ist. Es gilt, dass $\Phi_p(t)\cdot(t-1) = t^p - 1$.
	\begin{align*}
		&\Rightarrow f(\Phi_p(t))\cdot f(t-1) = f(t^p - 1) \\
		&\Rightarrow f(\Phi_p(t)) \cdot t = (t + 1)^p - 1 = t^p + \binom{p}{1}t^{p-1} + \dots + \binom{p}{p-1} t + \binom{p}{p} 1 - 1 \\
		&\Rightarrow f(\Phi_p(t)) = t^{p-1} + \binom{p}{1}t^{p-2} + \dots + \binom{p}{p-1}t^0 = a_{p-1}t^{p-1} + a_{p-2}t^{p-2} + \dots + a_0t^0
	\end{align*}
	Nun gilt $p \nmid a_{p-1}$ und $p \mid a_i$ für $0 \leq i \leq p-2$ und $p^2 \nmid a_0$. Somit sind die Kriterien für das \hyperref[thm:eisenstein]{Eisensteinsche Irreduzibilitätskriterium} erfüllt. Folglich ist $f\left( \Phi_p(t) \right) \in \IZ[t]$ irreduzibel. Damit ist auch $\Phi_p(t)$ irreduzibel.
\end{proof}

\paragraph{Anschauliche Interpretation von $\Phi_p(t)$.} Wir betrachten das Polynom $t^n-1 \in \IC[t]$ mit $n \in \IN$. Die Nullstellen sind dann genau die komplexen Zahlen $z \in \IC$ mit $z^n = 1$, also die $n$-ten Einheitswurzeln. 

%TODO: Insert Bild?

Wir erhalten also \glqq gleichverteilte\grqq\ Punkte auf dem Einheitskreis mit $(0,1) = \zeta_0$. Wir nennen $\zeta$ primitive $n$-te Einheitswurzel, falls $\zeta^n = 1$ und $\zeta^m \neq 1$ für alle $m > n$ ($m \in \IN$) gilt. Insbesondere ist dann $n = \ord (\xi)$ bezüglich der Multiplikation.

\begin{defi}
	Für $d \in \IN$ definieren wir 
	\begin{equation*}
		\Phi_d(t) = \prod_{\mathclap{\substack{\text{$z$ primitive $d$-te}\\\text{ Einheitswurzel $\neq 1$}}}} (t - z)
	\end{equation*}
	als das $d$-te Kreisteilungspolynom.
\end{defi}

\begin{bem}
	$\Phi_d(t) = \Phi_p(t)$ wie oben, falls $d$ eine Primzahl ist. 
\end{bem}

Wir wissen, dass $\Phi_p(t) \in \IZ[t]$ irreduzibel ist, falls $p$ eine Primzahl und primitiv ist. Nach dem \nameref{thm:gauss} ist deshalb auch $\Phi_p(t) \in \IQ[t] = \Quot(\IZ)[t]$ irreduzibel. Deshalb folgt nach \cref{lem:primired}, dass das von $\Phi_p(t)$ erzeuge Ideal $I := \left( \Phi_p(t)\right) $ maximal unter den Hauptidealen in $\IQ[t]$ ist. Da $\IQ[t]$ ein Hauptidealring ist (siehe Übungsblatt 5), ist $I$ auch ein maximales Ideal. Deshalb ist $\IQ[t] / I =: K_p $ ein Körper, auch der $p$-te Kreisteilungskörper genannt.

\begin{lem} \label{lem:koerhomoinj}
	Sei $K$ ein Körper und $R \neq \{0\}$ ein kommutativer Ring. Sei $\phi: K \to R$ ein Ringhomomorphismus. Dann ist $\phi$ injektiv.
\end{lem}

\begin{proof}
	Wir wissen, dass $\phi (1) = 1 \neq 0$. Sei $\ker \phi \neq \{0\}$. Dann existiert ein $a \in K$, welches im Kern liegt, sodass $1 = a^{-1}\cdot a$, was auch im Kern liegt, da $\ker \phi$ ein Ideal ist. Dann wäre aber $\phi(1) = 0$, was ein Widerspruch ist.
	
	Insbesondere können wir $K$ vermöge $\phi$ als Teilmenge von $R$ auffassen.  
\end{proof}

Deshalb können wir $\IQ$ als Teilmenge von $K_p$ vermöge der (injektiven) Einbettung \[\IQ \hookrightarrow \IQ[t] \xrightarrow{\can} \IQ[t]/(\Phi_p(t))\] auffassen. $K_p$ ist dadurch ein $\IQ$-Vektorraum (nachrechnen!).

\begin{lem} \label{lem:kreisteilkoerpervekraum}
	Sei $p$ prim. Dann ist $K_p$ ein $(p-1)$-dimensionaler $\IQ$-Vektorraum.
\end{lem}

\begin{proof}
	Wir wollen zeigen, dass
	\begin{equation*}
		B = \left\lbrace \ol{1}, \ol{t}, \ol{t}^2, \dots, \ol{t}^{p-2} \right\rbrace 
	\end{equation*}
	eine Basis ist.
	
	\begin{description}
		\item[\emph{Erzeugendensystem.}] Es reicht zu zeigen, dass $\ol{t}^k \in \Span_{\IQ}(B)$ für $k \in \IN_0$, damit $B$ ein Erzeugendensystem ist.
		\begin{enumerate}
			\item Für $0 \leq k \leq p-2$ ist das klar.
			\item Sei $k = p-1$. 
			\begin{align*}
			\ol{t}^{p-1} &= \ol{t}^{p-1}- \underbrace{\left( \ol{t}^{p-1} + \dots + \ol{t} + \ol{1} \right)}_{\ol{0} \in K_p} \\
			&= - \ol{t}^{p-2} + \dots - \ol{t} - \ol{1} \in \Span_{\IQ}(B)
			\end{align*}
			\item Sei $k > p-1$.
			\begin{align*}
			\ol{t}^k &= \ol{t} \cdot \ol{t}^{k-1} \in \ol{t} \cdot \Span_{\IQ}(B) \\
			&\subseteq \Span_{\IQ}\left\lbrace \ol{t}, \ol{t}^2, \cdots , \ol{t} ^ {p-1} \right\rbrace \\
			&\subseteq \Span_{\IQ}(B)
			\end{align*}
		\end{enumerate}
		Somit ist $B$ ein Erzeugendensystem.
		\item[\emph{Lineare Unabhängigkeit.}] Sei $\sum_{i=0}^{p-2}a_i\ol{t}^i = \ol{0}$ in $K_p$ mit $a_i \in \IO$. Wir betrachten $Q(X) := \sum_{i=0}^{p-2}a_iX^i \in \IQ[X]$. Es ist klar, dass $Q(t) \in \IQ[t]$ und $Q(t) \in K_p$. Wir wissen, dass
		\[
		\ol{0} = \sum_{i=0}^{p-2}a_i\ol{t}^i = \ol{\sum_{i=0}^{p-2}a_it^i} = \ol{Q(t)}
		\]
		Dann liegt $Q(t)$ in $I = \Phi_p(t)$ und folglich teilt $\Phi_p(t)$ $Q(t)$. Es folgt also, dass $\deg(Q(t)) \geq p-1$ oder $Q(t) = 0$, wobei ersteres ein Widerspruch zur Definition ist. Folglich gilt $Q(t) = 0$, womit alle $a_i = 0$ und $B$ ist damit linear unabhängig.
  \qedhere
	\end{description}
\end{proof}

Wir betrachten 
\begin{equation*}
	G := \left\lbrace \phi: K_p \to K_p \left| \substack{\text{$\phi$ ist Ringisomorphimus} \\ \text{und $\IQ$-linear, $\varphi|_\IQ = \id_\IQ$ }}\right. \right\rbrace .
\end{equation*}
$G$ ist eine Gruppe bezüglich der Komposition von Abbildungen. Man nennt $G$ die Galoisgruppe $\Gal(K_p \sslash \IQ)$ von $K_p$ über $\IQ$.
Es gilt, dass $\phi(\ol{1}) = \ol{1}$. Falls $\phi(\ol{t}) = z$, dann gilt $\phi\left( \ol{t} ^k \right) = z^k$, weil $\phi$ ein Ringhomomorphismus ist.
Wegen \cref{lem:kreisteilkoerpervekraum} bestimmt also $\phi(\ol t)$ die Abbildung $\phi$ bereits eindeutig, da $\phi$ $\IQ$-linear ist. Wir interessieren uns also nur für die Möglichkeiten für $z$.

\begin{lem} \label{lem:kreisteilkoerpernst}
	Sei $\phi \in G$ sowie $a_i = \IQ$ für alle (endlich vielen) $i$. Sei weiterhin
	\begin{equation*}
		P(X) = \sum_{i \geq 0} a_iX^i \in K_p[X].
	\end{equation*}
	Falls $y$ eine Nullstelle von $P(X)$ in $K_p$ ist, dann ist auch $\varphi(y)$ mit $\varphi \in G$ eine Nullstelle von $P(X)$.
\end{lem}

\begin{proof}
	Sei $y$ eine Nullstelle von $P(X)$. Dann ist $\sum_{i \geq 0} a_iy^i = 0$ in $K_p$. Folglich gilt 
	$$0 = \phi(0) = \phi\left( \sum_{i \geq 0} a_iy^i \right) = \sum_{i \geq 0} \phi(a_i) \phi(y^i) = \sum_{i \geq 0} a_i \phi(y)^i$$
	Somit ist $\phi(y)$ eine Nullstelle.
\end{proof}

Nun ist $\ol t$ eine Nullstelle von $X^p-1\in K_p[X]$, weil $X^p-1=\Phi_p(X)\cdot (X-1)$ und $\Phi_p(\ol{t})=0$ in $K_p$. Damit ist $\varphi(\ol t)$ eine Nullstelle von $X^p-1$ falls $\varphi \in G$.

Andererseits ist $\ol t^k$ für $0\le k \le p-1$ eine Nullstelle von $X^p -1$, weil $(\ol t^k)^p = (\ol t ^p)^k = \ol 1 = 1 \in K_p$. Die $\ol t^k$ für $0\le k \le p-2$ sind paarweise verschieden, da sie linear unabhängig sind.

Somit definiert die Zuordnung $\ol t \mapsto \ol t^k$ für $0\le k \le p-2$ dann $p-1$ paarweise verschiedene Elemente in $G$. Nach \cref{lem:kreisteilkoerpernst} gibt es aber höchstens $p$ Nullstellen, also $\abs{G}\le p$.

Also hat $G$ genau die Elemente gegeben durch \[ \varphi(\ol t) = \ol t^k \quad \text{für $1\le k \le p-2$}, \]
weil $\varphi(\ol t) = \ol t^0 = 1$ und $\varphi(\ol 1) = \ol 1 = 1$ unmöglich sind.

Zusammenfassend ist $G$ endlich und permutiert die Nullstellen von $X^p-1$. Wir werden zeigen:
\[
\Gal(K_p \sslash \IQ) \cong (\IZ/p\IZ)^\times
\]

\section{Galoistheorie}
\subsection{Körpererweiterungen}
\begin{defi}
	Es seien $K$ und $K'$ Körper. Dann heißt $\phi : K \to K'$ Körperhomomorphismus bzw. Körperisomorphismus, falls $\phi$ ein Ringhomomorphismus bzw. ein Ringisomorphismus ist.
\end{defi}

\begin{defi}
	Es sei $K$ ein Körper. Dann heißt $L \subseteq K$ Unterkörper, falls $L$ ein Unterring ist und für alle Elemente $x \in L$ das Inverse $x^{-1}$ bezüglich $x$ in $L$ liegt. Äquivalent dazu ist, dass $L$ ein Körper mit den von $K$ eingeschränkten Verknüpfungen ist.
\end{defi}

\begin{bem}
	Beliebige Schnitte von Unterkörpern sind Unterkörper.
\end{bem}

\begin{defi}
	Sei $K$ ein Körper sowie $N\subseteq L$ eine Teilmenge von $K$. Dann ist
	\[
	\langle N\rangle_\text{Körper} := \bigcap_{\mathclap{\substack{L \subseteq K\text{ Unterkörper}\\N\subseteq L}}} L
	\]
	der von $N$ erzeugte Unterkörper. Insbesondere heißt
	\[
	\langle \emptyset\rangle_\text{Körper} = \bigcap_{\mathclap{\substack{L \subseteq K\text{ Unter-}\\\text{körper}}}} L
	\]
	der Primkörper von $K$.
\end{defi}

\lecture{27. November 2017}

\begin{bem}
	Der Primkörper eines Körpers ist selbst ein Körper und zwar der kleinste Unterkörper von $K$. Es ist $0,1\in \langle\emptyset\rangle$ und damit auch $1+\dots+1 \in \langle \emptyset\rangle$ als die $n$-fache Addition der $1$ ($n\in\IN$).
\end{bem}

\begin{satz} Sei $K$ ein Körper. Dann ist der Primkörper von $K$ isomorph \textup(als Körper\textup) zu \[\begin{cases*} \IQ& falls $\chr K =0$,\\
	\IF_p& falls $\chr K = p > 0$.\end{cases*}\]
\end{satz}

\begin{proof}
	Betrachte $\phi\colon \IZ\to K$ mit 
	\[
		\phi(n) = \begin{cases*}
		n \cdot 1 & falls $n \in \IN$,\\
		0 & falls $n=0$,\\
		-(-n\cdot 1) & falls $-1 \in \IN$,
		\end{cases*}
	\]
	wobei klar ist, dass $\varphi$ ein Ringhomomorphismus ist und das Bild von $\varphi$ im Primkörper von $K$ liegt.
	
	\begin{description}
		\item[\emph{Fall 1: $\chr K = p > 0$.}] Dann liegt $p\cdot 1  = 0$ in $K$, es gilt also $\phi(pm) = 0$ für alle $m\in\IZ$, also folgt $p\IZ\subseteq \ker \phi$. Nun existiert nach dem \nameref{thm:homsatz_r} ein Ringhomomorphismus $\overline{\phi}\colon \IZ/p\IZ\to K$ mit $\overline{\phi}\circ \can  = \phi$. Da $p$ prim ist, ist $\IZ/p\IZ = \IF_p$ ein Körper. Nach \cref{lem:koerhomoinj} ist $\overline{\phi}$ folglich injektiv und $\im\ol\phi$ ist im Primkörper von $K$ enthalten, wobei $\im\overline{\phi} \cong \IF_p$ nach dem \nameref{thm:homsatz_r}. Weil der Primkörper der kleinste Unterkörper von $K$ und $\im\overline{\phi}$ ein Körper (isomorph zu $\IF_p$) ist, ist $\im \ol \phi$ der Primkörper von $K$.
		\item[\emph{Fall 2: $\chr K = 0$.}] Dann ist $\phi$ injektiv, denn aus $\phi(n) = \phi(m)$ folgt $(n-m)1 = 0$ in $K$, also gilt $n = m$ oder  $\chr K > 0$, wobei letzteres aber nicht der Fall ist. Also ist $\phi(n) \neq 0$ für alle $n\neq 0$, da $\phi$ ein Gruppenhomomorphismus ist. Damit gilt $\phi(n)\in K^{\times}$ für alle $n \in \IZ\setminus\{0\}$. Nach der \hyperref[thm:unieig_lokali]{universellen Eigenschaft der Lokalisierung} existiert ein Ringhomomorphismus $\hat{\phi}\colon \IQ= \Quot (\IZ)\to K, \frac ab \mapsto \phi(a)(\phi(b))^{-1}$. Mit \cref{lem:koerhomoinj} folgt dann, dass $\hat{\phi}$ injektiv ist, und weiter gilt nach dem \nameref{thm:homsatz_r} $\IQ/\ker\hat{\phi} \cong \im\hat{\phi}$, also $\IQ \cong \im \hat{\phi}$ (Isomorphie von Ringen bzw. Körpern). Wir wissen, dass $\im \phi$ im Primkörper von $K$ enthalten ist. Nach Definition von $\hat{\phi}$ liegt auch $\im\hat{\phi}$ im Primkörper von $K$, weil der Primkörper ein Körper ist. Damit haben wir nun einen Unterkörper $\im \hat{\phi}$ als Teilmenge des Primkörpers von $K$, also ist $\im \hat{\phi}$ der Primkörper, weil dieser minimal ist. Also ist $\IQ$ isomorph zum Primkörper von $K$.
  \qedhere
	\end{description}
\end{proof}

\begin{defi}
	Eine Körpererweiterung $L\sslash K$ ist ein Paar $L, K$ von Körpern, wobei $K\subseteq L$ ein Unterkörper von $L$ ist. Genauer: $L$ ist eine Körpererweiterung von $K$.	
\end{defi}

\begin{bsp}
	\leavevmode
	\begin{enumerate}
		\item $\IC\sslash\IR$ ist eine Körpererweiterung.
		\item Sei $L\sslash K$ eine Körpererweiterung und seien $\alpha_1, \dots, \alpha_n\in L$. Sei $M:=K\cup \{\alpha_1,\dots, \alpha_n\}$ Dann sind $K \subseteq \< M\>_{\text{Körper}}\subseteq L$ jeweils Körpererweiterungen. Wir bezeichnen $\<M\>_{\text{Körper}} = K(\alpha_1,\dots, \alpha_n)$ den von $K$ und $\alpha_1,\dots,\alpha_n$ erzeugten Unterkörper von $L$.
	\end{enumerate}
\end{bsp}
Beachte: $K\subseteq K[\alpha_1,\dots, \alpha_n]\subseteq K(\alpha_1,\dots, \alpha_n)\subseteq L$; wobei die erste Inklusion eine Inklusion von Ringen ist und die zweite im Allgemeinen tatsächlich echt ist. Die letzte Inklusion ist eine Inklusion von Körpern.

\begin{bsp}
	\leavevmode
	\begin{enumerate}
		\item $\IR\subseteq \IR[i] = \{a+b\ima\mid a,b\in\IR\} = \IR(\ima) = \IC$
		\item $\IR\subsetneq \IR[t]\subsetneq \IR(t) \subsetneq \IC(t)$
	\end{enumerate}
\end{bsp}

\begin{defi}
	Eine Körpererweiterung $L\sslash K$ heißt endlich erzeugt, falls $\alpha_1,\dots, \alpha_n\in L$ existieren, sodass $L = K(\alpha_1,\dots, \alpha_n)$. $L\sslash K$ heißt einfach (oder auch primitiv), falls ein $\alpha\in L$ mit $L = K(\alpha)$ existiert.
\end{defi}

\begin{defi}
	Sei $L\sslash K$ eine Körpererweiterung. Dann ist $L$ ein $K$-Vektorraum (in offensichtlicher Weise; siehe \cref{sec:kreisteil}). Wir nennen $\dim_KL = [L:K]$ den Grad der Körpererweiterung $L\sslash K$.
\end{defi}

\begin{bsp}
	$\IC\sslash\IR$: $[\IC:\IR] = 2$; $\IR(t)\sslash \IR$: $[\IR(t): \IR] = \infty$.
\end{bsp}
\begin{bem}
	Falls $L$ ein endlicher Körper sowie $L\sslash K$ eine Körpererweiterung ist, so gilt $|L| = |K|^{ [L:K]}$.
\end{bem}
\begin{satz}\label{thm:ke_grad}
	Sei $L\sslash K$ eine Körpererweiterung und $V$ ein $L$-Vektorraum. Dann ist $V$ durch Einschränkung der Verknüpfungen auch ein $K$-Vektorraum. Es gilt $\dim_KV= \dim_LV\cdot [L:K]$.
\end{satz}
\begin{kor}[Gradformel] \label{kor:gradformel}
	Seien $L_1\sslash L_2$ und $L_2\sslash L_3$ Körpererweiterungen. Dann gilt
	$$ [L_1: L_3] = [L_1:L_2]\cdot [L_2:L_3]$$
\end{kor}
\begin{proof} Folgt direkt aus \cref{thm:ke_grad}.
\end{proof}

\begin{proof}[Beweis von \cref{thm:ke_grad}]
	Sei $\{v_i\mid i\in I\}$ eine Basis von $V$ als $L$-Vektorraum und $\{w_j\mid j\in J\}$ eine Basis von $L$ als $K$-Vektorraum. Wir zeigen nun, dass $\{z_{(i,j)} = w_jv_i\mid i\in I, j\in J\}$ eine Basis von $V$ als $K$-Vektorraum ist. Damit folgt dann der Satz.
	\begin{description}
		\item[\emph{Erzeugendensystem.}] Sei $v\in V$, so existiert eine Darstellung $v = \sum_{i\in I} a_iv_i$ mit $a_i\in L$, wobei nur endliche viele $a_i\neq 0$ sind, und für festes $i\in I$ gilt außerdem
		\[a_i = \sum_{j\in J}b_{ij}w_j \text{ mit } b_{ij}\in K\]
		wobei wieder nur endlich viele $b_{ij}\neq 0$ sind. Nun gilt aber $v = \sum_{i\in I}\sum_{j\in J}b_{ij}w_jv_i$, und wir sind fertig.
		\item[\emph{Lineare Unabhängigkeit.}] Sei $\sum_{(i,j)\in I'}c_{ij}z_{(i,j)} = 0$ mit $c_{ij}\in K$ und endlichem $I'\subseteq I \times J$. Es ist zu zeigen, dass $c_{ij} = 0$ für alle $(i,j)\in I'$ gilt.
		
		Wir setzen $c_{ij} = 0$ für alle $(i,j)\in (I\times J) \setminus I'$. Nun folgt \[0 = \sum_{\mathclap{(i,j)\in I'}}c_{ij}z_{(i,j)} = \sum_{i\in I}\left(\sum_{j\in J}(c_{ij}w_j)\right)v_i,\]also $c_{ij}w_j\in L$, weil $K\subseteq L$ ein Unterkörper ist. Da $\{v_i\mid i\in I\}$ eine Basis von $V$ als $L$-Vektorraum ist, folgt $\sum_{j\in J}c_{ij}w_j = 0$ für alle $i \in I$. Da weiterhin $\{w_j\mid j\in J\}$ linear unabhängig über $K$ ist, gilt $c_{ij} = 0$ für alle $i \in I$ und $j \in J$.
  \qedhere
	\end{description}
\end{proof}

\subsection{Algebraische Körpererweiterung}
Sei $L\sslash K$ eine Körpererweiterung sowie $a\in L$. Die Inklusion $K\to L, \lambda\mapsto\lambda$ ist ein Ringhomomorphismus. Nach der \hyperref[thm:unieig_polyring]{universellen Eigenschaft des Polynomrings} existiert genau ein Ringhomomorphismus
\begin{eqnarray*}
	\ev_a\colon K[t] &\longto & L\\
	p(t) & \longmapsto & p(a).
\end{eqnarray*}
Wir nennen $a$ transzendent über $K$, falls $\ev_a$ injektiv ist, und algebraisch anderenfalls. Betrachte nun diese beiden Fälle.
\begin{description}
	\item[\emph{Fall 1: $a$ ist transzendent.}] Dann ist $\ev_a\colon  K[t]\to L$ injektiv und nach \cref{thm:evalpolyring} gilt $\im\ev_a = K[a]\subseteq L$.
	Also existiert ein Isomorphismus von Ringen $K[t]\to K[a]$ nach dem \nameref{thm:homsatz_r}. Nach Definition ist $K[a]\subseteq K(a)$ und somit gilt
	\[K(a) \supseteq \{fg^{-1}\mid f,g\in K[a], g\neq 0\} =: X.\]
	Da $K(a)$ der kleinste Unterkörper von $L$ ist, der $K$ und $a$ enthält und $X$ offensichtlich ein Körper ist, folgt Gleichheit.
	
	Folglich erhalten wir einen Isomorphismus von Ringen (bzw. Körpern) von $K(a)\cong K(t) = \Quot(K[t])$. Insbesondere sind $K(t)$, also auch $K(a)$ unendlichdimensionale als $K$-Vektorräume, also $[K(a):K] = \infty$.
	
	\item[\emph{Fall 2: $a$ ist algebraisch über $K$.}] Nach Definition ist $\ev_a$ nicht injektiv. Also ist $\ker(\ev_a)\neq \{0\}$ ein Ideal in $K[t]$. Da $K[t]$ ein Hauptidealring ist, existiert ein $p(t)\in K[t]$ mit $\ker(\ev_a) = (p(t))$. Sei ohne Beschränkung der Allgemeinheit $p(t)$ normiert (d.h. der Leitkoeffizient ist $1$).
	
	Wir wissen, dass $p(a) = \ev_a(p(t)) = 0$, also dass $a$ eine Nullstelle von $p(t)$ ist. Wähle nun ein normiertes Polynom minimalen Grades in $K[t]$, genannt $m_a(t)$, sodass $m_a(a) = 0$, also sodass eine Nullstelle ist. Wir nennen $m_a$ das Minimalpolynom zu $a$ (Existenz und Eindeutigkeit zeigt man wie für das Minimalpolynom in LA II). Ebenfalls wie in LA II zeigt man, dass $m_a(t)$ jedes Polynom $p(t)\in K[t]$ mit $p(a)=0$ teilt. Damit folgt, dass $\ker\ev_a = (m_a(t))$ ist.

	Vermöge des \nameref{thm:homsatz_r} erhalten wir schließlich einen Ringhomomorphismus $\overline{\ev_a}\colon K[t]/(m_a(t))\to \im(\ev_a)\subset L$.
	
\lecture{30. November 2017}
	
	Dabei ist $\im(\ev_a)\subseteq L$ und $L$ ein Körper und $\im(\ev_a)$ ein Ring. Folglich sind $\im\ev_a$ und $K[t]/(p(t))$ Integritätsbereiche. Damit ist $(p(t))$ ein Primideal und $p(t)$ ist prim. Da $K[t]$ ein Hauptidealring ist, ist $p(t)$ irreduzibel und $(p(t))$ maximal unter allen Hauptidealen; folglich ist $(p(t))$ maximal und $K[t]/(p(t))$ ein Körper.
	
	Da $\overline{\ev_a}\colon K[t]/(p(t))\to \im\ev_a = K[a]\subseteq L$ ein Ringhomomorphismus und auf einem Körper definiert ist, ist $\overline{\ev_a}$ injektiv und $K[a]$ somit ein Körper. Also gilt $K[a]\subseteq K(a) \subseteq L$. Da $K(a)$ der kleinste Körper ist, der $K$ und $a$ enthält, folgt $K[a] = K(a)$.
	
	\emph{Behauptung:} $[K(a):K] = d := \deg(m_a(t))$. Genauer: $1, a, a^2,\dots, a^{d-1}$ ist eine Basis von $K(a)$ als $K$-Vektorraum.
	\begin{proof}
		\leavevmode
		\begin{description}
			\item[\emph{Erzeugendensystem.}] Sei $k\geq 0$. Dann $t^{d+k} = t^km_a(t)+Q$ in $K[t]$, wobei $Q$ eine Linearkombination von Polynomen mit Grad kleiner als $d+k$ ist. Dann gilt in $K[t]/(p(t)) = K[t]/(m_a(t))$.
			\[\overline{t^{d+k}} = \overline{t^k}\overline{m_a(t)} + \overline{Q} = \overline{Q}.\]
			Damit folgt $\overline{t^{d+k}} \in \< \{1, \overline{t}, \dots, \overline{t^{d-1}}\}\>$. Folglich erzeugt $B :=  \{1, \overline{t}, \dots, \overline{t^{d-1}}\}$ dann $K[t]/(p(t))$ als $K$-Vektorraum. Wende nun die Evaluationsabbildung an:
			$\overline{\ev_a}(B)$ erzeugt $\im\overline{\ev_a} = K(a)$ als $K$-VR, da $\overline{\ev_a}$ insbesondere ein $K$-Vektorraum-Isomorphismus ist. Da $\overline{\ev_a}(B) = \{1, a, a^2,\dots, a^{d-1}\}$, folgt die Aussage.
			\item{\emph{Lineare Unabhängigkeit.}} Sei $\sum_{i =0}^{d-1}c_ia^i = 0$ (in $K(a)$) mit $c_i\in K$. Wir nehmen an, dass ein $i$ mit $c_i\neq 0$ existiert. Wähle $m$ maximal mit $c_m \neq 0$. Dann ist $\sum_{i = 0}^{m}c_ia^i = 0$. Sei ohne Beschränkung der Allgemeinheit $c_m = 1$. Dann wissen wir
			\[0 = \sum_{i = 0}^m c_ia^i = \overline{\ev_a}\left(\sum_{i=0}^mc_i\overline{t}^i\right)\Longrightarrow \ev_a\left(\sum_{i=0}^m c_it^i\right) = 0.\]
			Folglich hat $\sum_{i=0}^mc_it^i\in K[t]$ als Nullstelle $a$ und den Grad $m <d$ im Widerspruch zu Definition von $m_a(t)$, weil $m_a(t)$ das normierte Polynom kleinsten Grades mit $a$ als Nullstelle ist.
    \qedhere
		\end{description}
	\end{proof}
\end{description}

\begin{bsp}
	Betrachte $\IC\sslash\IR$ und $\ima\in\IC$. Dann ist $t^2+1\in \IR[t]$ das Minimalpolynom $m_i(t)$ zu $\ima$. Es gilt folglich $[\IR(i):\IR] = 2$.
\end{bsp}

\begin{satz}\label{thm:algebraisch_aequivalenzen}
	Sei $L\sslash K$ eine Körpererweiterung und $a\in L$. Dann sind äquivalent:
	\begin{enumerate}
		\item $a$ ist algebraisch über $K$.
		\item $K[a] = K(a) \subseteq L$
		\item $\dim_KK(a) = d = \deg(m_a(t))$.
	\end{enumerate}
	Im letzten Fall nennen wir $ d = [K(a):K]$ den Grad von $a$ über $K$.
\end{satz}
\begin{proof}
	siehe oben
\end{proof}
\begin{defi}
	Eine Körpererweiterung $L\sslash K$ heißt algebraisch, falls jedes $a\in L$ algebraisch über $K$ ist. Weiter heißt $L\sslash K$ endlich, wenn $L$ als $K$-Vektorraum endliche Dimension hat, also der Grad $[L:K]$ der Körpererweiterung endlich ist.
\end{defi}	
	
\begin{satz}\label{thm:algebraischekes}
	\leavevmode
	\begin{enumerate}
		\item Jede endliche Körpererweiterung ist algebraisch.\label{enumi:endlich=>algebraisch}
		\item Falls $L\sslash K$ algebraisch und endlich erzeugt ist, ist $L\sslash K$ endlich.\label{enumi:alg+endlicherz=>endlich}
		\item Seien $L_1\sslash L_2$ und $L_2\sslash L_3$ beide algebraisch. Dann ist auch $L_1\sslash L_3$ algebraisch \textup(Transitivität\textup).
	\end{enumerate}
\end{satz}
\begin{proof}
	\leavevmode
	\begin{enumerate}
		\item Sei $a\in L$. Dann existiert ein $m\in \IN$, sodass $1, a, a^2, \dots, a^m$ linear abhängig über $K$ ist, da nach Voraussetzung $\dim_KL<\infty$. Folglich existieren $c_i\in K$ ($0\leq i \leq m$), die nicht alle $0$ sind, mit
		$\sum_{i=0}^mc_ia^i = 0$. Also ist $\ev_a\colon K[t]\longto L$ nicht injektiv und $a$ somit algebraisch über $K$.
		\item Sei $L$ algebraisch und endlich erzeugt über $K$. Dann existieren Elemente $a_1,\dots, a_n\in L$ mit $L = K(a_1,\dots, a_n)$. Betrachte \[K \subseteq K(a_1) \subseteq K(a_1)(a_2) = K(a_1,a_2)\subseteq \dots \subseteq K(a_1,\dots, a_n) = L.\] Nach der \nameref{kor:gradformel} gilt $$[L:K] = [K(a_1,\dots, a_n):K] = \prod_{i = 1}^{n}[K(a_1, \dots, a_i):K(a_1,\dots, a_{i-1})]$$
		Nach Voraussetzung ist $a_i$ algebraisch über $K$, also insbesondere algebraisch über $K(a_1,\dots, a_{i-1})$, also ist $[K(a_1,\dots, a_i): K(a_1,\dots , a_{i-1})] < \infty$ nach \cref{thm:algebraisch_aequivalenzen}. Schließlich gilt $[L:K]< \infty$.
		\item Ist $a\in L_1$, so existiert ein $p(t)\in L_2[t]$ mit $p(t)\neq 0$ und $p(a) = 0$, da $L_1\sslash L_2$ algebraisch ist. Sei $p(t) = \sum_{i=0}^nb_it^i$. Betrachte den Unterkörper $K = L_3(b_0,b_1,\dots, b_n)$ von $L_2$. Dann ist $a$ offensichtlich algebraisch über $K$, also folgt $[K(a):K] \leq \deg(p(t)) <\infty$ nach \ref{enumi:alg+endlicherz=>endlich}; $K(a)\sslash K$ ist somit endlich. Andererseits ist $L_2\sslash L_3$ algebraisch, also ist erst recht $K\sslash L_3$ algebraisch.
		
		Nach Konstruktion ist $K\sslash L_3$ endlich erzeugt; mit \ref{enumi:alg+endlicherz=>endlich} folgt, dass $K\sslash L_3$ endlich ist. Mit der \nameref{kor:gradformel} erhalten wir $[K(a):L_3] = [K(a):K][K:L_3]$. Somit ist $[K(a):L_3] <\infty$, also ist $a$ algebraisch über $L_3$ nach \ref{enumi:endlich=>algebraisch}. Somit ist $L_1 \sslash L_3$ algebraisch.
  \qedhere
	\end{enumerate}
\end{proof}

\begin{lem}\label{lem:alguk}
	Seien $L\sslash K$ eine Körpererweiterung sowie $a,b\in L$ algebraisch über $K$. Dann sind $a+b$, $a\cdot b$, $a-b$ und $ab^{-1}$ (falls $b\neq 0$) auch algebraisch über $K$.
\end{lem}
\begin{proof}
	Betrachte $K(a,b) = K(a)(b)$. Dann $[K(a,b):K] = [K(a)(b):K(a)][K(a):K]$, wobei beide Faktoren endlich sind (der zweite ist endlich, weil $a$ algebraisch über $K$ ist, der erste ist endlich, weil $b$ algebraisch über $K$, also insbesondere über $K(a)$ ist). Folglich ist $K(a,b)\sslash K$ endlich und damit nach \cref{thm:algebraisch_aequivalenzen} algebraisch. Aber $K(a,b)$ enthält $a+b$, $a-b$, $ab$ und $ab^{-1}$ (falls $b\neq 0$) und somit sind diese Elemente algebraisch über $K$.
\end{proof}

\begin{bem}
	Insbesondere bilden die algebraischen Elemente über $K$ (in $L$) selbst einen Körper.
\end{bem}
	
\begin{bsp}
	\leavevmode
	\begin{enumerate}
	\item Sei $p$ eine Primzahl und $n \in\IN\setminus\{0\}$. Dann ist $f(t) = t^n-p\in \IQ[t]$ irreduzibel (nach dem \hyperref[thm:eisenstein]{Eisensteinschen Irreduzibilitätskriterium} irreduzibel in $\IZ[t]$ und primitiv, also irreduzibel in $\IQ[t]$ nach dem \nameref{thm:gauss}).
	Sei nun $a = \sqrt[n]{p}\in\IC$ eine Nullstelle von $f(t)$. Behauptung: $[\IQ(\sqrt[n]{p}): \IQ] = n$. Denn: Nach Definition des Minimalpolynoms $m_a(t)$ gilt $m_a(t)\mid f(t)$, also $f(t) = m_a(t)q(t)$ für ein $q(t)\in\IQ[t]$. Da $f(t)$ irreduzibel ist, folgt daraus $m_a(t)$ oder $q(t)\in \IQ[t]^{\times} = \IQ^{\times}$. Aber $m_a(t)$ kann keine Einheit sein; also $f(t) = m_a(t)\cdot \epsilon$ mit $\epsilon\in \IQ^{\times}$. Da $f(t)$ aber normiert ist, folgt $\epsilon = 1$. Mit \cref{thm:algebraischekes} folgt die Behauptung.
	\item\label{enumi:IQalgkoerp} Betrachte $\IC\sslash \IQ$. Sei $\IQ^{\text{alg}} = \{a\in\IC\mid a\text{ algebraisch über }\IQ\}$. Nach \cref{lem:alguk} ist $\IQ\subseteq\IQ^{\text{alg}}\subseteq \IC$ eine Körpererweiterung, nämliche der größte Unterkörper von $\IC$, der algebraisch über $\IQ$ ist.
	
	Aber $\IQ^{\text{alg}}$ ist nicht endlich über $\IQ$. Angenommen, $\IQ^{\text{alg}}\sslash \IQ$ wäre endlich, also $[\IQ^{\text{alg}}:\IQ] = m<\infty$. Wähle nun $n>m$. Nach dem ersten Beispiel ist $\sqrt[n]p \in \IQ^{\text{alg}}$ (mit $p$ wie oben) und $[\IQ(\sqrt[n]p) = \IQ] = n > m$. Aber dann folgt $\IQ(\sqrt[n]p)\subseteq \IQ^{\text{alg}}$ und damit $[\IQ^{\text{alg}}:\IQ]\geq n$.
	\end{enumerate}
\end{bsp}

Wir haben nun folgende Frage: Mit einem Körper $K$ und $p(t)\in K[t]$, existiert eine Körpererweiterung $L\sslash K$, sodass $p(t)$ eine Nullstelle in $L$ hat?

\begin{satz} \label{thm:15.4}
	Sei $K$ ein Körper und $f(t)\in K[t]$ irreduzibel. Dann existiert eine algebraische Körpererweiterung $L\sslash K$ mit $[L:K]  = d = \deg(f(t))$, sodass $f(t)$ eine Nullstelle in $L$ hat.
\end{satz}
\begin{proof}
	Da $f(t)$ irreduzibel ist, ist $(f(t))$ ein maximales Ideal in $K[t]$ und $K[t]/(f(t))$ somit ein Körper. Betrachte den Ringhomomorphismus 
	\begin{eqnarray*}
	\phi\colon K&\longto& K[t]\longto K[t]/(p(t)) = : L\\
	\lambda &\longmapsto & \text{konstantes Polynom $\lambda$},
	\end{eqnarray*}
	welcher injektiv ist, da $K$ ein Körper ist; wir können $K$ also als Unterkörper von $L$ auffassen. Es gilt $L = K(\overline{t})$ mit $\overline{t}  = \can (t)$. Setzt man $a: = \overline{t}$, so folgt $f(a) = f(\overline{t}) = \overline{f(t)} = \overline{0}\in L$. Es ist $a\in L$ also eine Nullstelle von $f(t)$. Da $f(t)$ irreduzibel ist, ist $f(t) = m_a(t)\epsilon$ für ein geeignetes $\epsilon \in K^{\times}$ (siehe oben). Also gilt $[L:K] = [K(a):K] = \deg(m_a(t)) = \deg(f(t)) = d$.
\end{proof}	



\lecture{4. Dezember 2017}
\begin{bem}
	Sei $f\in K[t]$ nicht notwendig irreduzibel. Dann können wir \cref{thm:15.4} auf irreduzible Faktoren von $f$, etwa $g\in K[t]$, anwenden. Wir schreiben $f = g\cdot h$ mit $h \in K[t]$. Dann existiert eine Körpererweiterung $L\sslash K$, sodass $g$ in $L$ eine Nullstelle hat, und damit auch $f$. Weiterhin gilt $[L:K] = \deg(g)\leq \deg (f)$.
\end{bem}
\begin{bsp} $K = \IR$, $f= t^2-1\in \IR[t]$ nicht irreduzibel, da $f = (t-1)(t+1) = gh$ mit $g = t-1$, $h = t+1\in\IR[t]$. Sei also $L = K = \IR$, und wir erhalten $[L:K] = 1 = \deg(g)<2 = \deg(f)$.

Im Gegensatz dazu sei $f = t^2+1$ (irreduzibel in $\IR[t]$). Mit $L =\IC$ erhält man $[L:K] = [\IC:\IR] = 2 = \deg(f)$.
\end{bsp}
	
\begin{satz}\label{thm:algab}
	Ein Körper $K$ heißt algebraisch abgeschlossen, falls eine der folgenden äquivalenten Aussagen gilt:
	\begin{enumerate}
		\item Jedes Polynom $f\in K[t]\setminus K$ hat eine Nullstelle in $K$. \label{enumi:algab:1}
		\item Jedes Polynom $f\in K[t]\setminus K$ ist Produkt von Polynomen von Grad $1$. \label{enumi:algab:2}
		\item Die normierten irreduziblen Polynome in $K[t]$ sind genau die Elemente der Form $t-a$ mit $a\in K$. \label{enumi:algab:3}
		\item Falls $L\sslash K$ eine algebraische Körpererweiterung ist, dann gilt schon $L = K$. \label{enumi:algab:4}
	\end{enumerate}
\end{satz}	

\begin{bsp}
	\leavevmode
	\begin{itemize}
		\item $\IR$ ist nicht algebraisch abgeschlossen, weil zum Beispiel $t^2+1$ keine Nullstelle in $\IR$ hat.
		\item $\IC$ ist algebraisch abgeschlossen (Fundamentalsatz der Algebra).
	\end{itemize}
\end{bsp}

\begin{proof}[Beweis von \cref{thm:algab}]
	\leavevmode
	\begin{description}
		\item[\glqq$1\Rightarrow2$\grqq:] Sei $f \in K[t]\setminus K$. Nach Annahme existiert eine Nullstelle $a \in K$. Wir zeigen Aussage \ref{enumi:algab:2} mit vollständiger Induktion.
		\begin{itemize}
			\item Der Induktionsanfang $\deg(f) = 1$ ist klar.
			\item $\deg(f)\geq 2$: Wir nehmen an, dass die Behauptung für alle Polynome kleineren Grades stimmt. Da eine $a$ Nullstelle von $f$ ist, teilt $t-a$ das Polynom $f$ nach Satz \ref{thm:nullstelle_eigenschaft}.
			Folglich gilt $f = (t-a)g$ für ein $g\in K[t]$. Es gilt $\deg(g)<\deg(f)$, da $K$ ein Integritätsbereich ist. Somit ist $g$ und damit auch $f$ ein Produkt von Linearfaktoren.
		\end{itemize}
		\item[\glqq$2\Rightarrow 3$\grqq:] Offensichtlich ist $t-a\in K[t]$ normiert und irreduzibel (aus Gradgründen). Sei $f\in K[t]$ normiert und irreduzibel.
		\begin{itemize}
			\item Falls $\deg(f) \in\{0,-\infty\}$, so folgt $f=0$ oder $f\in K^{\times} = K[t]^{\times}$; damit ist $f$ aber nicht irreduzibel, ein Widerspruch.
			\item Wenn $\deg(f) = 1$, ist $f = t-a$ für ein $a\in K$, weil $f$ normiert ist.
			\item Sei $\deg (f)\geq 2$. Nach \ref{enumi:algab:2} gilt dann $f = (t-a)g$ für ein $a\in K$ und $g\in K[t]$, wobei $g$ auch normiert ist. Da $(t-a)$ und $g$ aber aus Gradgründen keine Einheiten sind, steht dies im Widerspruch dazu, dass $f$ irreduzibel ist.
		\end{itemize}
		\item[\glqq$3\Rightarrow 4$\grqq:] Sei $L\sslash K$ eine algebraische Körpererweiterung und $b\in L$. Dann ist $b$ algebraisch über $K$, es existiert also ein $f\in K[t]\setminus K$ mit $f(b) = 0$. Folglich existiert ein Minimalpolynom $m_b(t)\in K[t]$ mit $m_b(b)= 0$. Nach der Definition des Normalpolynoms ist $m_b(t)$ normiert und irreduzibel. Mit \ref{enumi:algab:3} folgt, dass $m_b(t) = t-a$ für ein $a\in K$. Da $b$ eine Nullstelle von $m_b(t)$ ist, erhalten wir $b = a$ und damit $b\in K$. Somit ist bereits $L = K$.
		\item[\glqq$4\Rightarrow1$\grqq:] Sei $f\in K[t]\setminus K$. Nach Satz \ref{thm:15.4} existiert eine algebraische Körpererweiterung $L\sslash K$, sodass $f$ in $L$ eine Nullstelle hat. Aber nach Voraussetzung ist $L = K$. 
  \qedhere
	\end{description}
\end{proof}
	
\subsection{Algebraischer Abschluss}
\begin{satz}\label{thm:alg_ab}
	Sei $K$ ein Körper. Dann existiert ein algebraisch abgeschlossener Körper $L$ mit $K\subseteq L$.
\end{satz}

Zur Erinnerung: Für einen kommutativen Ring $R$ und eine Teilmenge $N\subseteq R$ bezeichnet $(N)$ das von $N$ erzeugte Ideal, also das kleinste Ideal in $R$, das $N$ enthält. Es gilt \[(N) = \left\{\left.\sum_{i = 1}^mr_in_i\right|m\in\IN, n_i\in N, r_i\in R \;\forall i\right\}.\]

\begin{proof}[Beweis von \cref{thm:alg_ab}]
	\leavevmode
	\begin{description}
		\item[\emph{1. Schritt:}] Wir \glqq bauen\grqq\ eine Körpererweiterung $L_1\sslash K$, sodass jedes $f\in K[t]$ mit $\deg(f)\geq 1$ eine Nullstelle in $L_1$ hat. Sei 
		\[J = K[t]\setminus K = \{\text{Polynome vom Grad $\geq 1$}\}\]
		Betrachte $R = K[X_f\mid f\in J]$ (siehe \href{http://www.math.uni-bonn.de/people/palmer/EIDA-Blatt-09.pdf}{Übungsblatt 9} für eine saubere Definition).
		Das ist ein Polynomring über $K$ in unendlich vielen Variablen.
		
		Sei $I := (f(X_f)\mid f\in J)\subseteq R$ nach der \hyperref[thm:unieig_polyring]{universellen Eigenschaft von $K[t]$}. Es ist klar, dass $I$ ein Ideal ist; wir behaupten, dass $I$ sogar ein echtes Ideal ist. Der Beweis hierfür folgt weiter unten.
		
		Nach \cref{thm:max_ideal_ex} existiert ein maximales Ideal $m\subseteq R$ mit $R\supseteq m \supseteq I$. Damit ist $R/m = L_1 $ ein Körper. Offenbar gilt $K \subseteq R$ und sogar $K \subseteq R/m$ (durch die Abbildung $K\to R\to R/m$). Somit haben wir eine Körpererweiterung $L_1\sslash K$. Sei nun $f\in J = K[t]\setminus K$ mit $f = \sum_{i = 0}^{\infty} b_i t^i$ und fast allen $b_i = 0$. Es ist zu zeigen, dass $f$ eine Nullstelle in $L_1$ hat.
		
		Wir wissen $f(\can(X_f)) = \sum_{i = 0}^{\infty}b_i\ol{X_f}^i$, wobei $b_i\in K$ ist. Also gilt \[f(\can(X_f)) = \sum_{i = 0}^{\infty}\ol{b_i}{X_f}^i = \ol{\sum_{i = 0}^{\infty}g_iX_f^i} = \can(f(X_f)),\] wobei $f(X_f)\in I\subseteq m$, also $f(\can(X_f)) = 0$. Also ist $\can(X_f)\in L_1$ eine Nullstelle von $f(t)$.
		
		\item[\emph{2. Schritt:}] Betrachte den \glqq Turm\grqq\ von Körpererweiterungen $K := L_0\subseteq L_1\subseteq L_2\subseteq\dots$, sodass $L_{i+1}\sslash L_{i}$ eine Körpererweiterung ist und jedes $f\in L_i[t]\setminus L_i$ eine Nullstelle in $L_{i+1}$ hat (existiert nach Schritt 1). Setze
		\[L : = \bigcup\limits_{i\geq 0} L_i, \]
		was natürlich ein Körper ist. Es ist nun zu zeigen, dass es für jedes $g\in L[t]\setminus L$ eine Nullstelle in $L$ gibt.
		
		Sei dafür $g = \sum_{i = 0}^{\infty}a_it^i$ mit nur endlich vielen $a_i\neq 0$. Folglich gibt es ein $n_0\in\IN$ mit $a_i\in L_{n_0}$ für alle $i$. Das heißt aber, dass wir $g$ als Polynom in $L_{n_0}[t]$ auffassen können. Dadurch hat $g(t)$ eine Nullstelle in $L_{n_0+1}\subseteq L$.
		
		Somit gibt es einen Körper $L$ mit $K\subseteq L$.
		\item[\emph{Beweis von $I\neq R$:}] Sei $I = R$, also liegt $1\in I$. Somit existieren gewisse $g_i\in R$ und $f_i\in J$ mit $1 = \sum_{i =1}^{n}g_if(X_{f_i})$. Nach wiederholter Anwendung von \cref{thm:15.4} existiert eine Körpererweiterung $L'\sslash K$, wobei $f_1,\dots, f_m$ Nullstellen in $L'$ haben, sagen wir $a_1,\dots, a_n$. Nach der \hyperref[thm:unieig_polyring]{universellen Eigenschaft für Polynomringe} existiert ein Ringhomomorphismus
		\begin{eqnarray*}
			\ev\colon R & \longto & L'[X_f\mid f\in J]
		\end{eqnarray*}
		mit \[\ev(X_f) = \begin{cases*} a_i & falls $f=f_i$ für ein $i$\\
		X_f & sonst
		\end{cases*}\]							
		und $\ev(\lambda)  = \lambda\in K\subseteq L'$ für $\lambda\in K$. Dann gilt $\ev(f_i(X_{f_i})) = f_i(a_i) = 0$, weil $a_i$ eine Nullstelle von $f_i$ ist. Also gilt für $1\in R$ 
		\[1 = \ev(1) = \ev\left(\sum_{i = 0}^{m}g_if_i(x_{f_i})\right) = \sum_{i = 0}^m\ev(g_i)\ev(f_i(X_{f_i})) = 0\]
		Folglich gilt $1 = 0$ in $L'[X_f\mid f\in J]$, was ein Widerspruch ist, also ist $1\notin I$.
  \qedhere
	\end{description}
\end{proof}
	
\lecture{7. Dezember 2017}	

\begin{satz}\label{thm:algabschexis}
	Sei $K$ ein Körper. Dann existiert eine algebraische Körpererweiterung $L\sslash K$, wobei $L$ algebraisch abgeschlossen ist.
\end{satz}
\begin{proof}
	Nach obigem Satz existiert ein Körper $L'$, sodass $K\subseteq L'$ ein Unterkörper und $L'$ algebraisch abgeschlossen ist. Betrachte
	\[K^{\text{alg}} = \{a\in L' \mid a\text{ algebraisch über } K\},\]
	was nach einem analogen Beweis wie \hyperref[enumi:IQalgkoerp]{hier} ein Körper ist.
	
	Nun ist $K\subseteq K^{\text{alg}}$, denn für $b\in K$ ist $b$ eine Nullstelle von $f = t-b\in K[t]$. Also sind $K\subseteq K^{\text{alg}}\subseteq L'$ jeweils Körpererweiterungen.
	
	Außerdem ist $K^{\text{alg}}\sslash K$ algebraisch, was nach Definition klar ist, da jedes Element $a\in K^{\text{alg}}$ algebraisch über $K$ ist.
	
	Wir zeigen schließlich, dass $K^{\text{alg}}$ algebraisch abgeschlossen ist. Sei nämlich $f\in K^{\text{alg}}[t]\setminus K^{\text{alg}}$. Es ist zu zeigen, dass $f$ eine Nullstelle in $K^{\text{alg}}$ hat. Wir wissen dabei, dass $K^{\text{alg}}\subseteq L'$ und $L'$ algebraisch abgeschlossen ist. Folglich hat $f$ eine Nullstelle $a\in L'$, also ist $a$ algebraisch über $K^{\text{alg}}$. Mit \cref{thm:algebraisch_aequivalenzen} folgt, dass $K^{\text{alg}}(a)\sslash K^{\text{alg}}$ eine algebraische Körpererweiterung ist. Aus der Transitivität von algebraischen Körpererweiterungen (\cref{thm:algebraischekes}) und der Algebraizität von $K^{\text{alg}}\sslash K$ folgt, dass $K^{\text{alg}}(a)\sslash K$  eine algebraische Körpererweiterung ist. Folglich ist jedes Element in $K^{\text{alg}}(a)$ algebraisch über $K$; insbesondere ist $a$ algebraisch über $K$, also $a\in K^{\text{alg}}$.
	
	Setze nun $L:= K^{\text{alg}}$.
\end{proof}

\begin{defi}
	Wir nennen den vermöge \cref{thm:algabschexis} existierenden Körper $L$ algebraischen Abschluss von $K$ und bezeichnen ihn oft mit $\overline{K}$.
\end{defi}
Wie sieht es nun mit der Eindeutigkeit von $\overline{K}$ aus (bis auf Isomorphie)?
\begin{defi}
	Sei $K$ ein Körper sowie $L_1\sslash K$ und $L_2, \sslash K$ Körpererweiterungen. Eine Abbildung $\phi\colon L_1\to L_2$ heißt $K$-Homomorphismus, falls $\phi$ ein Ring- und damit ein Körperhomomorphismus und $\phi|_K = \id_K$ ist. $\phi$ heißt $K$-Isomorphismus, falls zusätzlich $\phi$ bijektiv ist. Falls $L_1 = L_2$ und $\phi\colon L_1\to L_2 = L_1$ ein $K$-Isomorphismus ist, dann nennen wir $\phi$ auch $K$-Automorphismus.
\end{defi}

\begin{bsp}
	$L_1 = L_2$, $\phi = \id_{L_1}$ ist ein $K$-Automorphismus.
\end{bsp}

\begin{defi}
	Die Menge $\Aut(L\sslash K) = \{\phi\colon L\to L\mid \phi\ K\text{-Automorphismus}\}$ ist eine Gruppe bezüglich der Komposition von Abbildungen (für jede Körpererweiterung $L\sslash K$). Wir nennen sie die Automorphismengruppe von $L\sslash K$.
\end{defi}

\begin{bsp}
	Für $\IC\sslash \IR$ gilt $\Aut(\IC\sslash\IR) =\{\id_\IC, z\mapsto \overline{z}\}$.
\end{bsp}
\begin{proof}
	\leavevmode
	\begin{description}
		\item[\glqq$\supseteq$\grqq:] Klar.
		\item[\glqq$\subseteq$\grqq:] Sei $\phi\in \Aut(\IC\sslash\IR)$. Dann ist $\phi(a+b\ima) =\phi(a)+\phi(b)\phi(\ima) = a+b\phi(\ima)$, wobei $a,b\in\IR$. Also ist $\phi$ durch durch $\phi(\ima)$ bereits eindeutig bestimmt. Es gilt $-1 = \phi(-1) = \phi(\ima^2) = \phi(\ima)^2$. Daraus folgt, dass $\phi(\ima)$ eine Quadratwurzel von $-1$ sein muss, also $\phi(\ima)=\ima$ (also $\phi = \id_\IC$) oder $\phi(\ima) = -\ima$ (also ist $\phi$ die komplexe Konjugation).
  \qedhere
	\end{description}
\end{proof}

\noindent
Unser Ziel ist nun:
\begin{satz}[Eindeutigkeit des algebraischen Abschlusses]\label{thm:16.3}
	Sei $K$ ein Körper sowie $L_1\sslash K$ und $L_2\sslash K$ Körpererweiterungen, sodass $L_1,L_2$ algebraische Abschlüsse von $K$ sind. Dann existiert ein $K$-Isomorphismus $\phi\colon L_1\to L_2$.
\end{satz}

\begin{bem}
	Sei $\phi\colon L_1\to L_2$ ein $K$-Homomorphismus und $[L_1:K] = [L_2:K]<\infty$. Dann ist $\phi$ ein $K$-Isomorphismus, da $\phi$ ein Ringhomomorphismus und $L_i$ ein Körper ist, und somit $\phi$ injektiv ist. Außerdem ist $\phi(x+y) = \phi(x)+\phi(y)$ und $\phi(\lambda x) = \phi(\lambda)\phi(x) = \lambda\phi(x)$ für $\lambda\in K$ sowie $x,y\in L_i$. Also ist $\phi$ eine $K$-lineare Abbildung. Damit ist $\phi$ eine $K$-lineare injektive Abbildung zwischen endlichdimensionalen $K$-Vektorräumen der gleichen Dimension, also ist $\phi$ bijektiv und damit ein $K$-Isomorphismus.
\end{bem}

\paragraph{Übersicht} Sei $L\sslash K$ eine Körpererweiterung. Dann gilt folgendes bezüglich der Teilmengen von $\{\phi\colon L\to L\}$:

\begin{center}
	\begin{tikzcd}
		\left\{ \phi \middle| \substack{\text{$\phi$ Ringhomo-}\\\text{morphismus}}  \right\} \arrow[Equals]{d}{} && \left\{ \phi \mid \text{$\phi$ $K$-linear}  \right\} \\
		\left\{ \phi \middle| \substack{\text{$\phi$ Körperhomo-}\\\text{morphismus}}  \right\} & \left\{ \phi \middle| \substack{\text{$\phi$ Körperiso-}\\\text{morphismus}}  \right\} \arrow[Subseteq]{l}{} & \arrow[Subseteq]{u}{} \left\{ \phi \middle| \substack{\text{$\phi$ $K$-linear}\\\text{und injektiv}}  \right\} \\
		\left\{ \phi \middle| \substack{\text{$\phi$ $K$-Homo-}\\\text{morphismus}}  \right\} \arrow[Subseteq]{u}{} & \Aut(L \sslash K) \arrow[Subseteq]{ru}{} \arrow[Subseteq]{u}{} \arrow[Subseteq]{l}{}
	\end{tikzcd}
\end{center}

\begin{lem}\label{lem:khomonst}
	Es sei $\phi\colon L_1\to L_2$ ein $K$-Homomorphismus, und es sei $f\in K[t]$. Dann ist $a \in L_1$ genau dann eine Nullstelle von $f$, wenn $\phi(a) \in L_2$ eine Nullstelle von $f$ ist.
\end{lem}
\begin{proof}
Sei $f = \sum_{i = 0}^{\infty}b_it^i\in K[t]$ und $a$ eine Nullstelle von $f(t)$, also $\sum_{i = 0}^{\infty}b_ia^i = 0$. Das ist genau dann der Fall, wenn $\phi(\sum_{i = 0}^{\infty}b_ia^i) = \phi(0) = 0$, da $\phi$ als Ringhomomorphismus zwischen Körpern injektiv ist. Hierzu ist $\sum_{i = 0}^{\infty}	\phi(b_i)\phi(a)^i = 0$ äquivalent,  weil $\phi$ ein Ringhomomorphismus ist, was wiederum genau dann eintritt, wenn
$\sum_{i = 0}^{\infty}b_i\phi(a)^i =0$, da $\phi$ ein $K$-Homomorphismus ist, also ist $\phi(a)$ eine Nullstelle von $f$.
\end{proof}



\begin{lem}
	Sei $\phi\colon L_1\to L_2$ ein $K$-Homomorphismus. Wenn $a\in L_1$ algebraisch über $K$ ist, ist $\phi(a)$ algebraisch über $K$ und für die Minimalpolynome gilt $m_a(t) = m_{\phi(a)}(t)\in K[t]$.
\end{lem}

\begin{proof}
	Sei $a\in L_1$ algebraisch über $K$, es existiert also ein $f\in K[t]\setminus K$ mit $f(a) = 0$. $a$ ist folglich eine Nullstelle von $f \in K[t]$. Nach \cref{lem:khomonst} ist $\phi(a)$ eine Nullstelle von $f$. Daher ist $\phi(a)$ algebraisch über $K$ und Teil 1 des Lemmas folgt.
	
	Sei $m_a(t) \in K[t]$ das Minimalpolynom von $a$; $m_a(t)$ ist also normiert und von minimalem Grad, sodass $a$ Nullstelle ist. Nach \cref{lem:khomonst} ist $\phi(a)$ eine Nullstelle von $m_a(t)\in K[t]$. Folglich ist $m_{\phi(a)}$ Teiler von $m_a(t)$. Da analog auch $m_a(t)$ ein Teiler von $m_{\phi(a)}$ ist, gilt $m_a(t) = m_{\phi(a)} (t)$.
\end{proof}

\begin{bem}
	Beachte: $\phi$ bildet algebraische $a\in L_1$ auf Nullstellen von $m_a(t)$ ab\textbf{!!!}
\end{bem}

\begin{lem}
	Sei $K$ ein Körper sowie $L\sslash K$ und $L'\sslash K$ algebraische Körpererweiterungen und $a\in L$, $a'\in L'$ mit $m_a(t)=m_{a'}(t)\in K[t]$. Dann existiert ein eindeutiger $K$-Isomorphismus $\phi\colon K(a)\to K(a')$, sodass $\phi(a) = a'$.
\end{lem}
\begin{proof}
	Betrachte den Ringhomomorphismus
	\begin{eqnarray*}
		\ev_a\colon K[t]&\longto& L\\
		p & \longmapsto & p(a).
	\end{eqnarray*}
	Dann gilt $m_a(t)\mapsto m_a(a) = 0$, also $m_a(t) \in \ker(\ev_a)$, woraus $(m_a(t))\subseteq \ker(\ev_a)$ folgt. Nach dem \nameref{thm:homsatz_r} existiert genau ein Ringhomomorphismus \[\overline{\ev_a}\colon K[t]/(m_a(t))\to L,\] sodass $\overline{\ev_a}\circ \can = \ev_a$.
	
	Es ist klar, dass $\im\ev_a\subseteq K(a)$. Dann gilt auch $\im\overline{\ev_a}\subseteq K(a)$, und wir können $\overline{\ev_a}\colon K[t]/(m_a(t))\to K(a)$ betrachten. Wir wissen (siehe Vergleich transzendent vs. algebraisch), dass $(m_a(t))$ ein maximales Ideal ist. Also ist $K[t]/(m_a(t))$ ein Körper. Damit ist $\overline{\ev_a}$ injektiv und $\im\overline{\ev_a}\subseteq K(a)$. Außerdem ist $\im\overline{\ev_a}$ ein Körper, weil es das Bild eines Körpers unter einem Ringhomomorphismus ist, und $\im\overline{\ev_a}$ enthält offensichtlich $K$ und $a$. Somit gilt $\im(\overline{\ev_a}) = K(a)$. Also ist $\phi_1 = \overline{\ev_a}\colon K[t]/(m_a(t))\to K(a)$ ein Isomorphismus von Körpern. Nach Konstruktion gilt $\phi_1(\overline{t}) = a$ und $\phi_1|_K = \id_K$.
	
	Analog haben wir einen Isomorphismus von Körpern $\phi_2: K[t]/(m_{a'}(t))\to K(a')$ mit $\phi_2(\overline{t}) =a'$ und $\phi_2|_K  =\id_K$. Dabei ist $(m_a'(t)) = (m_a(t))$ (nach Voraussetzung). Nun ist $\phi = \phi_1\circ \phi_1^{-1}\colon K(a) \to K(a')$ ein Isomorphismus von Körpern und $\phi(a) = \phi_2(\overline{t}) = \phi(a')$ und $\phi|_K = \id_K$. Folglich erhalten wir einen $K$-Isomorphismus $\phi\colon K(a)\to K(a')$ mit $\phi(a) = a'$
	
	$\phi$ ist eindeutig, da $\phi|_K =\id_K$ und $\phi(a) =a'$; dadurch ist $\phi$ eindeutig bestimmt.
\end{proof}

\begin{satz}[Fortsetzungssatz] \label{thm:fortsetz}
	\leavevmode
	\begin{enumerate}
		\item Sei $L\sslash K$ eine algebraische Körpererweiterung sowie $L'$ ein algebraisch abgeschlossener Körper und $f\colon K\to L'$ ein Ringhomomorphismus. Dann existiert ein Ringhomomorphismus $\hat{f}\colon L\to L'$, sodass $\hat{f}|_K  = f$, also sodass das Diagramm
		\begin{center}
			\begin{tikzcd}
				L \arrow{r}{\exists \hat f} & L' \\
				K \arrow[Subseteq]{u}{} \arrow{ru}[swap]{f}
			\end{tikzcd}
		\end{center}
		kommutiert. \label{enumi:forsetz:1}
		\item Seien $K, K'$ Körper und $f\colon K\to K'$  ein Ringisomorphismus. Seien weiterhin $\overline{K}$ und $\overline{K'}$ algebraische Abschlüsse von $K$ beziehungsweise $K'$. Dann existiert ein Isomorphismus von Körpern $\hat{f}\colon \overline{K}\to \overline{K'}$ mit $\hat{f}|_K = f$. \label{enumi:forsetz:2}
	\end{enumerate}
	
\end{satz}

\lecture{11. Dezember 2017}

\begin{proof}
	\leavevmode
	\begin{enumerate}
		\item Betrachte
		\[Z =\left\{(M, g_M) \middle| \substack{\text{$K\subseteq M\subseteq L$ Zwischenkörper,}\\\text{$g_M\colon M\to L'$ Ringhomo-}\\\text{morphismus, $g_M|_K = f$}}\right\}.\]
		Nun: \begin{itemize}
			\item $Z$ ist nicht leer, da $(K,f)\in Z$.
			\item Wir definieren auf $Z$ durch $(M, g_M)\leq (M',g_{M'}):\Leftrightarrow M\subseteq M', g_{M'}|_M = g_M$ eine partielle Ordnung. Das Nachprüfen dieser Eigenschaft bleibt dem Leser überlassen.
			\item Sei $U\subseteq Z$ eine total geordnete Teilmenge. Dann hat $U$ eine obere Schranke $(S, g_S)$ in $Z$ bezüglich obiger Ordnung. Dafür sei $U  = \{(U_i,g_i)\mid i\in I\}\subseteq Z$. Dann setze $S := \bigcup_{i\in I} U_i$ und $g_S\colon S\to L'$ definiert durch $g_s(u_i) = g_i(u_i)$, falls $u_i\in U_i$. Nun gilt:
			\begin{description}
				\item[\emph{$S$ ist ein Körper.}] Da $U$ total geordnet ist, existiert für $x,y\in S$ ein $i\in I$ mit $x,y\in U_i$, da etwa für $x\in U_m, y\in U_j$ schon $U_m\leq U_j$ oder $U_j\leq U_m$ gilt. Da alle $U_k$ Körper sind, ist auch $S$ ein Körper.
				\item[\emph{$g_S$ ist wohldefiniert.}] Sei $u\in S$ mit $u\in U_i\cap U_j$. Da $U$ total geordnet ist, gilt $U_i\subseteq U_j$ oder $U_j\subseteq U_i$, und dann nach Definition der Ordnung sogar $g_i(u) = g_j(u)$, weil $(U_i,g_i)\leq (U_j, g_j)$ oder $(U_j,g_j)\leq (U_i,g_i)$ und $u\in U_i\cap U_j$ ist.
				\item[\emph{$g_S$ ist ein Ringhomomorphismus.}] Analog zu den obigen Beweisen ist dies sofort klar, da $U$ total geordnet ist und alle $g_i$ Ringhomomorphismen sind.
			\end{description}
			Somit liegt $(S, g_S)\in Z$ (beachte $g_i|_K = f$ für alle $i\in I$ nach Definition, also auch $g_S|_K = f$). Offensichtlich ist $(S,g_S)$ eine obere Schranke für $U$.
			\item Mit dem Lemma von Zorn folgt die Existenz eines maximalen Elementes $(M_{\max}, g_{\max})\in Z$.
			\item \emph{Behauptung:} $M_{\max} = L$. Daraus folgt direkt \ref{enumi:forsetz:1} mit dem Ringhomomorphismus $\hat{f} = g_{\max}\colon M_{\max} = L\to L'$, für den $\hat{f}|_K = g_{\max}|_K = f$ gilt, weil $(M_{\max}, g_{\max})\in Z$ liegt.
			\begin{proof}[Beweis der Behauptung]
				Da $(M_{\max}, g_{\max})\in Z$, gilt $M_{\max}\subseteq L$. Sei also $a\in L$, aber $a\notin M_{\max}$. Wir wollen nun einen Widerspruch herleiten. Da $L\sslash K$ algebraisch ist, ist erst recht $L\sslash M_{\max}$ algebraisch, da $K\subseteq M_{\max}$. Folglich existiert das Minimalpolynom $m_a(t)\in M_{\max}[t]\setminus M_{\max}$ von $a$. Sei $m_a = \sum_{i = 0}^{\infty}b_it^i$.  Sei $m'_a(t) = \sum_{i = 0}^{\infty}g_{\max}(b_i)t^i\in L'[t]\setminus L'$. Da $L'$ nach Voraussetzung algebraisch abgeschlossen ist, existiert eine Nullstelle $a'$ von $m_a'(t)$ in $L'$. Betrachte nun den Ringhomomorphismus
				\begin{eqnarray*}
					\ev_{a'}\colon M_{\max}[t] &\longto& L'\\
					\sum_{j = 0}^{\infty} c_it^i &\longmapsto& \sum_{i = 0}^{\infty} g_{\max}(c_i)(a')^i.
				\end{eqnarray*}
				Nach Konstruktion gilt hier:
				\begin{itemize}
					\item $(m_a(t))\subseteq \ker\ev_{a'}$.
					\item Es gilt $\ev_{a'}|_K = g_{\max}|_K = f$ nach der Definition von $Z$. Nach dem \nameref{thm:homsatz_r} erhalten wir einen Ringhomomorphismus
					\[\overline{\ev_{a'}}\colon M_{\max}[t]/(m_a(t))\to L'\enspace,\]
					sodass $\overline{\ev_{a'}}\circ \can = \ev_{a'}$. Da $a$ algebraisch ist, ist andererseits $\beta: M_{\max}(a) \xrightarrow{\sim} M_{\max}[t]/(m_a(t))$ ein Isomorphismus von Körpern und damit $M_{\max}\subsetneq M_{\max}(a)\cong M_{\max}[t]/(m_a(t))$. Wir erhalten also einen Körper $M_{\max}(a)\subseteq L$ mit $M_{\max}\subsetneq M_{\max}(a)$.
					
					Also ist $(M_{\max}, \ev_{a'}\circ \beta)\in Z$, da $\ev_{a'}\circ \beta\colon M_{\max}(a)\to L'$ offensichtlich ein Ringhomomorphismus mit $(\ev_{a'}\circ \beta)|_K = \ev_{a'}|_K = f$ nach Konstruktion ist. Das ist ein Widerspruch zur Maximalität von $(M_{\max}, g_{\max})$. Damit folgt die Behauptung und insgesamt Teil \ref{enumi:forsetz:1} des Satzes.
        \qedhere
				\end{itemize}
			\end{proof}
		\end{itemize}
		\item Betrachte $L := \overline{K}$ und $f\colon K\xrightarrow{\phi} K'\subseteq \overline{K'} =: L'$. Da $L\sslash K$ algebraisch (nach Definition des algebraischen Abschlusses) und $L' = \overline{K'}$ algebraisch abgeschlossen ist, können wir Teil \ref{enumi:forsetz:1} des Satzes anwenden und erhalten einen Ringhomomorphismus
		\[\hat{\phi}\colon L = \overline{K}\longto \overline{K'} = L'\]
		mit $\hat{\phi}|_K = \phi$. Da $\hat\phi$ ein Ringhomomorphismus und $\overline{K}$ ein Körper ist, ist $\hat\phi$ injektiv. Es bleibt die Surjektivität von $\hat{\phi}$ zu zeigen. Dabei wissen wir:
		\begin{itemize}
			\item $\im\hat{\phi} \subseteq L'$ ist ein Unterkörper, weil $\overline{K}$ ein Körper ist.
			\item $K' = \im\phi = f(K) = \hat{\phi}(K)\subseteq \im\hat{\phi}$, also $K'\subseteq \im\hat{\phi}\subseteq \overline{K'} = L'$.
			\item Da $\overline{K}$ algebraisch abeschlossen ist, ist auch $\im \hat{\phi}$ algebraisch abgeschlossen, da für $\sum_{i = 0}^{\infty}c_it^i\in \im \hat{\phi}[t]\setminus \im\hat{\phi}$ dann $a_i\in \overline{K}$ mit $c_i = \hat{\phi}(a_i)$ existieren, wobei $\sum_{i = 0}^{\infty} a_it^i\in \overline{K}[t]\setminus \overline{K}$ eine Nullstelle $x$ in $\overline{K}$ hat, da $\overline{K}$ algebraisch abgeschlossen ist, weshalb $\hat{\phi}(x)$ eine Nullstelle von $\sum_{i = 0}^{\infty}c_it^i$ ist.
			\item Die Körpererweiterung $\im\hat{\phi}\sslash K'$ ist algebraisch, da $\im\hat{\phi}\subseteq\overline{K'}$ und $\overline{K'}\sslash K'$ algebraisch per Definition ist, und somit insbesondere auch $\im\hat{\phi}\sslash K'$ algebraisch ist.
		\end{itemize} 
		Also ist $K'\subseteq  \im\hat {\phi}\subseteq \overline{K'}$; da $\im\hat{\phi}$ algebraisch abgeschlossen ist, folgt mit \cref{thm:algab} $\im\hat{\phi} = \overline{K'}$. Folglich haben wir einen Isomorphismus von Körpern
		\[\hat{\phi}\colon \overline{K}\xrightarrow{\sim} \overline{K'},\]
		wobei $\hat{\phi}|_K = f|_K = \phi$. Also folgt Teil \ref{enumi:forsetz:2}.
  \qedhere
	\end{enumerate}
\end{proof}

Damit folgt der Eindeutigkeitssatz \ref{thm:16.3}.

\medskip

Wie sehen nun zum Beispiel $\overline{\IQ}$ oder $\overline{\IF_2}$ aus?
\begin{bem}
	Wir wissen: $\IQ\subsetneq \overline{\IQ}$, da beispielsweise $t^2-2\in \IQ[t]$ keine Nullstelle in $\IQ$ hat. Wir können $\IQ^{\text{alg}} = \{z\in \IC\mid \text{$z$ algebraisch über $\IQ$}\}\subseteq \IC$ als algebraischen Abschluss nehmen, da $\IC$ algebraisch abgeschlossen ist. \glqq Man\grqq\ weiß, dass $\IQ^{\text{alg}} \subsetneq \IC$, weil es transzendente Zahlen gibt, z.B. $\pi\in\IC\setminus\IQ^{\text{alg}}$.
	Aus der Konstruktion des algebraischen Abschlusses kann man folgern, dass $\IQ^{\text{alg}}$ abzählbar ist.
	
	Für $\overline{\IF_2}$ ist die Lage besser. Wir brauchen aber ein besseres Verständnis endlicher Körper.
\end{bem}

\subsection{Endliche Körper}
Ziel: Gegeben eine Primzahl $p$ und $r\in\IN= \IZ_{>0}$. Dann existiert ein Körper $\IF$ mit $|\IF| = p^r$. Genauer gilt der folgende
\begin{satz}[Klassifikationssatz endlicher Körper]\label{thm:17.1}
	Es gibt eine Bijektion
	\begin{eqnarray*}
		\Phi\colon\left\{\IF\,\middle|\,\substack{\text{$\IF$ endlicher Körper}\\\text{\textup(bis auf Körperisomorphie\textup)}}\right\}& \xlongleftrightarrow{1:1}& \{p^r\mid \text{$p$ Primzahl}, r\in\IN\}\\
		\IF & \longmapsto & |\IF|
	\end{eqnarray*}
	Wir nennen dann den \textup(bis auf Isomorphie eindeutigen\textup) Körper mit $p^r$ Elementen $\IF_{p^r}$.
\end{satz}
\begin{bem}
	\leavevmode
	\begin{itemize}
		\item Falls $r = 1$ ist, kennen wir $\IF_p = (\IZ/p\IZ, +, \cdot)$ aus der linearen Algebra.
		\item Achtung: $\IF_4 \ncong \IZ/2\IZ\times \IZ/2\IZ$, da letzterer Nullteiler hat.
	\end{itemize}
\end{bem}
\begin{bsp}
	Betrachte $p(t) = t^2+t+1\in \IF_2[t]$. $p$ ist irreduzibel, da $p$ keine Nullstellen hat. Folglich ist $K = \IF_2[t]/(p(t))$ ein Körper, weil $(p(t))$ ein maximales Ideal ist. Als $\IF_2$-Vektorraum hat $K$ die Basis $\overline{1}, \overline{t}$. Also hat $K$ 4 Elemente $a\overline{1}+b\overline{t}$ mit $a,b\in \IF_2$; $K = \{0,\overline{ 1},\overline{ t}, \overline{1+t}\}$, wobei wir $x := \ol t$ und $y := \ol{1+t}$ setzen.
	
	Wir betrachten nun die Additions- und Multiplikationstafeln von $K$.
	\begin{center}
		\begin{tabular}{c||c|c|c|c}
			$+$ & $\ol 0$ & $\ol 1$ & $x$ & $y$  \\
			\hline \hline
			$\ol 0$ & $\ol 0$ & $\ol 1$ & $x$ & $y$ \\
			\hline
			$\ol 1$ & $\ol 1$ & $\ol 0$ & $y$ & $x$ \\
			\hline
			$x$ & $x$ & $y$ & $\ol 0$ & $\ol 1$ \\
			\hline
			$y$ & $y$ & $x$ & $\ol 1$ & $\ol 0$ \\
		\end{tabular} $\qquad$
		\begin{tabular}{c||c|c|c|c}
			$\cdot$ & $\ol 0$ & $\ol 1$ & $x$ & $y$  \\
			\hline \hline
			$\ol 0$ & $\ol 0$ & $\ol 0$ & $\ol 0$ & $\ol 0$ \\
			\hline
			$\ol 1$ & $\ol 0$ & $\ol 1$ & $x$ & $y$ \\
			\hline
			$x$ & $\ol 0$ & $x$ & $y$ & $\ol 1$ \\
			\hline
			$y$ & $\ol 0$ & $y$ & $\ol 1$ & $x$ \\
		\end{tabular}
	\end{center}
	
	\noindent Beobachtungen:
	\begin{itemize}
		\item $\chr{K} = 2$, weil $1+1 = 0$ und $\IF_2\subseteq K$ der Primkörper ist.
		\item Alle Elemente in $K^{\times}$ sind Nullstellen von $t^3-1$; alle Elemente in $K$ sind Nullstellen von $t^4 -t$.
		\item $K^{\times} = K\setminus \{0\}$ ist eine Gruppe bezüglich $\cdot$ sowie ismorph zu $\IZ/3\IZ$ und somit insbesondere zyklisch.
	\end{itemize}
\end{bsp}

\lecture{14. Dezember 2017}
\begin{lem}
	Sei $\IF$ ein endlicher Körper. Dann ist $|\IF| = p^r$ für eine Primzahl $p$ und $r\in\IN$ und es gilt $\chr \IF = p$.
\end{lem}

\begin{proof}
	$\IF$ hat den Primkörper $\IF_p$ (weil sonst $\IQ \cong \text{Primkörper} \subseteq \IF$, was ein Widerspruch zur Endlichkeit von $\IF$ wäre) für eine eindeutig bestimmte Primzahl $p$. Insbesondere ist $\IF_p\subseteq \IF$ ein Unterkörper. Damit ist $\IF$ ein $\IF_p$-Vektorraum und es gilt $\IF\cong (\IF_p)^n$ für ein $n\in\IN$ (Isomorphismus als $\IF_p$-Vektorraum). Folglich ist $|\IF| = p^n\Rightarrow$; wir setzen also $r = n = \dim_{\IF_p}\IF$.
\end{proof}
Also ist $\Phi$ aus \cref{thm:17.1} eine wohldefinierte Abbildung.

\begin{satz}\label{thm:17.3}
	Sei $K$ Körper sowie $H<(K^{\times},\cdot)$ eine endliche Untergruppe. Dann ist $H$ zyklisch.
\end{satz}
\begin{proof}
	\leavevmode
	\begin{description}
		\item[1. Behauptung:] Sei $G$ eine endliche abelsche Gruppe. Dann existiert ein Isomorphismus von Gruppen $G\cong G_{p_1}\times\dots\times G_{p_n}$, wobei die $G_{p_i}$ die $p_i$-Sylowuntergruppen von $G$ sind (genauer: $G_{p_1},\dots,G_{p_n}$ sind genau die Sylowuntergruppen von G). Insbesondere gilt dann
		\begin{equation}
			H\cong G_{p_1}\times\dots\times G_{p_n} \tag{*}\label{eq:h=prodg}
		\end{equation}
		mit $G_{p_i}$ als $p_i$-Sylowuntergruppe von $H$ für $1\le i \le n$.
		\item[2. Behauptung:] Alle $G_{p_i}$ ($1\leq i\leq n$) in \eqref{eq:h=prodg} für $1\le i\le  n$ sind zyklisch.
		\item[3. Behauptung:] Die Gruppe $G_{p_1}\times\dots\times G_{p_n}$ wie in \eqref{eq:h=prodg} ist zyklisch.
	\end{description}
	Damit ist $H$ zyklisch.
	\begin{description}
		\item[Beweis 1. Behauptung:] Sei $|G| = p_1^{c_1}\dots p_n^{c_n}$ mit paarweise verschiedenen Primzahlen $p_i$. Nach den \hyperref[thm:sylow]{Sylowsätzen} existiert eine $p_i$-Sylowuntergruppe $G_{p_i}$ von $G$. Diese ist eindeutig, da jede andere $p_i$-Sylowuntergruppe zu dieser konjugiert ist, woraus sogar Gleichheit folgt, da $G$ abelsch ist. Es ist nach dem \nameref{thm:lagrange} klar, dass $G_{p_i}\cap G_{p_j} = \{e\}$ für $i\neq j$ gilt. Betrachte
		\begin{eqnarray*}
			f\colon G_{p_1}\times\dots\times G_{p_n} & \longto & G\\
			(g_1,\dots, g_n) & \longmapsto & g_1\dots g_n.
		\end{eqnarray*}
		Im Folgenden seien $(g_1,\dots,g_n),(g'_1,\dots,g'_n)$ immer in $G_{p_1}\times\dots\times G_{p_n}$.
		\begin{itemize}
			\item $f$ ist ein Gruppenhomomorphismus, da \begin{align*}&f((g_1,\dots, g_n)(g_1',\dots, g_n')) = f((g_1g_1',\dots, g_ng_n')) = g_1g_1'g_2g_2'\dots g_ng_n'\\ &= g_1g_2\dots g_ng_1'\dots g_n' = f((g_1,\dots, g_n))f((g_1',\dots, g_n'))\end{align*} gilt.
			\item $f$ ist injektiv. Zum Beweis sei $f((g_1,\dots, g_n)) = f((g_1',\dots, g_n'))$. Dann ist $g_1\dots g_n = g_1'\dots g_n'$. Umformen ergibt
			\[ g_2\dots g_n(g_2'\dots g_n')^{-1} = g_1^{-1}g_1'\enspace,\]
			wobei $g_2\dots g_n(g_2'\dots g_n')^{-1}  = g_2(g_2')^{-1}\dots g_n(g_n')^{-1}\in f(\{e\}\times G_{p_2}\times \dots \times G_{p_n})$ und $g_1^{-1}g_1'\in f(G_{p_1})$.
			
			Sei $\tilde{G} = \{e\}\times G_{p_2}\times\dots\times G_{p_n}$. Nach dem \nameref{thm:homsatz_g} gilt $f(\tilde{G})\cong \tilde {G}/\ker f|_{\tilde{G}}$. Damit folgt $|f(\tilde{G})| = \frac{|\{e\}\times G_{p_2}\times\dots\times G_{p_n}|}{|\ker f|_{\tilde{G}}|}$ nach dem \nameref{thm:lagrange}. $|f(\tilde{G})|$ teilt somit $p_2^{c_2}\dots p_n^{c_n}$. Analog zeigt man, dass $|f(G_{p_1})|$ dann $p_1{c_1}$ teilt. 
			
			$\ord (g_2\dots g_n(g_2'\dots g_n')^{-1})$ teilt also $p_2^{c_2}\dots p_n^{c_n}$ und $\ord(g_1^{-1}g_1')$ teilt $p_1^{c_1}$. Da die $p_i$ paarweise verschieden sind, ist $\ord(g_1^{-1}g_1') = 1$, also $g_1^{-1}g_1' = e$ und schließlich $g_1' = g_1$. Völlig analog folgt $g_i = g_i'$ für alle $1\leq i\leq n$.
			\item $f$ ist surjektiv, da $f$ injektiv  ist und $|G_{p_1}\times\dots\times G_{p_n}| = p_1^{c_1}\dots p_n^{c_n} = |G|<\infty$ gilt.
		\end{itemize}
	Damit ist $f$ ein bijektiver Ringhomomorphismus und die 1. Behauptung folgt.
	\item[Beweis 2. Behauptung:] Wir nehmen an, $G_{p_i}$ wäre nicht zyklisch für ein $i=1,\dots,n$. Wir setzen $p = p_i$ und $c = c_i$ ($c_i$ wie oben). Es gilt $|G_p| = p^c$ und nach Annahme $\ord(g)<p^c$ für alle $g\in G_p$. Da $\ord(g)$ die Ordnung der Gruppe $|G_p|$ teilt, folgt $ord(g)\leq p^{c-1}$ und sogar $\ord(g) = p^m$ für ein $m\leq c-1$ für alle $g\in G_p$, wobei $m$ von $g$ abhängig ist.
	
	Also ist $g^{p^m} = 1$ und $g$ ist eine Nullstelle des Polynoms $t^{p^m}-1$, und somit auch von $t^{p^{c-1}}-1$. Damit ist $g$ eine Nullstelle von $t^{p^{c-1}}-1$ für alle $g\in G_p$. Aber $t^{p^{c-1}}-1$ hat höchstens $p^{c-1}$ verschiedene Nullstellen, was im Widerspruch dazu, dass $|G_p| = p^c> p^{c-1}$ ist. Also ist $G_{p_i}$ zyklisch für alle $1\leq i\leq n$.
	\item[Beweis 3. Behauptung:] Es ist zu zeigen, dass $G_{p_1}\times\dots\times G_{p_n}$ wie in \eqref{eq:h=prodg} zyklisch ist. Nach der 2. Behauptung wissen wir $G_{p_i}\cong \IZ/p_i^{c_i}\IZ$ mit der \nameref{thm:klasszykl}. Es ist also $H\cong \IZ/p_1^{c_1} \IZ\times\dots \times \IZ/p_n^{c_n}\IZ$ als Isomorphismus von Gruppen. Da die $p_i$ paarweise verschiedene Primzahlen sind, existieren $a,b\in\IZ$ so, dass $1 = ap_i^{c_i}+bp_j^{c_j}$ (Euklidischer Algorithmus). Damit ist $1\in p_i^{c_i}\IZ+p_j^{c_j}\IZ$ und wir können den Chinesischen Restsatz anwenden. Dadurch erhalten wir einen Isomorphismus von Ringen
	$$\IZ/p_1^{c_1}\IZ\times\dots \times\IZ/p_n^{c_n}\IZ \xlongrightarrow{\sim} \IZ/m\IZ$$
	mit $m = p_1^{c_1}\dots p_n^{c_n}$. Dies ist insbesondere Isomorphismus von Gruppen. Da $\IZ/m\IZ$ zyklisch ist, folgt schließlich, dass $H$ zyklisch ist.
	\qedhere
\end{description}
\end{proof}

\begin{bem} Insbesondere gilt: für endliche Körper $\IF$ ist $(\IF^{\times}, \cdot)$ eine zyklische Gruppe.
\end{bem}
\begin{lem}\label{lem:17.4}
	Sei $\IF$ ein endlicher Körper mit $|\IF| = n = p^r$, wobei $p$ eine Primzahl und $r\in\IN$ ist. Dann existiert ein Isomorphismus von Körpern $\IF\cong \IF_p[t]/(p(t))$, wobei $p(t)\in \IF_p[t]$ mit $\deg(p(t)) = r$ und $p(t)\mid t^n-t\in F_p[t]$ sowie irreduzibel ist. Umgekehrt existiert ein solcher Isomorphismus, wenn $p(t)$ soebige Eigenschaften erfüllt.
\end{lem}
\begin{proof}
	$(\IF^{\times}, \cdot)$ ist eine endliche ablesche Gruppe, also nach \cref{thm:17.3} zyklisch mit der Ordnung $n-1$. Folglich existiert ein $a\in\IF^{\times}$ mit $\IF^{\times} = \<a\>$. Insbesondere ist $\ord(a) = n-1$, also $a^{n-1} = 1\in\IF^{\times}\subseteq \IF$. Es gilt sogar $x^{n-1} = 1$ für alle $x\in\<a\> = \IF^{\times}$. Damit ist jedes $x\in \IF$ Nullstelle von $t^n-t\in\IF_p[t]$. Da es höchstens $n$ verschiedene Nullstellen gibt und $|\IF| = n $, folgt 
	\[\IF= \{\text{Nullstellen von $t^n-t\in\IF_p[t]$}\}\]
	und somit zerfällt $t^n-t$ über $\IF$ vollständig in Linearfaktoren. Sei $p(t)$ wie im Lemma definiert und o.B.d.A. $p(t)$ normiert. Da $x\in\IF^{\times}$ eine Nullstelle von $t^{n}-t$ ist und $p(t)$ das Polynom $t^{n}-t$ teilt, ist $p(t) = m_x(t)$ das Minimalpolynom für ein $x\in\IF^{\times}$, weil $p(t)$ irreduzibel ist. Betrachte nun den Ringhomomorphismus
	\begin{eqnarray*}
		\ev_x\colon \IF_p[t] &\longto & \IF\\
		\sum_{i = 0}^{\infty}b_it^i & \longmapsto & \sum_{i = 0}^{\infty}b_ix^i
	\end{eqnarray*}
	Dieser induziert, weil $p(x) = 0$ ist, nach dem \nameref{thm:homsatz_r} einen Ringhomomorphismus
	\[\overline{\ev_x}\colon \IF_p[t]/(m_x(t)) \to \IF.\]
	Wir wissen, dass $K :=\IF_p[t]/(p(t))$ ein Körper ist, weil $p(t)$ irreduzibel ist. Folglich ist $\overline{\ev_x}\colon K\to \IF$ injektiv und nach Konstruktion ein Ringhomomorphismus.
	
	Wir behaupten, dass $\overline{\ev_x}$ surjektiv ist. Wir wissen $\dim_{\IF_p}K = \deg(p(t)) = r$, da $\{1,\overline{t},\dots, \overline{t^{\deg(p(t))-1}}\}$ eine Basis ist. Also gilt $|K| = p^r = n = |\IF|$. Damit ist $\overline{\ev_x}$ eine injektive Abbildung zwischen endlichen Mengen derselben Kardinalität, also bijektiv.
	
	Folglich existiert ein $\IF_p[t]/(p(t))\xlongrightarrow{\cong}\IF$ Körperisomorphismus für alle $p(t)$ wie im Lemma.
	
	Es bleibt noch zu zeigen, dass ein solches $p(t)$ auch tatsächlich immer existiert; siehe dazu den \hyperref[nachtrag zu 17.4]{Nachtrag} zu Beginn der nächsten Vorlesung.
\end{proof}

\begin{bem}
	Wir betrachten das $\Phi$ aus \cref{thm:17.1}. \cref{lem:17.4} besagt insbesondere, dass zwei endliche Körper derselben Kardinalität isomorph sind; $\Phi$ ist also injektiv. Es bleibt zu zeigen, dass $\Phi$ surjektiv ist.
\end{bem}

\medskip

Beachte: Sei $\IF$ wie im \cref{lem:17.4} definiert. Dann gilt:
\begin{enumerate}
	\item $f(t) = t^n-t\in\IF_p[t]$ zerfällt in Linearfaktoren über $\IF$.
	\item $\IF= \IF_p(a_1,\dots, a_n)$, wobei $a_1,\dots, a_n$ die Nullstellen von $f(t)$ sind.
\end{enumerate}

\subsection{Zerfällungskörper}
\begin{defi}
	Sei $K$ ein Körper und $f(t)\in K[t]$ mit $\deg(f(t))\geq 1$. Sei $L\sslash K$ eine Körpererweiterung. Dann heißt $L$ (ein) Zerfällungskörper von $f(t)$ über $K$, falls
	\begin{enumerate}
		\item $f(t)$ in Linearfaktoren über $L$ zerfällt, es existiert also eine Darstellung
		\begin{equation}
			f(t) = c\prod_{i = 1}^{n}(t-a_i) \tag{*} \label{eq:zfklin}
		\end{equation}
		mit $a_1,\dots,a_n \in L$ und $c\in K$, und
		\item $L = K(a_1,\dots, a_n)\subseteq L$ mit $a_1,\dots, a_n$ wie in \eqref{eq:zfklin} gilt.
	\end{enumerate}
\end{defi}
\begin{bsp}
	Sei $f(t) = t^2+1\in\IR[t]$. Dann ist $\IC$ ein Zerfällungskörper über $\IR$, da $f(t) = (t-i)(t+i)$, also gilt 1., und $\IR(i,-i) = \IC$, also gilt 2.
\end{bsp}
\paragraph{Existenz von Zerfällungskörpern.} Sei $f(t) \in K[t]$. Betrachte den algebraischen Abschluss $\overline{K}$ von $K$. Dann existieren $a_1,\dots, a_n\in\overline{K}$ und $c\in K$ mit $f(t) = c\prod_{i = 1}^{n}(t-a_i)$. Setze $L: = K(a_1,\dots, a_n)$. Dies ist dann ein Zerfällungskörper von $f(t)\in K[t]$, der in $\overline{K}$ enthaltene Zerfällungskörper.

\lecture{18. Dezember 2017}

Wir wollen später noch zeigen:
\begin{satz}\label{thm:18.1}
	Sei $p$ eine Primzahl und $r\in\IN$. Dann existiert ein Körper $\IF$ mit $|\IF| = p^r$.
\end{satz}

\medskip

Zunächst noch ein {Nachtrag\label{nachtrag zu 17.4}} zu \cref{lem:17.4}. Beim dortigen Beweis bleibt noch zu zeigen, dass ein solches $p(t)$ existiert.
\begin{proof}
	Wir wissen:
	\begin{itemize}
		\item $\IF_p\subseteq \IF$ ist der Primkörper.
		\item $\IF^{\times}$ ist zyklisch; es existiert also ein $a\in\IF^{\times}$ mit $\IF^{\times} = \<a\>$.
		\item Für alle Einheiten $x\in \IF^{\times}$ gilt $x^n-x = 0$.
	\end{itemize}
	Insbesondere teilt $m_a(t)\in\IF_p[t]$ dann $t^n-t\in\IF_p[t]$ und es gilt $\IF_p(a) = \IF$. Betrachte den Ringhomomorphismus
	\begin{eqnarray*}
		\ev_a\colon \IF_p[t] & \longto & \IF\\
		\sum_{i = 0}^{\infty} b_it^i & \longmapsto &\sum_{i = 0}^{\infty}b_ia^i.
	\end{eqnarray*}
	Klar ist, dass $\im\ev_a = \IF_p(a) = \IF$ gilt; $\ev_a$ ist also surjektiv. Nach dem \nameref{thm:homsatz_r} existiert ein Ringhomomorphismus
	\[\overline{\ev_a}\colon K := \IF_p[t]/(m_a(t))\longto \IF\]
	mit $\ev_a = \overline{\ev_a}\circ\can$. Dieser ist $\overline{\ev_a}$ surjektiv und auch injektiv, da $K$ ein Körper ist. Somit ist $\overline{\ev_a}$ ist ein Isomorphismus von Körpern.
	
	Insbesondere ist $\deg(m_a(t)) = \dim_{\IF_p}K = \dim_{\IF_p}\IF= r$. Klar ist, dass $m_a(t)$ irreduzibel ist und $t^m-t$ teilt. Folglich ist $m_a(t)$ ein gesuchtes Polynom.
\end{proof}

\begin{defi}
	Sei $K$ ein Körper sowie $f\in K[t]$ mit $f = \sum_{i = 0}^n a_it^i$. Die formale Ableitung von $f(t)$ ist $f'(t) = \sum_{i = 1}^{n} ia_it^{i-1}\in K[t]$.
\end{defi}
\begin{bsp}
	\[
	f(t) = 2t^3+2t^2+1\in \IF_3[t] \quad \Longrightarrow \quad f'(t) = 6t^2 + 4t = t\in\IF_3[t]
	\]
\end{bsp}

\noindent
Übungsblatt: Die formale Ableitung erfüllt die Produktregel $(fg)' = f'g+fg'$.

\begin{satz}
	$L\sslash K$ eine Körpererweiterung sowie $f,g\in K[t]$ und $g\neq 0$. Dann:
	\begin{enumerate}
		\item Gilt $f = hg+r$ in $L[t]$ mit $\deg(r)<\deg(h)$, dann gilt auch $f = hg+r$ in $K[t]$.
		\item Falls $g$ das Polynom $f$ in $L[t]$ teilt, dann teilt es es auch in $K[t]$.
		\item Die normierten größten gemeinsamen Teiler in $L[t]$ und $K[t]$ sind gleich.
	\end{enumerate}
\end{satz}
\begin{proof}
	Da $K[t]$ euklidisch ist, existieren $h', r'\in K[t]$ mit $f = h'g+r'$, wobei $\deg(r')<\deg(g)$. Da $L[t]$ ebenfalls euklidisch ist, existieren eindeutige $h, r\in L[t]$ mit $f = hg+r$ und $\deg(r) <\deg(g)$. Aus der Eindeutigkeit folgen $h = h'$ und $r = r'$. Also gilt die 1. Aussage. Die 2. Aussage folgt aus der ersten mit $r = 0$. Die letzte Aussage bleibt dem aufmerksamen Leser als Übung überlassen.
\end{proof}

\begin{satz} \label{thm:18.3}
	Sei $K$ ein Körper sowie $p\in K[t]$ mit $p\neq 0$. Sei $L$ ein Zerfällungskörper von $p$ über $K$. Dann sind folgende Aussagen äquivalent:
	\begin{enumerate}
		\item $p$ hat eine eine mehrfache Nullstelle in $L$, es existiert also ein $a\in L$ mir $(t-a)^2\mid p$ in $L[t]$.
		\item $p$ und $p'$ haben einen gemeinsamen Teiler $h(t)\in L[t]\setminus L$.
		\item $p$ und $p'$ haben eine gemeinsame Nullstelle.
	\end{enumerate}
\end{satz}
\begin{proof}
	\leavevmode
	\begin{description}
		\item[\glqq$1\Rightarrow 2$\grqq:] Sei $a\in L$ eine mehrfache Nullstelle von $p$. Dann ist $p = (t-a)^2Q(t) = (t^2-2at+a^2)Q(t)$ für ein $Q(t)\in L[t]$. Nach der Produktregel ist $p' = 2(t-a)Q(t) + (t-a)^2Q'(t)$. Damit teilt $(t-a)$ sowohl $p$ als auch $p'$.
		\item[\glqq$2\Rightarrow 3$\grqq:] $p$ und $p'$ haben einen gemeinsamen Teiler $h(t)\in L[t]\setminus L$. Da $L$ ein Zerfällungskörper von $p$ ist, existiert ein $a\in L$, welches eine Nullstelle von $h(t)$ ist. Insbesondere ist $a$ eine gemeinsame Nullstelle von $p$ und $p'$.
		\item[\glqq$3\Rightarrow 1$\grqq:] Sei $a$ eine gemeinsame Nullstelle von $p$ und $p'$. Also ist $p = (t-a)Q(t)$ mit $Q(t)\in L[t]$. Es folgt $p' = Q(t) + (t-a)Q'(t)$. Da $(t-a)$ das Polynom $p'$ teilt, muss also $(t-a)$ auch $Q(t)$ teilen und somit gilt $(t-a)^2\mid p$. Folglich ist $a$ eine mehrfache Nullstelle von $p$.
	\end{description}
\end{proof}
\begin{kor}
	Sei $K$ ein Körper und $p\in K[t]\setminus K$ irreduzibel. Dann sind äquivalent:
	\begin{enumerate}
		\item $p$ hat eine mehrfache Nullstelle im Zerfällungskörper.
		\item $p' = 0$ \textup(Nullpolynom\textup).
		\item Es existiert ein $Q\in K[t]$ mit $p = Q(t^p)$ mit $0<p = \chr K$.
	\end{enumerate}
\end{kor}
\begin{proof}
	\leavevmode
	\begin{description}
		\item[\glqq$2\Rightarrow3$\grqq:] Sei $p = a_nt^n+\dots +a_1t+a_0$ mit $a_n \neq 0$, $n\geq 1$. Mit der Voraussetzung $p' = 0$ folgt $ja_j = 0$ für alle $1\leq j\leq n$. Da $K$ aber nullteilerfrei ist, gilt $n = 0$ in $K$. Folglich existiert eine Primzahl $q$ mit $q = \chr K$ und dann $n = qk$ für ein $k\in\IN$. Also ist $ja_j = 0$, falls $a_j = 0$ oder $j$ Vielfaches von $p$. Das heißt, $p = a_0+a_pt^p+a_{2p}t^{2p}+\dots+a_{pk}t^{pk}$. Damit ist $p = Q(t^p)$, wobei $Q(t)= a_0+a_pt+a_{2p}t^2+\dots a_{pk}t^k$.
		\item[\glqq$3\Rightarrow1$\grqq:] Sei $a$ eine Nullstelle von $Q(t)$ in einem Zerfällungskörper $L$ von $Q(t)$. Dann ist $Q(t) = (t-a)h(t)$ für ein $h(t)\in L[t]$; es gilt also $p = (t^p-a)h(t^p)$. Sei $b$ eine Nullstelle von $t^p-a$, also eine $p$-te Wurzel von $a$, in einem algebraischen Abschluss von $K$. 
		
		Nach der binomischen Formel \glqq für Dumme\grqq\ erhalten wir $p(t) = (t^p-b^p)h(t^p) = (t-b)^pn(t^p)$. Da $p\geq 2$ ist, hat $p(t)$ die mehrfache Nullstelle $b$ in einem algebraischen Abschluss von $K$, also auch in dem zugehörigen Zerfällungskörper.
		\item[\glqq$1\Rightarrow3$\grqq:] Sei $a$ eine mehrfache Nullstelle von $p$. Nach \cref{thm:18.3} ist $a$ eine gemeinsame  Nullstelle von $p$ und $p'$ bzw. $p$ und $p'$ haben einen gemeinsamen Teiler $h(t)\in L[t]\setminus L$. Es gilt $h(t)\neq 0$, da sonst schon $p(t) = 0$ im Widerspruch zur Irreduzibilität von $p$ ist. Da $p$ irreduzibel ist und $\deg(h(t)) \leq \deg(p')<\deg(p)$ im Fall von $p' \neq 0$, ist damit $h(t)$ eine Einheit im Widerspruch zu $h(t) \notin L$. Also ist $p' = 0$.
  \qedhere
	\end{description}
\end{proof}

\begin{defi}
	Sei $K$ ein Körper, $\chr K = p$. Dann ist $\Fr\colon K\to K, a\mapsto a^p$ ein Ringhomomorphismus, der Frobeniushomomorphismus. (Denn $\Fr(xy) = (xy)^p = x^py^p = \Fr(x)\Fr(y)$ und $\Fr(x+y) = (x+y)^p = x^p+y^p = \Fr(x) + \Fr(y)$ für alle $x,y\in K$.)
\end{defi}

\begin{proof}[Beweis von \cref{thm:18.1}]
	Sei $p(t) = t^n-t\in\IF_p[t]$. Sei $L$ ein Zerfällungskörper von $p(t)$ über $\IF_p$. Dann ist $\IF_p\subseteq L$ der Primkörper, also $\chr L = p$. Betrachte $\Fr^r : = \Fr\circ\dots\circ \Fr\colon L\to L$ als den $r$-ten Frobeniushomomorphismus. Sei weiterhin
	\[L^{\Fr^r} := \{x\in L\mid Fr^r(x) = x\} = \{x\in L\mid x^{p^r} = x\} = \{x\in L\mid x^n -x = 0\}\subseteq L.\]
	Man rechnet leicht nach, dass $L^{\Fr^r}\subseteq L$ ein Unterkörper von $L$ ist. Da $L$ ein Zerfällungskörper von $p(t ) = t^n-t$ ist und andererseits alle Nullstellen von $t^n-t$ schon in $L^{Fr^r}$ liegen, gilt $L^{\Fr^r} = L$. Daher ist $|L|\leq n$, weil $t^n-t$ höchstens $n$ verschiedene Nullstellen hat. Nun reicht es zu zeigen, dass $|L| = n$.
	
	Wir behaupten, dass $t^n-t$ genau $n$ verschiedene Nullstellen hat; damit ist dann $\IF := L$ der gesuchte Körper. Angenommen, es gibt weniger als $n$ verschiedene Nullstellen; dann existiert eine mehrfache Nullstelle $a$ in $L$, da $L$ ein Zerfällungskörper ist. Dann gilt aber $p'(a) = 0$. Andererseits ist $p'(t) = nt^{n-1}-1 = -1$, da $n = p^r$ und wir mit Charakteristik $p$ rechnen. Folglich ist $p'$ nie $0$ und wir erhalten einen Widerspruch. Also hat $t^n-t$ genau $n$ verschiedene Nullstellen.	
\end{proof}

Damit ist der \nameref{thm:17.1} gezeigt.

\begin{kor}
	Der Zerfällungskörper von $p(t) = t^n-t\in\IF_p[t]$ \textup(mit $p,n$ wie in \cref{thm:18.1}\textup) ist eindeutig bis auf Isomorphie.
\end{kor}
\begin{proof}
	$\Phi(p^r) := \IF$ ist (ein) Zerfällungskörper von $t^n-t$ und ist eindeutig bis auf Isomorphie von Körpern.
\end{proof}	

Allgemeiner lässt sich der folgende Satz formulieren:
\begin{satz} \label{thm:18.6}
	Seien $K$ und $K'$ Körper und $\phi\colon K\to K'$ ein Ringhomomorphismus. Sei $f(t)\in K[t]\setminus K$ und $L$ bzw. $L'$ Zerfällungskörper von $f(t)$ über $K$ bzw. von $\phi_{*}(f)\in K'[t]\setminus K'$, wobei 
	\begin{eqnarray*}
		\phi_{*}\colon K[t] & \longto & K'[t]\\
		\sum b_it^i & \longmapsto & \sum \phi(b_i)t^i.
	\end{eqnarray*}
	Falls $\phi$ ein Isomorphismus ist, dann existiert ein Ringisomorphismus $\hat \phi\colon L\to L'$ mit $\hat\phi|_K = \phi$.
\end{satz}
\begin{proof}
	Wähle den bis auf Isomorphie eindeutigen algebraischen Abschluss $\overline{K}$ bzw. $\overline{K'}$ von $K$ bzw. $K'$ so, dass $K\subseteq L\subseteq \overline{K}$ und $K'\subseteq L'\subseteq \overline{K'}$. Der \nameref{thm:fortsetz} sagt uns nun, dass ein Homomorphismus $\hat\phi\colon \overline{K}\to\overline{K'}$ mit $\hat\phi|_K = \phi$ existiert. Falls $\phi$ ein Isomorphismus ist, dann ist auch $\hat\phi$ Isomorphismus von Körpern.
	
	Andererseits existieren $a_1,\dots, a_n\in \overline{K}$ und $c\in K$ mit $f(t) = c\prod_{i = 1}^n(t-a_i)$ und damit $\hat\phi(f(t)) = \phi(c)\prod_{i = 1}^n(t-\hat\phi(a_i))$. Also ist $\hat\phi(a_i)$ für $1\leq i\leq n$ eine Nullstelle von $\hat\phi(f(t)) = \phi_{*}(f(t))$. Damit ist $L = K(a_1,\dots, a_n)$, da er ein Zerfällungskörper von $K$ ist, und wird unter $\hat\phi$ auf $L' = K'(\hat\phi(a_1),\dots ,\hat\phi(a_n))$ abgebildet. $\hat{\phi}$ ist offensichtlich surjektiv. Da $\hat{\phi}$ automatisch injektiv ist, ist $\hat\phi$ ein Isomorphismus von Körpern.
\end{proof}

\begin{satz}[Eindeutigkeit des Zerfällungskörpers]
	Sei $K$ ein Körper und $L, L'$ Zerfällungskörper von $K$. Dann existiert ein $K$-Isomorphismus $\hat\phi\colon L\to L'$. 
\end{satz}
\begin{proof}
	Wende \cref{thm:18.6} auf $K' = K$ und $\phi = \id_K$ an.
\end{proof}
\begin{satz}
	Sei $\IF$ ein endlicher Körper sowie $|\IF| = p^r = n$. Dann gilt:
	\begin{itemize}
		\item $\Gal(\IF\sslash\IF_p) = \{\phi\colon \IF\to \IF\mid\text{$\phi$ Ringisomorphismus}\}$.
		\item $\Gal(\IF\sslash\IF_p) \cong (\IZ/n\IZ, +)$ als Gruppe, erzeugt von $\Fr$. Wir werden zeigen, dass daraus folgt, dass eine Bijektion \textup(Galoiskorrespondenz\textup) existiert:
		\begin{eqnarray*}
			\{\text{Untergruppen von $\Gal(\IF\sslash\IF_p)$}\}&\xlongleftrightarrow{1:1}&\{\text{Unterkörper von $\IF$}\}\\
			U &\longmapsto & \IF^U = \{x\in\IF\mid \phi(x) = x \;\forall \phi\in U\}
		\end{eqnarray*}
	\end{itemize}
\end{satz}

\section{Galoistheorie for real}
Ziel: $L\sslash K$ \glqq gute\grqq\ Körpererweiterung. Dann gibt es eine 1:1-Korrespondenz zwischen Untergruppen der Galoisgruppe $Gal(L\sslash K)$ und Zwischenkörpern $K\subseteq$ ? $\subseteq L$.

\subsection{Normale und separable Körpererweiterungen}
\begin{satz}[Spezialfall der Galoiskorrespondenz]\label{thm:19.1} Sei $\IF$ ein endlicher Körper mit $|\IF| = n = p^r$ \textup($p$ Primzahl\textup). Dann definiere $G:= Gal(\IF\sslash\IF_p):=\{\phi\colon \IF\to\IF \text{ Körperisomorphismus}, \phi|_{\IF_p} = \id \}$. Dann existiert eine Bijektion von Mengen
\begin{eqnarray*}
\Phi\colon	\{\text{UG von  }G = Gal(\IF/\IF_p)\} & \xlongleftrightarrow{1:1} &\{\text{ Zwischenkörper }\IF_p\subseteq \text{ ? }\subseteq \IF\}\\
	H &\longmapsto & \IF^H
\end{eqnarray*}
wobei $\IF^H := \{x\in\IF|\phi(x) = x\forall \phi\in H\}$ den Fixkörper bezüglich $H$ bezeichnet.
\end{satz}
\begin{bem}
	\leavevmode
	\begin{enumerate}
		\item $\IF^H$ ist Körper, denn für $x,y\in\IF^H$ ist für alle $\phi\in H: \phi(x\pm y) = \phi(x)\pm\phi(y) = x\pm y$, $\phi(xy) = \phi(x)\phi(y) = xy$, $\phi(x^{-1}) = \phi(x)^{-1} = x^{-1}$, also ist $\Phi$ wohldefiniert
		\item $\phi\in H\Leftarrow \phi$ ist Ringhomomorphismus, also $\phi(1) = 1$ und $\phi|_{\IF_p} = \id$ automatisch (wir müssen es also eigentlich gar nicht fordern).
		
		Also $G = \{\phi\colon \IF\to \IF \text{ Körperhomomorphismus}\}$ in diesem Fall. Insbesondere: $\Fr\colon \IF\to\IF, x\mapsto x^p$ ist in $G$.
	\end{enumerate}
\end{bem}
Vorbereitung zum Beweis von \ref{thm:19.1}:
\begin{lem}
	Gleiche Voraussetzungen wie in Satz \ref{thm:19.1}. Dann ist $G = Gal(\IF/\IF_p)$ zyklisch der Ordnung $r$.
\end{lem}
\begin{proof}
	Wir wissen (nach der Konstruktion endlicher Körper) $\IF = \IF_p(a)$ für ein $a\in \IF$. Nach Bemerkung 2 ist $\phi\in G$ festgelegt durch $\phi(a)$. Sei $m_a(t)\in\IF_p[t]$ Minimalpolynom von $a$.
	
	\noindent 1. Behauptung: $m_a(t) = (t-a)(t-a^p)\dots(t-a^{p^{r-a}})$. Beweis später. Dabei sind die Koeffizienten von $m_a(t)$ in $\IF_p$, also $m_a(\phi(a)) = \phi(m_a(a)) = 0$, denn $\phi|_{\IF_p} = \id\Rightarrow \phi(a)$ Nullstelle von $m_a(t)$. Folglich existiert ein $s$ mit $0\leq s\leq r-1$ mit $\phi(a) = a^{p^s}$. Damit ist $\phi = \Fr^s$ (weil eindeutig bestimmt auf $a$) und $G = \<\Fr\> = \{\id, \Fr, \Fr^2,\dots\}$.
	
	\noindent 2. Behauptung: $\Fr^r = \id_{\IF}$, $\Fr^j\neq \Fr^i$ für $1\leq i<j\leq r$.
	Beweis: Sei $\Fr^j = \Fr^i\Rightarrow \Fr^{j-i} = \id_{\IF}\Rightarrow \Fr^m = \id_{\IF}$ für ein $m<r\Rightarrow x^{p^m} = x\forall x\in \IF\Rightarrow x^{p^m}-x = 0\forall x\in\IF$. Damit muss $p^m\geq|\IF| = p^r$ oder $p^m = 1$ sein, also $m\leq r$ oder $m = 0$ im Widerspruch zu $m<r$ und $i\neq j$.
	
	Wir hatten $\IF$ als Nullstellen von $f(t) = t^n-t$ konstruiert. Also gilt für $x\in \IF$ immer $x^{p^r} -x = 0$ ($p^r = n$), also $\Fr^r = \id_{\IF}$. Damit erhalten wir insgesamt $G = \{\Fr, \Fr^2,\dots, \Fr^r = \id_{\IF}\}$. Wie behauptet ist $G$ folglich zyklisch (erzeugt von $\Fr$) der Ordnung $r$.
\end{proof}

\begin{proof}[Beweis von \ref{thm:19.1}]
	Sei $H<G$; Da $G$ zyklisch ist, ist $H$ zyklisch der Ordnung $k$, wobei $k$ ein Teiler von $r$ ist, also $r = km$, dann $H = \{\Fr^m, \Fr^{2m},\dots, \Fr^{km} = \id\}$. Dann $\Phi(H) = \IF^H = \{x\in\IF|\phi(x) = x\forall\phi\in H\} = \{x\in \IF|\Fr^m(x) = x\} = \{x\in\IF|x^{p^m}-x = 0\}$. Nun betrachte $f\in \IF_p[t], f = t^{p^m}-t$.
	
	\noindent Behauptung: $f$ hat $p^m$ verschiedene Nullstellen in $\IF$. Denn $f' = p^mt^{p^m-1}-1 = -1$ hat keine Nullstelle. Damit ist $|\IF^H| = p^m$ (da $m$ Teiler von $r$ ist, liegen alle Nullstellen in $\IF$ nach Konstruktion von $\IF$). Folglich ist $\Phi$ injektiv.
	
	\noindent Surjektivität: Sei $K\subseteq \IF$ Unterkörper (dann gilt automatisch $\IF_p\subseteq K$). $\Rightarrow K$ ist $\IF_p$-Vektorraum $\Rightarrow |K| = p^m$ für ein $m\leq r$. Da $\IF$ auch $K$-Vektorraum ist, muss $m$ ein Teiler von $r$ sein. Nach der Konstruktion endlicher Körper gilt $x^{p^m}-x = 0$ für alle $x\in K$ und $K$ besteht genau aus diesen Nullstellen. Es folgt, dass $K = \IF^{\Fr^m} = \{x\in \IF|\Fr^m(x) = x\}$. Damit $K = \Phi(H)$ für $H = \<\Fr^m\>$. 
\end{proof}

\begin{satz} $\IF$ endlicher Körper, $\chr \IF = p$. Sei $1\neq a\in\IF$ und $r$ minimal, sodass $a^{p^r} =a$. Dann gilt:
	\begin{enumerate}
		\item $1, a, a^2,\dots, a^{p^{r-1}}$ paarweise verschieden
		\item $m_a(t) = (t-a)(t-a^p)\dots(t-a^{p^r-1}) =: f(t)\in\IF_p[t]$ Minimalpolynom von $a$ über $\IF_p$. \textup(Damit folgt dann die 1. Behauptung aus dem Beweis von Lemma 19.2\textup)
	\end{enumerate}
\end{satz}
\begin{proof}
	\leavevmode
	\begin{enumerate}
\item Folgt aus Punkt 2, weil $r$ minimal gewählt.
\item	Es gilt $m_a(t^p) = m_a(t)^p$ (Binomi für Dummies).
	Damit: Falls $z$ Nullstelle von $m_a(t)$, dann auch $z^p\Rightarrow$ Menge der Nullstellen von $m_a(t)$ sind stabil unter Frobeniusabbbildung $\Fr$.  Also gilt für jede Nullstelle $z$ von $m_a(t)$:  $Q = (t-z)(t-z^p)\dots(t-z^{p^r}-1)$ teilt $m_a(t)$, insbesondere für $z = a$. Also ist $Q_{z = a} = f(t) = m_a(t)$, weil $m_a(t)$ minimal mit Nullstelle $a$ und Leitkoeffizient 1, falls $q\in\IF_p[t]$.
	
	Noch zu zeigen: $Q(t) \in\IF_p[t]$. Wir wissen 
	\begin{equation}Q(t^p) = Q(t)^p\end{equation}
	 Falls $Q(t) = \sum_{i \geq 0}b_it^i\in\IF[t]$, dann $Q(t)^p = \sum_{i \geq 0} b_i^p(t^i)^p = \sum_{i\geq 0} \Fr(b_i)(t^i)^p$. Somit mit $(2)$: $b_i = \Fr(b_i)$ für alle $i$. Nach Konstruktion endlicher Körper ist dadurch $b_i\in \IF_p$ für alle $i$, womit wir schließlich $Q(t) \in\IF_p[t]$.
  \qedhere
	\end{enumerate}
\end{proof}
\begin{defi} $L\sslash K$ heißt einfach, falls $L\cong K(a)$ für ein $a\in L$.
\end{defi}
\begin{satz}
	Sei $K(a)\sslash K$ einfache, algebraische Körpererweiterung und $M\sslash K$ beliebige Körpererweiterung. Dann
	\begin{eqnarray*}
		\{K\text{-Homomorphismen } \phi\colon K(a)\to M\} & \xlongleftrightarrow{1:1} & \{\text{Nullstellen von }m_a(t)\in K[t]\text{ in }M\}\\
		\phi &\longmapsto & \phi(a)
	\end{eqnarray*}
	das heißt zu jeder Nullstelle $z\in M$ von $m_a(t)$ existiert genau ein $K$-Homomorphismus $\phi\colon K(a)\to M$ mit $\phi(a) = z$. 
\end{satz}
\begin{proof}
	\leavevmode
	\begin{description}
		\item[wohldefiniert:] Sei $\phi\colon K(a)\to M$ ein $K$-Homomorphismus, $\phi(a) = z$. Dann folgt $m_a(z) = m_a(\phi(a)) = \phi(m_a(a)) = 0$
		\item[injektiv:] $\phi$ ist eindeutig bestimmt durch $\phi(a)$, weil Körperhomomorphismus, $\phi|_K = \id$.
		\item[surjektiv:] Sei $z\in M$ Nullstelle von $m_a(t)$. Betrachte
		\begin{eqnarray*}
			\ev_z\colon K[t] & \longto & M \text{ Ringhomomorphismus}\\
			f(t) &\longmapsto & f(z)
		\end{eqnarray*}
		Da $z$ Nullstelle von $m_a(t)$ ist, induziert das Ringhomomorphismus
		$$\overline{\ev_z}\colon K[t]/(m_a(t))\longto M$$
		wobei $\beta\colon K[t]/(m_a(t)) \xrightarrow{\cong} K(a)\subseteq K$; $K(a)\sslash K$ algebraisch und $\overline{\ev_z}$ Körperhomomorphismus mit $\overline{\ev_z}\circ \beta|_K = \id_K$ nach Konstruktion. Setze $\phi:= \overline{\ev_z}\circ\beta$ und $\phi(a) = \overline{\ev_z}\circ\beta(a) = \overline{\ev_z}(\overline{t}) = z$.
  \qedhere
	\end{description}
\end{proof}
\begin{kor}
	$K$ Körper, $f\in K[t]$. Dann ist der Zerfällungskörper von $f$ eindeutig bis auf $K$-Isomorphismus.
\end{kor}
\begin{proof}
	Übung.
\end{proof}
\begin{defi} Eine Körpererweiterung $L\sslash K$ heißt normal, falls sie algebraisch ist und falls für $f\in K[t]$ irreduzibel gilt: $f$ hat Nullstelle in $L\Rightarrow f$ zerfällt in Linearfaktoren über $L$.
\end{defi}
\begin{bsp}
	\leavevmode
	\begin{enumerate}
		\item $\IQ(\sqrt{2})\sslash \IQ$ normal (Übung)
		\item $\IQ(\sqrt[3]{2})\sslash \IQ$ ist nicht normal, denn betrachte $f = t^3-2$. Es gilt $\IQ(\sqrt[3]{2})\subseteq \IR$ und $f$ hat Nullstellen $\sqrt[3]{2}\in \IQ(\sqrt[3]{2})$ und $\sqrt[3]{2}\xi, \sqrt[3]{2}\xi^2\notin \IR$, mit $\xi = e^{\frac{2\pi i}{3}}$
	\end{enumerate}
\end{bsp}
\begin{satz}[Normalität] Sei $L\sslash K$ eine endliche Körpererweiterung. Dann sind äquivalent:
\begin{enumerate}
	\item $L\sslash K$ normal
	\item $L$ ist Zerfällungskörper eines Polynoms $f\in K[t]$.
	\item Sei $M\sslash K$ eine Körpererweiterungen . Dann haben alle Körperhomomorphismen $\phi\colon L\to M$, die die Inklusion $K\to M$ fortsetzen, dasselbe Bild, das heißt, folgendes Diagramm kommutiert:
	\begin{tikzcd}
				L \arrow{r}{\phi} & M  \\
				K \arrow{u}\arrow{ru}\\
	\end{tikzcd}
	wobei $K\to L$, $K\to M$ jeweils die Inklusionen sind.
\end{enumerate}
\end{satz}
%\begin{bsp}
%	Das hier ist ein Beispiel-Diagramm:

%	\begin{tikzcd}
%		X \arrow{rd}[swap]{g\circ f} \arrow{r}{f} & Y \arrow{d}{g} \\
%		W \arrow{u}	& Z \\
%	\end{tikzcd}
%\end{bsp}

\end{document}
