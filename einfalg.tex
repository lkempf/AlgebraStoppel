\documentclass[12pt,a4paper]{scrartcl}

\usepackage{includes}
\usepackage{shortcuts}
\usepackage{numbering}

\author{Alexander Esgen \and Lukas Kempf \and Luise Puhlmann}
\title{Einführung in die Algebra}
\subtitle{Wintersemester 2017/18}


\begin{document}
\maketitle
\tableofcontents
\newpage

\lecture{9. Oktober 2017}

\url{http://www.math.uni-bonn.de/people/palmer/A1.html}

\paragraph {Organisatorisches}
\begin{itemize}
	\item Assistent: Martin Palmer
	\item Abgabe der Übungsblätter Donnerstag vor der Vorlesung
	\item Übungsgruppen Beginn nächste Woche
	\item Literatur siehe Homepage
\end{itemize}

\section{Gruppen}
\subsection{Grundlegendes}
\begin{defi} Eine Gruppe ist eine Menge $G$ zusammen mit einer Abbildung
\begin{align*}
	\circ\colon G\times G &\longto G\\
	 (g,h)&\longmapsto g\circ h
\end{align*}
(genannt Gruppenoperation), sodass gilt:
\begin{itemize}
	\item[(G1)]$(a\circ b)\circ c = a \circ (b\circ c) ~ \forall a,b,c\in G$ (Assoziativität)
	\item[(G2)] $\exists e\in G$ mit $g\circ e = g = e\circ g ~ \forall g\in G$ (Neutrales Element)
	\item[(G3)] $\forall g\in G\ \exists g^{-1}$ sodass $g\circ g^{-1} = e = g^{-1}\circ g$ (Inverse Elemente)
\end{itemize}
\end{defi}

\begin{bem}
\leavevmode
\begin{itemize}
	\item Neutrales Element $e$ ist eindeutig
	\item Inverse Elemente $g^{-1}$ sind eindeutig
	\item Es reicht sogar zu fordern: Existenz von Linksneutralem und Linksinversem oder Existenz von Rechtsneutralem und Rechtsinversem.
	\item Es gelten die Kürzungsregeln:
		\begin{eqnarray*}
			a\circ c = b\circ c &\Leftrightarrow& a = b\qquad \forall a,b,c\in G\\
			c\circ a = c\circ b &\Leftrightarrow& a = b \qquad \forall a,b,c\in G
		\end{eqnarray*}
\end{itemize}
\end{bem}

\begin{defi}
	$(G,\circ)$ heißt abelsch, falls $g\circ h = h\circ g$ für alle $g,h\in G$.
\end{defi}

\begin{bsp}
\leavevmode
\begin{itemize}
	\item $(\mathbb Z, +)$
	\item $(K,+,\cdot)$ Körper $\Rightarrow (K,+)$ und $(K^*=K\setminus \{0\}, \cdot)$ sind Gruppen
	\item $(V,+,\cdot)$ $K$-Vektorraum, dann ist $(V,+)$ eine Gruppe
	\item $K$ Körper, $n\in\mathbb N$; $G = \GL_n(K)$ ist Gruppe mit Matrixmultiplikation
	\item $M$ nichtleere Menge; $S_M := \{f\colon M\to M|f ~ \text{invertierbar}\}$ mit $\circ = $ Komposition von Abbildungen ist eine Gruppe; Spezialfall: $M = \{1,\dots n\},\ n\in\mathbb N$ ergibt die symmetrische Gruppe $S_n$ der Ordnung $n!$.
	\item Sei $(G,\circ)$ eine Gruppe und $a\in G$ fest gewählt. Dann ist $(G,\circ_a)$ eine Gruppe, wobei $g\circ_a h = g\circ a\circ h$.
\end{itemize}
\end{bsp}

\begin{defi} 
	$(G,\circ)$ Gruppe. Dann ist die Anzahl $\abs G$ der Elemente von $G$ die Ordnung von $G$.
\end{defi}

\begin{defi}
	 Sei $(G,\circ)$ Gruppe. Eine Teilmenge $H\subseteq G$ heißt Untergruppe (kurz UG), falls $H\neq\emptyset$ und $h_1,h_2\in H\Rightarrow h_1\circ h_2^{-1}\in H$. Wir schreiben dann: $H<(G,\circ)$ oder $H<G$.
\end{defi}

\begin{bem} $H<(G,\circ)$ gilt genau dann, wenn gilt:
	\begin{description}
	\item[(UG0)] $e\in H$
	\item[(UG1)] $h_1,h_2\in H\Rightarrow h_1\circ h_2\in H$
	\item[(UG2)] $h\in H\Rightarrow h^{-1}\in H$
	\end{description}
	Klar: Untergruppen sind Gruppen
\end{bem}

\begin{bsp}[selber nachprüfen!!!]
	\leavevmode
	\begin{itemize}
		\item $2\IZ < (\IZ, +)$
		\item $n\in\mathbb N$; $O(n) = \{A\in \GL_n(\IR)|AA^{T} = \mathds 1_n\}< \GL_n(\IR)$ die orthogonale Gruppe
		\item  $n\in\mathbb N$; $U(n) = \{A\in GL_n(\IC)|A\overline A^{T} = \mathds 1_n\}< \GL_n(\IC)$ die unitäre Gruppe
		\item $SL_n(K) = \{A\in \GL_n(K)|\det(A)=1\}<\GL_n(K)$
		\item $SO(n) = O(n)\cap SL_n(\IR)<O(n)$
		\item Spezielle Unitäre Gruppe
		\item $H(3,\IR) = \left\{\left(\begin{array}{ccc}
			1 & a & b \\ 
			0 & 1 & c \\ 
			0 & 0 & 1
		\end{array}\right) \right\}$: Obere Dreiecksmatrizen, nur 1en auf der Diagonalen (Heisenberggruppe)
	
	\end{itemize}
\end{bsp}

\begin{defi}
	Sei $(G,\circ)$ eine Gruppe. Sei $\emptyset\neq N\subseteq G$. Dann ist $\<N\>$ die kleinste (bzgl. Inklusion) Untergruppe von $G$, die $N$ enthält (also: $H<G$ mit $N\subseteq H\Rightarrow \<N\>\subseteq H$). Wir nennen $\<N\>$ die von $N$ erzeugte Untergruppe von $(G,\circ)$.
\end{defi}

\begin{bem}
	$\<N\>$ ist wohldefiniert, denn seien $H_1, H_2<G$ mit $N\subseteq H_1, N\subseteq H_2$, dann $N\subseteq H_1\cap H_2$ und $H_1\cap H_2<G$. Also existiert kleinste Untergruppe, die $N$ enthält; $\<N\>$ ist wohldefiniert.
\end{bem}

\begin{defi}
	$G$ Gruppe, $N\subseteq G$
	\begin{enumerate}
		\item $N$ erzeugt die Gruppe $G$, falls $\<N\> = G$. In diesem Fall heißt $N$ Erzeugendensystem der Gruppe $G$
		\item $(G,\circ)$ heißt endlich erzeugt als Gruppe, falls $\exists N\subseteq G$ mit $|N|$ endlich und $G = \<N\>$.
	\end{enumerate}
\end{defi}

\begin{bem}
	$(G,\circ)$ Gruppe, sei $N\subseteq G$. Dann gilt: $N$ erzeugt $G$ (also $G = \<N\>$) genau dann, wenn $\forall g\in G : \exists n_1,\dots,n_r\in G$ (mit $r\in \IN_0$), sodass $g = n_1\circ \dots \circ n_r$ (mit $g=e$, falls $r=0$) und $n_i\in N$ oder $n_i^{-1}\in N$ für alle $1\leq i\leq r$ (*).
\end{bem}

\begin{proof}
	\glqq$\Leftarrow$\grqq: Sei $g\in G$ und $g = n_1\circ\dots \circ n_r$ wie in (*). Daraus folgt $g\in \<N\>$, da $n_1,\dots,n_r\in \<N\>$ und dann auch $g$, weil $\<N\>$ Gruppe. Dadurch ist $G\subseteq \<N\>$, also $G = \<N\>$.\\
	\glqq$\Rightarrow$\grqq: Sei $G = \<N\>$. Behauptung: $H:=\{g\in G| g \mbox{ von der Form (*)}\}<G$. (dkddiermsü)
	
	Da $\<N\>\subseteq H$ nach Definition von $\<N\>$ gilt und $\<N\>$ eine Gruppe ist, muss also $\<N\> = H$ wegen Minimalität gelten, da $N\subseteq H$ gilt. Nach Voraussetzung folgt $G = H$. Also hat jedes $g\in G$ die Form (*).
\end{proof}

\begin{bsp}
	\leavevmode
	\begin{itemize}
		\item $\{$Transpositionen$\}\subseteq S_n$, d.h. $(i,j)$ mit $1\leq i<j\leq n$ erzeugen die Gruppe $S_n$
		\item $\{$Einfache Transpositionen$\}\subseteq S_n$, d.h. $(i,j)$ mit $1\leq i<j=i+1\leq n$ erzeugt $S_n$
	\end{itemize}
	
\end{bsp}

\begin{defi}
	Eine Gruppe $G$ heißt zyklisch, falls $\exists g\in G$, sodass $\gen{\{g\}} = G$ (d.h. falls $G$ von einem Element erzeugt wird).
\end{defi}

\noindent Beachte: $\gen{\{g\}} = \{e, g, g^{-1}, g^2, g^{-2},\dots\} = \{g^i|i\in\IZ\}$

\begin{bsp}

 $(\IZ,+)$ ist zyklisch mit $\IZ =\genSet{1} = \genSet{-1}$

\end{bsp}

\begin{defi}
	$(G,\circ)$ und $(G',\circ')$ seien Gruppen. Ein Gruppenhomomorphismus (kurz: Gruppenhomo) von $G$ nach $G'$ ist eine Abbildung $f\colon G\to G'$ mit $f(g\circ h) = f(g)\circ'f(h)\enspace \forall g, h\in G$.
	
	Er ist ein Gruppenisomorphismus (kurz: Gruppeniso), falls zusätzlich $f$ invertierbar ist. Wir schreiben $(G,\circ)\simeq (G',\circ')$, falls ein Gruppenisomorphismus von $G$ nach $G'$ existiert und nennen die Gruppen isomorph.
\end{defi}

\paragraph{Eigenschaften von Gruppenhomomorphismen} $f\colon G\to G'$ von $G$ nach $G'$ sei ein Gruppenhomomorphismus. Dann gilt:
\begin{description}
	\item[(E1)] $f$ Gruppenisomorphismus $\Leftrightarrow$ $f^{-1}$ Gruppenisomorphismus: Nach Definition existiert $f^{-1}$. Zu zeigen: $f^{-1}(g'\circ' h')= f^{-1}(g')\circ f^{-1}(h')$ für alle $g',h'\in G$. Sei $g', h' \in G' $. Daraus folgt $ \exists g, h\in G : f(g) = g', f(h) = h'$. Also:
	\begin{equation*}
		f^{-1}(g'\circ'h')= f^{-1}(f(g)\circ'f(h)) = f^{-1}(f(g\circ h))= g\circ h  = f^{-1}(g')\circ f^{-1}(h')
	\end{equation*} 
	
	\item[(E2)] $f$ bildet Neutrales auf Neutrales ab
	
	\lecture{12. Oktober 2017}
	\item[(E3)] $f$ bildet Inverse auf Inverse ab
	\item[(E4)] Sei $(G'',\circ'')$ eine weitere Gruppe; $f'\colon G'\to G''$ Gruppenhomomorphismus von $(G',\circ')$ nach $(G'',\circ'')$, dann ist $f'\circ f$ Gruppenhomomorphismus. Denn: 
	\begin{equation*}
		(f'\circ f)(g\circ h) = f'(f(g\circ h)) = f'(f(g)\circ'f(h)) = (f'\circ f)(g)\circ''(f'\circ f)(h)
	\end{equation*}
\end{description}

\begin{bsp}[Gruppenhomomorphismus]
	\leavevmode
	\begin{enumerate}
		\item $(G,\circ)$ mit $\mbox{id}\colon G\to G,\ g\mapsto g$ Gruppenhomomorphismus von $(G,\circ)$ nach $(G,\circ)$
		
		\textbf{Achtung} $\mbox{id}\colon G\to G,\ g\mapsto g$ ist kein Gruppenhomomorphismus von $(G,\circ)$ nach $(G,\circ_a)$, falls $a\neq e$
		
		\item $\det\colon \GL_n(K)\to K^*$ für einen Körper $K$ ist ein Gruppenhomomorphismus
		\item $f\colon \IR^*\to \IR_{\geq 0},\ x\mapsto |x|$ Gruppenhomomorphismus von $(\IR^*,\cdot)$ nach $(\IR_{\geq 0}, \cdot)$
		\item $x\mapsto \exp(x)$ Gruppenhomomorphismus von $(\IZ,+)$ nach $(\IR^*,\cdot)$
		\item Betrachte $G = \left\{\left.\left(\begin{array}{cc}
		1 & a \\ 
		0 & 1
		\end{array} \right)\right\vert a\in\IZ\right\}<\GL_n(\IR,\cdot)$ und $f\colon \IZ\to G,\ a\mapsto\left(\begin{array}{cc}
		1 & a \\ 
		0 & 1
		\end{array} \right)$. Das ist ein Gruppenhomomorphismus von $(\IZ,+)$ nach $(G,\mbox{Matrixmultiplikation})$. Es ist sogar ein Gruppenisomorphismus mit Inversem:$\left(\begin{array}{cc}
		1 & a \\ 
		0 & 1
		\end{array} \right)\mapsto a$
		\item \textit{Trivialer Gruppenhomomorphismus}: Schicke alles auf das neutrale Element
		\item Gegeben $(G,\circ)$ Gruppe, $a\in G$. Dann ist $f\colon G\to G,\ g\mapsto g\circ a^{-1}$ ein Gruppenhomomorphismus von $(G,\circ)$ nach $(G,\circ_a)$
	\end{enumerate}
\end{bsp}

\begin{lem}
	Sei $n\in \IZ$.
	\begin{enumerate}
		\item Dann $\exists!$ Gruppenhomomorphismus $can\colon \IZ\to \IZ/n\IZ$ von $(\IZ,+)$ nach $(\IZ/n\IZ,+)$ mit $can(1)=\overline{1}$
		\item Falls $n\neq 0$, existiert kein nichttrivialer Gruppenhomomorphismus $f\colon \IZ/n\IZ\to\IZ$ 
	\end{enumerate}
\end{lem}
\begin{proof}
	\leavevmode
	\begin{enumerate}
		\item
	Eindeutigkeit: Sei $f\colon \IZ\to \IZ/n\IZ$ ein Gruppenhomomorphismus. Dann $f(0)=\overline{0}$ nach (E2) und falls $f(1) = \overline{1}$, dann gilt $f(n)= f(1+\dots 1) = n\cdot f(1)$ für alle $n\in\IN$ und damit auch $f(-n) = -nf(1)$ nach (E5) $\Rightarrow$ $f$ eindeutig.
	
	\noindent Gruppenhomomorphismus: Es gilt dann $\text{can}(x) = \overline{x}$ für alle $x\in\IZ$ und da $\text{can}(x+y) = \overline{x+y} = \overline{x}+\overline{y} = \text{can}(x)+\text{can}(y)$ ist das auch ein Gruppenhomomorphismus.
	
	\item Sei $n\neq 0$. Sei $f\colon \IZ/n\IZ\to \IZ$ ein Gruppenhomomorphismus. Sei $f(\overline{1})= x$. Dann: (oBdA $n\in\IN$) $0=f(0)=f(\overline{n})= f(\overline{1}+\dots \overline{1})=nf(\overline{1})= nx\Rightarrow x=0$. Somit ist $f$ ein trivialer Gruppenhomomorphismus.
	\end{enumerate}
\end{proof}

\begin{lem}
	Sei $(G,\circ)$ eine Gruppe.
	\begin{enumerate}
		\item Sei $\text{Aut}(G) = \{f\colon G\to G| f \text{ Gruppeniso von $(G,\circ)$ nach }(G,\circ)\}$. Dann ist $\text{Aut}(G)$ eine Gruppe, die Automorphismengruppen von $G$
		\item Betrachte die Abbildung $\text{Konj}\colon G\to \text{Aut}(G) ,\ g\mapsto \text{Konj}(g)$, wobei $\text{Konj}(g)(h)= g\circ\ h\circ g^{-1}$ für alle $h\in G$. Dann ist Konj ein Gruppenhomomorphismus von $G$ nach $\text{Aut}(G)$. (Im Allgemeinen nicht injektiv.)
	\end{enumerate}
\end{lem}

\begin{proof}
	einfach nachrechnen
\end{proof}

\begin{bem}
	\leavevmode
	\begin{enumerate}
		\item 	Falls $(G,\circ)$ abelsch, dann ist jede Konjugation die Identität.
		\item $\text{Konj}(g) = \text{id}_G \Leftrightarrow g\in Z(G):=\{x\in G|x\circ y = y\circ x\enspace \forall y\in G\}$
	\end{enumerate}
\end{bem}

\noindent\textbf{Konvention:} Von jetzt an schreiben wir meist $gh$ statt $g\circ h$ und $G$ statt $(G,\circ)$.

\begin{satz} \label{thm:kerim_g}
	Sei $f\colon G\to G'$ Gruppenhomomorphismus. Dann gilt:	
	\begin{equation*}
	\begin{array}{llll}
		\ker(f) & := \{g\in G|f(g) = e\}                  & <G & \mbox{ Kern von }f\\
		\im(f)  & := \{g'\in G'|\exists g\in G\ f(g)=g'\} & <G'& \text{ Bild von }f
	\end{array}
	\end{equation*}
\end{satz}

\begin{proof}
	einfach nachrechnen
\end{proof}
\begin{bsp}
	\leavevmode
	\begin{enumerate}
		\item $\ker(\text{can}\colon \IZ\to \IZ/n\IZ) = n\IZ<\IZ$
		\item $\ker(\text{Konj}\colon G\to \text{Aut}(G)) = Z(G)<G$
		\item $\ker(\det\colon \GL_n(K)\to K^*) = SL_n(K)$
	\end{enumerate}
\end{bsp}

\paragraph{Übung:} $f$ Gruppenhomomorphismus; $f$ ist injektiv genau dann, wenn $\ker f = \{e\}$.

\begin{satz}[Satz von Cayley]
	Sei $G$ eine Gruppe. Dann ist
	\begin{align*}
		\Phi \colon G&\longto S_G\\
		g&\longmapsto \Phi(g)
	\end{align*} mit $\Phi(g)(h) = gh$ für alle $h\in G$ ein injektiver Gruppenhomomorphismus. (Damit kann man $G$ als Untergruppe einer Permutationsgruppe \glqq realisieren\grqq.)
\end{satz}

\begin{proof}
	\leavevmode
	\begin{enumerate}
	\item Wohldefiniert: $\Phi(g)$ ist invertierbar mit Inversem $h\mapsto g^{-1}h$.	
	\item Gruppenhomomorphismus: Zu zeigen: $\Phi(g_1g_2) = \Phi(g_1)\circ \Phi(g_2)$, also $\Phi(g_1g_2)(h) = \Phi(g_1)(\Phi(g_2)(h))$ für alle $h\in G$. Es gilt aber $\Phi(g_1g_2)(h) = g_1g_2h$ und $\Phi(g_1)(\Phi(g_2)(h)) = \Phi(g_1)(g_2h) = g_1g_2h$.
	
	\item Injektiv: Es reicht zu zeigen, dass der Kern trivial ist. Sei $g\in \ker\Phi\Leftrightarrow \Phi(g) = e = \text{id}_G \Leftrightarrow \Phi(g)(h)= h ~ \forall h\in G\Leftrightarrow gh = h ~ \forall h\in G\Leftrightarrow g= e$
	\end{enumerate}
\end{proof}


\subsection{Satz von Lagrange und Normalteiler}
\begin{defi}
	Sei $G$ eine Gruppe, $H<G$ eine Untergruppe und $a\in G$. Dann ist:
	\begin{itemize}
		\item[] $aH = \{ah|h\in H\}\subseteq G$ die Linksnebenklasse von $H$ zu $a$
		\item[] $Ha = \{ha|h\in H\}\subseteq G$ die Rechtsnebenklasse von $H$ zu $a$
	\end{itemize}
	Meist arbeiten wir mit Linksnebenklassen und nennen sie einfach Nebenklassen.
\end{defi}

\noindent
Aus der Linearen Algebra wissen wir folgendes: \begin{enumerate}
	\item Zwei Nebenklassen sind gleich oder disjunkt d.h. $aH\cap bH \neq \emptyset \Leftrightarrow aH = bH\Leftrightarrow b^{-1}a \in H$
	\item Die Abbildung $aH\to H,\ ah\mapsto h$ ist bijektiv $\Rightarrow$ alle Nebenklassen haben dieselbe Kardinalität
	\item $$ G = \bigcup\limits_{g\in G}gH = \overset{.}{\bigcup\limits_{b\in R} }bH$$, wobei $R\subseteq G$, sodass die $bH$ mit $b\in R$ genau ein Repräsentantensystem für die verschiedenen Nebenklassen bilden.
	\item $g\in aH\Leftrightarrow g^{-1}\in Ha^{-1}$ (dadurch ergibt sich eine Bijektion zwischen Links- und Rechtsnebenklassen)
	
\end{enumerate}

\begin{defi}
	Bezeichne mit $G/H$ die Menge der (Links)Nebenklassen von $G$ bezüglich $H$ und mit $ H\backslash G$ die Menge der Rechtsnebenklassen. Dann gilt $|G/H| = |H\backslash G|$ (nach (4)). Wir nennen diese Zahl den Index, auch $(G:H)$, von $H$ in $G$
\end{defi}

\begin{satz}[Satz von Lagrange] \label{thm:lagrange}
	Sie $G$ eine Gruppe, $H<G$ eine Untergruppe und gelte $|G|<\infty$. Dann gilt
	\begin{equation}
		|G| = |H|\cdot (G:H)
	\end{equation}
	Insbesondere: $|G| = p$ Primzahl $\Rightarrow H = \{e\}$ oder $H = G$.
\end{satz}

\begin{proof}
	Die Formel folgt direkt aus (3), (2) und der Definition des Index.
	Falls nun $|G| = p$ gilt, so muss auch $|H| = 1$ oder $|H| = p$ gelten, woraus $H = \{e\}$ oder $H = G$ folgt.
\end{proof}

\noindent Noch mehr Wissen aus der Linearen Algebra: Falls $G$ abelsch ist, dann ist $G/H$ wieder eine Gruppe mit Gruppenoperation
\begin{align*}
	\circ \colon G/H\times G/H &\longto G/H\\
	(aH,bH)&\longmapsto abH
\end{align*}
Im Allgemeinen (falls $G$ nicht abelsch ist) ist $\circ$ nicht wohldefiniert (siehe Übungsblatt 2).

\begin{defi}
	Sei $G$ eine Gruppe. Eine Untergruppe $H<G$ heißt Normalteiler, falls gilt: 
	\begin{equation*}
	\forall g\in G, h\in H: g\circ h\circ g^{-1}\in H
	\end{equation*}
	Wir schreiben dann: $H\vartriangleleft G$.
\end{defi}
\begin{bem}
	Falls $G$ abelsch, dann ist jede Untergruppe Normalteiler.
\end{bem}

\begin{lem} \label{lem:ker_nt}
	Sei $f\colon G\to G'$ ein Gruppenhomomorphismus.  Dann: $\ker(f)\nt G$.
\end{lem}
\begin{proof}
	Sei $g\in G$ und $h\in \ker f$. Dann gilt:
	\begin{align*}
		&f(ghg^{-1}) = f(g)f(h)f(g)^{-1} = f(g)f(g)^{-1} = e \\
		&\Rightarrow ghg^{-1}\in \ker f \\
		&\Rightarrow \ker f\nt G
	\end{align*}
\end{proof}



\lecture{16. Oktober 2017}

\begin{satz} \label{thm:nt}
	Sei $G$ eine Gruppe, $N\nt G$ ein Normalteiler. Dann gilt:
	\begin{enumerate}
		\item $G/N$ bilden Gruppe mit $\circ\colon G/N\times G/N \to G/N,\enspace (aN, bN)\mapsto abN$.
		\item Die Abbildung 
		\begin{eqnarray*}
			\can\colon G \longto& G/N\\
			g \longmapsto& gN
		\end{eqnarray*}
	ist ein surjektiver Gruppenhomomorphismus.
	\end{enumerate}
\end{satz}

\begin{proof}
	\leavevmode
	\begin{enumerate}
		\item Es gilt $(aN\circ bN)\circ cN = abN\circ cN = abc N = aN\circ (bN\circ cN) \Rightarrow$ (G1).
		Offensichtlich ist $eN = N$ neutrales Element $\Rightarrow$ (G2).
		$a^{-1}N$ ist das Inverse zu $aN$ $\Rightarrow$ (G3).
		
		Jetzt ist noch die Wohldefiniertheit zu zeigen. Sei also $a_1N = a_2N$ und $b_1N = b_2N$. Daraus sollte $a_1b_1N = a_2b_2N$ folgen.
		
		Tatsächlich gilt $a_1^{-1}a_2 \in N$ und $b_1^{-1}b_2\in N$. Dann gilt auch $(a_1b_1)^{-1}(a_2b_2) = b_1^{-1}a_1^{-1} a_2b_2$, wobei $a_1^{-1}a_2\in N$ und 
		\begin{align*}
		&b_1^{-1}a_1^{-1} a_2b_2 = b_1^{-1}b_2(b_2^{-1}a_1^{-1}a_2b_2)\in N \\
		&\Rightarrow (a_1b_1)^{-1}a_2b_2\in N \\
		&\Rightarrow a_1b_1N = a_2b_2N
		\end{align*}
		
		
		\item Surjektivität ist klar nach (3); um zu zeigen, dass das ein Gruppenhomomorphismus ist, muss man das einfach nachrechnen.
	\end{enumerate}
\end{proof}

\begin{bem}
	Somit gilt, dass Normalteiler genau die Kerne von Gruppenhomomorphismen sind.
\end{bem}

\begin{satz}[Homomorphiesatz] \label{thm:homsatz_g}
	Sei $f\colon G\to H$ ein Gruppenhomomorphismus. Sei $N\nt G$ ein Normalteiler. Dann: $N\subseteq \ker(f)\Leftrightarrow \exists!$ Gruppenhomomorphismus $\overline{f}\colon G/N\to H$, sodass $\overline{f}\circ \can = f$. Also 
	
	\begin{center}
	\begin{tikzcd}
		G \arrow{rd}[swap]{\can} \arrow{r}{f} & H  \\
				 	& G/N \arrow{u}[swap]{\exists! \overline{f}\text{ Gruppenhomo}} \\
	\end{tikzcd}
	\end{center}
\end{satz}


\begin{proof}
	\glqq $\Leftarrow$\grqq: $\ker(\can) = \{g\in G|gN = N\} = \{g\in G|g\in N\} = N \Rightarrow f(N) = \overline{f}(\can(N)) = \overline{f}(e) = e\Rightarrow N\subseteq \ker(f)$.
	
	\noindent \glqq $\Rightarrow$\grqq: Eindeutigkeit: Es muss für $\overline{f}$ gelten: $\overline{f}(aN)=\overline{f}(\can(a)) = f(a)\enspace \forall aN\in G/N\Rightarrow \overline{f}$ eindeutig bestimmt durch $f$.
	
	Existenz: Wir setzen $\overline{f}(aN): = f(a)\enspace \forall aN\in G/N$. Das ist offensichtlich wohldefiniert. Nachrechnen ergibt, dass es auch ein Gruppenhomomorphismus ist.
\end{proof}

\begin{kor}
	Sei $f\colon G\to H$ ein Gruppenhomomorphismus. Dann gilt $G/\ker f \cong \im f$.
\end{kor}
\begin{proof}
	$\ker f\nt G$ nach \cref{lem:ker_nt} $\Rightarrow G/\ker f$ ist eine Gruppe nach \cref{thm:nt}. $\im f$ ist eine Gruppe nach \cref{thm:kerim_g}. Setze $N:= \ker f$. Klar: $N\subseteq \ker f$. Also existiert nach \cref{thm:homsatz_g} ein $\overline{f}$, sodass
	
	\begin{center}
	\begin{tikzcd}
		G \arrow{rd}[swap]{\can} \arrow{r}{f} & H  \\
		& G/\ker f \arrow{u}[swap]{\exists! \overline{f}\text{ Gruppenhomo}} \\
	\end{tikzcd}
	\end{center}
	
	Also haben wir $\overline{f}\colon G/\ker f\to \mbox{im}f$ ein Gruppenhomomorphismus. Er ist surjektiv, weil $\can$ surjektiv ist. \\
	Behauptung: $\overline{f}$ ist injektiv.
	
	Es gilt $\overline{f}(aN)=f(a) = e \Leftrightarrow a\in \ker f = N$. Also $\ker \overline{f} = \{N\}$, was das neutrale Element in $G/\ker f$ ist. Also ist $\overline{f}$ injektiv. $\Rightarrow \overline{f}$ ist Gruppenisomorphismus.
\end{proof}

\begin{satz}[1. Isomorphiesatz] \label{thm:iso1_g}
	Sei $G$ eine Gruppe, $H<G$, $N\nt G$. Es gilt:
	\begin{enumerate}
		\item $HN:=\{hn|h\in H, n\in N\}<G$
		\item $N\nt HN$, $(H\cap N)\nt H$
		\item $H/(H\cap N) \cong HN/N$ mit dem Gruppenisomorphismus $h(H\cap N)\mapsto hN$
	\end{enumerate}
\end{satz}	
\begin{proof}
	\leavevmode
	\begin{enumerate}
		\item $HN\neq \emptyset$, da $e = ee\in HN$. Seien $h_1n_1,h_2n_2\in HN$ ($h_i\in H, n_i\in N$). Dann ist $h_1n_1(h_2n_2)^{-1} = h_1n_1n_2^{-1}h_2^{-1} = h_1h_2^{-1}h_2n_1n_2^{-1}h_2^{-1}$, wobei $n_1n_2^{-1}\in N$. Somit gilt auch $h_2n_1n_2^{-1}h_2^{-1}\in N$, da $N\nt G$. Da $h_1h_2^{-1}\in H$ ist der gesamte Ausdruck Element von $HN$.
		\item Zunächst zeigen wir, dass $N\nt HN$. Es gilt $N\subseteq HN$, da $n = en$. Daraus folgt, dass $N<HN$ weil $N<G$; analog auch $N\nt HN$, weil $N\nt G$.
		
		Noch zu zeigen: $(H\cap N)\nt H$. Es ist offensichtlich, dass $(H\cap N)\subseteq H$ und $(H\cap N)<H$, weil $(H\cap N)<G$. Sei $x\in H\cap N$, $h\in H$. Dann gilt $hxh^{-1}\in H$, weil $H<G$ und $hxh^{-1} \in N$ gilt, da $N\nt G$. Also $hxh^{-1}\in (H\cap N) \Rightarrow H\cap N\nt H$
		\item Betrachte 
		\begin{align*}
			f\colon H \longto& HN \xrightarrow{\can} HN/N\\
			h \longmapsto &he
		\end{align*}
		Es lässt sich leicht nachprüfen, dass $f$ ein Gruppenhomomorphismus ist. Für $x\in H$ gilt $x\in \ker(f)\Leftrightarrow xeN = N\Leftrightarrow x = xe\in \ker(\can) = N\Leftrightarrow x\in (H\cap N)$. Also existiert nach dem \nameref{thm:homsatz_g} ein Gruppenhomomorphismus $\overline{f}$:
		\begin{equation*}
			\overline{f}\colon H/(H\cap N)\longto (HN)/N
		\end{equation*}
		Dieser ist nach Konstruktion injektiv.
		
		Surjektiv: Sei $hnN\in (HN)/N$ mit $h\in H, n\in N$. Dann gilt aber: $hnN = hN$ und dann $f(h)=hN$. Somit gilt $\overline{f}\circ\can(h) = \overline{f}(\can(h)) = hN$ woraus folgt, dass $hN\in \im f$. Folglich ist $\overline{f}$ surjektiv und deshalb ein Gruppenisomorphismus.
	\end{enumerate}
\end{proof}

\noindent Anmerkung zu Beweis des Homomorphiesatzes: Wo wird in \glqq$\Rightarrow$\grqq verwendet, dass $N\subseteq \ker f$? Es wird benötigt für die Wohldefiniertheit von $\overline{f}$.

\begin{satz}[2. Isomorphiesatz] \label{thm:iso2_g}
	Sei $G$ eine Gruppe; $N_1\nt G$, $N_2\nt G$, $N_1\subseteq N_2$. Dann gilt $N_1\nt N_2$ und $N_2/N_1\nt G/N_1$ und es gilt:
	$$(G/N_1)/(N_2/N_1) \cong G/N_2$$ durch den Isomorphismus $(gN_1)N_2/N_1\mapsto gN_2$.
\end{satz}	

\begin{proof}
	$G/N_1$ ist eine Gruppe, weil $N_1\nt G$. Analog für $N_2$. Auch gilt $N_2/N_1\subseteq G/N_1$. Aus $N_1\subseteq N_2$ folgt, dass $N_1\nt N_2$, weil $N_1\nt G$. Sei
	\begin{align*}
		f\colon G/N_1  \longto &G/N_2\\
		gN_1  \longmapsto & gN_2
	\end{align*}
	Das ist wohldefiert: Seien $g, h\in G$.
	\begin{align*}
		&gN_1 = hN_1 \\
		&\Rightarrow g^{-1}h\in N_1\subseteq N_2 \\
		&\Rightarrow gN_2 = hN_2 \\
		&\Rightarrow \text{wohldefiniert}
	\end{align*}
	Klar: $f$ ist surjektiv und $gN_1\in \ker(f)\Leftrightarrow gN_2 = N_2\Leftrightarrow g\in N_2$. Also gilt $\ker(f) = \{gN_1|g\in N_2\} = N_2/N_1$. Also insbesondere $N_2/N_1\nt G/N_1$.
	Nach dem Korollar des Homomorphiesatzes erhalten wir einen Gruppenhomomorphismus
	$$ \overline{f}\colon (G/N_1)/\ker f(=N_2/N_1)\longto \im f = G/N_2 \mbox{ (da $f$ surjektiv)}$$
	Nach Kosntruktion ist $\overline{f}$ injektiv, also erhalten wir den gewünschten Gruppenisomorphismus mit $\overline{f}(gN_1\cdot (N_2/N_1)) = f(gN_1) = gN_2$.
\end{proof}

\paragraph{Anwendungen}
\begin{enumerate}
	\item \textit{Anzahlformel:} Sei $G$ eine endliche Gruppe, $H<G$, $N\nt G$. Dann gilt $$|HN| =\frac{ |H||N|}{|H\cap N|}$$ 
	Denn nach dem \nameref{thm:lagrange} ist $|H| = |H\cap N|(H:H\cap N)$ und $|HN| = |N|(HN:N)$. Nach dem \nameref{thm:iso1_g} ist $(H:H\cap N)=(HN:N)$. \hfill $\checkmark$
	\item Sie $(G,\circ) = (\IZ, +)$, $m,n\in\IN$ und $m|n$. Wir wissen: $m\IZ<\IZ$ und $n\IZ<\IZ$ (sogar Normalteiler, weil $G$ abelsch ist). Klar ist: $n\IZ\subseteq m\IZ$ (insbesondere auch $n\IZ\nt m\IZ$). Dann gilt 
	$$(\IZ/n\IZ)/(m\IZ/n\IZ) \cong \IZ/m\IZ$$
\end{enumerate}

\subsection{Zyklische Gruppen}
Wir schreiben kurz $\<g\>$ statt $\<\{g\}\>$.
\begin{satz}
	Untergruppen von zyklischen Gruppen sind zyklisch.
\end{satz}

\begin{proof}
	Sei $G$ eine zyklische Gruppe; $G = \<g\>$ mit $g\in G$. Sei $H<G$.\begin{description}
		\item[Fall 1] $H = \{e\} = \<e\>$, also zyklisch
		\item[Fall 2] $H\neq \{e\}\Rightarrow \exists m\in \IZ\setminus\{0\}: e\neq g^m\in H\Rightarrow \exists n\in \IN: e\neq g^n\in H$ (weil $H<G$). Wähle $n := \min\{j\in \IN|e\neq g^j\in H\}$. Behauptung: $H = \<g^n\>$.
		
		\glqq$\supseteq$\grqq: Klar, da $g^n\in H$
		
		\glqq$=$\grqq: Angenommen, Gleichheit gilt nicht. Also $\exists s\in \IZ: g^s\in H\setminus\<g^n\>$ (beachte $G = \<g\>$). Schreibe $s = an+r$ für $a,r\in \IZ$ und $0\leq r<n$. Falls $r = 0$, dann $s = an$ und $g^s = g^{an} = (g^n)^a\in \<g^n\>$ Widerspruch!
		
		Falls $r>0$: Dann $g^r = (g^{an})^{-1}g^{an}g^r = ((g^n)^a)^{-1}g^s\in H$ (Widerspruch zur Minimalität)
		
		Somit war die Annahme falsch und $H$ ist zyklisch.
	\end{description}
\end{proof}

\lecture{19. Oktober 2017}

\begin{lem}
	Bilder von zyklischen Gruppen und Gruppenhomomorphismen sind zyklisch.
\end{lem}

\begin{proof}
	Sei $f: G \rightarrow G'$ ein Gruppenhomomorphismus und sei $G$ zyklisch, also $G = \<g\>$ für ein $g \in G$ $\Rightarrow G = \left\lbrace g^i | i \in \IZ \right\rbrace$ also $f(G) = \left\lbrace f(g^i) | i\in \IZ \right\rbrace = \left\lbrace (f(g^i)) | i \in \IZ)\right\rbrace = \<f(g)\> \Rightarrow \im f = \<f(g)\>$ zyklisch.
\end{proof}

\begin{lem} \label{lem:ord}
	Sei $G$ endliche Gruppe $\abs G = n < \infty$. Sei $g \in G$ mit $G=\<g\>$ (also $G$ zyklisch).
	Sei $\ord(g) = \min \left\lbrace j \in \IN | g^j = e\right\rbrace $.
	Dann gilt: $\ord(g) = n$. 
\end{lem}

\begin{defi}
	Allgemeiner: Sei $G$ irgendeine Gruppe, $g \in G$. Dann definiere \begin{equation*}
		\ord(g) := \begin{cases*}
		\min \left\lbrace j \in \IN | g^j = e\right\rbrace &falls das existiert \\
		\infty &sonst
		\end{cases*}
	\end{equation*}
	Wir nennen $\ord(g)$ die Ordnung von $g \in G$.
\end{defi}

\begin{proof} [Beweis von Lemma \ref{lem:ord}]
	\begin{enumerate}
	\item Behauptung: $\ord(g)$ existiert. Angenommen es existiert nicht, also 
	$
		g^j \not = g ~ \forall j \in \IN \Rightarrow g^i \neq g^j \text{ falls } i \neq j, ~ i,j \in \IN 
	$
	(denn sonst gilt $g^{i-j}=e=g^{j-i}$ mit $i-j \in \IN$ oder $j-i \in \IN$).
	Also $\abs G = \infty \Rightarrow$ Widerspruch.
	
	Jetzt ist noch zu zeigen, dass $n = \ord(g)$ gilt. Dazu sei $S := \left\lbrace g, g^2, \ldots, g^{\ord(g)} = e \right\rbrace \subset G$.
	
	\item Behauptung: $S < G$. Klar: $e \in S$. Sei $g^a, g^b \in S$. Schreibe $a-b = k \cdot \ord(g) + r$, wobei $k, r \in \IZ, 0 \leq r < \ord(g)$. Daraus folgt
	\begin{equation*}
		g^a\left( g^b \right)^{-1}=g^{a-b}=g^{k \cdot \ord(g)+r}
		= \left(g^{\ord(g)} \right)^kg^r = e^kg^r=eg^r=g^r \in S
	\end{equation*}
	weil $0 \leq r < \ord(g)$.
	Da $g \in S$, gilt $\<g\>\subset S$. Weil $S < G$ ist klar, dass $S \subset \<g\>$, also $\<g\> = S$.
	\item Behauptung: $\abs S = \ord(g)$. Seien $g^i, g^j \in S$ mit $1 \leq i,j \leq \ord(g)$ und $g^i=g^j$. Also $g^{i-j} = e = g^{j-i}$, was ein Widerspruch zur Minimaltität von $\ord(g)$ ist außer $i=j$. Folglich sind die $g^i (1\leq i \leq \ord(g))$ paarweise verschieden, was die Behauptung zeigt.
	\end{enumerate}
\end{proof}

\begin{bem}
	Sei $G$ irgendeine Gruppe, $g\in G$. Dann gilt: $\ord(g) = \abs{\<g\>}$ und nach \nameref{thm:lagrange} dann $\ord(g)$ teilt $\abs G$, falls $\abs G$ endlich.
\end{bem}

\begin{satz}[Zyklische Gruppen]
	Je zwei zyklische Gruppen der selben Ordnung sind isomorph. Genauer gilt für $G$ zyklische Gruppe: 
	\begin{equation*}
		G \cong \begin{cases*}
			\IZ &falls $\abs G = \infty$ \\
			\IZ/n\IZ &falls $\abs G = n$
		\end{cases*}
	\end{equation*}
\end{satz}

\begin{proof}
	Sei $G = \<g\>$ mit $g \in G$. Sei $f: \IZ \rightarrow G : j \mapsto g^j$. Dann ist $f$ ein Gruppenhomomorphismus (nachrechnen) und surjektiv, da $G = \<g\>$.
	\begin{itemize}
		\item[Fall 1] $\abs G = \infty$. Dann muss $f$ injektiv sein, damit $f$ ein Isomorphismus ist und damit $\IZ \cong G$. Falls $f$ nicht injektiv ist, dann $\exists i,j \in \IZ, i\neq j$ mit $g^i=g^j$, als $g^{i-j} = e = g^{j-i}$. Folglich ist $\ord(g) < \infty$. Damit wäre $G$ nach \ref{lem:ord} endlich, was ein Widerspruch ist.
		\item[Fall 2] $\abs G = n$ endlich. Dann folgt aus \ref{lem:ord}: 
		\begin{equation*}
			\ord(g)=n \Rightarrow g^n = e \Rightarrow g^{nk} = (g^n)^k = e^k = e ~ \forall k\in \IZ \Rightarrow n\IZ \subset \ker F
		\end{equation*}
		Nach dem Homomorphiesatz gilt dann: 
		\begin{center}
		\begin{tikzcd}
					\IZ \arrow{rd}[swap]{\text{can}} \arrow{r}{f} & G  \\
						& \IZ/n\IZ \arrow{u}[swap]{\exists!\ \overline{f}\text{ Gruppenhomo}} \\
		\end{tikzcd}
		\end{center}
		 Also $\overline{f} : \IZ/n\IZ \rightarrow G$. Da $\abs{\IZ/n\IZ} = n = \abs G$ muss diese surjektive Abbildung schon ein Isomorphismus sein.
	\end{itemize}
\end{proof}

\subsection{Auflösbare Gruppen}
\begin{defi}
	Eine Normalreihe eine Gruppe $G$ ist eine Kette von Untergruppen der Form $\left\lbrace e\right\rbrace = G_0 \nt G_1 \nt \ldots \nt G_n = G$. Man nennt die Quotientengruppe $G_i/G_{i-1}$ die Faktoren der Normalreihe.
\end{defi}

\begin{defi}
	Eine Gruppe heißt auflösbar, falls eine Normalreihe mit abelschen Faktoren existiert.
\end{defi}

\begin{bsp}
	\leavevmode
	\begin{enumerate}
		\item Abelsche Gruppen sind auflösbar: $\left\lbrace e \right\rbrace \nt G$ und $G/\left\lbrace e \right\rbrace \cong G $, also abelsch
		\item Sei $G = \left\lbrace \begin{pmatrix}
		a & b \\ 
		0 & d
		\end{pmatrix} \in \GL_2(K) \right\rbrace < \GL_2(K) $. Behauptung: $G$ ist auflösbar. Dazu betrachtet man $G' = \left\lbrace \begin{pmatrix}
		a & 0 \\
		0 & d
		\end{pmatrix} \in \GL_2(K)\right\rbrace < \GL_2(K)$, wobei $G'$ insbesondere eine Gruppe ist. 
		\begin{equation*}
			f: G \longto G': \begin{pmatrix}
			a & b \\ 
			0 & d
			\end{pmatrix} \longmapsto \begin{pmatrix}
			a & 0 \\ 
			0 & d
			\end{pmatrix}
		\end{equation*}
		was ein Gruppenepimorphismus ist (nachrechnen). Es gilt: 
		\begin{equation*}
			\ker f = \left\lbrace \begin{pmatrix}
			1 & b \\ 
			0 & 1
			\end{pmatrix} \lvert b \in K \right\rbrace \nt G
		\end{equation*}
		Folglich gilt $\ker f \cong (K, +)$, sodass $\begin{pmatrix}
		1 & b \\ 
		0 & 1
		\end{pmatrix} \longmapsto b$, weil $\begin{pmatrix}
		1 & b \\ 
		0 & 1
		\end{pmatrix}\begin{pmatrix}
		1 & b' \\ 
		0 & 1
		\end{pmatrix}=\begin{pmatrix}
		1 & b + b' \\ 
		0 & 1
		\end{pmatrix}$ als Gruppenhomomorphismus offensichtlich bijektiv ist. Damit ist $\ker f$ abelsch und $G'$ somit auch.
		\begin{equation*}
			\Rightarrow \left\lbrace e\right\rbrace  = G_0 \nt \ker f = G_1 \nt G_2 = G
		\end{equation*}
		und $\ker f /\left\lbrace e\right\rbrace$ abelsch, sowie auch $G/\ker f \cong \im f = G'$ abelsch. Somit ist $G$ auflösbar.
		\item $S_4$ ist auflösbar. Betrachte
		\begin{equation*}
			S_4 > A_4 := \left\lbrace \pi \in S_4 | \sgn(\pi) = 1 \right\rbrace 
		\end{equation*}
		Nach LA 1 ist $\sgn$ ein Gruppenhomomorphismus und damit $A_4 = \ker(\sgn) < S_4$. Es gilt $S_4 \nt A_4$, weil $A_4 = \ker(\sgn)$ oder weil $(S_4 - A_4) = 2$, was dann nach Blatt 2 folgt. Betrachte nun 
		\begin{equation*}
			A_4 > V_4 := \left\lbrace e, \underbrace{(1,2)(3,4)}_a, \underbrace{(1,3)(2,4)}_b, \underbrace{(1,4)(2,3)}_c\right\rbrace
		\end{equation*}
		Gruppentafel:
		$\begin{array}{c||c|c|c}
			& a & b & c \\ \hline \hline 
			a & e & c & b \\ 
			\hline 
			b & c & e & a \\ 
			\hline 
			c & b & a & e \\ 
		\end{array}$ \\
		Dann gilt $A_4 \nt V_4$, da folgendes gilt: 
		\begin{equation*}
			\forall  \pi \in S_4: \pi \circ \underbrace{(a_1, a_2)(a_3, a_4)}_\tau \circ \pi^{-1} = (\pi(a_1), \pi(a_2))(\pi(a_3), \pi(a_3))
		\end{equation*}weil
		\begin{alignat*}{3}
			\pi(a_1)&\xmapsto{\pi^{-1}}&a_1&\xmapsto{\tau}&a_2&\xmapsto{\pi}\pi(a_2) \\
			\pi(a_2)&\longmapsto&a_2&\mapsto& a_1&\mapsto\pi(a_1) \\
			\pi(a_3)&\longmapsto&a_3&\mapsto& a_4&\mapsto\pi(a_4) \\
			\pi(a_4)&\longmapsto&a_4&\mapsto& a_3&\mapsto\pi(a_3) \\
		\end{alignat*} also $V_4 \nt A_4$. Folglich haben wir
		\begin{equation*}
			\left\lbrace e\right\rbrace = G_0 \nt V_4 = G_1 \nt A_4 = G_2 \nt S_4 = G_3
		\end{equation*}
		\begin{description}
			\item[$G_1/G_0 \cong V_4$] ablesch
			\item[$G_2/G_1 \cong \IZ/2\IZ$] also abelsch, da jede Gruppe $H$ der Ordunung 2 zyklisch mit $H = \<g\> (g \neq e)$ ist und dann nach Klassifikationssatz $H \cong \IZ/2\IZ$
			\item[$G_3/G_2$] Wir wissen, dass $\abs{G_3/G_2} = 3$. Dann behaupten wir, dass $G_3/G_2 \cong \IZ/3\IZ$. Jede Gruppe $H$ mit $\abs H = 3$ ist zyklisch, denn $\<g\> < H (g \neq e)$. Nach dem \nameref{thm:lagrange} gilt $\<g\> = H$, weil $\<g\> \neq e$ und 3 prim ist. Also folgt die Aussage aus dem Klassifikationssatz.
		\end{description}
		Daraus folgt, dass $S_4$ auflösbar ist.
	\end{enumerate}
\end{bsp}

	\begin{satz}
		Untergruppen und Bilder unter Gruppenhomomorphismen von auflösbaren Gruppen sind auflösbar.
	\end{satz}
	\begin{proof}
		Sei $G$ auflösbar. Dann existiert eine Auflösung 
		\begin{equation*}
			\left\lbrace e\right\rbrace = G_0 \nt G_1 \nt \ldots \nt G_n = G, \qquad  G_i/G_{i-1} \text{ abelsch.}
		\end{equation*}
		\begin{description}
			\item[Untergruppe:]~
				\begin{enumerate}
				\item Sei $U < G$. Behauptung: $\left\lbrace e\right\rbrace = G_0 \cap U \nt (G_1 \cap U) \nt \ldots \nt (G_n \cap U) = U $. Es ist klar, dass $(G_{i-1} \cap U) \subset (G_1 \cap U)$. Auch klar ist, dass $G_i \cap U$ eine Gruppe ist und $(G_{i-1} ) < (G_1 \cap U)$. Jetzt ist noch zu zeigen, dass $(G_{i-1} \cap U) \nt (G_i \cap U)$. Sei $x \in G_{i-1} \cap U$ und sei $y \in G_i \cap U$. Dann folgt, dass $\underbrace{yxy^{-1}}_{\in G_{i-1}} \in U$, weil $x, y \in U, U < G$, weil $x \in G_{i-1}, y \in G_i$ und $G_{i-1} \nt G_i$. Daraus folgt, dass $yxy^{-1} \in U \cap G_{i-1}$, was zu zeigen war.
				\item Behauptung: $G_i \cap U / G_{i-1} \cap U$ abelsch. Es gilt $G_i \cap U / G_{i-1} \cap U \stackrel{\text{1. Iso}}{\cong} (U \cap G_i)G_i/G_i \nt G_i/G_{i-1}$ abelsch. Daraus folgt die Behauptung.
			\end{enumerate}
\lecture{23. Oktober 2017}
			\item[Bild:]
				Sei $f\colon G\to G'$ Gruppenhomomorphismus. Behauptung: $\{e\} = f(G_0)\nt f(G_1)\nt \dots \nt f(G_n ) = f(G)$ ist eine Normalreihe mit $f(G_i)/f(G_{i-1})$ abelsch.
				
				Sei $y' = f(y)\in f(G_i)$ mit $y\in G_i$. Dann gilt $y'f(G_{i-1})(y')^{-1} = f(yG_{i-1}y^{-1})\subseteq f(G_i)\Rightarrow f(G_{i-1})\nt f(G_i)$ für alle $i$. Betrachte nun 
$$\alpha\colon G_i\xto{f} f(G_i)\xto{\text{can}}f(G_i)/f(G_{i-1})$$ Dies ist ein Gruppenhomomorphismus und offensichtlich surjektiv. Da $G_{i-1}\in \ker \alpha$ existiert Gruppenhomomorphismus $\overline{\alpha}\colon G_i/G_{i-1}\to f(G_i)/f(G_{i-1})$ nach dem \nameref{thm:homsatz_g}. $\overline{\alpha}$ ist surjektiv, da auch $\alpha$ surjektiv ist. Weil $G_i/G_{i-1}$ abelsch ist, ist auch $f(G_i)/f(G_{i-1})$ abelsch. Somit folgt die Behauptung.
		\end{description}
\end{proof}


\begin{defi}
	Sei $G$ eine Gruppe, $M := \{ghg^{-1}h^{-1}|g,h\in G\}$; dann heißt $[G,G] = \<M\>$ Kommutatorgruppe.
\end{defi}

\begin{bem}
	Nach dem 2. Übungsblatt gilt $[G,G] \nt G$. $[G,G]\nt G$ ist sogar der kleinster Normalteiler, sodass $G/[G,G]$ abelsch (denn: sei $N\nt G, a,b\in G, aNbN = bNaN\Leftrightarrow abN = baN\Leftrightarrow a^{-1}b^{-1}ab\in N\Leftrightarrow [G,G]\subseteq N$).
\end{bem}

\noindent
Betrachte zu einer Gruppe die abgeleitete Reihe:

\begin{equation}
\underbrace{G}_{D^0(G)} \tn \underbrace{[G,G]}_{D^1(G)} \tn \underbrace{[D^1(G),D^1(G)]}_{D^2(G)} \tn \ldots . \tag{*}\label{eq:abgeleitetereihe}
\end{equation}

\begin{satz}
	$G$ auflösbar $\Leftrightarrow \exists m\in\IN : D^m(G) = \{e\}$.
\end{satz}
\begin{proof}
	\leavevmode
	\begin{itemize}
		\item [\glqq $\Leftarrow$\grqq] Die abgeleitete Reihe \eqref{eq:abgeleitetereihe} ist nach Definition eine Normalreihe und die Faktoren sind abelsch nach der Bemerkung.
		\item [\glqq $\Rightarrow$\grqq] Sei $G$ auflösbar und $\{e\} = G_0\nt G_1\nt\dots G_n = G$ mit abelschen Faktoren. Nach der Bemerkung gilt $G_N/G_{n-1}$ abelsch $\Rightarrow [G_n,G_n]\subseteq G_{n-1}$.
		
		Behauptung: $D^i(G) \subseteq G_{n-i}$. Dies ist für $i = 0;1$. $D^{i+1}(G) = [D^i(G), D^i(G)]\subseteq [G_{n-i},{n-i}]\subseteq G_{n-i-1}$ nach Bemerkung. Also existiert ein $n\in N$ sodass $D^n(G) \subseteq G_0 = \{e\}\Rightarrow \exists m:=n$ mit $D^m(G) =\{e\}$.	
	\end{itemize}
\end{proof}


\subsection{Gruppenoperationen}
\begin{defi}
	Sei $G$ eine Gruppe, $X\neq \emptyset$ eine Menge. Eine Operation von $G$ auf $X$ ist eine Abbildung
	\begin{align*}
		\Phi\colon G\times X \longto & X\\
		(g,x) \longmapsto & g.x = \Phi(g,x)
	\end{align*}
	sodass
	\begin{description}
		\item[(O1)] $e.x = x$ für alle $x\in X$
		\item[(O2)] $g.(h.x) = (gh).x$ für alle $g,h\in G, x\in X$
	\end{description}
	Kurz: $G$ operiert auf $X$; wir schreiben $G\operates X$.
\end{defi}

\begin{bem}
	Existenz von $\Phi$ ist äquivalent zur Existenz von $\Phi'\colon G\to S_X$ Gruppenhomo mit $\Phi'(g)(x) := g.x$ (nachprüfen!)
\end{bem}

\begin{defi}
	Gegeben $G\operates X$, $G\operates Y$, $f\colon X\to Y$ Abbildung. $f$ heißt $G$-Homomorphismus, falls $f(g.x) = g.f(x)$ für alle $g\in G$ und $x\in X$.
\end{defi}

\begin{defi}
	$G\operates X$, $x\in X$. Dann\begin{enumerate}
		\item $G.x = \{g.x\mid g\in G\}$ Bahn von $x$
		\item $G_x = \{g\in G\mid g.x = x\}$ Stabilisator von $x$
		\item $X^G = \{x\in X\mid \forall g\in G\enspace g.x = x\}$ Menge der Fixpunkte
	\end{enumerate}
\end{defi}

\begin{bem}
	$x\sim y \Leftrightarrow : y\in G.x$ ist eine Äquivalenzrelation:\begin{itemize}
		\item $x\sim x$ klar, weil $x = e.x\in G.x$
		\item $x\sim y\Rightarrow \exists g\in G: g.x = y\Rightarrow x = g^{-1}.y\Rightarrow x\in G.y\Rightarrow y\sim x$
		\item $x\sim y, y\sim z\Rightarrow x\sim z$ klar nach (O2)
	\end{itemize}
	Also $X=\dot\bigcup\substack{\text{versch.}\\\text{Bahnen}}$.
\end{bem}

\begin{defi}
	$G$ operiert transitiv, falls genau eine Bahn existiert.
\end{defi}


\begin{bsp}~
	\begin{enumerate}
		\setcounter{enumi}{-1}
		\item $SO(\IR) \operates \IR^2$ durch Drehungen um $(0,0)$. Hier gibt es unendlich viele Bahnen.
		\begin{center}
			TOLLES BILD -  to be inserted
		\end{center}
		\item $G$ Gruppe, $H<G$, $X = G$. Es sei $H \operates G$ durch
		\begin{enumerate}[label=\alph*)]
			\item $h.x\coloneqq hx$ (linksreguläre Operation)
			\item $h.x\coloneqq xh^{-1}$ (rechtsreguläre Operation)
			\item $h.x\coloneqq hxh^{-1}$ (Konjugation)
		\end{enumerate}
		für alle $x\in X, h \in H$ definiert, wobei sich die folgenden Eigenschaften ergeben:
		\begin{enumerate}[label=\alph*)]
			\item \begin{itemize}[itemsep=-3pt]
				\item treu (nach \textsc{Cayley})
				\item transitiv $\Leftrightarrow G = H$
				\item Bahnen $=$ Rechtsnebenklassen
				\item $X^H = \emptyset \Leftrightarrow H \neq \{e\}$, sonst sind alle $x \in X$ Fixpunkte
			\end{itemize}
			\item wie a), außer Links- statt Rechtsnebenklassen als Bahnen
			\item \begin{itemize}[itemsep=-3pt]
				\item Bahnen $=$ Konjugationsklassen
				\item ${\displaystyle X^H = \{x \in X \mid \forall h \in H: h.x = x \} = \underbrace{\{x \in X \mid \forall h \in H: hxh^{-1} = x \}}_{\substack{\text{\emph{Zentralistator}}\\\text{bzgl. Konjugation auf }H}}}$
				\item Spezialfall $H=G$: $X^H = Z(G)$.
			\end{itemize}
		\end{enumerate}
		\item Sei $G$ eine Gruppe, $X = \{H < G \}$ und sei $G \operates X$ definiert durch Konjugation als \[ g.H \coloneqq gHg^{-1} = \{ ghg^{-1} \mid h \in H \} \in X. \]
		\begin{itemize}[itemsep=-3pt]
			\item Bahnen $=$ Konjugationsklassen von Untergruppen
			\item Stabilisator von $H\in X$: $G_H = \{ g.H = H \}$, heißt auch \emph{Normalisator} von $H$ in $G$, schreib $N_G(H)$.
			\item $X^G = \{ H < G \mid \forall g \in G: g.H = H \}=\{ H < G \mid \forall g \in G: gHg^{-1} = H \} = \{H \nt G \}$
		\end{itemize}
		\item Sei $G$ eine Gruppe, $H < G$, $X = G/H$. Dann $G \operates X$ durch $g.(aH) = gaH$ für alle $g\in G, a \in G$; heißt \emph{Linkstranslation}.
		\begin{itemize}[itemsep=-3pt]
			\item transitiv, da $\forall a,b \in G: \exists g \in G: g(aH)=bH$.
			\item Im Allgemeinen nicht treu, da \[\ker\Phi' = \underbrace{\bigcap_{x \in G}xHx^{-1}}_{\mathclap{\text{kleinster Normalteiler}}}.\]
		\end{itemize}
	\end{enumerate}
\end{bsp}

\begin{lem}
	$G\operates X$. Dann
	\begin{enumerate}
		\item $\forall x\in X: G_X<G$
		\item $f\colon G/G_X\to G.x, gG_x\mapsto g.x$ ist wohldefiniert, bijektiv und ein $G$-Homomorphismus (wobei $G$ links wie in Beispiel 2 oben und rechts durch $G\operates X$ operiert)
		\item $|G.x| = (G:G_x)$, wobei $(G:G_x) = \infty$, falls $|G/G_x |=\infty$
	\end{enumerate}
\end{lem}

\begin{proof}
	\leavevmode
	\begin{enumerate}
		\item Übung
		\item Es ist klar, dass $f$ surjektiv ist. Zur Injektivität: Sei $f(g_1G_x) = f(g_2G_x)$. Das ist äquivalent zu $ g_1.x = g_2.x\Leftrightarrow g_1^{-1}g_2.x = x \Leftrightarrow g_1^{-1}g_2\in G_x \Leftrightarrow g_1G_x = g_2G_x$ für alle $g_1,g_2 \in G, x\in X$. Also ist $f$ wohldefiniert und bijektiv.
		
		Nun muss noch gezeigt werden, dass $f$ ein $G$-Homorphismus ist: Es gilt $f(h.(gG_x)) = h.f(gG_x)$ für alle $x\in X, h,g\in G$. Aber $f(h.(gG_x)) = hgG_x = (hg).x = h.g.x = h.f(gG_x)$
		\item Es gilt nun $|G.x|\xeq{2.} |G/G_x| = (G:G_x)$
	\end{enumerate}
\end{proof}

\begin{satz}[Bahnenformel] \label{thm:bahnformel}
	Sei $G$ eine Gruppe, $X$ eine endliche Menge und sei $G\operates X$ eine Gruppenoperation. Dann gilt: $$|X| = \sum_{i\in I}(G:G_{x_i}) = \abs*{X^G} + \sum_{\mathclap{\substack{i\in I,\\x_i\notin X^G}}}(G:G_{x_i})\enspace,$$ wobei $(x_i)_{i\in I}$ Elemente in $X$ sind, sodass die Bahnen ein Repräsentantensystem der Bahnen bilden.
\end{satz}

\begin{proof} Es gilt $|X| = \abs*{\bigcup\limits_{i\in I} G.x_i} = \sum_{i\in I}|G.x_i| = \sum (G:G_{x_i})$, woraus der 1. Teil der Gleichung folgt. Teilt man nun die Bahnen $G.x_i$ in solche auf mit genau einem Element ($\Leftrightarrow x_i\in X^G$) und solchen mit $\geq 2$ Elementen, so folgt $x_i\in X^G\Leftrightarrow G_{x_i} = G\Leftrightarrow (G:G_{x_i}) = 1$. Daraus folgt sofort der 2. Teil der Gleichung.
\end{proof}

\begin{satz}
	Sei $G$ eine endliche Gruppe. $G\operates G$ durch Konjugation. Sei $(x_i)_{i\in I}$ so gewählt, dass die Bahnen ein Repräsentantensystem für Konjugationsklassen sind. Dann gilt: \[ |G|  = |Z(G)| + \sum_{\mathclap{\substack{i\in I,\\x_i\notin Z(G)}}}(G: C_G(x_i))\enspace ,\] wobei $C_G(x_i) = \{g\in G\mid gx_ig^{-1} = x_i\}$ der Zentralisatior von $x_i$ in $G$ ist.
\end{satz}

\begin{proof}
	Die Formel folgt direkt aus der Bahnenformel, da $C_G(x_i) = G_{x_i}$ mit der Konjugation als Operation und $x\in X^G\Leftrightarrow g.x = x~\forall g\in G\Leftrightarrow gxg^{-1} = x~\forall g\in G_{x_i}\Leftrightarrow x\in Z(G)$.
\end{proof}

\lecture{26. Oktober 2017}


\subsection{$p$-Gruppen und Sylow-Sätze}
\begin{defi}
	Sei $p$ prim (insbesondere $\geq 2$). Eine $p$-Gruppe ist eine Gruppe $G$ mit $|G| = p^r$ für ein $r\in\IN_0$. Insbesondere ist $|G|$ endlich.
\end{defi}
\begin{satz}
	Sei $G\neq\{e\}$ eine $p$-Gruppe. Dann gilt $\abs{Z(G)}\neq \abs*{\{e\}} = 1$. Insbesondere hat $G$ eine nicht-triviale abelsche Untergruppe.
\end{satz}
\begin{proof}
	Nach Satz 5.3 hat man \[ \underbrace{|G|}_{\mathclap{\text{durch $p$ teilbar}}}  = |Z(G)| + \underbrace{\sum_{\substack{i\in I,\\x_i\notin Z(G)}}(G: G_{x_i})}_{A}.\] Nach dem \nameref{thm:lagrange} ist $A$ durch $p$ teilbar oder gleich $1$, weil $G$ eine $p$-Gruppe ist. Letzters kann aber nicht sein, da $(G:G_{x_i}) = 1 \Leftrightarrow G = G_{x_i}\Leftrightarrow x_i\in Z(G)$, Widerspruch. Also sind die Summanden $(G:G_{x_i})$ und somit auch $A$ durch $p$ teilbar. Damit teilt $p$ auch $|Z(G)|\Rightarrow |Z(G)|\geq 2\Rightarrow Z(G)\neq \{e\}$.
\end{proof}

\begin{satz}
	Sei $G$ eine $p$-Gruppe. Dann existiert eine Normalreihe der Form
	$$ \{e\}\nt G_0\nt\dots\nt G_n = G$$ für ein $n\in\IN$, sodass $G_i/G_{i-1}\cong \IZ/p\IZ$ ($1\le i\le n$).
	Insbesondere ist $G$ auflösbar.
\end{satz}
\begin{proof}
	Übungsblatt 3.
\end{proof}

\begin{defi}
	Sei $G$ eine endliche Gruppe, $p$ eine Primzahl. Sei $|G| = p^rm$ mit $p\nmid m$. $H<G$ heißt $p$-Sylowgruppe, falls $|H| = p^r$. Wir definieren  $\Syl_p(G):=\{H<G\mid H\text{ ist Sylowgruppe}\}$.
\end{defi}

\begin{satz}[Sylowsätze]
	Sei $p$ eine Primzahl, $G$ eine endliche Gruppe, $|G| = p^rm$ mit $p\nmid m$.\begin{enumerate}
		\item $\forall 0\le k\le r \colon\exists H<G$ mit $|H| = p^k$
		\item Sei $U<G$ eine $p$-Gruppe. Dann $\exists g\in G$ und $S\in \Syl_p(G)$, sodass $U<gSg^{-1}$.
		\item Sei $n_p = |Syl_p(G)|$. Dann gilt \begin{itemize}
			\item $n_p\equiv 1\pmod p$
			\item $n_p\mid m$
		\end{itemize}
	\end{enumerate}
\end{satz}

\begin{proof}
	\leavevmode
	\begin{enumerate}
		\item Sei $1\leq k \leq r$. Der Fall $k = 0$ ist klar mit $H = \{e\}$. Sei $X = \{A\subseteq G \mid |A| = p^k\}$, wobei $|X| = \binom{p^rm}{p^k}$. Nach Übungsblatt 3 gilt: $p^{r-k+1}\nmid |X|$.
		
		Nun $G \operates X$ durch $g.A := gA=\{ga|a\in A\}$ für alle $g\in G, A\in X$. (klar: $|gA| = p^k $, also $gA\in X$). Nachrechnen: (O1), (O2) gelten (offensichtlich). 
		
		Nach \cref{thm:bahnformel} folgt $|X| = \sum_{i\in I}(G:G_{x_i})$, wobei $\exists i\in I$, sodass $p^{r-k+1} \nmid (G:G_{x_i})$, weil $p^{r-k+1}\nmid |X|$. Wähle solch ein $x_i = : A'\in X$.
		
		Behauptung: $G_{A'}<G$ mit $|G_{A'}| = p^k$. Dann folgt 1) mit $H = G_{A'}$. Klar: $G_{A'}<G$. Nach \textsc{Lagrange}: $|G| = |G_{A'}|(G:G_{A'})$, wobei $p^r$ die linke Seite der Gleichung teilt, und im Index auf der rechten Seite $p$ höchstens $r-k$-mal vorkommt.
		
		Deshalb: $p^k$ teilt $|G_{A'}|\Rightarrow p^k\leq |G_{A'}|$. Sei $a\in A'$. Dann $G_{A'}.a = \{g.a\mid g\in G_{A'}\}\subseteq G_{A'}.A'\subseteq A'$ nach Definition von $G_{A'}$.
		
		Also: $|G_{A'}| = |G_{A'}.a|\leq |A'| = p^k$.  (Def. von $G_{A'.a}$ und $A'\in X$).
		Also: $|G_{A'}| = p^k\Rightarrow$ Behauptung $\Rightarrow$ 1. Aussage.
		
		\item Sei $U<G$ mit $|U| = p^s$ für ein $s\in\IN$. Sei $S\in \Syl_p(G)$. $U \operates G/S$ nach (B3) durch Linksmultiplikation.
		\begin{align*}
			u.(gS) &= ugS\qquad \forall u\in U, g\in G \\
			m &= |G/S| = \sum_{i\in I}(U:U_{x_i})
		\end{align*}
		nach Definition von $S\in Syl_p(G)$ und \textsc{Lagrange}. Die zweite Gleichheit folgt aus Satz 5.2.
		
		Weil $p\nmid m$, existiert ein $i\in I$ sodass $p\nmid (U:U_{x_i})$. Wähle ein solches $x_i =: aS$. Nach \textsc{Lagrange} ist
		\[ p^s = |U| = |U_{aS}|(U:U_{aS}).\] Also $(U:U_{aS}) = 1$. Also $U = U_{aS}$. Damit
		\begin{center}
			$ \begin{array}{crclc}
			&u.aS &=& aS      						\qquad& \forall a\in U\\
			\Leftrightarrow& (ua)S &=& as		\qquad &\forall u\in U\\
			\Leftrightarrow& a^{-1}uaS &=& S \qquad &\forall u\in U\\
			\Leftrightarrow& a^{-1}ua&\in& S  \qquad &\forall u\in U\\
			\Leftrightarrow& u&\in& aSa^{-1}  \qquad &\forall u\in U
			\end{array}$
		\end{center}
		Setze $g := a$ und erhalte $U<gSg^{-1}$.
		
		\item Übungsblatt 3
	\end{enumerate}
\end{proof}

\paragraph{Konsequenzen.}
$G$ endliche Gruppe, $p$ Primzahl.
\begin{enumerate}
	\item Je zwei $p$-Sylowuntergruppen in $G$ sind zueinander konjugiert, also \[S, S'\in \Syl_p(G) \Rightarrow \exists g\in G: S' = gSg^{-1}.\]
	\begin{proof}
		Nach Sylowsatz 2 folgt $\exists g\in G$ mit $S'<gSg^{-1}$. Da $|S'| = |gSg'|$ nach Definition von $p$-Sylow gilt $S' = gSg^{-1}$.
	\end{proof}
	
	Beachte: Falls $n_p  = |\Syl_p(G)| = 1$, also $\exists!$ $p$-Sylowgruppe $S$, dann ist $S\nt G$. Denn $\forall g\in G$ ist $gSg^{-1}$ wieder $p$-Sylow, also $gSg^-1 = S$.
	
	\item (\textsc{Cauchy}) $p\mid|G|\Rightarrow \exists g\in G$ mit $\ord(g) = p$.
	\begin{proof}
		Nach Sylowsatz 1 existiert $H<G$ mit $|H| = p$. Wähle $g\in H$, $g\neq e$. Dann ist $\<g\> <H$ und $\<g\> \neq \{e\}$, also $\<g\> = H$ nach \textsc{Lagrange}. Aus Kapitel 3 folgt $ord(g) = |H| = p$.
	\end{proof}
	\item $G$ ist $p$-Gruppe $\Leftrightarrow $ Jedes Element $g\in G$ hat Ordnung $p^s$ für ein geeignetes $s\in \IN_0$ (abhängig von $g$).
	\begin{proof}~
		\begin{description}
			\item[\glqq$\Rightarrow$ \grqq] Sei $g\in G$. Sei $ord(g) = n$. Aus Satz 3.3 folgt $|\<g\>| = n$. $\Rightarrow n\mid |G|$ nach \textsc{Lagrange}. $\Rightarrow$ (da $G$ $p$-Gruppe) $n = p^s$ für ein $s \in \IN_0$.
			\item[\grqq $\Leftarrow$\grqq] zu zeigen: $|G| = p^r$ für ein $r\in\IN_0$.
			
			Annahme: $q\mid|G|$ für $q$ Primzahl $p\neq q$. Nach dem Satz von \textsc{Cauchy} existiert $g\in G$ mit $\ord(g) = q$. Das ist ein Widerspruch.
		\end{description}
	\end{proof}
	\begin{bem}
		$p$-Gruppen mit unendlicher Ordnung kann man definieren als Gruppen mit $ord(g) = p^r$, $r\in \IN_0$ von $p$ für alle $g\in G$.
	\end{bem}
	
\end{enumerate}

\paragraph{Anwendungen.}\leavevmode
Vorbemerkung: $G$ Gruppe, $|G| = p$ prim $\Rightarrow G\cong \IZ/p\IZ$. (Denn wähle $g\in G$, $g\neq e$. Dann $\<g\> <G$ und nach \textsc{Lagrange} ist $|\<g\>| = p = |G|$, also $G = \<g\>$ zyklisch, also $G\cong \IZ/p\IZ$ nach Klassifikation von zyklischen Gruppen.)
\begin{satz}
	$G$ Gruppe, $|G| = pq$ mit $p\neq q$ Primzahl. Dann ist $G$ auflösbar.
\end{satz}	
\begin{proof}
	Ohne Beschränkung der Allgemeinheit sei $p>q$. Nach Sylowsatz 3 gilt: $n_p\mid q$, also $n_p\in\{1,q\}$ und $n_p\equiv 1 \pmod p$.
	
	$\Rightarrow n_p = 1$, weil $p>q$. Nach Bemerkung in 1 gilt $\exists!$ $p$-Sylowgruppe $S$ und $S\nt G$. Nach Definition von $p$-Sylow und weil $|G| = pq$ gilt $|S| = p$. Also erhalten wir eine Normalreihe 
	$$ \{e\}\nt S\nt G$$ mit $S/\{e\}\cong S\cong \IZ/p\IZ$ und $|G/S| = q$, also $G/S\cong \IZ/q\IZ$.
	
	$\Rightarrow$ Faktoren sind abelsch $\Rightarrow$ $G$ ist auflösbar.
\end{proof}

\begin{satz}
	$G$ Gruppe, $|G| = pq$ mit $p,q$ prim sowie $p<q$ und $p \nmid q-1$. Dann folgt $G\cong\IZ/p\IZ \times \IZ/ q\IZ$.
\end{satz}
\begin{proof}
	Nach Sylowsatz 3 gilt $n_p\in \{1,q\}, n_q\in\{1,p\}$ und $n_p\equiv 1\pmod p$, $n_q\equiv 1\pmod q$. Da $p<q$ ist, gilt $n_q = 1$. Also existiert genau eine $q$-Sylowgruppe $Q\nt G$. Falls $n_p = q\Rightarrow q\equiv 1\pmod p$. Daraus folgt $p\mid (q-1)$ im Widerspruch zur Voraussetzung. Also ist $n_p = 1\Rightarrow \exists!$ $p$-Sylowgruppe $P\nt G$.
	
	1. Behauptung: $x\in P ,y\in Q$. Dann $xy = yx$. Denn $xyx^{-1}y^{-1}\in Q$, da $xyx^{-1}\in Q$ ($Q$ Normalteiler) und $y^{-1}\in Q$, analog $xyx^{-1}y^{-1}\in P$, da $x\in P$ und $yx^{-1}y^{-1} \in P$ ($P$ Normalteiler).
	
	$\Rightarrow$ $xyx^{-1}y^{-1} \in P\cap Q$. Aber $P\cap Q = \{e\}$, da $|P\cap Q|\mid p = |P|$ und $|P\cap Q| \mid q = |Q|$.
	
	$\Rightarrow$ 1. Behauptung.
	
	Betrachte nun $\Phi\colon P\times Q\to G, (x,y)\mapsto xy$. $\Phi$ ist ein wohldefinierter Gruppenhomomorphismus. Denn $\Phi((x,y)\circ (x',y')) = \Phi ((xx', yy')) = xx'yy'$; $\Phi((x,y))\circ \Phi((x',y')) = xyx'y' = xx'yy'$ (nach der 1. Behauptung).
	
	Außerdem ist $\Phi$ injektiv, denn $\Phi((x,y)) = e\Leftrightarrow xy = e\Leftrightarrow x = y^{-1} = e$, weil $P\cap Q = \{e\}$.
	
	$\Phi$ ist surjektiv, weil $|P\times Q| = |P|\cdot |Q| = pq = |G|$.
	
	$\Rightarrow$ $\Phi$ liefert Gruppenisomorphismus $P\times Q \cong G$, also $\IZ/p\IZ\times \IZ/q\IZ\cong G$.
\end{proof}


\begin{kor}
	$G$ Gruppe, $|G| = 15$. Dann $G\cong \IZ/3\IZ\times \IZ/5\IZ \cong \IZ/15\IZ$ und $G$ ist zyklisch.
\end{kor}

\begin{proof}
	Wir wissen $G \cong \IZ/3\IZ\times \IZ/5\IZ$. Behauptung: $ \IZ/3\IZ\times \IZ/5\IZ\cong \IZ/15\IZ$. Sei nämlich $g = (\overline{1}, \overline{1})\in  \IZ/3\IZ\times \IZ/5\IZ$. Dann gilt: $\ord(g) = \min \{j\mid (\overline{1},\overline{1})+\dots (\overline{1},\overline{1}) = (\overline{0}, \overline{0})\in  \IZ/3\IZ\times \IZ/5\IZ\}  =15$
	
	$\Rightarrow |\<g\>| = 15 \Rightarrow  \IZ/3\IZ\times \IZ/5\IZ$ ist zyklisch.
	
	$ \IZ/3\IZ\times \IZ/5\IZ\to \IZ/15\IZ, g\mapsto \overline{1}$ gibt den Isomorphismus.
\end{proof}

\lecture{30. Oktober 2017}

\subsection{Ringe}
\begin{defi} Ein Ring (mit 1) ist eine Menge $R$ zusammen mit zwei Abbildungen
	\begin{alignat*}{2}
		+,\cdot \colon R\times R &\longto  R &&\quad\\
		(a,b)&\longmapsto a+b \qquad &&\text{Addition}\\
		\text{bzw. }(a,b)&\longmapsto a\cdot b &&\text{Multiplikation,}
	\end{alignat*}
sodass gilt:
\begin{itemize}
	\item[(R1)] $(R,+)$ ist eine abelsche Gruppe.
	\item[(R2)] $\forall a,b,c\in R$ gilt $(a\cdot b)\cdot c = a\cdot (b\cdot c)$ (also $\cdot$ ist assoziativ)
	\item[(R3)] $\forall a,b,c\in R$ gilt:
	\begin{equation*}
		\begin{aligned}
		a\cdot(b+c) &= (a\cdot b)+ (a\cdot c)\\
		(b+c)\cdot a &= (b\cdot a)+(c\cdot a)
		\end{aligned} \tag{Distributivität}
	\end{equation*}
	\item[(R4)] $\exists 1 = 1_R\in R$, sodass $a\cdot 1 = a = 1\cdot a$ für alle $a\in R$ (Neutrales bezüglich $\cdot$)
\end{itemize}
\end{defi}

\begin{bem} \leavevmode
	\begin{enumerate}
		\item Wir bezeichnen mit $0$ oder $0_R$ das neutrale Element und mit $(-a)$ das Inverse zu $a\in R$ bezüglich $+$.
		\item Das Element $1\in R$ ist eindeutig (denn sei $1'$ ein anderes, dann ist $1 = 1\cdot 1' = 1'$).
		\item In einem Ring gilt: $a\cdot 0 = 0 = 0\cdot a$ für alle $a\in R$, denn $a\cdot 0 = a\cdot (0+0 = (a\cdot 0)+(a\cdot 0)\Rightarrow 0 = a\cdot 0$; analog für $0\cdot a$.
	\end{enumerate}
\end{bem}

\begin{defi}
	Ein Ring $(R, +,\cdot)$ heißt kommutativ, falls $a\cdot b = b\cdot a$ für alle $a,b\in R$
\end{defi}	

\begin{bsp}
	\leavevmode
	\begin{enumerate}
		\item Jeder Körper $(K,+,\cdot)$ ist ein kommutativer Ring (aber Ringe haben im Allgemeinen keine multiplikativ Inversen).
		\item (aus LA) Sei $V$ ein $K$-Vektorraum, $K$ ein Körper, dann ist $(\text{End}_K(V),+,\cdot)$ ein Ring mit $(f+g)(v) = f(v)+g(v)$ und $(f\cdot g)(v) = (f\circ g)(v)$ (Hintereinanderausführung) mit $f,g\in \text{End}_K(V), v\in V$ mit $0_{\text{End}_K(V)} =$ Nullabbildung; $1_{\text{End}_K(V)} = \text{id}_V$.
		\item Nullring: $R = \{0=1\}$ mit $0+0 = 0$ und $0\cdot 0 = 0$.
		\item Es gilt folgende Umkehrung von $1.$: wenn $(R,+,\cdot)$ ein kommutativer Ring ist, $R\neq \{0\}$, jedes $x\in R$ mit $x\neq 0$ besitzt Inverses $x^{-1}$ bezüglich $\cdot$; dann ist $(R, +, \cdot )$ Körper
		\item $(R, +, \cdot)$ Ring. Betrachte
		\[ R[t] = \left\{ \left. \sum_{i=0}^{\infty} a_it^i \right| a_i\in R, \text{ nur endlich viele }a_i\neq 0\right\}  = \left\{\left. \sum_{i=0}^{n} a_it^i \right| a_i\in R, n\in \IN_0\right\},\]
		die Polynome mit Koeffizienten in $R$. Dann ist $(R[t], +, \cdot)$ ein Ring mit $0_{R[t]} =$ Nullpolynom, d.h. $a_i = 0$ für alle $i$. $1_{R[t]}$ ist das Polynom $p(t) = \sum_{i = 0}^{\infty}a_it^i$ mit $a_0 = 1$ und $a_i = 0$ für $i\geq 1$.  Es gilt: $(R[t], +,\cdot)$ ist kommutativ $\Leftrightarrow$ $(R,+,\cdot)$ ist kommutativ.
	\end{enumerate}
\end{bsp}	

\begin{defi}
	$(R,+,\cdot)$ Ring. $R'\subseteq R$ heißt Unterring, falls
	\begin{itemize}
		\item[(UR1)] $1_R\in R'$
		\item[(UR2)] $\forall a, b\in R': a+(-b)\in R', a\cdot b \in R'$
	\end{itemize}
\end{defi}

\begin{bsp}
	$(R,+,\cdot)$ Ring. $Z(R) := \{a\in R|a\cdot x = x\cdot a \enspace \forall x\in R\}$ ist das Zentrum des Ringes; dieses ist ein Unterring.
	
	Warnung: $Z(R) \neq Z((R,+))$ im Allgemeinen
\end{bsp}	

\begin{defi}
	Seien $(R,+,\cdot)$ und $(S,+,\cdot)$ Ringe. Eine Abbildung $\phi\colon R\to S$ ist Ringhomomorphismus, falls gilt:
	\begin{itemize}
		\item[(RH1)] $\phi (a+b) = \phi (a)+\phi(b)$
		\item[(RH2)]$\phi(a\cdot b) = \phi(a)\cdot \phi(b)$
		\item[(RH3)]$\phi(1_R) = \phi(1_S)$
	\end{itemize}
	für alle $a,b\in R$.
	
	Falls $\phi$ zusätzlich bijektiv ist, ist $\phi$ ein Ringisomorphismus.
\end{defi}

\begin{bem}
	$\phi\colon R\to S$ Ringhomomorphismus $\Rightarrow$ $R\to S$ ist Gruppenhomomorphismus von $(R,+)$ nach $(S,+)$ wegen (RH1).
\end{bem}

\begin{lem}
	\leavevmode
	\begin{enumerate}
		\item $\phi\colon R\to S$ Ringisomorphismus $\Rightarrow \phi^{-1}\colon S\to R$ Ringisomorphismus
		\item $\phi_1\colon R\to S, \phi_2\colon S\to T$ Ringhomomorphismen $\Rightarrow \phi_2\circ\phi_1\colon R\to T$ Ringhomomorphismus
	\end{enumerate}
\end{lem}
\begin{proof}
	Nachrechnen.
\end{proof}


\begin{lem}
	Sei $\phi\colon R\to S$ Ringhomomorphismus. Dann ist $\im\phi\subseteq S$ ein Unterring.
\end{lem}

\begin{proof}
	Es gilt $\phi(1_R) = 1_S\in\im\phi\Rightarrow$ (UR1).
	
	Seien $s_1, s_2\in\im\phi\Rightarrow \exists r_1,r_2\in R: \phi (r_1) = s_1, \phi (r_2 )= s_2\Rightarrow s_1\cdot s_2 = \phi (r_1)\cdot \phi (r_2) = \phi(r_1\cdot r_2)\in \im\phi\Rightarrow s_1\cdot s_2\in \im\phi$.
	
	Außerdem ist $s_1+(-s_2) = \phi (r_1) + (-\phi (r_2)) = \phi (r_1)+\phi (-r_2) = \phi (r_1+(-r_2))\in\im\phi\Rightarrow s_1+(-s_2)\in \im\phi\Rightarrow$ (UR2).
\end{proof}

\noindent
\textbf{Warnung:} Wir setzen für $\phi\colon R\to S$ als Ringhomomorphismus
\[\ker\phi := \{r\in R\mid\phi(r)= 0_S\}.\] Dann ist $\ker\varphi\subseteq R$ genau dann Unterring, falls $S$ der Nullring ist. Denn:

\glqq $\Rightarrow$\grqq: $\ker\phi$ Unterring $\Rightarrow 1_R\in \ker \phi\Rightarrow 0_S = \phi (1_R) = 1_S\Rightarrow \forall s\in S: s = s\cdot 1_S = s\cdot 0_S = 0_S$.

\glqq $\Leftarrow$\grqq: $S = \{0\}\Rightarrow \ker\phi = R$ ist offensichtlich ein Unterring.

\begin{defi}
	Sei $(R,+,\cdot)$ ein Ring. $I\subseteq R$ heißt Ideal, falls gilt:
	\begin{itemize}
		\item[(I1)] $I<(R,+)$
		\item[(I2)] \begin{itemize}
			\item[a)] $a\cdot x \in I$ für alle $x\in I, a\in R$
			\item[b)] $x\cdot a \in I$ für alle $x\in I, a\in R$
		\end{itemize}
	\end{itemize}
	Falls nur (I1), (I2a) erfüllt sind, heißt $I$ Linksideal; falls nur (I1) und (I2b) erfüllt sind, heißt $I$ Rechtsideal.
\end{defi}

\begin{bsp}
	\leavevmode
	\begin{enumerate}
		\item $(\IZ,+,\cdot)$ ist Ring. Sei nun $n\in\IZ$, dann ist $I= n\IZ = \{nk\mid k\in\IZ\}\subseteq \IZ$ ein Ideal, denn: $n\IZ<(\IZ,+)$, also folgt (I1); und für $a\in\IZ$ und $x = nk\in n\IZ$ gilt: $ax = ank = nak\in I$; $xa = nka  = nak \in I$ und damit folgt (I2).
		
		\item $(R,+,\cdot)$ Ring; $(R[t],+,\cdot)$ wie in Beispiel oben; \[I = \left\{p(t)\in R[t]\left| p(t) = \sum_{i = 0}^{\infty}a_it^i \wedge a_0 = 0\right.\right\}\] enthält die Polynome ohne konstanten Term. Dann ist $I\subseteq R[t]$ ein Ideal (nachprüfen).
	\end{enumerate}
\end{bsp}


\begin{lem} $\phi\colon R\to S$ Ringhomomorphismus $\Rightarrow \ker\phi \subseteq R$ ist Ideal.
\end{lem}

\begin{proof}
	$\ker \phi<(R,+)$ nach 1.3 $\Rightarrow$ (I1). Sei nun $a\in R, x\in\ker \phi\Rightarrow \phi(ax) = \phi(a)\phi(x) = \phi(a)\cdot 0_S = 0_S\Rightarrow ax\in \ker \phi$. Genauso $xa\in\ker \phi\Rightarrow$ (I2).
\end{proof}

\begin{bsp}
	$(R[t],+,\cdot)$ wie in Beispiel 3. Sei $a\in R$.\begin{eqnarray*}
		\ev_a\colon R[t] & \to  R\\
		p(t) = \sum_{i=0}^{\infty}b_it^i &\mapsto& p(a) = \sum_{i=0}^{\infty} b_ia^i\\
		(b_i\in R; \text{ fast alle $b_i = 0$})&&\left(\text{mit }a^i = \prod_{k=1}^ia\right)
	\end{eqnarray*}
	heißt Auswertungs- oder Evaluationsabbildung.
	
	Nachrechnen: $\ev_a$ ist ein Ringhomomorphismus.
	
	$\ker (\ev_a) = \{p(t)\in R[t]\mid p(a) = 0_R\}$. Das sind genau die Polynome, die $a$ als Nullstelle haben. Wir wissen: $\ker (\ev_a) \subseteq R[t]$ Ideal nach 7.3.
	
	Spezialfall: $a = 0_R$. Dann gilt $\ker \ev_0  = I$ wie in Bsp. 3 Teil 2); (insbesondere $I$ Ideal).
\end{bsp}

Seien nun ein Ring $(R,+,\cdot)$ und ein Ideal $I\subseteq R$ gegeben. Insbesondere, nach $(I1)$, ist $I<(R,+)$, sogar $I\nt (R,+)$, weil $(R,+)$ abelsch.

$\Rightarrow$ $R/I$ ist wieder Gruppe mit den Nebenklassen in $(R,+)$ bezüglich $I$ als Elemente. Nebenklassen sind von der Form $\overline{a} = \{a+x\mid x\in I\}\enspace a\in R$ und die Gruppenoperation auf $R/I$ ist $\overline{a}\circ\overline{b} = \overline{a+b}$.

\begin{satz}
	Voraussetzungen: $R, I$ wie oben. Dann wird $(R/I, \circ)$ zu einem Ring $(R/I,+,\cdot)$, wobei $+=\circ$ und Multiplikation $\cdot  = \odot$ gegeben ist durch $\overline{a}\odot \overline{b} = \overline{a\cdot b}$, wobei letzteres die Multiplikation in R ist.
\end{satz}
\begin{proof}
	\leavevmode
	\begin{itemize}
		\item[(R1)] $(R/I,+)$ ist eine abelsche Gruppe (nach Kapitel 1.1).
		\item[(R2)] Seien $\overline{a},\overline{b},\overline{c}\in R/I$. $(\overline{a}\odot\overline{b})\odot\overline{c}  = (\overline{ab})\odot \overline{c} = \overline{(ab)c} = \overline{a(bc)}  = \overline{a}\odot \overline{bc} = \overline{a}\odot(\overline{b}\odot\overline{c})$. 
		\item[(R3)] Seien $\overline{a},\overline{b},\overline{c}\in R/I$. Dann $\overline{a}\odot(\overline{b}\circ \overline{c}) = \overline{a}\odot\overline{b+c} = \overline{a\cdot(b+c)} =\overline{ab+ac} = \overline{ab}\circ\overline{ac} = \overline{a}\odot \overline{b}\circ\overline{a}\odot\overline{c}$. Analog für den zweiten Teil von (R3).
		\item[(R4)] Sei $\overline{a}\in R/I$. Dann gilt $\overline{a}\odot\overline{1_R}  = \overline{a1_R} = \overline{a} = \overline{1_R\cdot a} = \overline{1_R}\odot \overline{a}$
		
		$\Rightarrow$ $\overline{1_R}$ ist neutrales Element für $\odot$.
		\item[] Noch zu prüfen: $\odot$ ist wohldefiniert! Also zu zeigen: für $\overline{a} = \overline{a'}$ und $\overline{b} = \overline{b'}$ folgt $\overline{a}\odot\overline{b} = \overline{a'}\odot\overline{b'}$ mit $\overline{a},\overline{b},\overline{a'},\overline{b'}\in R/I$. Sei also $\overline{a} = \overline{a'}$ und $\overline{b} = \overline{b'}$. 
		
		Dann existieren $x,y\in I$ mit $a+(-a') = x$ und $b+(-b') = y$ (1).
		
		Zu zeigen ist nun $\overline{ab} = \overline{a'b'}$. Es gilt $a\cdot b = (a'+x)\cdot (b'+y)  = (a'b')+(a'y)+(xb')+(xy)$, wobei $(a'y)+(xb')+(xy)\in I$, weil $I$ Ideal ist. $\Rightarrow (ab)+(-a'b')\in I\Rightarrow \overline{ab} = \overline{a'b'}$.
	\end{itemize}
\end{proof}



\lecture{2. November 2017}

Wir wissen: $(R,+,\cdot)$ Ring, $I\subseteq R$ Ideal; dann ist $(R/I,+,\odot)$ ein Ring mit $\overline{a}\odot\overline{b} = \overline{ab}$. Wir nennen $(R/I,+,\odot)$ den Quotientenring von $R$ nach/modulo $I$. Im Folgenden schreiben wir kurz $R$ statt $(R,+,\cdot)$ etc.

\begin{satz}[Homomorphiesatz]
	Sei $R$ Ring, $I\subseteq R$ Ideal. \begin{enumerate}
		\item Die Abbildung $\can\colon R\to R/I, a\mapsto \overline{a}$ ist Ringhomo.
		\item Sei $\phi\colon R\to S$ Ringhomo mit $I\subseteq \ker \phi$, dann $\exists!\;\overline{\phi}\colon R/I\to S$ Ringhomo, sodass $\overline{\phi}\circ \can  = \phi$. Also:
		\begin{center}
		\begin{tikzcd}
					R \arrow{rd}[swap]{\can} \arrow{r}{\phi} & S \\
					& R/I\arrow{u}[swap]{\exists! \; \overline{\phi}\text{ Ringhomo}} \\
		\end{tikzcd}
		\end{center}
	\end{enumerate}
\end{satz}

\begin{proof}
	Übungsblatt 3.
\end{proof} 

\begin{defi}
	Sei $R$ ein kommutativer Ring und $I, J\subseteq R$ Ideale. Dann ist \begin{itemize}
		\item $I+J = \{x+y|x\in I, y\in J\}\subseteq R$ die Summe von $I$ und $J$
		\item $I\cap J = \{x|x\in I, x\in J\}\subseteq R$ der Schnitt von $I$ und $J$
		\item $I\cdot J = \{\sum_{i = 1}^na_ib_i|a_i\in I, b_i\in J, n\in\IN\}$ das Produkt von $I$ und $J$
	\end{itemize}
	Das sind alles Ideale in $R$ (nachprüfen!).
\end{defi}

\begin{defi}
	Sei $R$ kommutativer Ring, $a,b\in R$. \begin{enumerate}
		\item $(a) = \{ra\mid r\in R\}$ ist das von $a$ erzeugte Ideal.
		\item $b$ teilt $a$ (in $R$), falls $\exists r\in R: a = br = rb$
		\item 
		$a$ ist Nullteiler, wenn $\exists r\in R, r \neq 0$ mit $ra = ar = 0$.
\textbf{}			
		$R$ ist nullteilerfrei, falls 0 der einzige Nullteiler ist.

		$R$ heißt Integritätsbereich, falls $R$ nullteilerfrei und $R \neq \{0\}$ ist.
		\item $a$ heißt Einheit, falls $\exists r\in R$ mit $ar = 1 = ra$. (also $a$ hat multiplikatives Inverses).
		
		$R^{\times} = \{c\in R \mid c \text{ Einheit in $R$}\}$ ist Einheitengruppe von $R$.
	\end{enumerate}
\end{defi}

\begin{bem}
	$(a)$ ist in der Tat ein Ideal:\begin{itemize}
		\item[(I2)] $r' \in R, x = ra\in (a)\Rightarrow r'x  = r' (ra) = (r'r)a \in (a)$ Genauso $xr'\in (a)$
		\item[(I1)] $x,y\in I$ $x = r'a, y = ra\Rightarrow x+(-y) = (r'a)+(-ra) = (r'+(-r))a \in (a)$ (benutze dabei $-(ra) = (-r)a$, weil $(ra)+((-r)a) = (r+(-r))a = 0a = 0$, also $-(ra) = (-r)a$ 
	\end{itemize}
\end{bem}
\begin{bem}
	$R^{\times}$ ist Gruppe bezüglich $\cdot$ (Übung).
\end{bem}

\begin{defi}
	Ein Ideal der Form $(a)$ wie oben heißt Hauptideal. Ein kommutativer Ring $R\neq\{0\}$ heißt Hauptidealring, wenn $R$ nullteilerfrei und jedes Ideal in $R$ ein Hauptideal ist (also von der Form $(a)$ ist): $\forall I\subseteq R\text{ Ideal }\exists a\in \IR: I = (a)$
\end{defi}

\begin{bsp}
	Betrachte $(\IZ,+,\cdot)$. Behauptung: Das ist ein Hauptidealring.\begin{itemize}
		\item $R \neq \{0\}$, $R$ nullteilerfrei (klar)
		\item Sei $I \subseteq \IZ$ ein Ideal. Falls $I = \{0\}$, dann $I = (0)$. Es sei also $I \neq \{0\}$. Dann $\exists x\in I, x\neq 0$. Wähle $n\in\IN$ minimal mit $n\in I$ (existiert weil $x\in I\Rightarrow -x \in I $ nach (I1)).
		
		Behauptung: $I = (n)$
		
		\begin{proof} \glqq $\supseteq$\grqq: nach (I2); \glqq $\subseteq$\grqq: Sei $y\in I$. Schreibe $y = bn+r$ mit $b,r\in\IZ$ mit $0\leq r< n\Rightarrow y+(-bn) = r\Rightarrow r\in I$; Widerspruch zur Minimalität von $n$ außer $r = 0$ $\Rightarrow y = bn\Rightarrow I\subseteq (n)$.
		\end{proof}
	\end{itemize}
\end{bsp}

\begin{bsp}
	$(1) = R$, $(0) = \{0\}$ sind Hauptideale.
\end{bsp}

\begin{lem}
	$R$ kommutativer Ring, $R\neq \{0\}$.\begin{enumerate}
		\item $R$ Körper $\Leftrightarrow R^{\times} = R\setminus\{0\}$
		\item Für $a,b,c\in R$ gilt \begin{enumerate}
			\item $(a)\subseteq (b)\Leftrightarrow$ $b$ teilt $a$
			\item $(a) = R \Leftrightarrow a\in R^{\times} \Leftrightarrow (a) = (1)$
			\item Falls $c$ nicht Nullteiler: $ac =bc \Rightarrow a =b$
			\item $a$ Nullteiler $\Rightarrow a\notin R^{\times}$
		\end{enumerate}
	\end{enumerate}
\end{lem}
\begin{proof}
	\leavevmode
	\begin{enumerate}
		\item $R$ Körper $\Rightarrow \forall x\in R, x\neq 0 \exists x^{-1}$ mit $xx^{-1} = 1 =x^{-1}x\Rightarrow x\in R^{\times}\Rightarrow R\setminus \{0\}\subseteq R^{\times}$
		
		%$x\in R^{\times} \Rightarrow \exists b\in R: xb = 1 = bx\Rightarrow$ $x$ invertierbar  un $ x\neq 0$ (weil $1\neq 0$) $\Rightarrow$ Jedes $x\in R, x\neq 0$ ist invertierbar
		
		Annahme: $R^{\times} = R\setminus\{0\}$. Sei $x\in R, x\neq 0\Rightarrow x\in R^{\times}\Rightarrow \exists r \in R$ mit $xr = 1 = rx\Rightarrow r = x^{-1}$ Inverses zu $x$ $\Rightarrow R$ Körper.
		
		\item \begin{enumerate}
			\item 
				\glqq $\Rightarrow$\grqq: Sei $(a)\subseteq (b)\Rightarrow a\in (b)\Rightarrow \exists r\in R$ mit $a = rb\Rightarrow$ $b$ teilt $a$.
		
				\glqq $\Leftarrow$\grqq: $b$ teilt $a$ $\Rightarrow \exists r\in R: a =rb\Rightarrow r'a = r'(rb) = (r'r)b\in (b)\forall r'\in R\Rightarrow (a)\subseteq (b)$
			\item
				Sei $(a) = R\Rightarrow 1\in (a)\Rightarrow \exists r\in R: 1 = ra = ar\Rightarrow a\in R^{\times}$.
			
				Sei $a\in R^{\times}\Rightarrow \exists r\in R: ra = 1 = ar\Rightarrow 1\in (a) \Rightarrow r'\cdot 1 = r'\in (a) \forall r'\in R \Rightarrow (a) = R$.
				
				Sei $a \in R^{\times}\Rightarrow 1\in (a) \Rightarrow R = (1) \subseteq (a) \Rightarrow (1)  = (a) $.
				
				Sei $(1) = (a) \Rightarrow1\in (a) \Rightarrow \exists r\in R: ra = 1 = ar\Rightarrow a\in R^{\times}$.
				
			\item 
				Sei $ac = bc \Rightarrow (a+(-b))c = 0\Rightarrow a+(-b) = 0\Rightarrow a = b$, da $c$ nicht Nullteiler.
			\item 
				Sei $a$ Nullteiler $\Rightarrow \exists r\neq 0$ mit $ra = 0$. Sei nun $a\in R^{\times}$, dann $\exists b$ mit $ab = 1\Rightarrow r = r\cdot 1 = rab = 0b = 0\Rightarrow r = 0$. Widerspruch!
		\end{enumerate}
	\end{enumerate}
\end{proof}


\begin{bsp}
	$R = \IR[t]$, $I = (t^2)$; Behauptung: $R/I$ ist kein Körper. Denn: $\overline{t} \neq \overline{0}$, weil $t\notin I$. Andererseits: $\overline{t^2} = \overline{0}$, weil $t^2\in I$ und somit $\overline{0} = \overline{t^2} = \overline{t}\odot\overline{t}$, also $\overline{t}$ Nullteiler in $R/I$ insbesondere nach Lemma 7.6 keine Einheit. $\Rightarrow (R/I)^{\times} \neq (R/I)\setminus\{0\}\Rightarrow R/I$ kein Körper.
\end{bsp}

\begin{bsp}
	$R = \IR [t], I = (t^2+1)$. Behauptung: $R/I$ ist ein Körper.
	\begin{proof}
		Klar: $R/I$ kommutativ, weil $R$ kommutativ. Sei $x\in R/I, x\neq 0$, zu zeigen $x\in (R/I)^{\times}$. In $R/I$ gilt $\overline{t^2+1} = \overline{0}$, weil $t^2+1\in I$ und damit für $j\in \IZ_{\geq 2}$:
		\begin{align*}
			\overline{t ^j} &= \overline{t^{j-2}(t^2+1)+(-t^{j-2})} \xeq[\text{in }R/I]{\text{Def von $+$ }} \overline{t^{j-2}(t^2+1)}+\overline{-t^{j-2}} \\
			&\xeq{\text{Def von $\odot$}}  \overline{t^{j-2}}\odot \overline{t^2+1} + \overline{-t^{j-2}} = \overline{-t^{j-2}} \xeq{\text{Def von $+$ in $R/I$}} - (\overline{t^{j-2}})
		\end{align*}
		Für $p = \sum_{i = 0}^{\infty}a_it^i\in\IR[t]$ gilt: $\overline{p} = \overline{b_01+b_1t}$ für gewisse $b_0,b_1\in R$. Also o.B.d.A. $x = \overline{b_01+b_1t}$ mit $b_0^2+b_1^2 \neq 0$.
		
		Sei $q: = \frac{1}{b_0^2b_1^2}(b_01-b_1t)\in \IR[t] = R$ wohldefiniert, da $b_0^2+b_1^2 \neq 0$.
		
		Setze $y = \overline{q}$. Behaupte $y = x^{-1}$ in $R/I$. Nun gilt 
		$$ (b_01 +b_1t) q = \frac{1}{b_0^2+b_1^2}(b_0^2+b_0b_1t-b_1b_0t-b_1^2t^2) = \frac{1}{b_0^2+b_1^2}(b_0^2-b_1^2t^2)$$
		Da $\overline{t^2+1} = \overline{0}$, also $\overline{-t^2} = \overline{1}$ gilt $xy = \overline{\frac{1}{b_0^2+b_1^2}(b_0^2+b_1^2)} = \overline{1}$. Also ist $x\in (R/I)^{\times}$ mit dem Inversen $y$.
		
		Übung: $R/I$ ist isomorph zu $\IC$ (als Körper).
	\end{proof}
\end{bsp}

\begin{defi}
	$R$ kommutativer Ring, $R\neq \{0\}$. $I\subseteq R$ Ideal heißt maximal, falls $I\neq R$ und $\nexists J\subseteq R$ Ideal mit $I\subset J\subset R$ (echte Teilmengen).
\end{defi}
\begin{lem}
	$R$ kommutativer Ring, $R\neq 0$. \[
	\text{$R$ ist ein Körper} \quad\Longleftrightarrow\quad \text{$\{0\}$ ist (einziges) maximales Ideal.}
	\]
\end{lem}
\begin{proof}
	\leavevmode
	\begin{itemize}
		\item[\glqq $\Leftarrow$\grqq:] Sei $\{0\}$ maximales Ideal. Dann ist es auch schon das einzige maximale Ideal, denn falls $I\neq \{0\}$ ein maximales Ideal ist, so folgt $\{0\}\subset I\neq R\Rightarrow \{0\} = I$. Widerspruch!
		
		Sei $a\in R, a\neq 0\Rightarrow (0) \subset (a) \Rightarrow (a) = R\Rightarrow \text{Lemma 6 } R^{\times} = R\setminus\{0\}\Rightarrow R$ Körper.
		\item[\glqq $\Rightarrow$\grqq:] Sei $R$ ein Körper. $I\subseteq R$ Ideal, $I\neq \{0\}$. Dann $\exists a\in I$ mit $ a\neq 0$. Weil $R$ ein Körper ist, ist $a\in R^{\times}\Rightarrow (a) = R \Rightarrow I = R\Rightarrow \{0\}$ ist einziges Ideal und somit maximal.
	\end{itemize}
\end{proof}

\begin{bsp}
	Betrachte $R = \IZ$ sowie ein Ideal $I\subseteq \IZ$. \[\text{$I$ maximal} \quad \Longleftrightarrow \quad  \text{$I = (n) $ mit $n = p$, $p$ prim}\]
	\begin{proof}
		\leavevmode
		\begin{itemize}
			\item[\glqq $\Rightarrow$\grqq:] Sei $I = (a)$ (weil $\IZ$ Hauptidealring) maximal. Falls $a\neq \pm p$ mit $p$ prim, so existiert $b\in \IZ$ mit $b\mid a$ und $b\neq a$, $b\neq  \pm 1$.
			
			$\Rightarrow$ $(a)\subset (b)\subset R$ (weil $b\neq \pm a$; $(b)\neq R$, weil $b\neq \pm 1$, also $b\notin \IZ^{\times}$), was im Widerspruch zu $I$ maximal steht, also ist $a  = \pm p$ mit $p$ prim.
			
			\item[\glqq $\Leftarrow$\grqq:] Sei $I = (a)$ mit $a = \pm p$, $p$ prim. Sei $J$ ein Ideal in $R$ sowie $I \subset J\subset R$. Dann gilt $J = (b)$ für ein $b\in\IZ$ (da $\IZ$ Hauptidealring) und nach Lemma 7.6 folgt $b\mid a$ und $b\neq \pm a$ (weil $I\subset J$) und $b \neq \pm 1$ (weil $(b)\neq R$). Das ist ein Widerspruch zu $a = \pm p$.
		\end{itemize}
	\end{proof}
\end{bsp}

\begin{satz} \label{satz8}
	$R$ kommutativer Ring, $R \neq\{0\}$, $I\subseteq R$ Ideal. \[\text{$R/I$ ist ein Körper.} \quad\Longleftrightarrow\quad \text{$I$ ist ein maximales Ideal.}\]
\end{satz}

\paragraph{Als Vorbereitung:}

\begin{satz}
	Sei $\phi\colon R\to S$ ein Ringepimorphismus. Dann gibt es eine Bijektion zwischen den Mengen
	\begin{alignat*}{2}
		\left\{\substack{\text{Ideale in $R$}\\\text{ mit $\ker \phi\subseteq I$} }\right\}&\longleftrightarrow \; &&\{\text{Ideale in $S$}\} \\\intertext{durch}
		I & \longmapsto &&\phi(I)\\
		\underbrace{\phi^{-1}(J)}_{\mathclap{\text{$\{r\in R\mid\phi(r) \in J\}$}}} & \longmapsfrom && J
	\end{alignat*}
\end{satz}

\begin{proof}
	\leavevmode
	Behauptung: $\phi(I)\subseteq S$ ist ein Ideal.
	\begin{itemize}
		\item[(I1)] klar, weil Bilder von Gruppen unter Gruppenhomomorphismen wieder Gruppen sind
		\item[(I2)] zu zeigen: $x\in \phi(I) , r\in S\Rightarrow rx\in \phi(I), xr\in \phi(I)$. Weil $\phi$ surjektiv $\exists r'\in R$ mit $\phi(r') = r$. Nach Anname $\exists y \in I$ mit $\phi(y) = x\Rightarrow \phi(r'y) = \phi (r')\phi(y)  = rx\in \phi(I)$, weil $y\in I$ und $I$ Ideal. Genauso $\phi(yr') = xr\in\phi(I)\Rightarrow \phi(I)$ Ideal.

		$\phi^{-1}(J)\subseteq R$ Ideal mit $\ker\phi\subseteq \phi^{-1}(J)$: $\ker\phi\subseteq \phi^{-1}(J)$, weil $\ker\phi = \phi^{-1}(\{0\})$ und $0\in J$ für jedes Ideal $J$. Klar: $\phi^{-1}(J)$ Untergruppe. Sei $r\in R, x\in\phi^{-1}(J)$; zu zeigen: $rx = xr\in\phi^{-1}(J)$. Es gilt $\phi(rx) = \phi(r)\phi(x)\in J$. Genauso $\phi(xr)\in J$. Also $rx,xr\in\phi^{-1}(J)$.
	\end{itemize}
	Sei nun $I \subseteq R$ ein Ideal mit $I \supseteq \ker\phi$ und $J \subseteq S$ ein Ideal. Es bleibt zu zeigen:
	\begin{description}
		\item[$\phi^{-1}(\phi(I)) = I$:] \leavevmode
		\begin{itemize}
			\item[\glqq $\supseteq$\grqq] $x\in I \Rightarrow \phi(x) \in \phi(I)\Rightarrow x\in \phi^{-1}(\phi(I))$
			\item[\glqq $\subseteq$\grqq] $x\in \phi^{-1}(\phi(I)) \Rightarrow \phi(x)\in \phi(I)\Rightarrow \exists y\in I$ mit $\phi(x) = \phi(y)\Rightarrow \phi(x+(-y)) = 0\xRightarrow[x+(-y)\in I]{y\in I} x\in I$
		\end{itemize}
		\item[$\phi(\phi^{-1}) = I$:] \leavevmode
		\begin{itemize}
			\item[\glqq $\supseteq$\grqq] $x\in J \xRightarrow{\phi\text{ surj}}\exists y\in R$ mit $\phi(y) = x\Rightarrow y\in\phi^{-1}(J)$ und $\phi(y) = x\Rightarrow x\in\phi(\phi^{-1}(J))$.
			\item[\glqq$\subseteq$\grqq] $x\in\phi(\phi^{-1}(J))\Rightarrow \exists y\in\phi^{-1}(J): \phi(y) = x\Rightarrow x\in J$
		\end{itemize}
	\end{description}
\end{proof}

\begin{bem} Beachte: Die Bijektion von Satz 7.9 ist inklusionserhaltend, also 
	\[
	\underbrace{I \subseteq I'}_{\mathclap{\substack{\text{Ideale in $R$,}\\\text{die $\ker\phi$ enthalten}}}} \quad \Longleftrightarrow \quad \phi(I) \subseteq \phi(I').
	\]
\end{bem}

\begin{proof}[Beweis von \cref{satz8}] %TODO: Das hier noch sauberer machen
	Setze $\phi = \can\colon R\to R/I$ (surjektiv mit $\ker\phi = I$).
	\begin{itemize}
		\item [\glqq $\Leftarrow$\grqq ] Voraussetzung: $R/I$ ist Körper.
		
		$\xRightarrow{7.7}\{0\}\subset R/I$ ist maximales Ideal.
		
		$\xRightarrow{7.9}\phi^{-1}(\{0\}) = I$ und $\phi^{-1}(R/I) = R$ sind die einzigen Ideale, die $I$ enthalten $\Rightarrow I $ maximales Ideal.
		
		\item[\glqq$\Rightarrow$\grqq] Voraussetzung: $I$ maximales Ideal
		
		$\Rightarrow$ es gibt genau $I$ und $R$ als Ideale, die $I$ enthalten
		
		$\xRightarrow{7.9}\{0\} = \phi(I)$ und $R/I = \phi(R)$ sind alle Ideale von $R/I$
		
		$\xRightarrow{7.7}R/I$ ist Körper
	\end{itemize}
\end{proof}

\begin{defi}
	Sei $R$ ein kommutativer Ring, $I\subseteq R$ Ideal. Dann heißt $I$ Primideal, falls $I\neq R$ und $\forall x, y\in R$ gilt: \[xy\in I \quad \Longrightarrow \quad x\in I\text{ oder }y\in I.\]
\end{defi}
\begin{bem}
	Falls $R$ ein Integritätsbereich ist, dann gilt: $\{0\}$ ist ein Primideal.
\end{bem}



\begin{bsp}
	$R = \IZ, a\in \IZ$. \[\text{$(a)$ Primideal.} \quad \Longleftrightarrow \quad \text{$a = \pm p$ mit $p$ prim oder $a = 0$} \]
\end{bsp}
\begin{proof}
	\leavevmode
	\begin{itemize}
		\item [\glqq $\Leftarrow$\grqq ]Sei $a = \pm p$, $p$ prim. Seien $x,y\in R$ mit $xy\in (a)$.

		$\Rightarrow$ $p$ teilt $xy$

		$\Rightarrow$ $p$ teilt $x$ oder $p$ teilt $y$

		$\Rightarrow$ $x\in (p) = (a)$ oder $y\in (p) = (a)$

		$\Rightarrow$ $(a)$ ist ein Primideal.
		\item [\glqq $\Rightarrow$\grqq ] Sei $(a) \neq (0)$ ein Primideal. Falls $a = \pm 1$, also $ (a) = R$, so ist das ein Widerspruch dazu, dass $(a)$ ein Primideal ist. Falls $a \neq \pm 1, a\neq \pm p$ mit $p$ prim, so $ \exists n, m$ mit $1<\abs{n}, \abs{m}<\abs{a}$ sodass $a = nm$. 
		Daraus folgt nun $ nm\in (a)$ und $n,m\notin (a)$, was wieder $(a)$ als Primideal widerspricht.
	\end{itemize}
\end{proof}

\begin{bem}
	Beachte: $(0)$ ist prim in $\IZ$, aber nicht maximal. In $\IZ$ gilt \glqq$I$ maximal $\Rightarrow I $ prim\grqq.
\end{bem}


\begin{satz}
	Sei $R$ ein kommutativer Ring, $R\neq\{0\}$ und $ I\subseteq R$ ein Ideal. Dann gilt
	\begin{align*} I \text{ Primideal} \quad &\Longleftrightarrow \quad R/I \text{ Integritätsbereich}. \\\intertext{Insbesondere folgt}
	 \text{$\{0\}$ Primideal} \quad&\Longleftrightarrow\quad \text{$R$ Integritätsbereich}.\end{align*}
\end{satz}

\begin{proof}
	\leavevmode
	\begin{itemize}
		\item[\glqq $\Rightarrow$\grqq] $R$ kommutativ $\Rightarrow R/I$ ist kommutativ.
		$I$ Primideal $\Rightarrow I\neq R\Rightarrow R/I \neq\{0\}$. 
		
		Seien $\overline{x}, \overline{y}\in R/I$, $\overline{x}\odot\overline{y} = \overline{0}\Rightarrow \overline{xy} = \overline{0}\Rightarrow xy\in I\Rightarrow x\in I$ oder $y\in I\Rightarrow \overline{x} = \overline{0}$ oder $\overline{y} = \overline{0}\Rightarrow R/I$ hat keine Nullteiler.
		
		\item[\glqq$\Leftarrow$\grqq] Voraussetzung: $R/I$ ist Integritätsbereich $\Rightarrow R/I \neq \{0\}\Rightarrow I \neq R$. Seien $x,y\in R$ mit $xy\in I$.
		
		$\Rightarrow \overline{x}\odot\overline{y} = \overline{xy} = \overline{y}$
		
		$\xRightarrow[\text{nullteilerfrei}]{R/I} \overline{x} = \overline{0}$ oder $\overline{y} = \overline{0}\Rightarrow x\in I$ oder $y\in I\Rightarrow I$ ist ein Primideal.
	\end{itemize}
\end{proof}

\begin{kor}
	$R\neq\{0\}$ kommutativer Ring, $I\subseteq R$ Ideal. Dann gilt: \[\text{$I$ maximal} \quad  \Longrightarrow \quad  \text{$I $ prim} \]
\end{kor}
\begin{proof}
	$I$ maximal $\xLeftrightarrow{7.8}  R/I$ Körper $\Rightarrow R/I$ Integritätsbereich $\xLeftrightarrow{7.10} I$ Primideal
\end{proof}


\begin{kor}
	Es sei $R=\IZ$.
	\begin{enumerate}
	\item Dann ist $\IZ/n\IZ$ Körper $\Leftrightarrow n = \pm p$ mit $p$ prim.
	\item  $\IZ/n\IZ$ Integritätsbereich $\Leftrightarrow n = \pm p$ und $p$ prim oder $n = 0$.
	\end{enumerate}
\end{kor}

\begin{proof}
	\leavevmode
	\begin{enumerate}
		\item Folgt aus Beispiel 7.9 und Satz 7.10.
		\item Für Ringe mit $R$ Integritätsbereich und alle Ideale sind Hauptideale, gilt: $I \neq \{0\}$ Primideal $\Rightarrow I $ maximal
		\end{enumerate}
\end{proof}	
	
\begin{satz}[Universelle Eigenschaft von Polynomringen 2]
	Sei $\phi\colon R\to S$ Ringhomomorphismus. Sei $a\in S$. Dann existiert ein eindeutiger Ringhomomorphismus
	\begin{eqnarray*}
		\ev_a\colon R[t] & \longto & S\\
		t & \longmapsto & a\\
		p(t )= \sum_{i = 0}^{\infty}b_it^i & \longmapsto& \sum_{i = 0}^{\infty} \phi(b_i)a^i
	\end{eqnarray*}
	
\end{satz}

\begin{proof}
	Da wir $\ev_a$ im Fall der Existenz bereits eindeutig charakterisiert haben, genügt es, die Existenz zu zeigen.
	\begin{align*}
		\ev_a(1_{R[t]}) &= \ev_a(1t^0) = \phi(1)\cdot a^0 = 1 \\
		\ev_a\left(\sum_{i= 0}^{\infty}b_it^i + \sum_{i = 0}^{\infty}c_it^i\right) &= \ev_a\left(\sum_{i= 0}^{\infty}(b_i+c_i)t^i\right)  = \sum_{i = 0}^{\infty} \phi(b_i+c_i)a^i \\
		&= \sum_{i = 0}^{\infty} \phi(b_i)a^i +\sum_{i=0}^{\infty}\phi(c_i)a^i=\ev_a\left(\sum_{i=0}^{\infty}b_it^i\right) + \ev_a \left(\sum_{i = 0}^{\infty} c_it^i\right) \\
		\ev_a\left(\left(\sum_{i = 0}^{\infty}b_it^i\right)\left(\sum_{i = 0}^{\infty}c_it^i\right)\right) &= \ev_a\left(\sum_{i = 0}^{\infty}\left(\sum_{k = 0}^{i} b_{i-k}c_k\right)t^i\right) = \sum_{i = 0}^{\infty} \left(\sum_{k = 0}^{i}\phi(b_{i-k})\phi(c_k)\right)a^i \\
		&= \sum_{i = 0}^{\infty} \left(\sum_{k = 0}^{i}\phi(b_{i-k})a^{i-k}\phi(c_k)a^k\right) = \left(\sum_{i = 0}^{\infty} \phi(b_i)a^i\right)\left(\sum_{i = 0}^{\infty}\phi(c_i)a^i\right)\\
		& = \ev_a\left(\sum_{i = 0}^{\infty}b_it^i\right)\cdot \ev_a\left(\sum_{i = 0}^{\infty}c_it^i\right)
	\end{align*}
	$\Rightarrow \ev_a$ ist tatsächlich ein Ringhomomorphismus.
\end{proof}

\begin{bsp}
	\begin{eqnarray*}
		\phi\colon \IR & \hookrightarrow & \IC\\%TODO: langer Einbettungspfeil (longhookrightarrow)
		x & \longmapsto & x
	\end{eqnarray*}
	Es existiert ein eindeutiger Ringhomomorphismus
	\begin{eqnarray*}
		\ev_\ima\colon \IR[t] & \longto & \IC\\
		t & \longmapsto & \ima\\
		p(t) & \longmapsto & p(i)
	\end{eqnarray*}
	Es gilt $\ev_\ima(t^2+1) = \ima^2 +1 = 0$. $\Rightarrow t^2+1\in\ker \ev_\ima\Rightarrow (t^2+1)\subset \ker \ev_\ima$. Mit dem Homomorphiesatz erhalten wir einen eindeutigen Ringhomomorphismus $\overline{\ev_\ima}\colon \IR[t]/(t^2+1)\to \IC$ mit $\overline{\ev_\ima}(\overline{b_1t+b_0}) = b_1i+b_0\Rightarrow \overline{\ev_\ima}$. Dieser ist offensichtlich surjektiv.
	
	Da jedes $\overline{p(t)}\in\IR[t]/(t^2+1)$ sich eindeutig schreiben lässt als $\overline{p(t)} = \overline{b_1t+b_0}$ gilt:
	$$\overline{\ev_\ima}(\overline{p(t)}) = \overline{\ev_\ima}(\overline{b_1t+b_0}) = b_1i +b_0 = 0\Longleftrightarrow b_1 = b_0 = 0$$
	$\Rightarrow\overline{\ev_\ima}$ ist injektiv.
	$\Rightarrow \overline{\ev_\ima}$ ist Isomorphismus von Ringen (insbesondere von Körpern).
\end{bsp}

\begin{defi}
	Ist $R$ ein kommutativer Ring, dann heißt $a\in R$ Nullstelle von $p(t) \in R[t]$, falls gilt: $p(a) := \ev_a(p(t)) = 0$ ($\phi\colon R\to R = \id \colon R\to R$).
\end{defi}


\begin{satz}
	$R$ kommutativer Ring, $a\in R$. Dann:
	\begin{enumerate}
		\item $a$ Nullstelle von $p(t) \Leftrightarrow t-a$ teilt $p(t)$ in $R[t]$.
		\item $R$ Integritätsbereich $\Rightarrow R[t]$ ist Integritätsbereich und $\deg(p(t)q(t)) = \deg (p(t)) + \deg (q(t))$
		\item $\deg (p(t)) = n, R$ Integritätsbereich $\Rightarrow$ $p(t)$ hat maximal $n$ verschiedene Nullstellen.
	\end{enumerate}
\end{satz}


%\begin{bsp}
%	Das hier ist ein Beispiel-Diagramm:

%	\begin{tikzcd}
%		X \arrow{rd}[swap]{g\circ f} \arrow{r}{f} & Y \arrow{d}{g} \\
%		W \arrow{u}	& Z \\
%	\end{tikzcd}
%\end{bsp}

\end{document}